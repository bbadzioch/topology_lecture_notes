% !TEX TS-program = lualatex
% !TEX root = mth419_lecture_notes.tex
\documentclass[letterpaper]{article}

\usepackage{amsmath, amssymb,amsthm,}
\usepackage{amsfonts}
\usepackage{amssymb}
\usepackage{rotating}
\usepackage[tracking=true]{microtype}
\usepackage[
linkcolor=black, 
citecolor=black, 
urlcolor= black, 
colorlinks=true, 
hypertexnames=false]{hyperref}


%---------------------------------------------------
%  Multiaudience
%---------------------------------------------------

\usepackage{multiaudience}

\SetNewAudience{SKELETON}
\SetNewAudience{FULL}

\NewDocumentEnvironment{SK}{+b}
  {
    \begin{shownto}{SKELETON}
      #1
    \end{shownto}
  }{}
  
\NewDocumentEnvironment{FL}{+b}
  {
    \begin{shownto}{FULL}
      #1
    \end{shownto}
  }{}


%---------------------------------------------------
%  Cross-references
%---------------------------------------------------
\usepackage{cleveref}

%---------------------------------------------------
% Page geometry
%---------------------------------------------------
\usepackage[
                      top    = 1in,
                      bottom = 1in,
                      left   = 0.8in,
                      right  = 0.8in]
                      {geometry}


%---------------------------------------------------
%  List formatting
%---------------------------------------------------
\usepackage{enumitem}
\setlist{noitemsep,topsep=5pt,parsep=0pt,partopsep= 0pt, itemindent=0pt}
\renewcommand{\labelenumi}{\textbf{\theenumi )}} % redefines enumerate labels to 1), 2) 3) etc.

\newcommand{\benu}{\begin{enumerate}}
\newcommand{\eenu}{\end{enumerate}}

\newcommand{\bitem}{\begin{itemize}}
\newcommand{\eitem}{\end{itemize}}


%---------------------------------------------------
% TiKZ
%---------------------------------------------------
\usepackage{tikz} % NOT COMPATIBLE
\usetikzlibrary{calc,through,intersections, arrows, shapes, matrix, patterns, calligraphy, positioning, math}
\usetikzlibrary {shapes.geometric}
\usetikzlibrary{decorations.pathmorphing, decorations.markings}
\usepackage{tikz-3dplot}
\usetikzlibrary{arrows.meta}
\usetikzlibrary{bending}
\usepackage{pgf, pgfplots}
\usepgflibrary{plotmarks}
\pgfplotsset{compat=1.12}
\tikzset{>=latex}



\NewDocumentEnvironment{tikzpic}{ O{artifact} O{} +b}
{
\begin{equation*}
\begin{tikzpicture}[#1, #2]
#3
\end{tikzpicture}
\end{equation*}
}
{}





%---------------------------------------------------
% Colors
%---------------------------------------------------
\definecolor{myblue}{rgb}{0,.25,.6}
\definecolor{light}{gray}{.95}
\definecolor{lines}{RGB}{60, 60, 60}
\definecolor{shadecolor}{rgb}{.95,1,1}



%---------------------------------------------------
% Fonts
%---------------------------------------------------




\usepackage{iwona}
\usepackage{iwonamath}
\usepackage{pifont}
\usepackage[no-math]{fontspec}
%\usepackage{unicode-math}

\setmainfont{iwona}


%\setmainfont{NagwaTKText}
%\setmathfont{NagwaTK Math}

%---------------------------------------------------
%  Larger fonts 
%---------------------------------------------------
\usepackage{scrextend}
\changefontsizes{14pt}


%---------------------------------------------------
% Section headings
%---------------------------------------------------

\usepackage[explicit]{titlesec} % NOT COMPATIBLE
\titleformat{\section}
  {\bfseries\selectfont}
  {}{0pt}
  {\begin{tcolorbox}[
      enhanced,
      boxrule=0pt,
      arc=0pt,
      outer arc=0pt,
      width=\textwidth,
      top=2pt,
      bottom = 2pt,
      interior code={\fill[red!70!black] (frame.north west) rectangle (frame.south east);},
    ]
    \color{white} MTH 419 \hfill  
    \thesection. 
    #1
    \end{tcolorbox}}




%---------------------------------------------------
% New section macro
%---------------------------------------------------
% include section numbers in page numbers, 
% reset page counter in each section
\renewcommand{\thepage}{\thesection-\arabic{page}}
\newcommand{\lecture}[1]{\newpage\section{#1}\setcounter{page}{1}}

%---------------------------------------------------
%  Colorboxes
%---------------------------------------------------
\usepackage{tcolorbox}
\tcbuselibrary{skins, breakable, theorems}

% Note: the name of the counter for all these environments is tcb@cnt@definition

\newtcbtheorem
  [auto counter, number within=section]% init options
  {definition}% name
  {Definition}% title
  {
    width = \textwidth,
    fonttitle = \bfseries,
    arc = 0mm,
    %toptitle = 0mm,
    beforeafter skip= 20pt,
    bottomtitle = 0mm,
    enhanced,
    boxsep=4pt,
    left=8pt,
    right=8pt,
    top=5pt,
    bottom = 5pt,
    colframe = red!70!black,
    colback  = red!10!white,
    separator sign={\ },
    label type=def
  }% options
  {def}% prefix

\crefname{def}{definiton}{definitions}

\newtcbtheorem
  [use counter from=definition, number within=section]% init options
  {thm}% name
  {Theorem}% title
  {
    width=\textwidth,
    fonttitle=\bfseries,
    arc = 0mm,
    %toptitle = 0mm,
    beforeafter skip= 20pt,
    bottomtitle = 0mm,
    enhanced,
    breakable,
    boxsep=4pt,
    left=8pt,
    right=8pt,
    top=5pt,
    bottom = 5pt,
    colframe = red!70!black,
    colback  = red!10!white,
    separator sign={\ },
    label type=thm
  }% options
  {thm}% prefix

\crefname{thm}{theorem}{theorems}

\newtcbtheorem
  [use counter from=definition, number within=section]% init options
  {lemma}% name
  {Lemma}% title
  {
    width=\textwidth,
    fonttitle=\bfseries,
    arc = 0mm,
    %toptitle = 0mm,
    beforeafter skip= 20pt,
    bottomtitle = 0mm,
    enhanced,
    breakable,
    boxsep=4pt,
    left=8pt,
    right=8pt,
    top=5pt,
    bottom = 5pt,
    colframe = red!70!black,
    colback  = red!10!white,
    separator sign={\ },
    label type=lemma
  }% options
  {lemma}% prefix

\crefname{lemma}{lemma}{lemmas}

\newtcbtheorem
  [use counter from=definition, number within=section]% init options
  {cor}% name
  {Corollary}% title
  {
    width=\textwidth,
    fonttitle=\bfseries,
    arc = 0mm,
    %toptitle = 0mm,
    beforeafter skip= 20pt,
    bottomtitle = 0mm,
    enhanced,
    breakable,
    boxsep=4pt,
    left=8pt,
    right=8pt,
    top=5pt,
    bottom = 5pt,
    colframe = red!70!black,
    colback  = red!10!white,
    separator sign={\ },
    label type=cor
  }% options
  {cor}% prefix

\crefname{cor}{corollary}{corollaries}



% Exercise environment
\newtheoremstyle{eexercise}% name of the style to be used
 {-1mm}% measure of space to leave above the theorem. E.g.: 3pt
 {-1mm}% measure of space to leave below the theorem. E.g.: 3pt
 {}% name of font to use in the body of the theorem
 {0pt}% measure of space to indent
 {\bfseries}% name of head font
 {}% punctuation between head and body
 {5pt plus 1pt minus 1pt}% space after theorem head; " " = normal interword space
 {#1 #2}% Manually specify head


\theoremstyle{eexercise}
\newtheorem*{exercise}{Exercise.}
%\newtheorem{exercise}[tcb@cnt@definition]{Exercise}



%---------------------------------------------------
%  Paragraph formatting
%---------------------------------------------------
\usepackage{parskip}
%\setlength{\parindent}{0pt}
%\setlength{\parskip}{10pt}




%---------------------------------------------------
%  Math macros
%---------------------------------------------------


% Matrices - the macro below produces matrices with evenly spaces columns
% with column spacing determined by the width of matrix entries
\usepackage{mathtools,collcell,eqparbox}

\newcounter{BMatrix}

\newlength{\maxwd}
\newcommand{\setmaxwd}[1]{%
  \eqmakebox[BM-\theBMatrix][\BMalign]{$#1$}%
}

\MHInternalSyntaxOn
\newenvironment{BMatrix}[1][r]{
  \def\BMalign{#1}
  \stepcounter{BMatrix}
  \left[
  \MH_let:NwN \@ifnextchar \MH_nospace_ifnextchar:Nnn
  \array{ #1 *{\numexpr\c@MaxMatrixCols-1} {>{\collectcell\setmaxwd}#1<{\endcollectcell}}}
  }{
  \endarray 
  \right]
}
\MHInternalSyntaxOff
\makeatother

\newcommand{\bbm}[1][r]{\begin{BMatrix}[#1]}
\newcommand{\ebm}{\end{BMatrix}}


% Number theory
\DeclareMathOperator{\lcm}{lcm}
\DeclareMathOperator{\sign}{sign}


% Groups
\newcommand{\lrang}[1]{\langle #1 \rangle}
\DeclareMathOperator{\id}{id}
\DeclareMathOperator{\Orb}{Orb}
\DeclareMathOperator{\Stab}{Stab}
\DeclareMathOperator{\Aut}{Aut}
\DeclareMathOperator{\Fix}{Fix}


% Rings and fields

\newcommand{\N}{{\mathbb N}}
\newcommand{\Z}{{\mathbb Z}}
\newcommand{\Q}{{\mathbb Q}}
\newcommand{\R}{{\mathbb R}}
\newcommand{\C}{{\mathbb C}}
\newcommand{\Poly}{{\mathbb P}}


% Vector space bases
\newcommand{\BB}{{\mathcal B}}
\newcommand{\CC}{{\mathcal C}}
\newcommand{\DD}{{\mathcal D}}
\newcommand{\EE}{{\mathcal E}}
\newcommand{\FF}{{\mathcal F}}
\newcommand{\MM}{{\mathcal M}}

% Vectors

\newcommand{\bb}{{\bf b}}
\newcommand{\cc}{{\bf c}}
\newcommand{\dd}{{\bf d}}
\newcommand{\ee}{{\bf e}}
\newcommand{\mm}{{\bf m}}
\newcommand{\nn}{{\bf n}}
\newcommand{\vv}{{\bf v}}
\newcommand{\ww}{{\bf w}}
\newcommand{\uu}{{\bf u}}
\newcommand{\xx}{{\bf x}}
\newcommand{\yy}{{\bf y}}
\newcommand{\zz}{{\bf z}}
\newcommand{\zzero}{{\bf 0}}


% Arrows
\newcommand{\ra}{\rightarrow}
\newcommand{\lra}{\longrightarrow}
\newcommand{\la}{\leftarrow}
\newcommand{\lla}{\longleftarrow}
\newcommand{\Ra}{\Rightarrow}


% Vector spaces
\newcommand{\Span}{\text{Span}}
\newcommand{\Col}{\text{Col}}
\newcommand{\Row}{\text{Row}}
\newcommand{\Nul}{\text{Nul}}
\newcommand{\Ker}{\text{Ker}}
\newcommand{\Img}{\text{Im}}
\newcommand{\rank}{\text{rank}}
\newcommand{\dist}{\text{dist}}
\newcommand{\proj}{\text{proj}}

\DeclarePairedDelimiter{\innprod}{\langle}{\rangle}


