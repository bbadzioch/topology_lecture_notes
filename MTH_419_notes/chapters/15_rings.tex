% !TEX TS-program = lualatex-dev
% !TEX root = ../mth419_lecture_notes.tex

\lecture{Rings}

\begin{definition}{}{RINGS DEF}
A \emph{ring} is set $R$ equipped with two binary operations:

\medskip

\bitem
\item \emph{addition}, denoted $a+b$ \\[-3mm]
\item \emph{multiplication}, denoted $a\cdot b$
\eitem

\medskip

satisfying the following properties:
\medskip

\benu
\item $R$ taken with addition is an abelian group (with the identity element $0\in R$).\\[-3mm]
\item Multiplication is associative: $(a\cdot b)\cdot c = a\cdot (b\cdot c)$ for any $a, b, c\in R$.\\[-3mm]
\item For any $a, b, c\in R$ we have $(a+b)c = ac + bc$ and $(b+c)a = ba + ca$. \\[-3mm]
\eenu
\end{definition}


\begin{definition}{}{COMM RINGS DEF}
We say that a ring $R$ is \emph{commutative} if $ab = ba$ for any $a, b\in R$. 

\bigskip

We  say that $R$ is a \emph{ring with unity} if there is an element $1\in R$ such that 
$1\cdot a = a\cdot 1 = a$ for all $a\in R$.
\end{definition}


\begin{FL}
\textbf{Example.}
$\Z$, $\Q$, $\R$, $\C$ with the usual addition and multiplication are commutative 
rings rings with unity. 

\textbf{Example.} $\Z_{n}$ with the addition and multiplication modulo $n$ is 
a commutative ring with unity.


\textbf{Example.} Let $M_{n}(\R)$ denote the set of $n\times n$ matrices with coefficients 
in $\R$. This is ring with addition and multiplication given by the usual addition 
and multiplication of matrices. This is a ring with unity (given by the identity matrix), 
but it is not commutative. In the same way with can define rings  $M_{n}(\Z)$,  
$M_{n}(\Q)$ and $M_{n}(\C)$  of $n\times n$ matrices with integer, rational and complex
coefficients. 


\textbf{Example.} Let $R$ be a commutative ring. By $R[x]$ we
denote the ring of polynomials with coefficients in $R$. Elements of $R[x]$ 
are polynomials of the form
\[
p(x) = a_{0} + a_{1}x + {\dots} + a_{n}x_{n}
\]
where $a_{i}\in R$ and $n\geq 0$. Addition and multiplication are the usual addition 
and multiplication of polynomials. The ring $R[x]$ is commutative if $R$ is commutative
If $R$ is a ring with unit $1\in R$ then $R[x]$ is a ring with unity given by the polynomial
$p(x) = 1$. 


\textbf{Example.} In a similar way as in the last example, from any ring $R$ we 
obtain a ring $R[x_{1}, \dots, x_{m}]$ of polynomials of $m$ variables.

\textbf{Example} Let $S$ denote the set of all polynomials
\[
p(x) = 0 + a_{1}x + a_{2}x^{2} + {\dots} + a_{n}x^{n}
\]
such that $a_{i}\in \Z$, $n\geq 1$. The set $S$ is a commutative ring with the 
usual addition and multiplication of polynomials, but it does not have a unity.
\end{FL}

\showto{SKELETON}{\newpage}
\ 
\showto{SKELETON}{\vskip 120mm}

\begin{thm}{}{RING UNITY UNIQUE}
If a ring $R$ has a unity, the the unity is unique.
\end{thm}

\begin{FL}
\begin{proof}
Assume that $1, 1'$ are two unities in $R$, so that 
\[
1\cdot a = a\cdot 1 = a \text{\ \ \ and \ \ \ } 1'\cdot a = a\cdot 1' = a 
\]
for all $a\in R$. Then $1 = 1\cdot 1' = 1'$.
\end{proof}
\end{FL}

\showto{SKELETON}{\newpage}


\begin{thm}{}{RING ZERO MULTIPL}
If a ring $R$. For any $a, b\in R$ we have:

\benu
\item $0\cdot a = a \cdot 0 = 0$. \\[-4mm]
\item $a(-b) = (-a)b = -(ab)$. \\[-4mm]
\item $(-a)(-b) = ab$ \\[-4mm]
\item if $R$ has a unity $1\in R$ then $(-1)a = a(-1) = -a$.
\eenu
\end{thm}

\begin{FL}
\begin{proof}
\textbf{1)} We have
\[
0\cdot a = (0+0)a = 0\cdot a + 0\cdot a
\]
Subtracting $0\cdot a$ from both sides we obtain $0\cdot a = 0$. By the same argument, 
$a\cdot 0 = 0$. 

\textbf{2)} We have 
\[
a(-b) + ab = a(b - b) = a\cdot 0 = 0
\]
Thus $a(-b) = -ab$. In the same way, $a(-b) = -(ab)$

\textbf{3)} Using part 2) we obtain
(-a)(-b) = -(a(-b)) = -(-(ab)) = ab


\textbf{4)} We have 
\[
0 = 0\cdot a = (1 + (-1))a = 1\cdot a + (-1)a = a + (-1)a
\]
Thus $(-1)a = -a$. Similarly, $a(-1) = -a$.
\end{proof}
\end{FL}
\showto{SKELETON}{\newpage}

\begin{definition}{}{SUBRING}
Let $R$ be a ring. A \emph{subring} of $R$ is a subset $S\subseteq R$ such that 
$S$ is a ring with respect to the addition and multiplication in $R$.  
\end{definition}

\showto{SKELETON}{\vskip 40mm}


\begin{FL}
\textbf{Example.} $\Z$ is a subring of $\Q$. 

\textbf{Example.} Let $2\Z$ denote the set of even integers. Then $2\Z$ is a subring of 
$\Z$. 
\end{FL}

\begin{thm}{}{SUBRING CRITERION}
Let $R$ be a ring. A subset $S\subseteq R$ is a subring of $R$ if and only if the 
following conditions are satisfied:
\benu
\item $0 \in S$ \\[-4mm]
\item if $a, b\in S$ then $a+b\in S$ and $ab\in S$ \\[-4mm]
\item if $a\in S$ then $(-a)\in S$ \\[-4mm]  
\eenu 
\end{thm}

\begin{definition}{}{RING DIR PRODUCT}
The \emph{direct product} of rings $R_{1}$, $R_{1}$ is a ring $R_{1}\times R_{2}$
defined as follows:
\bitem
\item Elements of $R_{1}\times R_{1}$ are ordered tuples $(a_{1}, a_{2})$ 
where $a_{i}\in R_{i}$ \\[-4mm]
\item Addition and multiplication are given by
\begin{align*}
(a_{1}, a_{2}) + (b_{1}, b_{2}) = & \  (a_{1}+b_{1},\  a_{2}+b_{2})\\
(a_{1}, a_{2}) \cdot (b_{1}, b_{2}) = & \  (a_{1}b_{1},\  a_{2}b_{2}) \\
\end{align*}
\eitem
\end{definition}

\begin{FL}
\textbf{Note.} If $R_{1}, R_{2}, \dots, R_{n}$ are rings that the direct product
$R_{1}\times R_{2}\times {\dots} R_{n}$ is defined analogously as as 
in \Cref{def:RING DIR PRODUCT}. 
\end{FL}

