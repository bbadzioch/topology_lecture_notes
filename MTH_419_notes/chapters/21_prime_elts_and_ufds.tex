% !TEX TS-program = lualatex-dev
% !TEX root = ../mth419_lecture_notes.tex

\lecture{Prime elements and UFDs}


\begin{definition}{}{RING PRIME ELT}
Let $R$ be an integral domain, and let $a, b\in R$. We say that $a$ \emph{divides} $b$
if $b = ac$ for some $c\in R$. We then write: $a\mid b$.
\end{definition}



\begin{thm}{}{ASSOCIATE DIVISION}
If $R$ is an integral domain and $a, b\in R$ are non-zero elements then $a\sim b$ 
and only if $a\mid b$ and $b\mid a$.
\end{thm}

\begin{proof}
If $a\mid b$ and $b\mid a$ then $b = ca$ and $a = db$. This gives $b = cdb$. 
Since $R$ is an integral domain, we obtain that $cd = 1$, so $c, d$ are units and 
$d = c^{-1}$. Therefore $a\sim b$. 

Conversely, if $a\sim b$ then $b = ua$ for some unit $u$. Then $a \mid b$. 
Also, $a = u^{-1}b$, so $b\mid a$.
\end{proof}


\bigskip


\textbf{Example}

\bitem
\item In $\Z$ we have: 
\[
\{\text{prime elements}\} = \{\pm \text{ prime numbers}\} = \{\text{irreducible elements}\}
\]

\item By the proof of \Cref{thm:Z MINUS SQRT FIVE NOT UFD}, in $\Z[\sqrt{-5}]$ the element 
$\alpha = 2+\sqrt{5}i$ is irreducible. On the other hand $\alpha$ is not a prime element since 
$\alpha\mid (3\cdot 3)$ but $\alpha \nmid 3$.
\eitem



\begin{thm}{}{PRIME IS IRRED}
If $R$ is an integral domain and $a\in R$ is a prime element then $a$ is irreducible. 
\end{thm}


\begin{proof}
Let $a\in R$ be a prime element and let $a= bc$. We want to show that either $b$ or $c$
must be a unit in $R$.

We have $a\mid (bc)$. Since $a$ is a prime element it implies
that $a\mid b$ or $a\mid c$. 

We can assume  that $a\mid b$.  Since also $b\mid a$, thus by \Cref{thm:ASSOCIATE DIVISION} 
we obtain that $a\sim b$, i.e. $a = bu$ for some unit $u\in R$. Therefore we have
\[
bc = a = bu
\]
Since $R$ is an integral domain, this gives $u=c$, and so $c$ is s unit.
\end{proof}


\begin{thm}{}{IRRED IS PRIME IN UFD}
If $R$ is a UFD and  $a\in R$  then $a$ is an irreducible element if and only if  
$a$ is a prime element. 
\end{thm}



\begin{proof}

($\Leftarrow$) This follows from \Cref{thm:PRIME IS IRRED}.

($\Rightarrow$) Assume that $a\in R$ is irreducible and that $a\mid (bc)$. We want 
to show that either $a\mid b$ or $a\mid c$.

If $b=0$, then $b= a\cdot 0$ so $a\mid b$. If $b$ is a unit, then $c = b^{-1}bc$ so $a\mid c$. 

As a consequence we can assume that $b, c$ are non-zero, non-units. 

Since $a\mid (bc)$ there is $d\in R$ such that $bc =ad$. Assume that $d$ is not a unit. 
Since $R$ is a UFD we have decompositions:
\[
b = b_{1}\cdot {\dots} \cdot b_{m}, \ \ \ \  
c = c_{1}\cdot {\dots} \cdot c_{n}, \ \ \ \ 
d =  d_{1}\cdot {\dots} \cdot d_{p}
\]
where $b_{i}, c_{j}, d_{k}$ are irreducible.
This gives
\[
b_{1}\cdot \dots {\cdot} b_{m}\cdot  c_{1}\cdot \dots {\cdot} c_{n} = a\cdot  d_{1}\cdot \dots {\cdot} d_{p}
\]
By the uniqueness of decomposition in UFDs this implies that  either $a\sim b_{i}$ for some $i$
or $a\sim c_{j}$ for some $j$. In the first case we get $a\mid b$, and in the second case $a\mid c$.

If $d$ is a unit the argument is similar. 
\end{proof}



\begin{thm}{}{UFD VIA PRIMES}
An integral domain $R$ is a UFD if and only if the following conditions are satisfied:

\bigskip
 
\benu
\item every non-zero, non-unit element of $R$ is a product 
of irreducible elements.\\[-3mm]
\item every irreducible element in $R$ is a prime element. 
\eenu
\end{thm}




\begin{proof}


($\Rightarrow$) This follows from the definition of UFD and 
\Cref{thm:IRRED IS PRIME IN UFD}.


($\Leftarrow$) Assume that $R$ satisfies conditions 1) - 2) of the theorem. We only need 
to show that  if $b_{1}, \dots, b_{k}$, $c_{1}, \dots, c_{l}$ are irreducible elements in $R$
such that 
\[
b_{1}\cdot {\dots} \cdot b_{k} = c_{1}\cdot {\dots} \cdot c_{l}
\]
then $k=l$, and after reordering of the factors we have 
$b_{1}\sim c_{1}$, $\dots$, $b_{k}\sim c_{k}$.

We argue by induction with respect to $k$. 
 
If $k=1$ then we have  $b_{1}= c_{1}\cdot {\dots} \cdot c_{l}$. Since $b_{1}$ is irreducible, 
this implies that $l=1$, and so $b_{1}=c_{1}$.


Next, assume that the uniqueness property holds for some $k$ and that we have 
\[
b_{1}\cdot {\dots} \cdot b_{k}\cdot b_{k+1} = c_{1}\cdot {\dots} \cdot c_{l}
\]
where $b_{i}$, $c_{j}$ are irreducible elements. This implies that 
$b_{k+1} \mid (c_{1}\cdot {\dots} \cdot c_{l})$. By condition 2) we get that $b_{k+1}$ 
is a prime element. It follows that $b_{k}\mid c_{j}$ for some $1\leq j\leq l$. We can assume that 
$b_{k+1}\mid c_{l}$. Then $c_{l} = ab_{k+1}$ for some $a\in R$. Since $c_{l}$, $b_{k+1}$
are irreducible,  $a$ must be a unit. This shows that $b_{k+1}\sim c_{l}$. Furthermore, we obtain from here that 
\[
b_{1}\cdot {\dots} \cdot b_{k}\cdot b_{k+1} = c_{1}\cdot {\dots} \cdot c_{l-1}\cdot a b_{k+1}
\]
Since $R$ is an integral domain this gives
\[
b_{1}\cdot {\dots} \cdot b_{k} = c_{1}\cdot {\dots} \cdot c_{l-1}a
\]
Since $b_{k}$ is irreducible and $a$ is a unit, the product $c_{l-1}a$ is an irreducible element. Therefore, by the inductive assumption we get that $k=l-1$, and that 
after reordering of factors we have
\[
b_{1}\sim c_{1}, \dots, \ \   b_{k-1}\sim c_{k-1}, \ \ b_{k}\sim c_{l-1}a \sim c_{l-1}
\]
\end{proof}


