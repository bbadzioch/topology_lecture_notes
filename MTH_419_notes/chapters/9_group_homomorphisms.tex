% !TEX TS-program = lualatex-dev
% !TEX root = ../mth419_lecture_notes.tex

\lecture{Homomorphisms}


\begin{definition}{}{GP HOMOMORPHISM}
Let $G$, $H$ be groups. A group homomorphism is a function 
\[
f\colon G \to H
\]
which for any $a, b \in G$ satisfies $f(a\cdot b) = f(a)\cdot f(b)$
\end{definition}


\begin{thm}{}{GP HOMOM PROPS}
Let $f\colon G \to H$ be a groups homomorphism. Then:
\bitem
\item $f(e_{G}) = e_{H}$ where $e_{G}$ and $e_{H}$ are the identity elements 
in $G$ and $H$, respectively. 
\item $f(a^{-1}) = f(a)^{-1}$ for any $a\in G$. 
\eitem 
\end{thm}

\begin{FL}
\begin{proof}
1) We have
\[
f(e_{G}) = f(e_{G}\cdot e_{G}) = f(e_{G})\cdot f(e_{G})
\]
This gives:
\[
e_{H} = f(e_{G})\cdot f(e_{G})^{-1} = (f(e_{G})\cdot f(e_{G}))\cdot f(e_{G})^{-1}
= f(e_{G}) 
\]
2) We have 
\[
e_{H} = f(e_{G}) = f(a\cdot a^{-1}) = f(a)\cdot f(a^{-1})
\]
which gives:
\[
f(a)^{-1} =  f(a)^{-1} \cdot e_{H} = f(a)^{-1}\cdot (f(a)\cdot f(a^{-1})) = f(a^{-1})
\]
\end{proof}
\end{FL}

\showto{SKELETON}{\newpage}
\showto{SKELETON}{{\bf Examples.}}

\begin{FL}
{\bf Example.} For $n\geq 2$ take the function $f\colon \Z \to \Z_{n}$ given by 
$f(k) = k \mod n$. Then $f$ is a group homomorphism. 


{\bf Example.} Recall that for $n\geq 1$, the general linear group $GL(n, \R)$ 
is a group that consists of $n\times n$ invertible matrices with matrix multiplication
as the group operation. Recall also that $\R^{\ast}$ is the group of non-zero
real numbers with multiplication. For an invertible matrix $A$ its determinant 
is a non-zero number $\det A$. Moreover, $\det AB = (\det A) \cdot (\det B)$. 
This means that the determinant defines a homomorphism of groups
\[
\det\colon GL(n, \R) \to \R^{\ast}
\]


{\bf Example.} Let $S_{n}$ be the symmetric group on $n$ letters and let 
$\sign \colon S_{n} \to \Z_{2}$ be defined by 
\[
\sign(\alpha) = 
\begin{cases}
0 & \text{is $\alpha$ is an even permutation} \\
1 & \text{is $\alpha$ is an odd permutation} \\
\end{cases} 
\]
This gives a homomorphism of groups.

{\bf Example.} For a group $G$, consider the function $f\colon G\to G$ given by 
$f(a) = a^{-1}$. In general this function is not a homomorphism. For example, 
take $G = S_{3}$, the symmetric group on 3 letters, let $\alpha, \beta \in S_{3}$
be given by  $\alpha = (1, 2)$,  $\beta = (2, 3)$. Then we have:
\begin{align*}
& f(\alpha\circ \beta) = ((1, 2)\circ (2, 3))^{-1} = (2, 3)^{-1}\circ(1, 2)^{-1} 
= (2, 3)\circ(1, 2) = (1, 3, 2) \\
& f(\alpha)\circ f(\beta) =  (1, 2)^{-1}\circ (2, 3)^{-1} = (1, 2)\circ (2, 3) = (1, 2, 3)
\end{align*}
and so $f(\alpha\circ\beta) \neq f(\alpha)\circ f(\beta)$

On the other hand, if $G$ is an abelian group then for any $a, b\in G$ we get
\[
f(ab) = (ab)^{-1} = b^{-1}a^{-1} = a^{-1}b^{-1} = f(a)f(b)
\]
Thus for an abelian group $f(a) = a^{-1}$ defines a homomorphism.

{\bf Example.} Recall that by $\R$ we denote the group of all real numbers with 
addition and by $\R^{\ast}$ the group of non-zero real numbers with multiplication. 
Define  $g \colon \R \to \R^{\ast}$ by $g(a) = 2^{a}$. For  $a, b\in \R$ we have 
\[
g(a+b) = 2^{a+b} = 2^{a}\cdot 2^{b} = g(a)\cdot g(b)
\]
This means that $g$ is a homomorphism of groups.

{\bf Example.} For any group $G$ and an element $a\in G$ there is exactly one
homomorphism $f\colon \Z \to G$ such that $f(1) = a$. This homomorphism if given 
by $f(m) = a^{m}$. 

{\bf Example.} For any group $G$ the identity function $\id_{G}\colon G\to G$, 
given by $\id_{G}(a) = a$ for all $a\in G$ is a homomorphism. 
\end{FL}

\showto{SKELETON}{\newpage}

\begin{thm}{}{GP HOMOM ORDER}
Let $f\colon G \to H$ be a homomorphism of groups and let $a\in G$. If $|a| < \infty$
then $|f(a)|$ divides $|a|$.  
\end{thm}


\begin{FL}
\begin{proof}
If $|a| = n$ then
\[
f(a)^{n} = f(a^{n}) = f(e_{G}) = e_{H}
\]
This means that $|f(a)|$ divides $n$ (see \Cref{thm:ORDER GP ELT DIVIDES}).
\end{proof}


{\bf Example.} Let $G$ be a group and let $a\in G$ be an element such that $|a| = n$. 
Then for each $k=1,2, \dots$ there is exactly one homomorphism $f\colon\Z_{kn} \to G$
such that $f(1) = a$. This homomorphism is given by $f(m) = a^{m}$.
\end{FL}

\showto{SKELETON}{\vskip 100mm}

\begin{definition}{}{GP HOMOM KER IM}
Let $f\colon G \to H$ be a group homomorphism. The \emph{kernel of $f$} is the subset
of $G$ defined by 
\[
\Ker(f) = \{ g\in G \ | \ f(g) = e\}
\]
The \emph{image of $f$} is the subset of $H$ given by
\[
\Img(f) = \{ f(g) \ | \ g \in G  \}
\]
\end{definition}

\begin{thm}{}{KER IM ARE SUBGPS}
If $f\colon G \to H$ is a homomorphism of groups then $\Ker(f)$ is a subgroup of 
$G$ and $\Img(f)$ is a subgroup of $H$. 
\end{thm}

\begin{FL}
\begin{proof}
Exercise. 
\end{proof}
\end{FL}

\showto{SKELETON}{\newpage}

\showto{SKELETON}{{\bf Examples.}}
\showto{SKELETON}{\newpage}


\begin{FL}
{\bf Example.} Let $f\colon \Z \to \Z_{n}$, $f(k) = (k \mod n)$. Then 
$\Img(f) = \Z_{n}$ and $\Ker(f) = \{ nq \ |\ q \in \Z \}$. This 
subgroup of $\Z$ is often denoted by $n\Z$. 

{\bf Example.} Take the determinant homomorphism $\det\colon GL(n, R) \to \R^{\ast}$. 
Then $\Img(\det) = \R^{\ast}$. Also, $\Ker(\det) = \{ A \in GL(n, \R) \ | \ \det A = 1 \}$
This group is called the \emph{special linear group} and it is denoted by 
$SL(n, \R)$.

{\bf Example.} For the homomorphism $\sign\colon S_{n} \to \Z_{2}$ we have 
$\Img(\sign) = \Z_{2}$ and $\Ker(\sign) = A_{n}$, where $A_{n}$ is the alternating 
group.


{\bf Example.} Let $G$ be an abelian group and let $f\colon G \to G$ be given by
$f(a) = a^{-1}$. Then $\Img(f) = G$ and $\Ker(f) = \{e\}$.

{\bf Example.} Let $g\colon \R \to \R^{\ast}$, $g(a) = 2^{a}$. Then $\Img(g) = \R^{+}$
and $\Ker(g) = \{ 0 \}$.
\end{FL}

\begin{thm}{}{KER AND 1TO1}
If $f\colon G\to H$ is a homomorphism then $f(a) = f(b)$ if and only if 
$b = ak$ for some $k\in \Ker(f)$.
\end{thm}

\begin{FL}
\begin{proof}
If $f(a) = f(b)$ then
\[
e = f(a)^{-1}f(b) = f(a^{-1})f(b) = f(a^{-1}b)
\]
so $a^{-1}b\in Ker(f)$. Taking $k = a^{-1}b$ we then get $ak = a(a^{-1}b) = b$.
Conversely, if $k\in\Ker(f)$ then $f(ak) = f(a)f(k) = f(a)e = f(a)$.
\end{proof}
\end{FL}

\showto{SKELETON}{\vskip 70mm}


\begin{cor}{}{TRIVIAL KER 1TO1}
A homomorphism of groups $f\colon G\to H$ is 1-1 if and only if $\Ker(f) = \{e\}$.
\end{cor}

\showto{SKELETON}{\vskip 40mm}

\begin{cor}{}{GP HOMOM INV IMAGE}
If $f\colon G\to H$ is a homomorphism of groups, and $f(a) = b$ for some 
$a\in G$, $b\in H$ then
\[
f^{-1}(b) = \{ ak \ | \ k\in Ker(f) \ \}
\]
\end{cor}

\begin{FL}
{\bf Note.} If $G$ is a group and $H\subseteq G$ is a subgroup, then there exists a homomorphism 
\[
f\colon K \to G
\]
such that $\Img(f) = H$ Indeed, we can take $f\colon H \to G$, $f(a) = a$. 

We will show, however that, in general, not for every subgroup $H\subseteq G$ there 
is a homomorphism $f\colon G \to K$ such that $H = \Ker(f)$. 
\end{FL}
\showto{SKELETON}{\newpage}


\begin{thm}{}{KER NORMALITY}
Let $f\colon G \to H$ is a homomorphism of groups then $g\in \Ker(f)$ if and only if 
for each $a\in G$ we have $aga^{-1}\in \Ker(f)$.
\end{thm}

\begin{FL}
\begin{proof}
Since $f(g) = e$ we obtain: 
\[
f(aga^{-1}) = f(a)f(g)f(a^{-1}) = f(a)ef(a)^{-1} = f(a)f(a)^{-1} = e
\]
and so $gag^{-1}\in \Ker(f)$.  
\end{proof}
\end{FL}

\showto{SKELETON}{\vskip 60mm}

\begin{definition}{}{NORMAL SUBGP}
Let $G$ be a group. We say that a subgroup $H\subseteq G$ is a \emph{normal subgroup}
of $G$ if for any $h\in H$ and $g\in G$ we have $ghg^{-1} \in H$. 

\medskip

We write $H \triangleleft G$ to denote that $H$ is a normal subgroup of $G$. 
\end{definition}


\begin{cor}{}{KERNEL IS NORMAL}
If $f\colon G\to H$ is a homomorphism of groups then $\Ker(f)$ is a normal subgroup
of $G$.
\end{cor}


\begin{FL}
\begin{proof}
This follows from \Cref{thm:KER NORMALITY}.
\end{proof}
\end{FL}

\showto{SKELETON}{\newpage}

\begin{FL}
{\bf Example.} $G$ is an abelian group then any subgroup $H\subseteq G$ is normal since for 
$h\in H$ and $g\in G$ we have 
\[
ghg^{-1} = gg^{-1}h = h \in H
\]

{\bf Example.} Recall that the alternating group $A_{n}$ is a subgroup of 
the symmetric group $S_{n}$ consisting of all even permutations. This subgroup is normal, 
since if $\alpha$ is an even permutation and $\beta$ is any permutation then
$\beta\circ\alpha\circ\beta^{-1}$ is an even permutation.
\end{FL}
 
{\bf Example.} Consider the dihedral group $D_{4}$:

\bigskip

\begin{center}
{\small
\tagpdfsetup{table/header-rows={1}}
\begin{tabular}{l !{\vrule width 2pt} llllllll}
\circ     & \ $I$       & $R_{90}$  & $R_{180}$ & $R_{270}$ & $H$       & $V$       & $D$       & $D'$      \\[1mm]
\noalign{\hrule height 2pt} \\[-3mm]
$I$       & \ $I$       & $R_{90}$  & $R_{180}$ & $R_{270}$ & $H$       & $V$       & $D$       & $D'$      \\
$R_{90}$  & \ $R_{90}$  & $R_{180}$ & $R_{270}$ & $I$       & $D'$      & $D$       & $H$       & $V$       \\
$R_{180}$ & \ $R_{180}$ & $R_{270}$ & $I$       & $R_{90}$  & $V$       & $H$       & $D'$      & $D$       \\
$R_{270}$ & \ $R_{270}$ & $I$       & $R_{90}$  & $R_{180}$ & $D$       & $D'$      & $V$       & $H$       \\
$H$       & \ $H$       & $D$       & $V$       & $D'$      & $I$       & $R_{180}$ & $R_{90}$  & $R_{270}$ \\
$V$       & \ $V$       & $D'$      & $H$       & $D$       & $R_{180}$ & $I$       & $R_{270}$ & $R_{90}$  \\[1mm]
$D$       & \ $D$       & $H$       & $D'$      & $V$       & $R_{270}$ & $R_{90}$  & $I$       & $R_{180}$ \\
$D'$      & \ $D'$      & $V$       & $D$       & $H$       & $R_{90}$  & $R_{270}$ & $R_{180}$ & $I$       \\
\end{tabular}
}
\end{center}

\begin{FL}
The set $\{I, V\}$ is a subgroup of $D_{4}$. However, this is not a normal subgroup 
since, for example, we have 
\[
DVD^{-1} = DVD = R_{90}D = H 
\]
and $H\not\in \{I, V\}$. 


\begin{exercise} Check that the subgroup of rotations  
$G = \{I, R_{90}, R_{180}, R_{270}\}$ is a normal subgroup of $D_{4}$. 
\end{exercise}

{\bf Note.} We will see later that for any normal subgroup $H\triangleleft G$ there
is a homomorphism $f\colon G \to K$ such that $\Ker(f) = H$. 
\end{FL}


