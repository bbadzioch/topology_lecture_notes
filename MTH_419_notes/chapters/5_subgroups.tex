% !TEX TS-program = lualatex-dev
% !TEX root = ../mth419_lecture_notes.tex


\lecture{Subgroups}

\begin{definition}{}{SUBGROUP}
Let $G$ be a group. A  subset $H\subseteq G$ is 
a \emph{subgroup} of $G$ if it is a group under the operation in $G$. 
\end{definition}

\begin{FL}
{\bf Examples.}


\bitem
\item $\Z$ and  $\Q$ are subgroups of $\R$.\\[0mm] 
\item $\Z$ is a subgroup of $\Q$.\\[0mm] 
\item Let $H\subseteq \Z$ be the set of all odd integers.
This is not a subgroup of $\Z$ since e.g. $3, 5\in H$ but 
$3+5 \not\in H$.  \\[0mm]
\item Let $H\subset GL(2, \R)$ be a set consisting of all 
invertible matrices with integer entries. Then $H$ is not 
not a subgroup of $GL(2, \R)$ since, for example, 
\[
A = 
\bbm
1 & 2 \\
3 & 4 \\
\ebm
\in H
\hskip 10mm
\text{but}
\hskip 10mm
A^{-1} = 
\bbm
-2 & 1 \\
\frac{3}{2} & -\frac{1}{2} \\
\ebm
\not\in H
\] 

\eitem 
\end{FL}

\showto{SKELETON}{\vfill}

{\bf Note.} If $G$ is a group, then the largest subgroup of $G$ is the group $G$
itself. The smallest subgroup of $G$ is the group $\{e\}$ consisting of the identity 
element of $G$ only.



\begin{thm}{}{SUBGROUP CRITERION}
Let $G$ be a group. A  subset $H\subseteq G$ is 
a subgroup of $G$ if and only if the following conditions are
satisfied:
\benu
\item[1)] The identity element $e$ belongs to $H$.
\item[2)] If $a, b\in H$ then $a\cdot b \in H$.
\item[3)] If $a\in H$ then $a^{-1}\in H$.
\eenu
\end{thm}


\newpage

{\bf Exercise.} The dihedral group $D_{4}$ has the following multiplication table:

\bigskip

\begin{center}
{\small
\tagpdfsetup{table/header-rows={1}}
\begin{tabular}{l !{\vrule width 2pt} llllllll}
\circ     & \ $I$       & $R_{90}$  & $R_{180}$ & $R_{270}$ & $H$       & $V$       & $D$       & $D'$      \\[1mm]
\noalign{\hrule height 2pt} \\[-3mm]
$I$       & \ $I$       & $R_{90}$  & $R_{180}$ & $R_{270}$ & $H$       & $V$       & $D$       & $D'$      \\
$R_{90}$  & \ $R_{90}$  & $R_{180}$ & $R_{270}$ & $I$       & $D'$      & $D$       & $H$       & $V$       \\
$R_{180}$ & \ $R_{180}$ & $R_{270}$ & $I$       & $R_{90}$  & $V$       & $H$       & $D'$      & $D$       \\
$R_{270}$ & \ $R_{270}$ & $I$       & $R_{90}$  & $R_{180}$ & $D$       & $D'$      & $V$       & $H$       \\
$H$       & \ $H$       & $D$       & $V$       & $D'$      & $I$       & $R_{180}$ & $R_{90}$  & $R_{270}$ \\
$V$       & \ $V$       & $D'$      & $H$       & $D$       & $R_{180}$ & $I$       & $R_{270}$ & $R_{90}$  \\[1mm]
$D$       & \ $D$       & $H$       & $D'$      & $V$       & $R_{270}$ & $R_{90}$  & $I$       & $R_{180}$ \\
$D'$      & \ $D'$      & $V$       & $D$       & $H$       & $R_{90}$  & $R_{270}$ & $R_{180}$ & $I$       \\
\end{tabular}
}
\end{center}

\bigskip


Find all subgroups of $D_{4}$. 

\showto{FULL}{\vskip 10mm}
\showto{SKELETON}{\newpage}


\begin{definition}{}{GROUP CENTER}
The \emph{center} of a group $G$ is a set $Z(G)\subset G$ consisting of 
elements that commute with all elements of $G$:
\[
Z(G) = \{ g\in G \ | \ ag = ga \text{ for all $a\in G$} \}
\]
\end{definition}



\begin{thm}{}{GROUP CENTER IS SUBGROUP}
If $G$ is a group then the center $Z(G)$ of $G$ is a subgroup of $G$. 
\end{thm}

\begin{FL}
\begin{proof}
1) For the identity element  $e\in G$ we have 
\[
ea = a = ae
\]
for any $a\in G$, so $e \in Z(G)$



2) Assume that $g, h\in Z(G)$. We will show that then $gh\in Z(G)$. Indeed, for any 
element $a\in G$ we have 
\[
a(gh) = (ag)h = (ga)h = g(ah) = g(ha) = (gh)a
\] 


3) Assume that $g\in Z(G)$. We need to show that then $g^{-1}\in Z(G)$. For any 
$a\in G$ we have 
\[
ag^{-1} = (ga^{-1})^{-1} = (a^{-1}g)^{-1} = g^{-1}a^{-1}
\] 

\end{proof}
\end{FL}


\showto{SKELETON}{\newpage}
\begin{exercise}
Find the center of the dihedral group $D_{4}$.
\end{exercise}

\showto{SKELETON}{\vskip 100mm}


\begin{definition}{}{GROUP CENTRALIZER}
Let $G$ a group and let $a\in G$. The \emph{centralizer} of $a$ in $G$
is the set $C(a)\subseteq G$, which consists of all elements of $G$ 
that commute with $a$:

\[
C(a) = \{ g\in G \ | \ ag = ga \}
\]
\end{definition}



{\bf Exercise.} Find the centralizer of the element $V$ in $D_{4}$.


\showto{SKELETON}{\vfill}

\begin{thm}{}{CENTRALIZER IS SUBGROUP}
If $G$ is a group and $a\in G$ then the centralizer $C(a)$ of $a$ in $G$ is a subgroup of $G$. 
\end{thm}

\begin{proof}
Similar as for \Cref{thm:GROUP CENTER IS SUBGROUP}.
\end{proof}

\showto{FULL}{\vskip 10mm}
\showto{SKELETON}{\newpage}


\begin{definition}{}{SUBGROUP GENERATED BY SET}
If $G$ is a group and $S$ is a non-empty subset of $G$, 
then $\langle S \rangle$ denotes the smallest subgroup of $G$ 
containing all elements of $S$:
\[
\langle S \rangle = 
\{ a_{1}^{k_{1}}\cdot a_{2}^{k_{2}} \dots \cdot a_{n}^{k_{n}} \ | \ 
n \geq 1 \text{ and for each $i$ we have } g_{i}\in S \text{ and } k_{i}\in \Z   
\}
\]
We say that $\langle S \rangle$ is the \emph{subgroup of G generated by the set $S$}.
\end{definition}

\vskip 5mm 

{\bf Note. } If $a\in G$ then $ \langle a \rangle = \{ a^{k} \ | \ k\in \Z \}$.

\vskip 5mm 

{\bf Exercise.} Find the subgroup $\langle 2 \rangle$ in $\Z_{10}$.

\showto{SKELETON}{\vskip 40mm}


{\bf Exercise.} Find the subgroup $\langle 2 \rangle$ in $\Z_{9}$.

\showto{SKELETON}{\vskip 40mm}

{\bf Exercise.} Find the subgroup $\langle V, R_{180} \rangle$ in $D_{4}$.


\showto{FULL}{\vskip 10mm}
\showto{SKELETON}{\newpage}


{\bf Recall:}
\bitem
\item The order of a group $G$ is the number of elements of $G$. 
It is denoted by $|G|$. \\[0mm]
\item The order of an element $a$ of a group $G$ is the smallest 
integer $n>0$ such that $a^{n} = e$. It is denoted by $|a|$. 
\eitem


\begin{thm}{}{ORDER ELT VS GROUP}
Let $G$ be a group, let $a\in G$ and let $\langle a \rangle$ be the subgroup of 
$G$ generated by $a$. Then 
\[
|a| = |\lrang{a}|
\]
\end{thm}

\begin{FL}
\begin{proof}
Assume that $|a| = n$. We will show that the group $\lrang{a}$ consists of 
$n$ distinct elements: $e = a^{0}, a^{1}, a^{2}, \dots, a^{n-1}$ and so 
$|\lrang{a}| = n$.

First, notice that all these elements are different. Indeed, if 
$0 \leq k < l < n$ and $a^{k} = a^{l}$ then $0 < l-k < n$ and 
\[
a^{l-k} = a^{l}\cdot a^{-k} = a^{k}\cdot a^{-k} = e.  
\]
This is impossible since $l-k$ is smaller than the order of $a$. 

Next, we will show that $\lrang{a}$ does not contain any elements other 
than  $a^{0}, a^{1}, \dots, a^{n-1}$. Each element of  $\lrang{a}$
is of the form $a^{k}$ for some $k\in \Z$. 
We have $k = qn + r$ for some $q, r\in \Z$, $0 \leq r < n$. 
Then $a^{k} = a^{r}$. 
\end{proof}
\end{FL}



