% !TEX TS-program = lualatex-dev
% !TEX root = ../mth419_lecture_notes.tex

\lecture{Ring homomorphisms}


\begin{definition}{}{RING HOMOM DEF}
A \emph{homomorphism} from a ring $R$ to a ring $S$ is a function $f\colon R \to S$ such 
that for any $a, b\in R$ we have
\bitem
\item $f(a + b) = f(a) + f(b)$
\item $f(ab) = f(a)\cdot f(b)$
\eitem
\end{definition}


\begin{FL}
\textbf{Note.} Since a homomorphism of rings $f\colon R \to S$ 
is a homomorphism of their additive groups, thus we have $f(0) = 0$ and 
$f(-a) = -f(a)$ for any $a\in R$. 

On the other hand if $R$ and $S$ are rings with unity, then it need not be true 
in general that $f(1) = 1$. Take for example $R = \Z$, $S = \Z\times\Z$, and let 
$f\colon \Z \to \Z\times\Z$ be given by $f(n) = (n, 0)$. Then $f$ is a ring 
homomorphism, but $f(1) = (1, 0)$ which is not the unity in $\Z\times\Z$.

To avoid such situations, usually, when working with rings with unity, it is 
additionally assumed that homomorphisms preserve the unity, $f(1) = 1$. 

\textbf{Example.} For $n > 1$ the function $f\colon \Z \to \Z_{n}$ given by 
$f(k) = k \mod n$ is a ring homomorphism.

\textbf{Example.} Let $R$ be a ring and let $a\in R$. The function 
$f\colon R[x] \to R$ defined by $f(p(x)) = p(a)$ is a homomorphism of rings. 

\textbf{Example.} If $R$ is a ring and $I \triangleright R$ then the function 
$q\colon R \to R/I$ given by $q(a) = a + I$ is a homomorphism of rings.
\end{FL}

\showto{SKELETON}{\newpage}

\begin{definition}{}{RING ISO DEF}
A \emph{isomorphism} of rings is a homomorphism $f\colon R \to S$ which is 
a bijection. 

\bigskip

If there exists an isomorphism between rings $R$ and $S$ then we say that 
these rings are \emph{isomorphic} and we write $R \cong S$.
\end{definition}

\begin{thm}{}{RING ISO INVERSE}
If $f\colon R \to S$ is an isomorphism of rings then the inverse function 
$f^{-1}\colon S \to R$ is also an isomorphism of rings. 
\end{thm}

\begin{FL}
\begin{proof}
Exercise.
\end{proof}
\end{FL}

\showto{SKELETON}{\vskip 20mm}

\begin{definition}{}{KER IM RING HOMOM}
Let $f\colon R \to S$ be a homomorphism of rings. The \emph{image} of $f$
is the set $\Img(f) \subseteq S$ defined by 
\[
\Img(f) = \{ f(r) \ | \ r \in R\}
\]
The \emph{kernel} of $f$ is the set $\Ker(f) \subseteq R$ given  by 
\[
\Ker(f) = \{ r\in R \ | \ f(r) = 0\}
\]
\end{definition}

\begin{thm}{}{KER IS IDEAL}
Let $f\colon R \to S$ be homomorphism of rings. Then 
\benu
\item $\Img(f)$ is a subring of $S$
\item $\Ker(f)$ is an ideal of $R$.
\eenu
\end{thm}

\begin{FL}
\begin{proof}\ 

\textbf{1)} Exercise. 

\textbf{2)} One can check that $\Ker(f)$ is a subring of $R$ (exercise). Also, 
if $a\in \Ker(f)$ and $r\in R$ then 
\[
f(ra) = f(r)\cdot f(a) = f(r)\cdot 0 = 0
\]
so $ra\in \Ker(f)$. Similarly, $ar\in\Ker(f)$. This show that $Ker(f)\triangleleft R$. 
\end{proof}
\end{FL}

\showto{SKELETON}{\newpage}

\begin{thm}{(First Isomorphism Theorem for Rings)}{RING FIRST ISO THM}
Let $f\colon R \to S$ be a ring which is onto. Then $S \cong R/\Ker(f)$.
\end{thm}


\begin{FL}
\begin{proof}
Define $g\colon R/\Ker(f) \to S$ by $g(a + \Ker(f)) = f(a)$. One can check that 
this is a well defined function, which gives an isomorphism of rings.
\end{proof}

\textbf{Example.} Recall that for $n> 1$ we have an onto homomorphism 
$f\colon \Z \to \Z_{n}$, $f(k) = k\mod n$. Notice that 
\[
\Ker(f) = \{nk \ | \ k\in \Z \} = n\Z
\]
This gives $\Z/n\Z \cong \Z_{n}$.


\textbf{Example.} Take the homomorphism $f\colon R[x] \to R$ defined by $f(p(x)) = p(0)$.
This homomorphism of onto, since if $a\in R$ then $r = f(p(x))$ for the polynomial 
$p(x) = a$. We have 
\begin{align*}
\Ker(f) & = \{ p(x) \ | \ p(x) = 0 \} \\
& = \{ a_{1}x + {\dots} + a_{n}x^{n} \ | \ a_{i}\in R, n \geq 0 \} \\
& = xR[x]
\end{align*}
This shows that $R[x]/xR[x] \cong R$.
\end{FL}

\showto{SKELETON}{\vskip 100mm}



\begin{thm}{}{EVERY IDEAL IS KER}
If $R$ is a ring and $I\triangleleft R$ then there exists a ring homomorphism
$f\colon R \to S$ such that $\Ker(f) = I$.  
\end{thm}

\begin{FL}
\begin{proof}
Take $S = R/I$ and $f\colon R \to R/I$ defined by $f(a) = a + I$.
\end{proof}
\end{FL}





