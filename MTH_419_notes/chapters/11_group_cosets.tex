% !TEX TS-program = lualatex-dev
% !TEX root = ../mth419_lecture_notes.tex

\lecture{Cosets and Lagrange theorem}

\begin{definition}{}{GP COSET}
Let $G$ be a group and $H\subseteq G$ a subgroup. For $a\in G$ the 
\emph{left coset of $H$ in $G$ containing $a$} is the subset of 
$G$ given by 
\[
aH = \{ ah \ | \ h\in H\}
\]
Similarly, the \emph{right coset of $H$ in $G$ containing $a$} is the subset
\[
Ha = \{ ha \ | \ h\in H\}
\]
\end{definition}

{\bf Example.} Consider the group $D_{4}$:

\begin{center}
{\small
\tagpdfsetup{table/header-rows={1}}
\begin{tabular}{l !{\vrule width 2pt} llllllll}
\circ     & \ $I$       & $R_{90}$  & $R_{180}$ & $R_{270}$ & $H$       & $V$       & $D$       & $D'$      \\[1mm]
\noalign{\hrule height 2pt} \\[-3mm]
$I$       & \ $I$       & $R_{90}$  & $R_{180}$ & $R_{270}$ & $H$       & $V$       & $D$       & $D'$      \\
$R_{90}$  & \ $R_{90}$  & $R_{180}$ & $R_{270}$ & $I$       & $D'$      & $D$       & $H$       & $V$       \\
$R_{180}$ & \ $R_{180}$ & $R_{270}$ & $I$       & $R_{90}$  & $V$       & $H$       & $D'$      & $D$       \\
$R_{270}$ & \ $R_{270}$ & $I$       & $R_{90}$  & $R_{180}$ & $D$       & $D'$      & $V$       & $H$       \\
$H$       & \ $H$       & $D$       & $V$       & $D'$      & $I$       & $R_{180}$ & $R_{90}$  & $R_{270}$ \\
$V$       & \ $V$       & $D'$      & $H$       & $D$       & $R_{180}$ & $I$       & $R_{270}$ & $R_{90}$  \\[1mm]
$D$       & \ $D$       & $H$       & $D'$      & $V$       & $R_{270}$ & $R_{90}$  & $I$       & $R_{180}$ \\
$D'$      & \ $D'$      & $V$       & $D$       & $H$       & $R_{90}$  & $R_{270}$ & $R_{180}$ & $I$       \\
\end{tabular}
}
\end{center}

\begin{FL}
Take the subgroup $K = \{I, H\}$ of $D_{4}$. Here are some left and right cosets of 
$K$ in $D_{4}$:
\begin{center}
\tagpdfsetup{table/tagging=presentation}
\begin{tabular}{lll}
$R_{90} K =  \{R_{90}, D'\}$ &\ \ \ & $K  R_{90} =  \{R_{90}, D \}$ \\
$D'K = \{D', R_{90}\}$       &\ \ \ & $K  D' =  \{D',  R_{270} \}$ \\
$DK =  \{ D, R_{270} \}$     &\ \ \ & $KD = \{D, R_{90} \}$ \\
\end{tabular}
\end{center}
Notice that: 
\vskip -12mm
\ 
\bitem
\item Cosets defined by different elements may be the same. E.g. $R_{90}K = D'K$. 
\item Left coset of a given element may be different that the right coset. 
For example, $R_{90}K\neq KR_{90}$.
\eitem
\end{FL}
\showto{SKELETON}{\newpage}

\begin{thm}{}{GP COSET PROPS}
Let $G$ be a group, $H\subseteq G$ a subgroup, and let $a, b\in G$. Then:
\benu
\item $a\in aH$. \\[-5mm]
\item either $aH = bH$ or $aH\cap bH = \varnothing$. \\[-5mm]
\item $aH = bH$ if and only if $a^{-1}b \in H$. \\[-5mm]
\item $|aH| = |H|$, where $|aH|$ denotes the number of elements in $aH$. 
\eenu 
Analogous properties hold for right cosets.
\end{thm}



\begin{FL}
\begin{proof}\ 


\textbf{1)} Since $e\in H$ thus $a = ae \in aH$.
 
\textbf{2)} Assume that $aH \cap bH \neq \varnothing$ and let $g\in aH \cap bH$. Then 
$ah_{1} = g = bh_{2}$
Then for $h \in H$ we have 
\[
ah = ah_{1}(h_{1}^{-1}h) = bh_{2}(h_{1}^{-1}h) \in bH
\]
This shows that $aH \subseteq bH$. By a similar argument $bH \subseteq aH$, so 
$aH = bH$. 

\textbf{3)} If $aH = bH$ then $b = ah$ for some $h\in H$, so $a^{-1}b = h \in H$. 
Conversely, if $a^{-1}b = h \in H$ then $b \in aH \cap bH$. By part 2) this gives 
$aH = bH$. 

\textbf{4)} It is enough to notice that the function $f\colon H \to aH$, 
$f(h) = ah$ is a bijection. 
\end{proof}
\end{FL}


\showto{SKELETON}{\newpage}


\begin{definition}{}{GP COSETS SET}
For a group $G$ and a subgroup $H\subseteq G$
by $G/H$ we denote the set of left cosets of $H$ in $G$ and by 
$H{\,}\reflectbox{/} G$ we denote the set of right cosets. 
\end{definition}



\begin{FL}
{\bf Example.} Cosets of $K = \{I, H\}$ in $D_{4}$:





\bigskip

\begin{center}
\begin{minipage}[t]{0.15\textwidth}
\begin{center}
\tagpdfsetup{table/tagging=presentation}
\begin{tabular}{!{\vrule width 1pt} ll !{\vrule width 1pt}}
\noalign{\hrule height 1pt}
$I$ & $H$ \\
\noalign{\hrule height 1pt}
$R_{90}$ & $D'$ \\
\noalign{\hrule height 1pt}
$R_{180}$ & $V$ \\
\noalign{\hrule height 1pt}
$R_{270}$ & $D$ \\
\noalign{\hrule height 1pt}
\end{tabular}
\bigskip

\emph{$G/H$ \\ left cosets}
\end{center}
\end{minipage}
\hskip 20mm
\begin{minipage}[t]{0.15\textwidth}
\begin{center}
\tagpdfsetup{table/tagging=presentation}
\begin{tabular}{!{\vrule width 1pt} ll !{\vrule width 1pt}}
\noalign{\hrule height 1pt}
$I$ & $H$ \\
\noalign{\hrule height 1pt}
$R_{90}$ & $D$ \\
\noalign{\hrule height 1pt}
$R_{180}$ & $V$ \\
\noalign{\hrule height 1pt}
$R_{270}$ & $D'$ \\
\noalign{\hrule height 1pt}
\end{tabular}
\bigskip

\emph{$H{\,}\reflectbox{/} G$ \\ right cosets}
\end{center}
\end{minipage}
\end{center}
\end{FL}



\showto{SKELETON}{\vskip 90mm}


\begin{thm}{}{NUM COSETS LEFT EQ RIGHT}
If $G$ is a group and $H\subseteq G$ is a subgroup, then $|G/H| =|H{\,}\reflectbox{/} G|$. 
\end{thm}

\begin{FL}
\begin{proof}
The function $f\colon G/H \to H {\,}\reflectbox{/} G$ given by $f(aH) = Ha^{-1}$ is a bijection.
\end{proof}
\end{FL}

\showto{SKELETON}{\vskip 50mm\newpage}


\begin{definition}{}{SUBGP INDEX}
If $G$ is a group and $H\subseteq G$ is a subgroup then the \emph{index}
of $H$, denoted $[G:H]$, is the number of left cosets of $H$ in $G$ 
(or, equivalently, the number of right cosets):
\[
[G:H] = |G/H| =  | H {\,}\reflectbox{/} G|
\]
\end{definition}


\begin{FL}
{\bf Example.} If $K = \{I, H\}\subseteq D_{4}$ then $[D_{4}: K] = 4$.
\end{FL}
\showto{SKELETON}{\vskip 40mm}
 
\begin{thm}{(Lagrange Theorem)}{LAGRANGE THM}
If $G$ is a finite group and $H\subseteq G$ is a subgroup then 
\[
|G| = [G:H]\cdot |H|
\]
\end{thm}

\begin{FL}
\begin{proof}
By \Cref{thm:GP COSET PROPS} each element of $G$ belongs to exactly one left coset 
of $H$. Thus, if $a_{1}H,\ a_{2}H, \dots, a_{n}H$ are all distinct cosets, then 
\[
|G| = |a_{1}H| + |a_{2}H| + {\dots} + |a_{n}H|
\] 
Moreover, since each coset consists of $|H|$ elements and there are $[G\colon H]$
cosets, we obtain that $|G| = [G:H]\cdot |H|$. 
\end{proof}
\end{FL}
\showto{SKELETON}{\vfill}

\begin{cor}{}{SUBGP ORD DIV GP}
If $G$ is a finite group and $H\subseteq G$ is a subgroup then the order of 
$H$ divides the order of $G$.  
\end{cor}

\showto{SKELETON}{\newpage}

\begin{cor}{}{ELT ORD DIV GP}
If $G$ is a finite group and $a\in G$ then the order $|a|$ of $a$ divides the order of $G$.
\end{cor}

\begin{FL}
\begin{proof}
Recall that by \Cref{thm:ORDER ELT VS GROUP} we have $|a| = |\langle a \rangle|$
where $\langle a \rangle$ is the subgroup of $G$ generated by $a$. Also, by 
\Cref{cor:SUBGP ORD DIV GP}, $|\langle a \rangle|$ divides $|G|$.

\end{proof}
\end{FL}

\begin{FL}
{\bf Note.} It is not true that if $G$ is a group and $k$ divides $|G|$ then 
$G$ contains an element of order $k$. Take for example the symmetric group $S_{4}$. 
By looking at possible disjoint cycle decompositions of elements of $S_{4}$, we 
can see that every element of $S_{4}$ has order 1,2,3 or 4. This means that $S_{4}$
does not contain any element of order 6, even though 6 divides the order of $S_{4}$.


{\bf Note.} It is also not true that if $k$ divides the order of a group $G$, then 
$G$ contains a subgroup $H$ of order $k$. We will see an example of that later.   


{\bf Example.} Let $G$ be a group of order $p$ where $p$ is a prime number. 
Then every element of $G$ is of order either 1 (i.e. it is the identity element)
or $p$. Thus if $a\in G$  and $a\neq e$ then $|a| = |G|$. This means 
that $G$ is a cyclic group generated by $a$, and so $G\cong \Z_{p}$. 


{\bf Example.} We will show if $G$ is a group of order $4$, then $G$ is isomorphic 
either to $\Z_{4}$ or to $\Z_{2}\oplus\Z_{2}$. By \Cref{cor:ELT ORD DIV GP}, if 
$a\in G$ then $|a|=1$ (which means that $a=e$), $|a|=2$, or $|a|=4$. If 
$G$ contrains an element of order $4$, then it is cyclic, and so $G\cong\Z_{4}$. 
Otherwise, $G$ contains the trivial element $e$ and three elements, (which we will 
denote $a, b, c$) of order 2. Notice that $ab = ba = c$ (since $ab = b$ would give  
$a=e$,  $ab = a$ would give $b=e$, and $ab = e = aa$ would imply that $b=a$). 
Similarly, we obtain that $ac = ca = b$ and $bc = bc = a$. This shows that the function 
$f\colon G \to \Z_{2}\oplus\Z_{2}$, given by $f(a) = (1, 0)$, $f(b) = (0, 1)$ and 
$f(c) = (1, 1)$ is an isomorphism of groups.
\end{FL}
