% !TEX TS-program = lualatex-dev
% !TEX root = ../mth419_lecture_notes.tex

\lecture{Ideals and quotient rings}

\begin{definition}{}{IDEALS}
Let $R$ be a ring. An \emph{(both sided) ideal} of $R$ is a subring
$I\subseteq R$ such that for any $a\in I$ and $r\in R$ we have
$ra\in I$ and $ar\in I$. 

\bigskip

We write $I \triangleleft R$ to denote that $I$ is an ideal of $R$.
\end{definition}

\begin{FL}
\textbf{ Example.} For $n\geq 1$ let $n\Z$ denote the set of integers that 
are multiples of $n$: 
\[
n\Z = \{ nk \ | \ k\in \Z \}
\]
Then $n\Z$ is an  ideal of $\Z$. 

In general, if $R$ is a commutative ring and $a\in R$ then define
\[
aR = \{ ar \ | \ r \in R\}
\]
Ideals of this form are called \emph{principal ideals}.

\textbf{ Example.} Let $I, J$ be ideals of a ring $R$. Define
\[
I + J = \{ a + b \ | \ a \in I,\ b\in J \}
\]
Then $I+J$ is an ideal of $R$. Notice that $I, J \subseteq I + J$. 


\textbf If $R$ is a commutative ring and $a_{1}, a_{2}, \dots, a_{n}\in R$
then denote:
\[
\lrang{a_{1}, a_{2}, \dots a_{n}} = a_{1}R + a_{2}R +{\dots} + a_{n}R 
\]
We say that $\lrang{a_{1}, a_{2}, \dots a_{n}}$ is the ideal 
\emph{generated by the elements $a_{1}, a_{2}, \dots a_{n}$}. 
This ideal consists of all elements of $R$ of the form 
\[
a_{1}r_{1} + a_{2}r_{2} + {\dots} + a_{n}r_{n}
\]
for any $r_{1}, r_{2}, \dots, r_{n}\in R$. This is the smallest ideal of 
$R$ containing $a_{1}, a_{2}, \dots, a_{n}$.


\textbf{Example.} The ring $\Z$ is a subring of $\Q$, but it is not 
an ideal of $\Q$.


\textbf{Example.} Let $R[x]$ be the ring of polynomials with coefficients on 
a ring $R$. Define 
\[
I = \{ 0 + a_{1}x + {\dots} + a_{n}x^{n} \ | \ a_{i}\in R,\ n\geq 0 \}
\]
Then $I$ is an ideal of $R[x]$. Notice that this ideal is a principal ideal:
$I = xR[x]$.

\textbf{Example.} In any ring $R$ the smallest ideal is the ideal $\{ 0 \}$, and 
the largest ideal is $R$. 

\textbf{Example} If $R$ is a ring with unity and $I \triangleleft R$ is an ideal 
such that $1\in I$ then $I = R$. Indeed, for any $a\in R$ we have $a = a\cdot 1 \in I$.


\textbf{Example.} If $F$ is a field and $I \triangleleft F$ then either $I = \{0\}$
or $I = F$. Indeed, assume that there is some $a\neq 0$ such that $a\in I$. 
Then for any $b\in F$ we have $b = (ba^{-1})a$ and so $b\in I$. 


\textbf{Note.} Let $R$ be a ring and $S$ be a subring. Since $R$, taken with addition 
is an abelian group and $S$ is its normal subgroup, we can consider its 
quotient group $R/S$. The elements of $R/S$ are left cosets $a+S$ for $a\in R$ 
and addition is given by $(a + S) + (b + S) = (a + b) + S$.
\end{FL}

\showto{SKELETON}{\newpage\  \vskip 100mm}


\begin{thm}{}{QUOTIENT RING MULTIPLICATION}
Let $R$ be a rings and $I \triangleleft R$. Let $a_{1}, a_{2}, b_{1}, b_{2}\in R$
be elements such that $a_{1} + I = a_{2} + I$ and $b_{1} + I = b_{2} + I$. Then 
$(a_{1}b_{1}) + I = (a_{2}b_{2}) + I$.  
\end{thm}
 
 
\begin{FL} 
\textbf{Note.} \Cref{thm:QUOTIENT RING MULTIPLICATION} is not true if $I$ is a subring 
of $R$ which is not an ideal. Take for example $R = \Q$ and $I = \Z$. 
Let $a_{1} = \frac{1}{2}$, $a_{2} = \frac{1}{2}$, $b_{2} = 0$, $b_{2} = 1$. 
We have $a_{1} + \Z = a_{2} + \Z$ and $b_{1} + \Z = b_{2} + \Z$.
However, $a_{1}b_{1} + \Z = 0 + \Z$ and $a_{2}b_{2} = \frac{1}{2} + \Z$, so 
$a_{1}b_{1} + \Z  \neq a_{2}b_{2} + \Z$.
  
\begin{proof}[Proof of \Cref{thm:QUOTIENT RING MULTIPLICATION}]
If $a_{1} + I = a_{2} + I$ then $a_{2} = a_{1} + r$ for some $r\in I$. 
Similarly, $b_{2} = b_{1} + r'$ for some $r'\in R$ This gives 
\[
a_{2}b_{2} = (a_{1} + r)(a_{2} + r') = a_{1}a_{2} + a_{1}r' + ra_{2} + rr'
\]
Since $I$ is an ideal, thus $a_{1}r' + ra_{2} + rr' \in I$, and so 
$a_{2}b_{2} + I = a_{1}b_{1} + I$.
\end{proof}
\end{FL}

\showto{SKELETON}{\newpage}

\begin{definition}{}{QUOTIENT RING DEF}
Let $R$ be a ring and let $I\triangleleft R$. The \emph{quotient ring}
$R/I$ is the ring defined as follows:
\bitem
\item Elements of $R/I$ are cosets $a + I$ for $a\in R$. 
\item Addition: $(a + R) + (b + R) = (a+b) + R$.
\item Multiplication: $(a + R) \cdot (b + R) = (ab) + R$.
\eitem 
\end{definition}

\begin{FL}
\textbf{Note.} Let $R$ be a ring and $I\triangleleft R$. If $R$ is commutative 
then the quotient ring $R/I$ is commutative. If $R$ is a ring with unity 
$1\in R$ then $R/I$ is a ring with unity $(1 + I) \in R/I$.
\end{FL}

\showto{SKELETON}{\vskip 30mm}

\begin{definition}{}{PRIME IDEAL}
Let $R$ be a commutative ring with unity. A \emph{prime ideal} of $R$ is
an ideal $I \triangleleft R$ such that $I\neq R$ and if $ab\in I$ then 
either $a\in I$ or $b\in I$. 
\end{definition}

\begin{FL}
\textbf{Example.} Take the ring of integers $\Z$. Recall that 
\[
n\Z = \{ nk \ | \ k\in \Z \}
\]
If $p$ is a prime number, then $p\Z$ is a prime ideal. Indeed, if $ab\in p\Z$
then $ab = pk$ for some $k\in \Z$. This means that $p$ divides $ab$, and so 
it must divide either $a$ (in which case $a\in p\Z$) or $b$ (and then $b\in p\Z$). 

On the other if $n > 0$ is not a prime, then $n = km$ for some $0 < k, m <n$. 
This gives $k \not\in nZ$, $m \not\in n\Z$ but $km = n \in n\Z$, so $n\Z$ is not 
a prime ideal.

\textbf{Example.} The zero ideal $\{ 0 \} \triangleleft R$ is prime if and only if 
$R$ is an integral domain. Indeed, if $\{0\}$ is not prime if and only if there 
are non-zero elements $a, b\in R$ such that $ab = 0$. 
\end{FL}

\showto{SKELETON}{\newpage}


\begin{thm}{}{PRIME ID QUOT IS INT DOMAIN}
If $R$ is a commutative ring with unity and $I \triangleleft R$ then 
$I$ is a prime ideal if and only if $R/I$ in an integral domain.  
\end{thm}

\begin{FL}
\begin{proof}
Assume that $R/I$ is not an integral domain. Then there are $a, b\in R$
such that $a + I \neq 0 + I$ and $b + I\neq 0 + I$ but $ab + I = 0 + I$. 
This means that $a\not\in I$, $b\not\in I$ but $ab\in I$, which shows that 
$I$ is not a prime ideal. 

The prove of the converse is similar. 
\end{proof}
\end{FL}

\showto{SKELETON}{\newpage}



\begin{definition}{}{MAX IDEAL}
Let $R$ be a commutative ring with unity. A \emph{maximal ideal} of $R$ is
an ideal $I \triangleleft R$ such that $I\neq R$ if $J \triangleleft R$ is any 
other ideal such that $J\neq R$ and $I\subseteq J$, then $I = J$. 
\end{definition}

\begin{FL}
\textbf{Example.} The ideal $4\Z \triangleleft \Z$ is not maximal since 
$4\Z \subseteq 2\Z$. 


\textbf{Example.} Take the ring $\Z$. We will show that if $p$ is a prime then $p\Z$
is a maximal ideal. Indeed, assume that there is some ideal $J \triangleleft \Z$
such that $p\Z \subseteq J$. Let $a \in J$ and $a\not\in p\Z$. 
Then $a$ is not divisible by $p$, so $\gcd(p, a) = 1$. As a consequence 
we can find $k, l \in\Z$ such that $pk + al = 1$. Since $p, a \in J$, 
this gives $1\in J$, and so $J = \Z$.
\end{FL}


\showto{SKELETON}{\vskip 50mm}


\begin{thm}{}{MAX ID QUOT IS FIELD}
f $R$ is a commutative ring with unity and $I \triangleleft R$ then 
$I$ is a maximal ideal if and only if $R/I$ in a field.  
\end{thm}


\begin{FL}
\begin{proof} Assume that $R/I$ is a field and let $J \triangleleft R$ be an 
ideal such that $I\subseteq J$ but $I \neq J$. We will show that $J = R$. 
Indeed, take $a\in J$ such that $a\not\in I$. Then $a + I$ is a non-zero 
element in $R/I$, so it is a unit in $R/I$. In other words, there is $b\in R$ such 
that $(ab + I) = (a + I)(b + I) = 1 + I$. This means that 
$1 = ab + c$ for some $c\in I$. Since $a\in J$ and $c \in J$ we obtain 
that $1 \in J$, and so $J = R$.  

Conversely, assume that $I$ is a maximal ideal and let $a + I \in R/I$
be a non-zero element. Take $J = I + aR$. Then $I \subseteq J$ and $I \neq J$
since $a\not\in I$. By maximality of $I$ we obtain that $J = R$. This means 
that $1\in J$, so $1 = c + ab$ for some $c\in I$ and $b\in R$. This gives:
\[
1 + I = ab + I = (a + I)(b + I)
\]
so $a + I$ is a unit in $R/I$ with $(a + I)^{-1} = b+ I$.
\end{proof}
\end{FL}

\showto{SKELETON}{\newpage}



\begin{thm}{}{MAX IDEAL ARE PRIME}
If $R$ is a commutative ring with unity then every maximal ideal of $R$ is a prime 
ideal.
\end{thm}


\begin{FL}
\begin{proof}
If $I \triangleleft R$ is a maximal ideal then $R/I$ is a field
by \Cref{thm:MAX ID QUOT IS FIELD},  and so by \Cref{thm:FIELDS ARE INT DOM} it 
is an integral domain. Then, \Cref{thm:PRIME ID QUOT IS INT DOMAIN} implies that 
$I$ is a prime ideal.   
\end{proof}


\textbf{Example.} In the ring $\Z[x]$ define 
\[
I = \{ p(x)\in \Z[x] \ | \ p(0) = 0\}
\] 
This is an ideal of $\Z[x]$. We claim that $I$ is a prime ideal. 
Indeed, if $p(x), q(x) \in I$ are polynomials such that $p(x)q(x) \in I$, 
then $p(0)\cdot q(0) = 0$, so either $p(0) = 0$ or $q(0) = 0$. This means that 
either $p(x)\in I$ or $q(x) \in I$. 

On the other hand, $I$ is not a maximal ideal. Take, for example 
\[
J = \{ p(x)\in \Z[x] \ | \ p(0) \text{ is an even number}\}
\]
One can check that $J$ is an ideal of $\Z[x]$. We have $I \subseteq J$, 
but $J \neq \Z[x]$, since e.g. $p(x) = 1+x \not\in J$. 
\end{FL}



