% !TEX TS-program = lualatex-dev
% !TEX root = ../mth419_lecture_notes.tex

\lecture{Permutation groups}


\begin{definition}{}{PERMUTATION DEF}
A \emph{permutation} of a set $A$ is a function $f\colon A \to A$
which is a bijection. 
\end{definition}

\vskip 10mm

\begin{tikzpic}
\draw[rounded corners=10pt] (-0.25,0) rectangle (1.25,3);
\draw[rounded corners=10pt] (2.75,0) rectangle (4.25,3);
\foreach \x/\y in {0/d, 1/c, 2/b, 3/a}{
\fill (0.5, {0.5 + (2/3)*\x}) circle (0.1) node[left] {\small $\y$};
\fill (3.5, {0.5 + (2/3)*\x}) circle (0.1) node[right] {\small $\y$};
}
\draw[line width = 1, ->, >=latex, shorten <= 2mm, shorten >= 2mm] 
(0.5, {0.5 + (2/3)*0}) -- (3.5, {0.5 + (2/3)*3});
\draw[line width = 1, ->, >=latex, shorten <= 2mm, shorten >= 2mm] 
(0.5, {0.5 + (2/3)*1}) -- (3.5, {0.5 + (2/3)*0});
\draw[line width = 1, ->, >=latex, shorten <= 2mm, shorten >= 2mm] 
(0.5, {0.5 + (2/3)*2}) -- (3.5, {0.5 + (2/3)*2});
\draw[line width = 1, ->, >=latex, shorten <= 2mm, shorten >= 2mm] 
(0.5, {0.5 + (2/3)*3}) -- (3.5, {0.5 + (2/3)*1});
\node[anchor= east] at (-0.25, 2.65) {\small $A$};
\node[anchor= west] at (4.25, 2.65) {\small $A$};
\node[anchor = south] at (2, 2.5) {\small $f$};
\end{tikzpic}


\begin{FL}
{\bf Note.} For every permutation $f$ we have the inverse function $f^{-1}$
such that $f\circ f^{-1} (x) = x$ and  $f^{-1}\circ f (x) = x$ for all $x\in A$.

\begin{tikzpic}
\draw[rounded corners=10pt] (-0.25,0) rectangle (1.25,3);
\draw[rounded corners=10pt] (2.75,0) rectangle (4.25,3);
\foreach \x/\y in {0/d, 1/c, 2/b, 3/a}{
\fill (0.5, {0.5 + (2/3)*\x}) circle (0.1) node[left] {\small $\y$};
\fill (3.5, {0.5 + (2/3)*\x}) circle (0.1) node[right] {\small $\y$};
}
\draw[line width = 1, <-, >=latex, shorten <= 2mm, shorten >= 2mm] 
(0.5, {0.5 + (2/3)*0}) -- (3.5, {0.5 + (2/3)*3});
\draw[line width = 1, <-, >=latex, shorten <= 2mm, shorten >= 2mm] 
(0.5, {0.5 + (2/3)*1}) -- (3.5, {0.5 + (2/3)*0});
\draw[line width = 1, <-, >=latex, shorten <= 2mm, shorten >= 2mm] 
(0.5, {0.5 + (2/3)*2}) -- (3.5, {0.5 + (2/3)*2});
\draw[line width = 1, <-, >=latex, shorten <= 2mm, shorten >= 2mm] 
(0.5, {0.5 + (2/3)*3}) -- (3.5, {0.5 + (2/3)*1});
\node[anchor= east] at (-0.25, 2.65) {\small $A$};
\node[anchor= west] at (4.25, 2.65) {\small $A$};
\node[anchor = south] at (2, 2.5) {\small $f^{-1}$};
\end{tikzpic}
\end{FL}


\showto{SKELETON}{\vfill}


\begin{definition}{}{PERMUTATION GP DEF}
Let $A$ be a set. The \emph{permutation group} of $A$ is a group $S(A)$
defined as follows:
\bitem
\item {\bf Elements of $S(A)$:} permutations $f\colon A \to A$.
\item {\bf Group operation:} composition of functions $g\circ f$. 
\item {\bf The identity element:} the function $\varepsilon\colon A \to A$, 
$\varepsilon(x) = x$ for all $x\in A$.  
\item {\bf The inverse of $f$:} the inverse permutation $f^{-1}$.   
\eitem
\end{definition}

\newpage


\begin{definition}{}{SN GP DEF}
For $n \geq 1$ the group $S_{n}$ is the group of permutations of the set 
$A = \{1, 2, \dots, n\}$. This group is called the \emph{symmetric group 
on $n$ letters}.
\end{definition}


\vskip 5mm

{\bf Matrix notation of permutations:}
\begin{tikzpic}[alt={Permutation 1 to 3, 2 to 2, 3 to 4, 4 to 1}]
\draw[rounded corners=10pt] (-0.25,0) rectangle (1.25,3);
\draw[rounded corners=10pt] (2.75,0) rectangle (4.25,3);
\foreach \x/\y in {0/4, 1/3, 2/2, 3/1}{
\fill (0.5, {0.5 + (2/3)*\x}) circle (0.1) node[left] {\small $\y$};
\fill (3.5, {0.5 + (2/3)*\x}) circle (0.1) node[right] {\small $\y$};
}
\draw[line width = 1, ->, >=latex, shorten <= 2mm, shorten >= 2mm] 
(0.5, {0.5 + (2/3)*0}) -- (3.5, {0.5 + (2/3)*3});
\draw[line width = 1, ->, >=latex, shorten <= 2mm, shorten >= 2mm] 
(0.5, {0.5 + (2/3)*1}) -- (3.5, {0.5 + (2/3)*0});
\draw[line width = 1, ->, >=latex, shorten <= 2mm, shorten >= 2mm] 
(0.5, {0.5 + (2/3)*2}) -- (3.5, {0.5 + (2/3)*2});
\draw[line width = 1, ->, >=latex, shorten <= 2mm, shorten >= 2mm] 
(0.5, {0.5 + (2/3)*3}) -- (3.5, {0.5 + (2/3)*1});
\node[anchor = south] at (2, 2.5) {\small $\alpha$};
\end{tikzpic}

\begin{FL}
\[
\alpha = 
\bbm
1 & 2 & 3 & 4 \\
3 & 2 & 4 & 1 \\
\ebm
\phantom{\alpha =}
\]
\end{FL}

\showto{SKELETON}{\vskip 20mm}


{\bf Composition:}

\begin{tikzpic}
\draw[rounded corners=10pt] (-0.25,0) rectangle (1.25,3);
\draw[rounded corners=10pt] (2.75,0) rectangle (4.25,3);
\draw[rounded corners=10pt] (5.75,0) rectangle (7.25,3);
%
\foreach \x/\y in {0/4, 1/3, 2/2, 3/1}{
\fill (0.5, {0.5 + (2/3)*\x}) circle (0.1) node[left] {\small $\y$};
\fill (3.5, {0.5 + (2/3)*\x}) circle (0.1) node[right] {\small $\y$};
\fill (6.5, {0.5 + (2/3)*\x}) circle (0.1) node[right] {\small $\y$};
}
\draw[line width = 1, ->, >=latex, shorten <= 2mm, shorten >= 2mm] 
(0.5, {0.5 + (2/3)*0}) -- (3.5, {0.5 + (2/3)*3});
\draw[line width = 1, ->, >=latex, shorten <= 2mm, shorten >= 2mm] 
(0.5, {0.5 + (2/3)*1}) -- (3.5, {0.5 + (2/3)*0});
\draw[line width = 1, ->, >=latex, shorten <= 2mm, shorten >= 2mm] 
(0.5, {0.5 + (2/3)*2}) -- (3.5, {0.5 + (2/3)*2});
\draw[line width = 1, ->, >=latex, shorten <= 2mm, shorten >= 2mm] 
(0.5, {0.5 + (2/3)*3}) -- (3.5, {0.5 + (2/3)*1});
\node[anchor = south] at (2, 2.5) {\small $\alpha$};
%
\draw[line width = 1, ->, >=latex, shorten <= 5mm, shorten >= 2mm] 
(3.5, {0.5 + (2/3)*0}) -- (6.5, {0.5 + (2/3)*3});
\draw[line width = 1, ->, >=latex, shorten <= 5mm, shorten >= 2mm] 
(3.5, {0.5 + (2/3)*1}) -- (6.5, {0.5 + (2/3)*1});
\draw[line width = 1, ->, >=latex, shorten <= 5mm, shorten >= 2mm] 
(3.5, {0.5 + (2/3)*2}) -- (6.5, {0.5 + (2/3)*0});
\draw[line width = 1, ->, >=latex, shorten <= 5mm, shorten >= 2mm] 
(3.5, {0.5 + (2/3)*3}) -- (6.5, {0.5 + (2/3)*2});
\node[anchor = south] at (5, 2.5) {\small $\beta$};
\end{tikzpic}

\begin{FL}
\[
\beta \circ \alpha = 
\bbm
1 & 2 & 3 & 4 \\
2 & 4 & 3 & 1 \\
\ebm
\circ
\bbm
1 & 2 & 3 & 4 \\
3 & 2 & 4 & 1 \\
\ebm
\phantom{\alpha \circ \beta =}
\]
\[
= 
\bbm
1 & 2 & 3 & 4 \\
3 & 4 & 1 & 2 \\
\ebm
\]
\end{FL}

\vfill 

\begin{thm}{}{SN GP ORDER}
For any $n \geq 1$ we have $|S_{n}| = n!$
\end{thm}

\newpage

\underline{\bf Dihedral groups and permutation groups}

\begin{FL}
Let $P_{n}$ be a regular polygon with $n$ vertices. Label the vertices with 
numbers $1, 2, \dots, n$. Since every symmetry of $P_{n}$ sends vertices to 
vertices, it defines a certain permutation of vertices:




\begin{tikzpic}[alt={Diagonal reflection of a square.}]
 \node[name=s1, 
  regular polygon, 
  regular polygon sides=4, 
  minimum size=2cm, 
  draw, 
  color=red, 
  fill=red!20,
  line width=2pt, 
  ] 
  at (0,0) {};
  \draw[line width=1, black!60] (-0.9, -0.9) -- (0.9, 0.9);
  \node[anchor = south west] at (s1.corner 1) {\small\bf $1$};
  \node[anchor = north west] at (s1.corner 4) {\small\bf $2$};
  \node[anchor = north east] at (s1.corner 3) {\small\bf $3$};
  \node[anchor = south east] at (s1.corner 2) {\small\bf $4$};
 %
  \node[red, anchor = north east] at (s1.corner 1) {\small $\bf 1$};
  \node[red, anchor = south east] at (s1.corner 4) {\small $\bf 2$};
  \node[red, anchor = south west] at (s1.corner 3) {\small $\bf 3$};
  \node[red, anchor = north west] at (s1.corner 2) {\small $\bf 4$};
%
  \begin{scope}
  \draw[->, line width=1, >=latex] (1.5, 0) -- node[above] {$D$} (3.5, 0);
  \end{scope}
 % 
  \begin{scope}[xshift = 5cm]
  \begin{scope}[
  yscale=-1, 
  rotate=-90, 
  transform shape,
  ]
  \node[name=s2, 
  regular polygon, 
  regular polygon sides=4, 
  minimum size=2cm, 
  draw, 
  color=red, 
  fill=red!20,
  line width=2pt, 
  ] 
  at (0,0) {};
 \end{scope}
  \node[anchor = south west] at (s2.corner 1) {\small\bf $1$};
  \node[anchor = south east] at (s2.corner 4) {\small\bf $4$};
  \node[anchor = north east] at (s2.corner 3) {\small\bf $3$};
  \node[anchor = north west] at (s2.corner 2) {\small\bf $2$};
% 
  \node[red, anchor = north east] at (s2.corner 1) {\small $\bf 1$};
  \node[red, anchor = north west] at (s2.corner 4) {\small $\bf 2$};
  \node[red, anchor = south west] at (s2.corner 3) {\small $\bf 3$};
  \node[red, anchor = south east] at (s2.corner 2) {\small $\bf 4$};
 \end{scope}
\end{tikzpic}

\[
D = 
\bbm
1 & 2 & 3 & 4 \\
1 & 4 & 3 & 2 \\
\ebm
\phantom{D =}
\]

\ 

Since composition of symmetries corresponds to composition of permutations of vertices, 
we can identify the dihedral group $D_{n}$ with a subgroup of the group of permutations 
$S_{n}$. Note that not every permutation in $S_{n}$ comes from a symmetry of $P_{n}$. E.g.:


\[
\alpha = 
\bbm
1 & 2 & 3 & 4 \\
1 & 2 & 4 & 3 \\
\ebm
\phantom{\alpha =}
\]


\begin{tikzpic}[][scale=1.5]
  \draw[red, 
  fill=red!20,
  line width=2pt
  ] (0, 0) -- (1, 0) -- (1, 1) -- (0, 1) -- cycle;
%
  \node[anchor = south west] at (1, 1) {\small\bf $1$};
  \node[anchor = south east] at (0, 1) {\small\bf $2$};
  \node[anchor = north east] at (0, 0) {\small\bf $3$};
  \node[anchor = north west] at (1, 0) {\small\bf $4$};
%
  \begin{scope}
  \draw[->, line width=1, >=latex] (1.7, 0.5) -- node[above] {???} (3.3, 0.5);
  \end{scope}
%
  \begin{scope}[xshift = 4cm]
  \draw[red, 
  fill=red!20,
  line width=2pt
  ] (0, 0) -- (1, 0) -- (0, 1) -- (1, 1) -- cycle;
%
  \node[anchor = south west] at (1, 1) {\small\bf $1$};
  \node[anchor = south east] at (0, 1) {\small\bf $2$};
  \node[anchor = north east] at (0, 0) {\small\bf $4$};
  \node[anchor = north west] at (1, 0) {\small\bf $3$};
 \end{scope}
\end{tikzpic}

\end{FL}
\showto{SKELETON}{\newpage}


{\bf Note.} The groups $S_{n}$ are non-abelian for $n > 2$\showto{FULL}{, e.g:}

\begin{FL}
\[
\overset{
{\color{red}\alpha}}
{\bbm
1 & 2 & 3 \\
2 & 1 & 3 \\
\ebm
}
\circ
\overset{
{\color{red}\beta}}
{\bbm
1 & 2 & 3 \\
1 & 3 & 2 \\
\ebm
}
= 
\overset{
{\color{red}\alpha\circ \beta}}
{
\bbm
1 & 2 & 3 \\
2 & 3 & 1 \\
\ebm
}
\]
\[
\overset{
{\color{red}\beta}}
{
\bbm
1 & 2 & 3 \\
1 & 3 & 2 \\
\ebm
}
\circ
\overset{
{\color{red}\alpha}}
{
\bbm
1 & 2 & 3 \\
2 & 1 & 3 \\
\ebm
}
= 
\overset{
{\color{red}\beta\circ \alpha}}
{
\bbm
1 & 2 & 3 \\
3 & 1 & 2 \\
\ebm
}
\]
\end{FL}

\showto{SKELETON}{\vskip 50mm}
\showto{FULL}{\newpage}


\begin{definition}{}{SN MOVING ELT}
Let $\alpha \in S_{n}$ and let $i \in \{1, \dots, n\}$. We will say that 
$\alpha$ \emph{moves} $i$ if $\alpha(i) \neq i$. If $\alpha(i) = i$
we will say that $\alpha$ \emph{fixes} $i$. 
\end{definition}

\begin{FL}
{\bf Example.} The permutation
\[
\bbm
1 & 2 & 3 & 4 \\
2 & 4 & 3 & 1 \\
\ebm
\]
moves 1, 2 and 4, and fixes 3. 
\end{FL}
\showto{SKELETON}{\vskip 30mm}


\begin{definition}{}{SN DISJOINT PERM}
We will say that permutations $\alpha, \beta \in S_{n}$ are \emph{disjoint}
if there is no $i\in \{1, \dots, n\}$. which is moved by both $\alpha$ and $\beta$. 
\end{definition}

{\bf Example.}
\begin{tikzpic}[alt={Disjoint permutations.}]
\draw[rounded corners=10pt] (-0.25,0) rectangle (1.25,4);
\draw[rounded corners=10pt] (2.75,0) rectangle (4.25,4);
\foreach \x/\y in {0/5, 1/4, 2/3, 3/2, 4/1}{
\fill (0.5, {0.5 + (2/3)*\x}) circle (0.1) node[left] {\small $\y$};
\fill (3.5, {0.5 + (2/3)*\x}) circle (0.1) node[right] {\small $\y$};
}
\draw[red, line width = 1, ->, >=latex, shorten <= 2mm, shorten >= 2mm] 
(0.5, {0.5 + (2/3)*0}) -- (3.5, {0.5 + (2/3)*3});
\draw[red, line width = 1, ->, >=latex, shorten <= 2mm, shorten >= 2mm] 
(0.5, {0.5 + (2/3)*1}) -- (3.5, {0.5 + (2/3)*0});
\draw[line width = 1, ->, >=latex, shorten <= 2mm, shorten >= 2mm] 
(0.5, {0.5 + (2/3)*2}) -- (3.5, {0.5 + (2/3)*2});
\draw[red, line width = 1, ->, >=latex, shorten <= 2mm, shorten >= 2mm] 
(0.5, {0.5 + (2/3)*3}) -- (3.5, {0.5 + (2/3)*1});
\draw[line width = 1, ->, >=latex, shorten <= 2mm, shorten >= 2mm] 
(0.5, {0.5 + (2/3)*4}) -- (3.5, {0.5 + (2/3)*4});
\node[anchor = south] at (2, 3.5) {\small $\alpha$};
\node at (2, -1)
{
$
\bbm
1 & {\color{red} 2} & 3 & {\color{red} 4} & {\color{red} 5}\\
1 & {\color{red} 4} & 3 & {\color{red} 5} & {\color{red} 2}\\
\ebm
$
};
%
\begin{scope}[xshift=6cm]
\draw[rounded corners=10pt] (-0.25,0) rectangle (1.25,4);
\draw[rounded corners=10pt] (2.75,0) rectangle (4.25,4);
\foreach \x/\y in {0/5, 1/4, 2/3, 3/2, 4/1}{
\fill (0.5, {0.5 + (2/3)*\x}) circle (0.1) node[left] {\small $\y$};
\fill (3.5, {0.5 + (2/3)*\x}) circle (0.1) node[right] {\small $\y$};
}
\draw[line width = 1, ->, >=latex, shorten <= 2mm, shorten >= 2mm] 
(0.5, {0.5 + (2/3)*0}) -- (3.5, {0.5 + (2/3)*0});
\draw[line width = 1, ->, >=latex, shorten <= 2mm, shorten >= 2mm] 
(0.5, {0.5 + (2/3)*1}) -- (3.5, {0.5 + (2/3)*1});
\draw[red, line width = 1, ->, >=latex, shorten <= 2mm, shorten >= 2mm] 
(0.5, {0.5 + (2/3)*2}) -- (3.5, {0.5 + (2/3)*4});
\draw[line width = 1, ->, >=latex, shorten <= 2mm, shorten >= 2mm] 
(0.5, {0.5 + (2/3)*3}) -- (3.5, {0.5 + (2/3)*3});
\draw[red, line width = 1, ->, >=latex, shorten <= 2mm, shorten >= 2mm] 
(0.5, {0.5 + (2/3)*4}) -- (3.5, {0.5 + (2/3)*2});
\node[anchor = south] at (2, 3.5) {\small $\beta$};
\node at (2, -1)
{
$
\bbm
{\color{red} 1} & 2 & {\color{red} 3} & 4 & 5\\
{\color{red} 3} & 2 & {\color{red} 1} & 4 & 5\\
\ebm
$
};
\end{scope}
\end{tikzpic}



\begin{thm}{}{DISJOINT PERM COMMUTE}
If $\alpha, \beta \in S_{n}$ are disjoint permutations then 
\[
\alpha\circ\beta = \beta\circ\alpha
\]
Moreover, 
\[
\alpha\circ\beta(i) = 
\begin{cases}
\alpha(i) & \text{if $i$ is moved by $\alpha$} \\
\beta(i) & \text{if $i$ is moved by $\beta$} \\
i & \text{otherwise} \\
\end{cases}
\]
\end{thm}

\begin{FL}
\begin{proof}
Assume that $i\in \{1, \dots, n\}$ is an element moved by $\alpha$. Then 
$\alpha$ also moves $\alpha(i)$. It follows that both $i$ and $\alpha(i)$
are fixed by $\beta$, so we have 
\[
\beta\circ\alpha(i) = \alpha(i) = \alpha\circ\beta(i)
\]
By the same argument, if $i$ is moved by $\beta$ then 
\[
\alpha\circ\beta(i) = \beta(i) =  \beta\circ\alpha(i)
\]
Finally, it both $\alpha$ and $\beta$ fix $i$ then 
\[
\alpha\circ\beta(i) = i =  \beta\circ\alpha(i)
\]
\end{proof}
\end{FL}

\showto{SKELETON}{\newpage}


\begin{definition}{}{SN CYCLE}
A permutation $\alpha\in S_{n}$ is a \emph{cycle of length $r$} (or \emph{$r$-cycle}) 
if there are distinct elements $i_{1}, i_{2},\dots i_{r}\in \{1, 2, \dots, n\}$ such that
\[
\alpha(i_{1}) = i_{2}, \ \ 
\alpha(i_{2}) = i_{3}, \ \ 
\dots \ \ 
\alpha(i_{r-1}) = i_{r}, \ \ 
\alpha(i_{r}) = i_{1} \\ 
\]
and $\alpha$ fixes all other elements of $\{1, \dots, n\}$.
\end{definition}


\begin{FL}
{\bf Example.}

\[
\alpha=
\bbm
1 & 2 & 3 & 4 & 5 & 6 \\
3 & 2 & 6 & 4 & 1 & 5\\
\ebm
\phantom{\alpha=}
\]
\begin{tikzpic}
  \path (90 :1cm) node {1}
        (0:1cm) node {3}
        (270:1cm) node {6}
        (180:1cm) node {5}
        ;
  \draw [red, line width=1pt, ->, >=latex] (0, 0) +(-15:1) arc(-15:-75:1);
  \draw [red, line width=1pt, ->, >=latex] (0, 0) +(-105:1) arc(255:195:1);
  \draw [red, line width=1pt, ->, >=latex] (0, 0) +(165:1) arc(165:105:1);
  \draw [red, line width=1pt, ->, >=latex] (0, 0) +(75:1) arc(75:15:1);
\end{tikzpic}

\vskip 10mm

{\bf Cycle notation.} A permutation $\alpha$ such that 
\[
\alpha(i_{1}) = i_{2}, \ \ 
\alpha(i_{2}) = i_{3}, \ \ 
\dots \ \ 
\alpha(i_{r-1}) = i_{r}, \ \ 
\alpha(i_{r}) = i_{1} \\ 
\]
and which fixes all other elements is denoted by $(i_{1}, i_{2}, \dots, i_{r})$.


\vskip 10mm

{\bf Example.}

\[
\bbm
1 & 2 & 3 & 4 & 5 & 6 \\
3 & 2 & 6 & 4 & 1 & 5\\
\ebm
= (1, 3, 5, 6) = (3, 5, 6, 1) = (5, 6, 1, 3) = (6, 1, 3, 5)
\]
\end{FL}
\showto{SKELETON}{\vfill}



\begin{thm}{}{SN CYCLE DECOMP}
Every permutation in $S_{n}$ is either a cycle or a product of disjoint cycles.
\end{thm}


\begin{FL}
{\bf Example.}

\[
\bbm
1 & 2 & 3 & 4 & 5 & 6 & 7 & 8 \\
3 & 8 & 6 & 2 & 1 & 5 & 7 & 4 \\
\ebm
= (1, 3, 6, 5)\circ(2, 8, 4)\circ (7) = (1, 3, 6, 5)\circ(2, 8, 4)
\]
\begin{tikzpic}
  \path (90 :1cm) node {1}
        (0:1cm) node {3}
        (270:1cm) node {6}
        (180:1cm) node {5}
        ;
  \draw [red, line width=1.5pt, ->, >=latex] (0, 0) +(-15:1) arc(-15:-75:1);
  \draw [red, line width=1.5pt, ->, >=latex] (0, 0) +(-105:1) arc(255:195:1);
  \draw [red, line width=1.5pt, ->, >=latex] (0, 0) +(165:1) arc(165:105:1);
  \draw [red, line width=1.5pt, ->, >=latex] (0, 0) +(75:1) arc(75:15:1);
%  
   \path (3, 0) +(90 :1cm) node {2}
         (3, 0) +(-30:1cm) node {4}
         (3, 0) +(210:1cm) node {8}
        ;
  \draw [red, line width=1.5pt, ->, >=latex] (3, 0) +(75:1) arc(75:-15:1);
  \draw [red, line width=1.5pt, ->, >=latex] (3, 0) +(-45:1) arc(-45:-135:1);
  \draw [red, line width=1.5pt, ->, >=latex] (3, 0) +(-165:1) arc(-165:-255:1);
%
  \path (6,0) +(90:1cm) node {7};
  \draw [red, line width=1.5pt, ->, >=latex] (6, 0) +(75:1) arc(75:-255:1);
\end{tikzpic}
\end{FL}
\showto{SKELETON}{\vskip 40mm\newpage}



\begin{lemma}{}{SN ISOLATED CYCLE}
Let $\alpha \in S_{n}$, and let $i_{0} \in \{1, \dots, n\}$ be an element moved 
by $\alpha$. Then:
\benu
\item There exists $r>1$ such that $\alpha^{r}(i_{0}) = i_{0}$
\item If $r>1$ is the smallest integer satisfying $\alpha^{r}(i_{0}) = i_{0}$ then 
all elements 
\[
i_{0}, \alpha(i_{0}), \alpha^{2}(i_{0}), \dots, \alpha^{r-1}(i_{0})
\]
are distinct. 
\eenu
\end{lemma}


\begin{FL}
\begin{proof}
Consider the sequence  
\[
i_{0} = \alpha^{0}(i_{0}),\  \alpha^{1}(i_{0}),\ \alpha^{2}(i_{0}),\  \dots
\]
Since all elements of this sequence come from the finite set $\{1, \dots, n\}$, 
there must an integer $r\geq 1$ such that the elements 
$i_{0}, \alpha(i_{0}), \alpha^{2}(i_{0}), \dots, \alpha^{r-1}(i_{0})$ are distinct
and $\alpha^{r}(i_{0})$ is equal to one of the previous elements. We will show 
that $\alpha^{r}(i_{0}) = i_{0}$. Indeed, otherwise 
$\alpha^{r}(i_{0}) = \alpha^{k}(i_{0})$ for some $1 \leq k < r$. This gives 
\[
\alpha(\alpha^{r-1}(i_{0})) = \alpha(\alpha^{k-1}(i_{0}))
\] 
and since $\alpha$ is a 1-1 function, we obtain
\[
\alpha^{r-1}(i_{0}) = \alpha^{k-1}(i_{0})
\]
This contradicts the assumption that the elements 
$i_{0}, \alpha(i_{0}), \dots, \alpha^{r-1}(i_{0})$ are distinct.
\end{proof}
\end{FL}
\showto{SKELETON}{\newpage}


\showto{SKELETON}{\emph{Poof of \Cref{thm:SN CYCLE DECOMP}}.\newpage}
\begin{FL}
\begin{proof}[Proof of \Cref{thm:SN CYCLE DECOMP}]
Let $\alpha\in S_{n}$. We will argue that $\alpha$ can be written as a product of 
cycles by induction with respect to the number $k$ of elements of $\{1, \dots, n\}$
moved by $k$. If $k=0$ then $\alpha$ fixes all elements, so it is the identity 
permutation, which is a $1$-cycle. 

Assume then that all permutations moving $k$ or fewer elements can be written as a product 
of disjoint cycles and that $\alpha$ moves $k+1$ elements. Let $i_{0}\in \{1, \dots, n\}$ 
be an element moved by $\alpha$. By \Cref{lemma:SN ISOLATED CYCLE} there is $r>1$ such 
that the elements
\[
i_{0}, \ \alpha(i_{0}), \ \alpha^{2}(i_{0}), \dots, \alpha^{r-1}(i_{0})
\]
are all distinct and $\alpha^{r}(i_{0}) = i_{0}$. Denote for $k=1, \dots, r-1$
denote $i_{k} = \alpha^{k}(i_{0})$. Notice that $\alpha(i_{k}) = i_{k+1}$ for $k<r-1$ and 
$\alpha(i_{r-1}) = i_{0}$. Let $\beta\in S_{n}$ be a permutation defined as follows:

\[
\beta(i) = 
\begin{cases}
i & \text{if $i \in \{i_{0}, \dots, i_{r-1}\}$} \\
\alpha(i) & \text{otherwise}
\end{cases}
\]
Notice that the cycle $(i_{0}, i_{1}, \dots, i_{r-1})$ and $\beta$ are disjoint
permutations. Thus, we can use \Cref{thm:DISJOINT PERM COMMUTE} to show that 
$\alpha = (i_{0}, i_{1}, \dots, i_{r-1})\circ \beta$. Then, since $\beta$ moves
fewer elements than $\alpha$, by the inductive assumption we can write 
$\beta$ as a product of disjoint cycles:
\[
\beta = \gamma_{1}\circ \dots \circ \gamma_{m}
\]
Therefore we obtain a decomposition of $\alpha$ into a product of disjoint cycles:
\[
\alpha =  (i_{0}, i_{1}, \dots, i_{r-1})\circ \gamma_{1}\circ \dots \circ \gamma_{m}
\]
\end{proof}

\vskip 5mm
\end{FL}

Recall that the least common multiple of integers 
$n_{1}, n_{2}, \dots, n_{k} \geq 1$  is the smallest positive integer
$\lcm(n_{1}, \dots, n_{k})$ which is divisible by each of these numbers.


\begin{thm}{}{SN ELEMENT ORDER}
Assume that a permutation $\alpha\in S_{n}$ has a decomposition into disjoint cycles
\[
\alpha = \gamma_{1}\circ \dots \circ \gamma_{m}
\]
where $\gamma_{i}$ is a cycle of length $r_{i}>1$. 
Then the order of $\alpha$ is given by 
\[
|\alpha| = \lcm(r_{1}, r_{2}, \dots, r_{m}) 
\]
\end{thm}


\begin{FL}
\begin{proof}
First, notice that if $\gamma$ is an $r$-cycle then $|\gamma| = r$. Let
\[
\alpha = \gamma_{1}\circ \dots \circ \gamma_{m}
\]
where $\gamma_{i}$ is an $r_{i}$-cycle, and let $p = \lcm(r_{1}, \dots, r_{m})$.
By \Cref{thm:DISJOINT PERM COMMUTE} disjoint cycles commute, so 
\[
\alpha^{p} = (\gamma_{1}\circ \dots \circ \gamma_{m})^{p} 
=  \gamma_{1}^{p}\circ \dots \circ \gamma_{m}^{p} = \varepsilon
\]
where $\varepsilon$ is the identity permutation.
By \Cref{thm:ORDER GP ELT DIVIDES} we obtain that $|\alpha|$ divides $p$. 

Next, we claim that $\gamma_{i}^{|\alpha|} = \varepsilon$ for each $i$. 
Indeed, since the cycles are disjoint, elements moved by $\gamma_{i}^{|\alpha|}$ 
are fixed by $\gamma_{j}^{|\alpha|}$ for all $j\neq i$, so if $\gamma_{i}^{|\alpha|}$ 
moves some element, then the same element is moved by 
$\gamma_{1}^{|\alpha|}\circ \dots \circ \gamma_{m}^{|\alpha|}$. This however 
cannot happen because 
\[
\gamma_{1}^{|\alpha|}\circ \dots \circ \gamma_{m}^{|\alpha|}
= (\gamma_{1}\circ \dots \circ \gamma_{m})^{|\alpha|} = \alpha^{|\alpha|} = \varepsilon 
\]
In this way we obtained that $r_{i}$ divides $|\alpha|$ for $i=1, \dots, m$, 
and so $p$ divides $|\alpha|$. Therefore $|\alpha| = p$.

\end{proof}
\end{FL}
\showto{SKELETON}{\newpage}


\begin{exercise}
Compute the order of the following permutation in $S_{8}$:
\[
\bbm
1 & 2 & 3 & 4 & 5 & 6 & 7 & 8 \\
3 & 8 & 6 & 2 & 1 & 5 & 7 & 4 \\
\ebm
\]
\end{exercise}

\showto{SKELETON}{\vskip 50mm}

\begin{exercise}
Find all possible orders of elements of $S_{5}$.
\end{exercise}
\showto{SKELETON}{\vskip 60mm}

\begin{exercise}
Compute the number of permutations of order 10 in $S_{8}$.
\end{exercise}

\begin{exercise}
Compute the number of permutations of order 3 in $S_{7}$.
\end{exercise}
\showto{SKELETON}{\newpage}


\begin{definition}{}{SN TRANSPOSITION}
A \emph{transposition} in $S_{n}$ is a cycle  $(i_{1}, i_{2})$ of length 2. 
\end{definition}


\begin{thm}{}{CYCLE IS PROD OF TRANSP}
Every permutation in $S_{n}$ can be written as a product of transpositions.
\end{thm}


\begin{FL}
\begin{proof}
By \Cref{thm:SN CYCLE DECOMP} every permutation is product of cycles, so 
it is enough to show that every cycle can be written as a product of 
transpositions. This is true, since if $(i_{1}, i_{2}, \dots, i_{r})$ is a cycle 
in $S_{n}$ then 
\[
(i_{1}, i_{2}, \dots, i_{r}) = (i_{1}, i_{r})\circ (i_{1}, i_{r-1})
\circ \dots \circ (i_{1}, i_{2})
\]
\end{proof}


{\bf Note.} A permutation can be written as a product of cycles in many different ways:
\medskip
\begin{align*}
\bbm
1 & 2 & 3 & 4 \\
2 & 3 & 1 & 4 \\
\ebm 
&\  = \ \\
          &\  =\ (1, 3)\circ (1, 2) \\  
          &\  =\  (2, 3)\circ (1, 3) \\
          &\  =\  (1, 3)\circ (4, 2)\circ (1, 2)\circ(1, 4) \\
          &\  =\  (2, 4)\circ (1, 2)\circ (2, 3)\circ(1, 4) \\
          &\ = \ \dots \\
\end{align*}

\vskip -15mm
\ 
\end{FL}
\showto{SKELETON}{\newpage}

\begin{thm}{}{PARITY OF PERMUTATION}
Let $\alpha\in S_{n}$ and let 
\[
\alpha = \beta_{1}\circ\beta_{2}\circ \dots \circ \beta_{r}
\]
be a decomposition of $\alpha$ into a product of transpositions. 

\ 

\textbullet\  If the number $r$
is even, then every other decomposition of $\alpha$ into transpositions consists of 
an even number of transpositions. 

\medskip

\textbullet\  If $r$ is odd, then every other decomposition 
of $\alpha$ into transpositions consists of an odd number of transpositions. 
\end{thm}
\showto{SKELETON}{\newpage}

\begin{lemma}{}{TRIVIAL PERM EVEN}
Let $\beta_{1}, \dots, \beta_{r}$ be transpositions in $S_{n}$ such that 
\[
\beta_{1}\circ\beta_{2}\circ \dots \circ \beta_{r} = \varepsilon
\]
where $\varepsilon$ is the identity permutation. Then $r$ is an  even number. 
\end{lemma}

\begin{FL}
\begin{proof}
We will prove by induction with respect to $k$ the following statement:

\begin{center}
\emph{For any $k\geq 2$, if
$\beta_{1}\circ\beta_{2}\circ \dots \circ \beta_{r} = \varepsilon$
and $r \leq k$ then $r$ is an even number.} 
\end{center}

If $k=2$ this holds, since
the only way to write $\varepsilon$ as a product of 1 or 2 transpositions 
is $\beta\circ\beta^{-1}$, which uses 2 transpositions.  

For the inductive step, assume then that the statement holds for some $k$. 
We need to show that it also holds for $k+1$. 
Let then $\beta_{1}, \dots, \beta_{r}$ be transpositions such that $r\leq k+1$ and
\begin{equation}
\tag{$\ast$}
\beta_{1}\circ\beta_{2}\circ \dots \circ \beta_{r} = \varepsilon
\end{equation} 
Assume that one of the transpositions $\beta_{i}$ is of the form $(a, b)$ for some 
$a, b\in \{1, \dots, n\}$.
One can check that the following identities hold:
\bitem
\item $(a, b)\circ (c, d) = (c, d)\circ (a, b)$
\item $(a, b)\circ (b, c) = (b, c)\circ (a, c)$
\eitem
Here $a, b, c, d$ are distinct elements of the set $\{1, \dots, n \}$. These 
identities say that when multiplying transpositions, we can move the transposition
involving $a$ toward the right side without changing the number of transpositions. 
Using this observation, we can rewrite the equation ($\ast$) as follows:
\[
\tag{$\ast \ast$}
\gamma_{1}\circ \dots \circ \gamma_{r-k}\circ (a, b_{1}) \circ (a, b_{2})
\circ {\dots} \circ (a, b_{k}) = \varepsilon
\]
where $\gamma_{1}, \dots, \gamma_{r-k}$ are transpositions that do not involve $a$.  
If $b_{1}\neq b_{i}$ for $i=2, \dots k$, then the permutation on the left hand 
side of the equation ($\ast\ast$) would send $b_{1}$ to $a$, which is impossible.
This means that there is $i > 1$ such that $b_{1}\neq b_{2}, \dots, b_{i-1}$ and 
$b_{1} = b_{i}$. We will need one more identity:
\bitem
\item $(a, b)\circ (a, c) = (b, c)\circ (a, b)$ for distinct elements $a, b, c$.
\eitem
Using this identity, we can bring the equation ($\ast \ast$) to the following form:
\[
\gamma_{1}\circ {\dots} \circ \gamma_{r-k}\circ 
(b_{1}, b_{2}) \circ {\dots} \circ 
(b_{1}, b_{i-1})\circ (a, b_{1})\circ(a, b_{i})\circ(a, b_{i+1})\circ {\dots} \circ (a, b_{n})
= \varepsilon
\]
Since $b_{1} = b_{i}$ we have $(a, b_{1})\circ(a, b_{i}) = \varepsilon$, and the above 
equation becomes
\[
\gamma_{1}\circ {\dots} \circ \gamma_{r-k}\circ 
(b_{1}, b_{2}) \circ {\dots} \circ 
(b_{1}, b_{i-1})\circ (a, b_{i+1})\circ {\dots} \circ (a, b_{n})
= \varepsilon
\]
This expresses $\varepsilon$ as a product of $r-2$ transpositions. Since $r \leq k+1$, 
thus $r-2 \leq k$, and so, by the inductive assumption, $r-2$ must be an even number. 
Therefore $r$ is an even as well.
\end{proof}


\begin{proof}[Proof of \Cref{thm:PARITY OF PERMUTATION}]
Assume that a permutation $\alpha$ can be written as a product of transpositions
in two different ways:
\begin{align*}
\alpha = & \beta_{1}\circ\beta_{2}\circ \dots \circ \beta_{r} \\
\alpha = & \gamma_{1}\circ\gamma_{2}\circ \dots \circ \gamma_{s} \\
\end{align*}
Then we have 
\begin{align*}
\varepsilon & = \alpha \circ \alpha^{-1}  \\
            & = (\beta_{1}\circ\beta_{2}\circ \dots \circ \beta_{r})
            \circ (\gamma_{1}\circ\gamma_{2}\circ \dots \circ \gamma_{s})^{-1} \\
            & = (\beta_{1}\circ\beta_{2}\circ \dots \circ \beta_{r}) \circ 
             (\gamma_{s}^{-1}\circ\gamma_{s-1}^{-1}\circ \dots \circ \gamma_{1}^{-1}) \\ 
            & = (\beta_{1}\circ\beta_{2}\circ \dots \circ \beta_{r}) \circ
             (\gamma_{s}\circ\gamma_{s-1}\circ \dots \circ \gamma_{1})          
\end{align*}
This means that $\varepsilon$ is a product of $r+s$ transpositions. 
Since by \Cref{lemma:TRIVIAL PERM EVEN}, $r+s$ is an even number, thus
either both $r$ and $s$ are even numbers or they are both odd.  
\end{proof}
\end{FL}
\showto{SKELETON}{\newpage}


\begin{definition}{}{PERM SIGN}
A permutation $\alpha\in S_{n}$ is \emph{even} if it can be written as a product of 
even number of transpositions and it is \emph{odd} if it can be written as a product
of an odd number of tranpositions.
\end{definition}


\begin{thm}{}{EVEN PERM ARE SUBGP}
The subset of $S_{n}$ consisting of all even permutations is a subgroup of $S_{n}$.
\end{thm}

\showto{SKELETON}{\vskip 50mm}


\begin{definition}{}{ALT GP}
The subgroup of $S_{n}$ consisting of even permutations is called an 
\emph{alternating group on $n$ letters} and it is denoted by $A_{n}$
\end{definition}

\begin{thm}{}{ALT GP ORDER}
For $n\geq 2$ the alternating group $A_{n}$ has order $\frac{n!}{2}$.
\end{thm}

\begin{FL}
\begin{proof}
Let $B_{n}$ be the set of all odd permutations in $S_{n}$. Since $|S_{n}| = n!$, 
it is enough to show that $|A_{n}| = |B_{n}|$, i.e. that there exists a bijection 
$f\colon A_{n} \to B_{n}$. Such bijection can be defined by $f(\alpha) = (1, 2)\circ \alpha$.
\end{proof}
\end{FL}
\showto{SKELETON}{\newpage}

\begin{definition}{}{SN PERM SIGN}
The \emph{sign} of a permutation $\alpha\in S_{n}$ is defined as follows:
\[
\sign(\alpha) = 
\begin{cases}
+1 & \text{if $\alpha$ is even} \\
-1 & \text{if $\alpha$ is odd}
\end{cases}
\]
\end{definition}

\begin{FL}
{\bf Note.} Recall that for a square matrix $A$ we can compute its determinant
$\det A$. The determinant can be defined using permutations and their signs as follows. 
For a matrix
\[
A = 
\bbm
a_{1,1} & {\dots} & a_{1,n} \\
\vdots  &            &  \vdots \\
a_{n,1} & {\dots} & a_{n,n} \\
\ebm
\] 
we set
\[
\det A = \sum_{\alpha\in S_{n}} \sign(\alpha)\cdot
a_{1,\alpha(1)}\cdot a_{2,\alpha(2)}\cdot{\dots}\cdot a_{n,\alpha(n)}
\]
\end{FL}

