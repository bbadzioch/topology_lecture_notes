% !TEX TS-program = lualatex-dev
% !TEX root = ../mth419_lecture_notes.tex

\lecture{Isomorphisms of groups}


\begin{definition}{}{GROUP ISO}
An \emph{isomorphism of groups} is a group homomorphism which is both onto and 1-1. 
\end{definition}

\begin{tikzpic}
\draw[rounded corners=10pt] (-0.35,0) rectangle (1.25,3);
\draw[rounded corners=10pt] (2.75,0) rectangle (4.35,3);
\foreach \x/\y/\z in {1/b/{b'}, 2/a/{a'}, 3/{e_{G}}/{e_{H}}}{
\fill (0.5, {0.5 + (2/3)*\x}) circle (0.1) node[left] {\small $\y$};
\fill (3.5, {0.5 + (2/3)*\x}) circle (0.1) node[right] {\small $\z$};
\draw[line width = 1, ->, >=latex, shorten <= 2mm, shorten >= 2mm] 
(0.5, {0.5 + (2/3)*\x}) -- (3.5, {0.5 + (2/3)*\x});
}
\node at (0.5, 0.7) {$\vdots$};
\node at (3.5, 0.7) {$\vdots$};
%
\node[anchor= east] at (-0.35, 2.65) {\small $G$};
\node[anchor= west] at (4.35, 2.65) {\small $H$};
\node[anchor = south] at (2, 2.5) {\small $f$};
\end{tikzpic}


\begin{thm}{}{INVERSE GP ISO}
If $g\colon G \to H$ is an isomorphism then the inverse function $f^{-1}\colon H \to G$
is also an isomorphism. 
\end{thm}

\begin{FL}
\begin{proof}
The inverse function is 1-1 and onto, so it remains to prove that it is a homomorphism 
of groups. 

Let $a', b' \in H$. We need to show that $f^{-1}(a'b') = f^{-1}(a')f^{-1}(b')$. 
Since $f$ is a 1-1 function it will suffice to show that 
$f(f^{-1}(a'b')) = f(f^{-1}(a')\cdot f^{-1}(b'))$.

We have $f(f^{-1}(a'b')) = a'b'$. Also, since $f$ is a homomorphism, we have 
\[
f(f^{-1}(a')f^{-1}(b')) = f(f^{-1}(a'))\cdot f(f^{-1}(b')) = a'b'
\] 
\end{proof}
\end{FL}

\showto{SKELETON}{\vfill}

\begin{thm}{}{ISO KER IMG}
A homorphism of groups $f\colon G\to H$ is an isomorphism if and only if 
$\Img(f) = H$ and $\Ker(f) = \{e\}$.
\end{thm}

\showto{SKELETON}{\newpage}


\begin{definition}{}{ISOMORPHIC GPS}
We say the group $G$ is \emph{isomorphic} to a group $H$ if there exists an isomorphism 
$f\colon G \to H$. Then we write $G \cong H$.
\end{definition}

\begin{thm}{}{GP ISO IS EQUIV}
Isomorphism of groups is an equivalence relation:
\benu
\item For any group $G$ we have $G\cong G$.
\item If $G, H$ are groups such that $G\cong H$ then $H\cong G$. 
\item If $G, H, K$ are groups such that $G\cong H$ and $H\cong K$, then $G\cong K$.
\eenu
\end{thm}


\begin{FL}
\begin{proof}
1) The identity function $\id\colon G \to G$, $\id(a) = a$ is an isomorphism, 
so $G\cong G$.

2) If $G\cong H$ then there is an isomorphism $f\colon G\to H$. In such case the 
function $f^{-1}\colon H \to G$ is also an isomorphism, so $H\cong G$.

3) If $G\cong H$ and $H\cong K$ then we have isomorphisms $f\colon G \to H$ and 
$g\colon H \to K$. Then the composition $g\circ f \colon G \to K$ is also an 
isomorphism and so $G \cong K$.
\end{proof}
\end{FL}

\begin{FL}
{\bf Example.} Recall that for a set $A$ the group $S(A)$ of permutations of $A$
consists of all permutations $f\colon A\to A$ with their composition as the 
group operation. We will show that if $B$ is another set with the same number 
of elements as $A$ then $S(A) \cong S(B)$. Indeed, let $\varphi\colon A\to B$
be a function which is onto and 1-1. We define 
\[
\Phi \colon S(A) \to S(B)
\]
by $\Phi(\alpha) = \varphi \circ \alpha \circ \varphi^{-1}$. One can check that 
this is an isomorphism of groups. 

In particular, recall that the symmetric group on $n$ letters $S_{n}$ is defined 
as the group of permutations of the set $\{1, 2, \dots, n\}$.  We obtain 
that if $A$ is a set consists of $n$ elements, then $S(A) \cong S_{n}$.

{\bf Example.} If $G$, $H$ are cyclic groups and $|G| = |H|$ then $G \cong H$. 
Indeed, let $G = \lrang{a}$ and $H = \lrang{b}$ then the function $f\colon G \to H$, 
$f(a^{k}) = b^{k}$ is an isomorphism. 

As a consequence, we obtain that every cyclic group is isomorphic either to 
$\Z$ or to $\Z_{n}$ for some $n\geq 1$. 

{\bf Example.} Let $m, n>0$ be integers such that $\gcd(m, n) =1$. Recall that by \Cref{thm:CYCLIC DIR SUM} the group $\Z_{m}\oplus\Z_{n}$ is cyclic of order $mn$. 
This gives: $\Z_{m}\oplus\Z_{n} \cong \Z_{mn}$.


{\bf Note.} Let $f\colon G \to H$ be a homomorphism which is 1-1 
(i.e. $\Ker(f) = \{ e\}$). Then we can obtain an isomorophism from $f$ by 
replacing $H$ with $\Img(f)$:
\[
f\colon G \to \Img(f)
\]
Thus any 1-1 homomorphism $f\colon G \to H$ defines an isomorphism between 
$G$ and the subgroup $\Img(f)$ of $H$.   
\end{FL}

\showto{SKELETON}{\newpage}


\begin{thm}{(Cayley's Theorem)}{CAYLEYS THM}
Let $G$ be a finite group of order $n$. Then $G$ is isomorphic to 
a subgroup of the symmetric group $S_{n}$. 
\end{thm}

\begin{FL}
\begin{proof}
Let $\overline{G}$ denote the set of elements of $G$. Since $\overline{G}$
consists of $n$ elements, the symmetric group $S_{n}$ is isomorphic
to the group $S(\overline{G})$ of permutations of $\overline{G}$. Thus it will 
suffice to prove that $G$ is isomorphic to a subgroup of $S(\overline{G})$. 

To show this, it is enough to construct a 1-1 homomorphism 
$\Phi\colon G \to S(\overline{G})$.  We will do it as follows.
For an element $a\in G$ define a function $t_{a}\colon \overline{G} \to \overline{G}$
by $t_{a}(x) = ax$. This function is a permutation of the set $\overline{G}$, 
so it is an element of the group $S(\overline{G})$. We define: 
$\Phi(a) = t_{a}$. 


We will show that the function $\Phi$ is a group homomorphism, 
i.e. that for any $a, b\in G$ we have $\Phi(ab) = \Phi(a)\circ \Phi(b)$. 
Indeed, $\Phi(ab) = t_{ab}$ and $t_{ab}\colon \overline{G} \to \overline{G}$ 
is the permutation given by $t_{ab}(x) = abx$. On the other hand, 
\[
\Phi(a)\circ \Phi(b) =t_{a}\circ t_{b}
\] 
where $t_{a}(x) = ax$ and $t_{b}(x) = bx$ 
We have 
\[
t_{a}\circ t_{b}(x) = t_{a}(bx) = abx = t_{ab}(x)
\]
Therefore $\Phi(ab) = \Phi(a)\circ \Phi(b)$. 

It remains to show that $\Phi$ is 1-1, i.e. that $\Ker(\Phi) = \{e\}$. 
Assume then that $a\in \Ker(\Phi)$. This means that $\Phi(a) = t_{a}$
is the identity permutation, $t_{a}(g) = g$ for $g\in G$. 
But then $ag = g$ for all $g\in G$, which means that $a = e$. 

\end{proof}
\end{FL}


\showto{SKELETON}{\newpage}

\begin{definition}{}{GP AUTOM}
An \emph{automophism} of a group $G$ is an isomorphism $f\colon G \to G$. 
\end{definition}

\begin{FL}
{\bf Example.} Given a group $G$ and $a\in G$, define $f_{a}\colon G\to G$ by 
$f_{a}(g) = aga^{-1}$. This function is an automorphism of $G$. Automorphisms 
of this form are called \emph{inner automorphisms}.

Note that if $G$ is an abelian group then $G$ for any $a\in G$ we have 
$f_{a}(g) = g$. Thus the only inner automorphism of $G$ is the identity function. 
\end{FL}

\showto{SKELETON}{\vskip 70mm}

\begin{definition}{}{GP AUTOM GP}
Let $G$ be a group. The \emph{group of automorphisms} of $G$ is the group
$\Aut(G)$ whose elements are automorphisms of $G$ and the group operation is
given by composition of automorphisms.  
\end{definition}


