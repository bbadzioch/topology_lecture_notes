% !TEX TS-program = lualatex-dev
% !TEX root = ../mth419_lecture_notes.tex

\lecture{Order of an element of a group}


\underline{\bf Exponentiation}

\begin{FL}
\medskip 

Let $G$ be a group and let $g\in G$. For an integer $n > 0$ we denote:


\bitem
\item $g^{n} = \underbrace{g\cdot g \cdot {\dots}\cdot g}_{\text{$n$ times}}$
\hfill (additive notation: $ng = \underbrace{g + g + {\dots} + g}_{\text{$n$ times}}$)
\\[0mm]
\item $g^{-n} = \underbrace{g^{-1}\cdot g^{-1} \cdot {\dots}\cdot g^{-1}}_{\text{$n$ times}}$ 
\hfill (additive notation: $(-n)g = \underbrace{-g - g - {\dots} - g}_{\text{$n$ times}}$)
\\[0mm]
\item $g^{0} = e$ 
\hfill (additive notation: $0g = 0$) \\
\eitem
\vskip 15mm
\end{FL}


\showto{SKELETON}{\vskip 70mm}

\underline{\bf Properties of exponentiation}

\begin{FL}
\bitem
\item $g^{m + n} = g^{m}\cdot g^{n}$ 
\hfill (additive notation: $(m+n)g = (mg) + (ng)$)
\\[0mm]
\item $g^{mn} = \left(g^{m}\right)^{n}$ 
\hfill (additive notation: $(mn)g = m(ng)$)
\\[0mm]

\eitem 

\vskip 10mm
\end{FL}

\showto{SKELETON}{\vskip 40mm}

\begin{definition}{}{GROUP ELEMENT ORDER}
Let $G$ be a group. An \emph{order} of an element $g\in G$ is the smallest 
integer $n \geq 1$ such that $g^{n} = e$. We write: $|g| = n$. 

\bigskip

If $g^{n} \neq e$ for all $n\geq 1$ then we say that $g$ is an element of an 
\emph{infinite order} and we write $|g| = \infty$. 
 
\end{definition} 

\begin{FL}
{\bf Note.} If $g^{n} = e$ then  $g^{-1} = g^{n-1}$. In particular, if $g^{2} = e$ 
then $g = g^{-1}$.
\end{FL}

\showto{SKELETON}{\newpage}


\begin{exercise}
Recall that the multiplication table of the dihedral group $D_{4}$
is as follows: 
\begin{center}
{\small
\tagpdfsetup{table/header-rows={1}}
\begin{tabular}{l !{\vrule width 2pt} llllllll}
\circ     & \ $I$       & $R_{90}$  & $R_{180}$ & $R_{270}$ & $H$       & $V$       & $D$       & $D'$      \\[1mm]
\noalign{\hrule height 2pt} \\[-3mm]
$I$       & \ $I$       & $R_{90}$  & $R_{180}$ & $R_{270}$ & $H$       & $V$       & $D$       & $D'$      \\
$R_{90}$  & \ $R_{90}$  & $R_{180}$ & $R_{270}$ & $I$       & $D'$      & $D$       & $H$       & $V$       \\
$R_{180}$ & \ $R_{180}$ & $R_{270}$ & $I$       & $R_{90}$  & $V$       & $H$       & $D'$      & $D$       \\
$R_{270}$ & \ $R_{270}$ & $I$       & $R_{90}$  & $R_{180}$ & $D$       & $D'$      & $V$       & $H$       \\
$H$       & \ $H$       & $D$       & $V$       & $D'$      & $I$       & $R_{180}$ & $R_{90}$  & $R_{270}$ \\
$V$       & \ $V$       & $D'$      & $H$       & $D$       & $R_{180}$ & $I$       & $R_{270}$ & $R_{90}$  \\[1mm]
$D$       & \ $D$       & $H$       & $D'$      & $V$       & $R_{270}$ & $R_{90}$  & $I$       & $R_{180}$ \\
$D'$      & \ $D'$      & $V$       & $D$       & $H$       & $R_{90}$  & $R_{270}$ & $R_{180}$ & $I$       \\
\end{tabular}
}
\end{center}
\bigskip

Find the order of every element of $D_{4}$.
\end{exercise}


\showto{SKELETON}{\vskip 70mm}

\begin{exercise}
Find the order of every element in the  group $\Z_{6}$. 
\end{exercise}


\showto{SKELETON}{\newpage}


\begin{thm}{}{ORDER IN FINITE GP IS FINITE}
If $G$ is a finite group and $g\in G$ then $|g| < \infty$. 
\end{thm} 

\begin{FL}
\begin{proof}
Consider the sequence 
\[
g^{1}, g^{2}, g^{3}, \dots \subseteq G
\]
Since $G$ consists of finitely many elements, we must have $g^{m} = g^{n}$ for some 
$n > m$. This gives 
\begin{align*}
g^{-m}g^{m}  =\  & g^{-m}g^{n} \\
e  =\  & g^{n-m} \\
\end{align*}
Thus $|g| \leq n-m < \infty$.
\end{proof}
\end{FL}

\showto{SKELETON}{\vskip 90mm}


\begin{thm}{}{ORDER GP ELT DIVIDES}
If $G$ is a group, $g\in G$ and $n\geq 1$ is an integer such that $g^{n} = e$, then 
$|g|$ divides $n$.  
\end{thm} 


\begin{FL}
\begin{proof}
We have 
\[
n = |g|\cdot q + r
\]
for some integers $q\geq 0$ and $0\leq r < |g|$. We want to show that $r=0$. Assume 
that it is not true. Then we have 
\[
e = g^{n} = g^{|g|\cdot q + r} = g^{|g|\cdot q}\cdot g^{r} 
= \left(g^{|g|}\right)^{q} \cdot g^{r}
= e \cdot g^{r}
= g^{r}
\]
We obtain that $g^{r} = e$. This is however impossible, since $r < |g|$.
\end{proof}

\vskip 10mm
\end{FL}

\showto{SKELETON}{\newpage}

\begin{thm}{}{ORDER OF PRODUCT}
If $G$ is a group, and $a, b\in G$ are elements such that $|a|, |b| < \infty$
and $ab = ba$ then $|ab|$ divides $|a|\cdot |b|$.   
\end{thm} 

\begin{FL}
\begin{proof}
Let $|a| = m$ and $|b| = n$. We have 
\[
(ab)^{mn} = a^{mn}b^{mn} = \left( a^{m} \right)^{n}\cdot \left( b^{n} \right)^{m}
= e^{n}\cdot e^{m} = e 
\]
By \Cref{thm:ORDER GP ELT DIVIDES} we get that $|ab|$ divides $mn = |a|\cdot |b|$.
\end{proof}

\vskip 10mm 

{\bf Example.} In the dihedral group $D_{4}$ take $a = R_{90}$, $b=R_{180}$. 
Then $a\cdot b = R_{90}\cdot R_{180} = R_{270}$
We have $|R_{90}| = 4$, $|R_{180}| = 2$, so $|R_{90}|\cdot |R_{180}| = 8$ 
which is divisible by $|R_{270}| = 4$. 

\vskip 10mm

{\bf Example.} \Cref{thm:ORDER OF PRODUCT} is not true in general if $ab\neq ba$. 
Take for example $a, b$ to be two different reflections in the dihedral group $D_{3}$.
Then $|a| = |b| = 2$, so $|a|\cdot |b| = 4$, but $|ab| = 3$.   

As another example, in the group $GL(2, \R)$ take matrices
\[
A = 
\bbm
0 & 2 \\
\frac{1}{2} & 0\\
\ebm, 
\ \ \ \ 
B = 
\bbm
1 & 0 \\
0 & 1 \\
\ebm
\] 
Then $|A| = 2$ and $|B|=2$ but $|AB| = \infty$.

\end{FL}

\showto{SKELETON}{\newpage}

\begin{thm}{}{ORDER OF POWER}
If $G$ is a group, and $a\in G$ is element such that $|a| = n < \infty$
then 
\[
|a^{k}| = \frac{n}{\gcd(n, k)}
\] 
\end{thm} 


\begin{exercise}
Compute the order of the element $6 \in \Z_{10}$. 
\end{exercise}

\begin{FL}
\vskip 10mm

\begin{proof}[Proof of \Cref{thm:ORDER OF POWER}]
Let $|a^{k}| = r$. Then $r$ is the smallest positive integer such that 
$\left(a^{k}\right)^{r} = a^{kr} = e$. 
Using \Cref{thm:ORDER GP ELT DIVIDES} we obtain that $r$ is the smallest
positive integer such that $n|kr$. This means that $kr$ is the least 
common multiple of $n$ and $k$: $kr = \lcm(n, k)$. This gives: 
\[
kr = \lcm(n,k) = \frac{nk}{\gcd(n, k)}
\]
and so $r=\dfrac{n}{\gcd(n, k)}$.
\end{proof}
\end{FL}


