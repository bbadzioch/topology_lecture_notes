% !TEX TS-program = lualatex-dev
% !TEX root = ../mth419_lecture_notes.tex

\lecture{Dihedral groups}

\ 

{\bf Regular polygons $P_{n}$ with $n$ sides:}


\begin{tikzpic}[alt=Regular polygons.][scale=2]
\begin{scope}
\draw[red, fill=red!20, line width=2pt] ({-1/sqrt(3)}, 0) -- ({1/sqrt(3)}, 0) -- (0, 1)  -- cycle;
\node[below, yshift=-2mm] at (0, 0) {$P_{3}$};
\end{scope}
%
\begin{scope}[xshift=2cm]
\draw[red, fill=red!20, line width=2pt] 
(0.5, 0) 
-- (0.5, 1) 
-- (-0.5, 1) 
-- (-0.5, 0) 
-- cycle;
\node[below, yshift=-2mm] at (0, 0) {$P_{4}$};
\end{scope}
%
\begin{scope}[xshift=4.2cm]
\pgfmathsetmacro{\afive}{{360/5}}
\pgfmathsetmacro{\rfive}{{1/(1+sin(90 - \afive/2))}}
\pgfmathsetmacro{\xfive}{{\rfive * sin(\afive)/sin(90 - \afive/2)}}
\draw[red, fill=red!20, line width=2pt]  
(-\xfive/2, 0) 
-- (\xfive/2, 0) 
-- ({\xfive * sin(90 - \afive/2)}, {\xfive*sin(\afive)})
-- (0, 1)
-- ({-\xfive * sin(90 - \afive/2)}, {\xfive*sin(\afive)}) 
-- cycle;
\node[below, yshift=-2mm] at (0, 0) {$P_{5}$};
\end{scope}
%
\begin{scope}[xshift=6.2cm]
\pgfmathsetmacro{\asix}{{360/6}}
\pgfmathsetmacro{\rsix}{{1/(2*sin(90 - \asix/2))}}
\pgfmathsetmacro{\xsix}{{\rsix * sin(\asix)/sin(90 - \asix/2)}}
\draw[red, fill=red!20, line width=2pt]  
(-\xsix/2,0) 
-- (\xsix/2, 0)
-- (\rsix, 0.5)
-- (\xsix/2, 1)
-- (-\xsix/2, 1)
-- (-\rsix, 0.5)
-- cycle;
\node[below, yshift=-2mm] at (0, 0) {$P_{6}$};
\end{scope}
\end{tikzpic}


{\bf Symmetries of $P_{4}$}:


\begin{tikzpic}[alt=Symmetries of the square.]
% R0
\begin{scope}
\node[name=s1, 
  regular polygon, 
  regular polygon sides=4, 
  minimum size=2cm, 
  draw, 
  color=red, 
  fill=red!20,
  line width=2pt, 
  ] 
  at (0,0) {};
  \node[red] at (s1.center) {\Huge \bf F};
%
  \begin{scope}
  \draw[->, line width=1, >=latex] (1.5, 0) -- node[above] {$I$} (3.5, 0);
  \end{scope}
%
  \begin{scope}[xshift = 5cm]
  \begin{scope}[
  %xscale=-1, 
  rotate=0, 
  transform shape,
  ]
  \node[name=s2, 
  regular polygon, 
  regular polygon sides=4, 
  minimum size=2cm, 
  draw, 
  color=red, 
  fill=red!20,
  line width=2pt, 
  ] 
  at (0,0) {};
  \node[red] at (s2.center) {\Huge \bf F};
 \end{scope}
 \end{scope}
\end{scope}
%
% R90
\begin{scope}[xshift = 90mm]
\node[name=s1, 
  regular polygon, 
  regular polygon sides=4, 
  minimum size=2cm, 
  draw, 
  color=red, 
  fill=red!20,
  line width=2pt, 
  ] 
  at (0,0) {};
  \node[red] at (s1.center) {\Huge \bf F};
% 
  \begin{scope}
  \draw[->, line width=1, >=latex] (1.5, 0) -- node[above] {$R_{90}$} (3.5, 0);
  \end{scope}
%
  \begin{scope}[xshift = 5cm]
  \begin{scope}[
  %xscale=-1, 
  rotate=-90, 
  transform shape,
  ]
  \node[name=s2, 
  regular polygon, 
  regular polygon sides=4, 
  minimum size=2cm, 
  draw, 
  color=red, 
  fill=red!20,
  line width=2pt, 
  ] 
  at (0,0) {};
  \node[red] at (s2.center) {\Huge \bf F};
 \end{scope}
 \end{scope}
\end{scope}
%
% R180
\begin{scope}[yshift = -30mm]
\node[name=s1, 
  regular polygon, 
  regular polygon sides=4, 
  minimum size=2cm, 
  draw, 
  color=red, 
  fill=red!20,
  line width=2pt, 
  ] 
  at (0,0) {};
  \node[red] at (s1.center) {\Huge \bf F};
%
  \begin{scope}
  \draw[->, line width=1, >=latex] (1.5, 0) -- node[above] {$R_{180}$} (3.5, 0);
  \end{scope}
%
  \begin{scope}[xshift = 5cm]
  \begin{scope}[
  %xscale=-1, 
  rotate=-180, 
  transform shape,
  ]
  \node[name=s2, 
  regular polygon, 
  regular polygon sides=4, 
  minimum size=2cm, 
  draw, 
  color=red, 
  fill=red!20,
  line width=2pt, 
  ] 
  at (0,0) {};
  \node[red] at (s2.center) {\Huge \bf F};
 \end{scope}
 \end{scope}
\end{scope}
%
% R270
\begin{scope}[yshift = -30mm, xshift=90mm]
\node[name=s1, 
  regular polygon, 
  regular polygon sides=4, 
  minimum size=2cm, 
  draw, 
  color=red, 
  fill=red!20,
  line width=2pt, 
  ] 
  at (0,0) {};
  \node[red] at (s1.center) {\Huge \bf F};
%
  \begin{scope}
  \draw[->, line width=1, >=latex] (1.5, 0) -- node[above] {$R_{270}$} (3.5, 0);
  \end{scope}
%
  \begin{scope}[xshift = 5cm]
  \begin{scope}[
  %xscale=-1, 
  rotate=-270, 
  transform shape,
  ]
  \node[name=s2, 
  regular polygon, 
  regular polygon sides=4, 
  minimum size=2cm, 
  draw, 
  color=red, 
  fill=red!20,
  line width=2pt, 
  ] 
  at (0,0) {};
  \node[red] at (s2.center) {\Huge \bf F};
 \end{scope}
 \end{scope}
\end{scope}
%
% H
\begin{scope}[yshift = -60mm, xshift=0mm]
\node[name=s1, 
  regular polygon, 
  regular polygon sides=4, 
  minimum size=2cm, 
  draw, 
  color=red, 
  fill=red!20,
  line width=2pt, 
  ] 
  at (0,0) {};
  \node[red] at (s1.center) {\Huge \bf F};
  \draw[line width=1, dashed] (-1, 0) -- (1, 0);
%
  \begin{scope}
  \draw[->, line width=1, >=latex] (1.5, 0) -- node[above] {$H$} (3.5, 0);
  \end{scope}
%
  \begin{scope}[xshift = 5cm]
  \begin{scope}[
  yscale=-1, 
  %rotate=-270, 
  transform shape,
  ]
  \node[name=s2, 
  regular polygon, 
  regular polygon sides=4, 
  minimum size=2cm, 
  draw, 
  color=red, 
  fill=red!20,
  line width=2pt, 
  ] 
  at (0,0) {};
  \node[red] at (s2.center) {\Huge \bf F};
 \end{scope}
 \end{scope}
\end{scope}
%
% V
\begin{scope}[yshift = -60mm, xshift=90mm]
\node[name=s1, 
  regular polygon, 
  regular polygon sides=4, 
  minimum size=2cm, 
  draw, 
  color=red, 
  fill=red!20,
  line width=2pt, 
  ] 
  at (0,0) {};
  \node[red] at (s1.center) {\Huge \bf F};
    \draw[line width=1, dashed] (0, -1) -- (0, 1);
%
  \begin{scope}
  \draw[->, line width=1, >=latex] (1.5, 0) -- node[above] {$V$} (3.5, 0);
  \end{scope}
%
  \begin{scope}[xshift = 5cm]
  \begin{scope}[
  xscale=-1, 
  %rotate=-270, 
  transform shape,
  ]
  \node[name=s2, 
  regular polygon, 
  regular polygon sides=4, 
  minimum size=2cm, 
  draw, 
  color=red, 
  fill=red!20,
  line width=2pt, 
  ] 
  at (0,0) {};
  \node[red] at (s2.center) {\Huge \bf F};
 \end{scope}
 \end{scope}
\end{scope}
%
% D
\begin{scope}[yshift = -90mm, xshift=0mm]
\node[name=s1, 
  regular polygon, 
  regular polygon sides=4, 
  minimum size=2cm, 
  draw, 
  color=red, 
  fill=red!20,
  line width=2pt, 
  ] 
  at (0,0) {};
  \node[red] at (s1.center) {\Huge \bf F};
  \draw[line width=1, dashed] (-1, -1) -- (1, 1);
%
  \begin{scope}
  \draw[->, line width=1, >=latex] (1.5, 0) -- node[above] {$D$} (3.5, 0);
  \end{scope}
%
  \begin{scope}[xshift = 5cm]
  \begin{scope}[
  yscale=-1, 
  rotate=-90, 
  transform shape,
  ]
  \node[name=s2, 
  regular polygon, 
  regular polygon sides=4, 
  minimum size=2cm, 
  draw, 
  color=red, 
  fill=red!20,
  line width=2pt, 
  ] 
  at (0,0) {};
  \node[red] at (s2.center) {\Huge \bf F};
 \end{scope}
 \end{scope}
\end{scope}
%
% D'
\begin{scope}[yshift = -90mm, xshift=90mm]
\node[name=s1, 
  regular polygon, 
  regular polygon sides=4, 
  minimum size=2cm, 
  draw, 
  color=red, 
  fill=red!20,
  line width=2pt, 
  ] 
  at (0,0) {};
  \node[red] at (s1.center) {\Huge \bf F};
  \draw[line width=1, dashed] (-1, 1) -- (1, -1);
%
  \begin{scope}
  \draw[->, line width=1, >=latex] (1.5, 0) -- node[above] {$D'$} (3.5, 0);
  \end{scope}
%
  \begin{scope}[xshift = 5cm]
  \begin{scope}[
  xscale=-1, 
  rotate=-90, 
  transform shape,
  ]
  \node[name=s2, 
  regular polygon, 
  regular polygon sides=4, 
  minimum size=2cm, 
  draw, 
  color=red, 
  fill=red!20,
  line width=2pt, 
  ] 
  at (0,0) {};
  \node[red] at (s2.center) {\Huge \bf F};
 \end{scope}
 \end{scope}
\end{scope}
\end{tikzpic}


\newpage 


{\bf Composition of symmetries:}

\begin{FL}
\vskip 5mm

\begin{tikzpic}[alt=Example of a composition of symmeties of a square.]
\node[name=s1, 
  regular polygon, 
  regular polygon sides=4, 
  minimum size=2cm, 
  draw, 
  color=red, 
  fill=red!20,
  line width=2pt, 
  ] 
  at (0,0) {};
  \node[red] at (s1.center) {\Huge \bf F};
   \draw[line width=1, dashed] (-1, -1) -- (1, 1);
%
  \begin{scope}
  \draw[->, line width=1, >=latex] (1.5, 0) -- node[above] {$D$} (3.5, 0);
  \end{scope}
%
  \begin{scope}[xshift = 5cm]
  \begin{scope}[
  yscale=-1, 
  rotate=-90, 
  transform shape,
  ]
  \node[name=s2, 
  regular polygon, 
  regular polygon sides=4, 
  minimum size=2cm, 
  draw, 
  color=red, 
  fill=red!20,
  line width=2pt, 
  ] 
  at (0,0) {};
  \node[red] at (s2.center) {\Huge \bf F};
 \end{scope}
\draw[line width=1, dashed] (0, -1) -- (0, 1);
 \end{scope}
%
  \begin{scope}[xshift=5cm]
  \draw[->, line width=1, >=latex] (1.5, 0) -- node[above] {$V$} (3.5, 0);
  \end{scope}
%
  \begin{scope}[xshift = 10cm]
  \begin{scope}[
  rotate=-270, 
  transform shape,
  ]
  \node[name=s2, 
  regular polygon, 
  regular polygon sides=4, 
  minimum size=2cm, 
  draw, 
  color=red, 
  fill=red!20,
  line width=2pt, 
  ] 
  at (0,0) {};
  \node[red] at (s2.center) {\Huge \bf F};
 \end{scope}
 \end{scope}
%
\begin{scope}
\draw[->, line width=1, >=latex] (0, -1) to[out=-20,in=200] node[below] {$V\circ D = R_{270}$} (10, -1);
\end{scope}
\end{tikzpic}


\vskip 5mm

\begin{tikzpic}[alt=Example of a composition of symmeties of a square.]
\node[name=s1, 
  regular polygon, 
  regular polygon sides=4, 
  minimum size=2cm, 
  draw, 
  color=red, 
  fill=red!20,
  line width=2pt, 
  ] 
  at (0,0) {};
  \node[red] at (s1.center) {\Huge \bf F};
   \draw[line width=1, dashed] (0, -1) -- (0, 1);
 %
  \begin{scope}
  \draw[->, line width=1, >=latex] (1.5, 0) -- node[above] {$V$} (3.5, 0);
  \end{scope}
 %
  \begin{scope}[xshift = 5cm]
  \begin{scope}[
  xscale=-1, 
  %rotate=-90, 
  transform shape,
  ]
  \node[name=s2, 
  regular polygon, 
  regular polygon sides=4, 
  minimum size=2cm, 
  draw, 
  color=red, 
  fill=red!20,
  line width=2pt, 
  ] 
  at (0,0) {};
  \node[red] at (s2.center) {\Huge \bf F};
 \end{scope}
\draw[line width=1, dashed] (-1, -1) -- (1, 1);
 \end{scope}
%
  \begin{scope}[xshift=5cm]
  \draw[->, line width=1, >=latex] (1.5, 0) -- node[above] {$D$} (3.5, 0);
  \end{scope}
%
  \begin{scope}[xshift = 10cm]
  \begin{scope}[
  rotate=-90, 
  transform shape,
  ]
  \node[name=s2, 
  regular polygon, 
  regular polygon sides=4, 
  minimum size=2cm, 
  draw, 
  color=red, 
  fill=red!20,
  line width=2pt, 
  ] 
  at (0,0) {};
  \node[red] at (s2.center) {\Huge \bf F};
 \end{scope}
 \end{scope}
%
\begin{scope}
\draw[->, line width=1, >=latex] (0, -1) to[out=-20,in=200] node[below] {$D\circ V = R_{90}$} (10, -1);
\end{scope}
\end{tikzpic}

\vskip 5mm
\end{FL}


\showto{SKELETON}{\vfill}


{\bf Composition table of symmetries of a square:}

\bigskip


\begin{center}
{\small
\tagpdfsetup{table/header-rows={1}}
\begin{tabular}{l !{\vrule width 2pt} llllllll}
\circ     & \ $I$       & $R_{90}$  & $R_{180}$ & $R_{270}$ & $H$       & $V$       & $D$       & $D'$      \\[1mm]
\noalign{\hrule height 2pt} \\[-3mm]
$I$       & \ $I$       & $R_{90}$  & $R_{180}$ & $R_{270}$ & $H$       & $V$       & $D$       & $D'$      \\
$R_{90}$  & \ $R_{90}$  & $R_{180}$ & $R_{270}$ & $I$       & $D'$      & $D$       & $H$       & $V$       \\
$R_{180}$ & \ $R_{180}$ & $R_{270}$ & $I$       & $R_{90}$  & $V$       & $H$       & $D'$      & $D$       \\
$R_{270}$ & \ $R_{270}$ & $I$       & $R_{90}$  & $R_{180}$ & $D$       & $D'$      & $V$       & $H$       \\
$H$       & \ $H$       & $D$       & $V$       & $D'$      & $I$       & $R_{180}$ & $R_{90}$  & $R_{270}$ \\
$V$       & \ $V$       & $D'$      & $H$       & $D$       & $R_{180}$ & $I$       & $R_{270}$ & $R_{90}$  \\[1mm]
$D$       & \ $D$       & $H$       & $D'$      & $V$       & $R_{270}$ & $R_{90}$  & $I$       & $R_{180}$ \\
$D'$      & \ $D'$      & $V$       & $D$       & $H$       & $R_{90}$  & $R_{270}$ & $R_{180}$ & $I$       \\
\end{tabular}
}
\end{center}


For $n \geq 3$ the dihedral group $D_{n}$ is defined as follows:
\bitem
\item {\bf Elements of  $D_{n}$:} symmetries of the regular polygon with $n$ sides.\\[-8mm] 
\item {\bf Group operation:} Composition of symmetries (e.g. $V\circ D = R_{270}$). \\[-8mm] 
\item {\bf The identity element:} The identity symmetry $I$. 
\eitem

\newpage

\begin{definition}{}{GROUP ORDER}
The \emph{order} of a group $G$ is the number of elements of $G$. It is denoted by 
$\left | G \right |$.

\medskip

If $G$ has infinitely many elements, we write $\left | G \right | = \infty$.
\end{definition}


\begin{FL}
{\bf Examples.}


\bitem
\item $|D_{4}| = 8$
\item $|\Z_{n}| = n$
\item $|\Z| = \infty$
\item $|\Q| = \infty$
\item $|\R| = \infty$
\item $|GL(n, \R)| = \infty$
\eitem
\end{FL}

\showto{SKELETON}{\vskip 50mm}


\begin{thm}{}{DIHEDRAL GROUP ORDER}
For $n \geq 3$ we have $|D_{n}| = 2n$.
\end{thm}



\begin{FL}
\begin{proof}
Let $P_{n}$ be a regular polygon with vertices $a_{1}, a_{2}, \dots, a_{n}$:
\begin{tikzpic}[alt=Regular polygon.]
    \node[
    name = s,
    regular polygon, 
    regular polygon sides=7, 
    minimum size=2.0cm, 
    color=red, 
    fill=red!20,
    line width=2pt,
    draw] 
    at (0,0) {};
%
\node[red] at (s.center) {\Huge \bf F};
\node[above] at (s.corner 1) {$a_{1}$};
\node[anchor=south west] at (s.corner 7) {$a_{2}$};
\node[right] at (s.corner 6) {$a_{3}$};
\node[anchor=north west] at (s.corner 5) {$a_{4}$};
\end{tikzpic}
Each symmetry of $P_{n}$ is uniquely determined once we know where is sends vertices 
$a_{1}$ and $a_{2}$. For every $1 \leq i \leq n$ there is a symmetry that sends 
$a_{1}$ to $a_{i}$ and $a_{2}$ to $a_{i+1}$ given by a rotation:
\begin{tikzpic}[alt=Rotation symmetry of a regular polygon.]
\begin{scope}
\node[name=s1, 
  regular polygon, 
  regular polygon sides=7, 
  minimum size=2.0cm, 
  draw, 
  color=red, 
  fill=red!20,
  line width=2pt, 
  ] 
  at (0,0) {};
  \node[red] at (s1.center) {\Huge \bf F};
  \node[above] at (s1.corner 1) {$a_{1}$};
  \node[anchor=south west] at (s1.corner 7) {$a_{2}$};
%
  \begin{scope}
  \draw[->, line width=1, >=latex] (2, 0) -- node[above] {} (4, 0);
  \end{scope}
%
  \begin{scope}[xshift = 6cm]
  \begin{scope}[
  %xscale=-1, 
  rotate=-360*2/7, 
  transform shape,
  ]
  \node[name=s2, 
  regular polygon, 
  regular polygon sides=7, 
  minimum size=2.0cm, 
  draw, 
  color=red, 
  fill=red!20,
  line width=2pt, 
  ] 
  at (0,0) {};
  \node[red] at (s2.center) {\Huge \bf F};
 \end{scope}
   \node[right] at (s2.corner 1) {$a_{1}$};
  \node[anchor=north west] at (s2.corner 7) {$a_{2}$};
 \end{scope}
\end{scope}
\end{tikzpic}
Also, for every $i$ we there is a symmetry that sends 
$a_{1}$ to $a_{i}$ and $a_{2}$ to $a_{i-1}$. This is given by a composition of 
a reflection with respect to a line that passes through $a_{1}$ and a rotation:

\begin{tikzpic}[alt=Composition of reflection and a rotation of a regular polygon.]
\begin{scope}
\node[name=s1, 
  regular polygon, 
  regular polygon sides=7, 
  minimum size=2.0cm, 
  draw, 
  color=red, 
  fill=red!20,
  line width=2pt, 
  ] 
  at (0,0) {};
  \node[red] at (s1.center) {\Huge \bf F};
  \node[above] at (s1.corner 1) {$a_{1}$};
  \node[anchor=south west] at (s1.corner 7) {$a_{2}$};
%
  \begin{scope}
  \draw[->, line width=1, >=latex] (2, 0) -- node[above] {} (4, 0);
  \end{scope}
%
  \begin{scope}[xshift = 6cm]
  \begin{scope}[
  xscale=-1, 
  %rotate=-360*2/7, 
  transform shape,
  ]
  \node[name=s2, 
  regular polygon, 
  regular polygon sides=7, 
  minimum size=2.0cm, 
  draw, 
  color=red, 
  fill=red!20,
  line width=2pt, 
  ] 
  at (0,0) {};
  \node[red] at (s2.center) {\Huge \bf F};
 \end{scope}
   \node[above] at (s2.corner 1) {$a_{1}$};
  \node[anchor=south east] at (s2.corner 7) {$a_{2}$};
 \end{scope}
%
  \begin{scope}
  \draw[->, line width=1, >=latex] (8, 0) -- node[above] {} (10, 0);
  \end{scope}
%
  \begin{scope}[xshift = 12cm]
  \begin{scope}[
  xscale=-1, 
  rotate=360*2/7, 
  transform shape,
  ]
  \node[name=s2, 
  regular polygon, 
  regular polygon sides=7, 
  minimum size=2.0cm, 
  draw, 
  color=red, 
  fill=red!20,
  line width=2pt, 
  ] 
  at (0,0) {};
  \node[red] at (s2.center) {\Huge \bf F};
 \end{scope}
   \node[anchor=west] at (s2.corner 1) {$a_{1}$};
  \node[anchor=south west] at (s2.corner 7) {$a_{2}$};
 \end{scope}
%
\end{scope}
\end{tikzpic}

Altogether, this gives $2n$ possible symmetries of $P_{n}$. 
\end{proof}
\end{FL}


\showto{SKELETON}{\newpage}

\vskip 10mm

{\bf Note.} The dihedral group $D_{n}$ consists of the following elements:
\benu
\item[1)] $n$ rotations by the angles of $k\cdot \frac{360}{n}$ degrees 
for $k=0, \dots, n-1$. For $k=0$ this gives the rotation by 0 degrees, 
i.e. the identity symmetry. \\[0mm]

\item[2)] $n$ reflections with respect to different symmetry axes. If $n$ is 
odd, there is one symmetry axis for each vertex of the polygon $P_{n}$:

\ 

\begin{tikzpic}[alt=Axes of symmetry of a regular pentagon.][scale=2]
\pgfmathsetmacro{\afive}{{360/5}}
\pgfmathsetmacro{\rfive}{{1/(1+sin(90 - \afive/2))}}
\pgfmathsetmacro{\xfive}{{\rfive * sin(\afive)/sin(90 - \afive/2)}}
\pgfmathsetmacro{\hfive}{{\rfive * sin(90 - \afive/2)}}
\draw[red, fill=red!20, line width=2pt]  
(-\xfive/2, 0) 
-- (\xfive/2, 0) 
-- ({\xfive * sin(90 - \afive/2)}, {\xfive*sin(\afive)})
-- (0, 1)
-- ({-\xfive * sin(90 - \afive/2)}, {\xfive*sin(\afive)}) 
-- cycle;
\pgfmathsetmacro{\x}{{0.5}}
\foreach \i in {0, 1, 2, 3, 4}
{
\draw[line width=1, shorten >= -\x cm, shorten <=-\x cm, 
rotate around={\i*\afive:(0,\hfive)}] (0, 1) -- (0, 0);
}
\end{tikzpic}


\ 

If $n$ is even, there are $\frac{n}{2}$ symmetry axes passing through pairs of
opposite vertices and $\frac{n}{2}$ symmetry axes crossing opposite sides 
of $P_{n}$: 

\ 

\begin{tikzpic}[alt=Axes of symmetry of a square.][scale=2]
\pgfmathsetmacro{\x}{{0.5}}
\draw[red, fill=red!20, line width=2pt] 
(-0.5, -0.5) 
-- (0.5, -0.5) 
-- (0.5, 0.5) 
-- (-0.5, 0.5) 
-- cycle;
\draw[line width=1, shorten >= -\x cm, shorten <=-\x cm] (-\x, 0) to (\x, 0);
\draw[line width=1, shorten >= -\x cm, shorten <=-\x cm] (0, -\x) -- (0, \x);
\draw[line width=1, shorten >= -\x cm, shorten <=-\x cm] (-\x, -\x) -- (\x, \x);
\draw[line width=1, shorten >= -\x cm, shorten <=-\x cm] (-\x, \x) -- (\x, -\x);
\end{tikzpic}


\eenu


\newpage



\begin{definition}{}{GROUP GENERATORS}
Let $G$ be a group, and let $S\subseteq G$ be a subset of $G$. 
We say that the set $S$ \emph{generates} G if every element 
of $G$ can be obtained as a product of some elements of $S$ 
and inverses of elements of $S$.   
\end{definition}


\begin{FL}
{\bf Examples.} 

\bitem
\item
Let $P_{n}$ be a regular polygon with vertices $a_{1}, \dots, a_{n}$. 
In the proof of  \Cref{thm:DIHEDRAL GROUP ORDER} we have seen that every 
symmetry of $P_{n}$ can be obtained by composing some rotation 
of $P_{n}$ and a reflection $D$ with respect to the line that passes through 
the vertex $a_{1}$. Moreover, every rotation of $P_{n}$ can be obtained by 
composing some number of times the rotation $R$ by the angle of $\frac{360}{n}$ 
degrees. Thus, every symmetry of $P_{n}$ can be obtainted as some product of 
$R$ and $D$. This means that the set $\{R, D\}$ generates the dihedral group $D_{n}$.  
\\[0mm]


\item
The group of integers $\Z$ is generated by a set $\{ 1 \}$ consisting of 
single element $1\in \Z$, since every element of $\Z$ can be obtained by adding some 
number of times $1$ and $-1$. 
\\[0mm]


\item
The group of integers $\Z_{n}$ is generated by a set $\{ 1 \}$ consisting of 
single element $1\in \Z$. 
\\[0mm]

\item
The set $\{2\}$ generates $\Z_{3}$, but it does not generate $\Z_{4}$. 
\\[0mm]
\eitem
\end{FL}

