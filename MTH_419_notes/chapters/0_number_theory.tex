% !TEX TS-program = lualatex-dev
% !TEX root = ../mth419_lecture_notes.tex

\lecture{Some number theory}

\underline{\bf Divisibility of integers.}

If $m, n$ are integers then we say that $m$ \emph{divides} $n$ if 
$n = km$ for some integer $k$. We write: $m|n$.

\bigskip

{\bf Some properties of divisibility:}
\bitem
\item For every $n$ we have $1|n$. \\[-3mm]
\item For every $m\neq 0$ we have $m|0$ \\[-3mm]
\item If $k|m$ and $m|n$ then $k|n$ \\[-3mm]
\item If $m|n$ and $n|m$ then either $m=n$ or $m=-n$
\eitem

\vskip 20mm 

\underline{\bf The Greatest Common Divisor}

The \emph{greatest common divisor} of integers $n_{1}, \dots, n_{k}$ is the 
greatest integer $m$ such that $m|n_{i}$ for $i=1,\dots, k$. We denote this 
integer by $\gcd(n_{1}, \dots, n_{k})$.

{\bf Some properties of gcd:}
\bitem
\item If $m|n_{i}$ for $i=1, \dots, k$ then $m|gcd(n_{1}, \dots, n_{k})$.
\item If $n > 0$ then $\gcd(0, n) = n$  
\item $\gcd(m, n) = \gcd(m, n-m)$ \\[-3mm]
\item If $m, n$ are non-zero integers that there exists integers $a, b$
such that 
\[
\gcd(m, n) = am + bn
\]
Moreover, $\gcd(m, n)$ is the smallest positive integer of such form.
\eitem

If $\gcd(m, n) = 1$ then we say that $m$ and $n$ are \emph{relatively prime}.

\newpage


\underline{\bf The Least Common Multiple}

The \emph{least common multiple} of integers $n_{1}, \dots, n_{k}$ is the 
smallest positive integer $m$ such that $n_{i}|m$ for $i=1,\dots, k$. We denote this 
integer by $\lcm(n_{1}, \dots, n_{k})$.

{\bf Some properties of lcm:}
\bitem
\item If $m$ is an integer such that $n_{i}|m$ for $i=1, \dots, k$ then 
$\lcm(n_{1}, \dots, n_{k})| m$. \\[-2mm] 
\item $\lcm(n_{1}, \dots, n_{k}) 
= \dfrac{n_{1}\cdot{\dots}\cdot n_{k}}{\gcd(n_{1}, \dots, n_{k})}$
\eitem

\vskip 20mm 



\underline{\bf Prime numbers.}

An integer $p>1$ is prime if the only positive integers dividing $p$ are 
$p$ and $1$. 

\bigskip

{\bf Some properties of primes:}
\bitem
\item If $p$ is a prime and $p|mn$ then  either $p|m$ or $p|n$. \\[-1mm]
\item If $n>1$ is any integer then there is a unique way of writing 
$n$ as a product of primes:
\[
n = p_{1}p_{2}\cdot{\dots}\cdot p_{k}
\]
such that $p_{1}\geq p_{2}\geq {\dots} \geq p_{k}$.
 \\[-3mm]

\eitem
