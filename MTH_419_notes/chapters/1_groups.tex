% !TEX TS-program = lualatex-dev
% !TEX root = ../mth419_lecture_notes.tex

\lecture{Groups}

\vskip -7mm

\begin{definition}{}{DEF:GROUP}
A \emph{group} is set $G$ equipped with an operation that assigns to 
each pair of elements $a, b \in G$ an element $a\cdot b \in G$ in such 
way, that the following conditions are satisfied:
\benu
\item[{\bf 1)}] For any $a, b, c\in G$ we have 
\[
(a\cdot b)\cdot c = a\cdot (b \cdot c)
\]
(associativity).
\\[-1mm]
\item[{\bf 2)}] There exists an element $e\in G$ such that 
\[
e\cdot a = a \cdot e = a
\]
for all $a\in G$. The element $e$ is called the \emph{identity element} or the 
\emph{trivial element}.
\\[-1mm]
\item[{\bf 3)}] For each element $a\in G$ there exists an element $b\in G$ such that 
\[
a\cdot b = b \cdot a = e
\]
Such element $b$ is called the \emph{inverse} of $a$ and it is denoted by $a^{-1}$. 
\eenu 
\end{definition}

\begin{FL}
{\bf Note.} The first condition in the definition of a group implies that if we want 
to multiply several elements, then it does not matter how we place parentheses. E.g.:
\[
((ab)c)d = (ab)(cd) = a(b(cd)) = a((bc)d) = (a(bc))d
\]
\end{FL}


\showto{SKELETON}{\newpage}

\begin{definition}{}{DEF:ABGROUP}
A \emph{abelian group} is a group $G$ where the multiplication is commutative: 
\[
a \cdot b = b \cdot a
\] 
for all $a, b\in G$. 
\end{definition}



\showto{FULL}{\newpage}
\showto{SKELETON}{\vskip 30mm}

{\bf \underline{Notation}}


{\bf Multiplicative:} 

\benu
\item[\textbullet] $a\cdot b$, \ $a \times b$, \ $a \ast b$, \ 
$a \circ b$,\  $a \odot b$, \dots \\[-5mm]
\item[\textbullet] {\bf Inverse element:} $a^{-1}$.  \\[-5mm]
\item[\textbullet] {\bf The identity element:} $e$,  $1$, \dots
\eenu



{\bf  Additive:} 

\benu
\item[\textbullet] $a + b$ \\[-5mm]
\item[\textbullet] {\bf Inverse element:} $-a$.  \\[-5mm]
\item[\textbullet] {\bf The identity element:} $0$.
\eenu

{\bf Note:} The additive notation is only used for abelian groups.


\showto{SKELETON}{\newpage}
\showto{FULL}{\vskip 10mm}

{\bf \underline{Some examples of groups}}

\begin{shownto}{FULL}
\bitem
\item $\Z$ - the group of integers (with addition) \\[-5mm]
\item $\Q$ - the group of rational numbers (with addition)  \\[-5mm]
\item $\R$ - the group of real numbers (with addition)  \\[-5mm]
\item $\C$ - the group of complex numbers (with addition)  \\[-5mm]
\eitem



\bitem
\item $\Q^{\ast}$ - the group of non-zero rational numbers (with multiplication)  \\[-5mm]
\item $\R^{\ast}$ - the group of non-zero real numbers (with multiplication)  \\[-5mm]
\item $\C^{\ast}$ - the group of non-zero complex numbers (with multiplication)  \\[-5mm]
\eitem



\bitem
\item $\Q^{+}$ - the group of positive rational numbers (with multiplication)  \\[-5mm]
\item $\R^{+}$ - the group of positive real numbers (with multiplication)  \\[-5mm]
\eitem


\bitem
\item The trivial group $\{e\}$.
\eitem

\end{shownto}


\showto{FULL}{\vskip 5mm}
\showto{SKELETON}{\newpage}

{\bf \underline{Example:} General linear groups $GL(n, \R)$.}

\begin{FL}
\bitem
\item {\bf Elements of  $GL(n, \R)$:} invertible $n\times n$ matrices with real entries.
\item {\bf Group operation:} matrix multiplication.
\item {\bf The identity element:} the identity matrix $I_{n}$. 
\eitem
\vskip 10mm
\end{FL}


\showto{SKELETON}{\vskip 80mm}

{\bf \underline{Example:} Groups $\Z_{n}$}

\begin{FL}
{\bf Notation:} Let $n > 0$ be an integer. For any integer $m$ we have 
\[
m = qn + r
\]
where $q, r\in \Z$ and $0 \leq r < n$. Then we write 
\[
m \bmod n = r
\]

\vskip 10mm

For an integer $n\geq 2$ the group $\Z_{n}$ is defined as follows:

\bitem
\item {\bf Elements of  $\Z_{n}$:} numbers $0, 1, \dots, n-1$
\item {\bf Group operation $\oplus$:} For $k, l\in \Z_{n}$ we set
\[
k \oplus l \coloneq (k+l) \bmod n
\]
\item {\bf The identity element:} $0$. 
\item {\bf Inverses:} The inverse of an element $k\in \Z_{n}$ is the element $n-k$. 

\eitem


\vskip 10mm

\end{FL}

\showto{SKELETON}{\newpage}


{\bf \underline{Example:} Groups $U(n)$}

\bigskip

{\bf Recall:} 
\bitem
\item If $m, n$ are integers then  $\gcd(m, n)$, is the greatest integer that divides 
both $m$ and $n$. \\[0mm]


\item For any $m, n\neq 0$ there exists $a, b\in \Z$ such that 
\[
am + bn = \gcd(m, n)
\] 
Moreover, $\gcd(m, n)$ is the smallest positive integer that can be obtained for 
any choice of $a$ and $b$. 
\eitem


\begin{FL}
\vskip 10mm 

For an integer $n\geq 2$ the group $U(n)$ is defined as follows:

\bitem
\item {\bf Elements of  $U(n)$:} integers $1 \leq k < n$ such that $\gcd(k, n) = 1$ 
\item {\bf Group operation $\odot$:} For $k, l\in U(n)$ we set
\[
k \odot l \coloneq (k\cdot l) \bmod n
\]
\item {\bf The identity element:} $1$. 
\item {\bf Inverses:} If $k$ is en element of $U(n)$, then we can find $p, q \in \Z$
such that  $pk + qn = \gcd(k, n) = 1$. Let $\overline{p} = p \mod n$.  
Notice that we have $\gcd(p, n) = 1$, so also $\gcd(\overline{p}, n) = 1$. Also, 
\[
(\overline{p}\cdot k) \mod n =  (pk) \mod n = (pk + qn) \mod n = 1
\]
This means that $\overline{p} = k^{-1}$ in the group $U(n)$.

\eitem

\end{FL}

