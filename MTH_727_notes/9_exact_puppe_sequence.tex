% !TEX root = mth727_lecture_notes.tex


\chapter[Exact Puppe Sequence]{Exact Puppe Sequence}
\chaptermark{Exact Puppe Sequence}
\label{EXACT PUPPE SEQUENCE CHAPTER}
\thispagestyle{firststyle}

Recall that a sequence of pointed sets 
\[
(S_{2}, s_{2}) \overset{f}{\lra} (S_{1}, s_{1}) \overset{g}{\lra} (S_{0}, s_{0})
\]
is \emph{exact at $S_{1}$} if $f(S_{2}) = g^{-1}(s_{0})$. 


For pointed spaces $(X, x_{0})$ and $(Y, y_{0})$ let $[X, Y]_{\ast}$ denote the set of 
pointed homotopy classes of maps $X\to Y$. This is a pointed set, with the basepoint 
represented by the constant function $c_{y_{0}}\colon X \to Y$, $c_{y_{0}}(x) = y_{0}$
for all $x\in X$. 


\begin{definition}
A pointed space $(X, x_{0})$ is well-pointed if the pair $(X, x_{0})$ has the homotopy 
extension property.
\end{definition}


\begin{definition}
A sequence of maps of spaces 
\[
(X_{0}, x_{0}) \overset{f_{0}}{\lra} (X_{1}, x_{1}) \overset{f_{1}}{\lra} (X_{2}, x_{2})
\]
is \emph{exact at $X_{1}$} is for any well-pointed space $(Y, y_{0})$ the sequence 
pointed sets 
\[
[Y, X_{0}]_{\ast} \overset{f_{0\ast}}{\lra}
[Y, X_{1}]_{\ast} \overset{f_{1\ast}}{\lra}
[Y, X_{2}]_{\ast}
\]
is exact at $[Y, X_{1}]_{\ast}$.
\end{definition}


\begin{proposition}
If $p\colon E\to B$ is a Hurewicz fibration, $e_{0}\in E$, $b_{0} = p(e_{0})\in B$
$F = p^{-1}(b_{0})$, and $i\colon F \to E$ is the inclusion map 
then the sequence $(F, e_{0}) \overset{i}{\lra}  (E, e_{0}) 
\overset{p}{\lra} (B, b_{0})$ is exact at $E$.
\end{proposition}

\begin{proof}
Exercise.
\end{proof}

Let $f\colon (X, x_{0}) \to (Y, y_{0})$ be any pointed map. Consider the sequence 
\[
\hofib f \overset{i(f)}{\lra} X \overset{f}{\lra} Y
\]
where 
\[
i(f)\colon 
\hofib_{y_{0}}f = \{(x, \omega) \in X \times PY \ |\ 
f(x)=\omega(0), \ \omega(1) = y_{0}\} \lra X
\]
is given by $i(f)(x, \omega) = x$.
Since this sequence is homotopy equivalent to a sequence given by a Hurewicz fibration, 
it is exact at $X$. We can continue this construction inductively, by taking  
consecutive homotopy fibers:
\begin{equation*}
\label{ITERATED HOFIB EQ}
\tag{$\ast$}
\dots \lra
\hofib i^{3}(f) \overset{i^{4}(f)}{\lra} 
\hofib i^{2}(f) \overset{i^{3}(f)}{\lra} 
\hofib i(f) \overset{i^{2}(f)}{\lra} 
\hofib f \overset{i(f)}{\lra} X \overset{f}{\lra} Y
\end{equation*}
In this way we obtain a sequence which is exact at all spaces. As it turns out, 
this sequence has a more convenient description. The starting point for it is 
the following fact:

\begin{proposition}
\label{HOFIB MAP IS FIBRATION PROP}
Let $f\colon X\to Y$ be a map and $y_{0}\in Y$. Then the map 
$i(f)\colon \hofib_{y_{0}}f \lra X$ is a Hurewicz fibration.
\end{proposition} 

\begin{proof}
Exercise.
\end{proof}


\begin{corollary}
\label{HOFIB I(F) IS LOOP SPACE COR}
For any map of pointed spaces $f\colon (X, x_{0}) \to (Y, y_{0})$ we have 
a commutative diagram 
\begin{equation*}
\begin{tikzpicture}
\matrix (m) 
[matrix of math nodes, row sep= 2em, column sep=2em, text height=1.5ex, text depth=0.25ex]
{
\hofib i(f)  & \hofib f & X & Y \\
\Omega Y     & \\ 
};
\path[->, thick, font=\scriptsize]
(m-1-1) 
edge node[above] {$i^{2}(f)$} (m-1-2)
(m-1-2)

edge node[above] {$i(f)$} (m-1-3)
(m-1-3)
edge node[above] {$f$} (m-1-4)
(m-2-1)
edge node[anchor= east] {$g$} node[anchor= west] {$\simeq$} (m-1-1)
edge node[anchor= north west] {$j$} (m-1-2)
; 
\end{tikzpicture}
\end{equation*}
\end{corollary}


\begin{proof}
We have 
\[
i(f)^{-1}(x_{0}) = \{(x_{0}, \omega) \in X \times PY \ |\ 
\omega(0) = f(x_{0}) = y_{0}, \ \omega(1) = y_{0}\} \cong \Omega Y
\]
Thus $\Omega Y$ can be identified with the fiber of $i(f)$ over $y_{0}$,
and the map $j\colon \Omega Y \to \hofib f$, $j(\omega) = (x_{0}, \omega)$
with the inclusion of the fiber. By Proposition \ref{HOFIB MAP IS FIBRATION PROP}
and Corollary \ref{FIBER VS HOFIBER COROLLARY} we obtain a homotopy 
equivalence $g\colon \Omega Y \to \hofib i(f)$ such that the above diagram commutes.
\end{proof}

\begin{note}
The homotopy equivalence in Corollary \ref{HOFIB I(F) IS LOOP SPACE COR}
can be explicitly described as follows. Up to a homeomorphism we have 
\[
\hofib i(f) = 
\{ (\omega, \tau)\in PX \times PY \ | \ 
f\omega(0) = \tau(0), \ \omega(1) = y_{0}, \ \tau(1) = x_{0}\}
\]
Then $i^{2}(f)\colon \hofib i(f) \to \hofib f$ is given by 
$(\omega, \tau) = (\omega(0), \tau)$ and $g(\tau) = (c_{x_{0}}, \tau)$. 
\end{note}

Applying Corollary \ref{HOFIB I(F) IS LOOP SPACE COR} iteratively to the sequence 
(\ref{ITERATED HOFIB EQ}) we get homotopy equivalences
\begin{align*}
& \hofib i(f) \overset{\simeq}{\longleftarrow}\Omega Y  \\
& \hofib i^{2}(f) \overset{\simeq}{\longleftarrow}\Omega X  \\
& \hofib i^{3}(f) \overset{\simeq}{\longleftarrow}\Omega \hofib f  \\
& \hofib i^{4}(f) \overset{\simeq}{\longleftarrow}\Omega \hofib i(f) \simeq \Omega^{2} Y \\
& \hofib i^{5}(f) \overset{\simeq}{\longleftarrow}\Omega \hofib i^{2}(f) \simeq \Omega^{2} X \\
& \dots \ \ \ \dots \ \ \ \dots \ \ \ \ \dots \ \ \ \dots \\
\end{align*}
Moreover, one can check that the following diagram commutes up to homotopy:

\begin{equation*}
\label{PUPPE EXACT SEQ EQ}
\tag{$\ast\ast$}
\begin{tikzpicture}[baseline=(current  bounding  box.center)]
\matrix (m) 
[matrix of math nodes, row sep=3em, column sep=1.5em, text height=1.5ex, text depth=0.25ex]
{
{\dots}  & \hofib i^{4} (f) & \hofib i^{3} (f) & \hofib i^{2} (f) & \hofib i(f) & \hofib f & X & Y \\
{\dots}  & \Omega^{2} Y & \Omega \hofib f & \Omega X & \Omega Y & \hofib f & X & Y \\
};
\path[->, thick, font=\scriptsize]
(m-1-1)
edge (m-1-2)
(m-1-2)
edge node[above] {$i^{5}(f)$} (m-1-3)
(m-1-3) 
edge node[above] {$i^{4}(f)$} (m-1-4)
(m-1-4) 
edge node[above] {$i^{3}(f)$} (m-1-5)
(m-1-5) 
edge node[above] {$i^{2}(f)$} (m-1-6)
(m-1-6) 
edge node[above] {$i(f)$} (m-1-7)
(m-1-7) 
edge node[above] {$f$} (m-1-8)

(m-2-1)
edge (m-2-2)
(m-2-2) 
edge node[below] {$\Omega j$} (m-2-3)
edge node[pos=0.45, anchor = south, rotate=90] {$\simeq$} (m-1-2)
(m-2-3) 
edge node[below] {$\Omega i(f)$} (m-2-4)
edge node[pos=0.45, anchor = south, rotate=90] {$\simeq$} (m-1-3)
(m-2-4) 
edge node[below] {$\Omega f$} (m-2-5)
edge node[pos=0.45, anchor = south, rotate=90] {$\simeq$} (m-1-4)
(m-2-5) 
edge node[below] {$j$} (m-2-6)
edge node[anchor=south, rotate=90] {$\simeq$}(m-1-5)
(m-2-6) 
edge node[below] {$i(f)$} (m-2-7)
edge node[pos=0.45, anchor = south, rotate=90] {$=$} (m-1-6)
(m-2-7) 
edge node[below] {$f$} (m-2-8)
edge node[pos=0.45, anchor = south, rotate=90] {$=$} (m-1-7)
(m-2-8) 
edge node[pos=0.45, anchor = south, rotate=90] {$=$} (m-1-8)
;
\end{tikzpicture}
\end{equation*}

Since the upper row of this diagram is exact, the same is true for the lower row. 

\begin{definition}
The sequence in the lower row of the diagram (\ref{PUPPE EXACT SEQ EQ})
is called the \emph{Puppe exact sequence} associated to the map $f$.
\end{definition}

As a consequence, for any map of pointed spaces $f\colon (X, x_{0}) \to (Y, y_{0})$
and any well-pointed space $(Z, z_{0})$ we obtain a long exact sequence of sets: 
\begin{multline*}
\label{PUPPE EXACT HOMOTOPY CLASSES SEQ EQ}
\tag{\maltese}
\dots 
\overset{\Omega^{2} f_{\ast}}{\lra} [Z, \Omega^{2} Y]_{\ast}
\overset{\Omega j_{\ast}}{\lra} [Z, \Omega \hofib f ]_{\ast}
\overset{\Omega i(f)_{\ast}}{\lra} [Z, \Omega X ]_{\ast} 
\overset{\Omega f_{\ast}}{\lra} [Z, \Omega Y ]_{\ast} 
\\
\overset{j_{\ast}}{\lra} [Z, \hofib f ]_{\ast} 
\overset{i(f)_{\ast}}{\lra} [Z, X ]_{\ast} 
\overset{f_{\ast}}{\lra} [Z, Y ]_{\ast} 
\end{multline*}

\begin{note}
\label{LOOP SPACE TO GROUP STRUCTURE NOTE}
For any pointed space $(X, x_{0})$ and $n\geq 1$ the loop space $\Omega^{n} X$ 
is quipped with a multiplication map 
$\mu\colon \Omega^{n} X \times \Omega^{n} X \to \Omega^{n} X$
given by concatenation of loops. 
For any pointed space $(Z, z_{0})$ this defines a multiplication 
\[
\mu_{\ast}\colon [Z, \Omega^{n} X]_{\ast} \times [Z, \Omega^{n} X]_{\ast} 
\to [Z, \Omega^{n} X]_{\ast}
\]
given by $\mu_{\ast}([\varphi], [\psi]) = [\mu\circ (\varphi\times \psi)]$.
This equips the set $[Z, \Omega^{n} X]_{\ast}$ with a group structure.
Moreover, for $n\geq 2$ the multiplication $\mu$ commutes up to homotopy, 
and in effect $[Z, \Omega^{n} X]_{\ast}$ becomes an abelian group.

As a result the exact sequence (\ref{PUPPE EXACT HOMOTOPY CLASSES SEQ EQ})
becomes an exact sequence of groups starting at $[Z, \Omega Y]_{\ast}$
and its groups are abelian starting with $[Z, \Omega^{2} Y]_{\ast}$
\end{note}




\begin{nn}{\bf Loop spaces and suspensions.} There is a different way of interpreting 
group structures appearing in the sequence (\ref{PUPPE EXACT HOMOTOPY CLASSES SEQ EQ}), 
which uses suspensions of a space in place of loop spaces.
\end{nn}




\begin{definition}
Let $X$ be a space. The \emph{unreduced suspension} of $X$ if the space
\[
SX = X \times [0, 1]/(X\times \{0, 1\})
\]
\end{definition}





\begin{note}
Any map $f\colon X \to Y$ defines a map $Sf\colon SX \to SY$ given by $Sf([x, t]) = [f(x), t]$. 
This map is called the suspension of $f$. In this way we obtain the suspension functor
\[
S \colon \Top \to \Top
\]
This functor preserves homotopy classes of maps: if $f, g\colon X \to Y$ and $f\simeq g$
then $Sf \simeq Sg$. 
\end{note}

\begin{example}
For a sphere $S^{n}$ we have $SS^{n} \cong S^{n+1}$. 
\end{example}


\begin{definition}
Let $(X, x_{0})$ be a pointed space. The \emph{reduced suspension} of $X$ is the 
pointed space
\[
\Sigma X = X \times [0, 1]/(X\times \{0, 1\}\cup \{x_{0}\}\times [0, 1])
\]
or equivalently $\Sigma X = SX/\{[x_{0}, t] \ | \ t\in [0, 1]\}$.
The basepoint in $\Sigma X$ is given by $[x_{0}, 0]\in \Sigma X$. 
\end{definition}


\begin{note}
If $(X, x_{0})$ is a well-pointed space, then  Proposition \ref{CONTR QUOTIENT WITH HEP PROP}
implies that the quotient map $SX \to \Sigma X$ is a homotopy equivalence. In particular, 
for any basepoint $x_{0}\in S^{n}$ we have $\Sigma S^{n} \simeq SS^{n} \cong S^{n+1}$. 
One can show that actually there is a homeomorphism $\Sigma  S^{n} \cong S^{n+1}$ 
\end{note}


\begin{note}
\label{SUSPENSION LOOP ADJUNCTION NOTE}
Any map  $f\colon (X, x_{0}) \to (Y, y_{0})$ of pointed spaces, defines a map  
$\Sigma f\colon \Sigma X \to  \Sigma Y$ given by $\Sigma f ([x, t]) = [f(x), t]$. This 
defines the suspension functor
\[
\Sigma \colon \Top_{\ast} \to \Top_{\ast}
\] 
\end{note}

Similarly as for the unreduced suspension, the reduced suspension preserves 
homotopy classes: if $f, g\colon (X, x_{0}) \to (Y, y_{0})$ are maps of pointed spaces 
and  $f\simeq g$ then $\Sigma f \simeq \Sigma g$. 

Let $X$ be a Hausdorff space. By properties of mapping spaces 
(\ref{MAPPING SPACE PROPERTIES NN}) the adjunction map $\adj(\omega) = \omega^{\sharp}$
defines a homeomorphism $\adj\colon \Map(X\times [0, 1], Y) \to \Map(X, PY)$. Let 
$(X, x_{0})$ and  $(Y, y_{0})$ be pointed spaces. Consider $\Omega_{y_{0}} Y$ as a subspace 
of $PY$ and let $\Map_{\ast}(X, \Omega_{y_{0}} Y)$ be the subspace of    
$\Map(X, PY)$ consisting of basepoint preserving maps. Then $\adj$ 
restricts to a homeomorphism between this subspace and the subspace of 
$\Map(X\times [0, 1], Y)$ consisting of all maps $f\colon X\times [0, 1]\to Y$ such that 
$f(X\times \{0, 1\}\cup \{x_{0}\}\times [0, 1]) = y_{0}$. Such maps are in a bijective 
correspondence with basepoint preserving maps $\Sigma X \to Y$. In this way we 
obtain a homeomorphism 
\[
\adj\colon \Map_{\ast}(\Sigma X, Y) \overset{\cong}{\lra} \Map_{\ast}(X, \Omega Y)
\]
On the level of homotopy classes of maps this gives a bijection 
\[
\adj\colon [\Sigma X, Y]_{\ast} \overset{\cong}{\lra} [X, \Omega Y]_{\ast}
\]
The set of the right hand side has a group structure induced by concatenation of 
loops. A group structure on the left hand side can be defined using the pinch map 
$\Sigma X \to \Sigma X \vee \Sigma X$. In this way the above bijection becomes 
an isomorphism of groups. 

As a result, the exact sequence (\ref{PUPPE EXACT HOMOTOPY CLASSES SEQ EQ}) can be 
equivalently written as 

\begin{multline*}
\dots 
\overset{f_{\ast}}{\lra} [\Sigma^{2} Z, Y]_{\ast}
\overset{j_{\ast}\adj}{\lra} [\Sigma Z,  \hofib f ]_{\ast}
\overset{i(f)_{\ast}}{\lra} [\Sigma Z,  X ]_{\ast} 
\overset{f_{\ast}}{\lra} [\Sigma Z, Y ]_{\ast} 
\overset{j_{\ast}\adj}{\lra} [Z, \hofib f ]_{\ast} 
\overset{i(f)_{\ast}}{\lra} [Z, X ]_{\ast} 
\overset{f_{\ast}}{\lra} [Z, Y ]_{\ast} 
\end{multline*}

Consider this sequence with $Z=S^{0}$. Since $\Sigma^{n} S^{0}\cong S^{n}$ we obtain

\begin{multline*}
\dots 
\overset{f_{\ast}}{\lra} [S^{2}, Y]_{\ast}
\overset{j_{\ast}\adj}{\lra} [S^{1},  \hofib f ]_{\ast}
\overset{i(f)_{\ast}}{\lra} [S^{1},  X ]_{\ast} 
\overset{f_{\ast}}{\lra} [S^{1}, Y ]_{\ast} 
\overset{j_{\ast}\adj}{\lra} [S^{0}, \hofib f ]_{\ast} 
\overset{i(f)_{\ast}}{\lra} [S^{0}, X ]_{\ast} 
\overset{f_{\ast}}{\lra} [S^{0}, Y ]_{\ast} 
\end{multline*}

Since $[S^{n}, Y]_{\ast} = \pi_{n}(Y)$ we recover the long 
exact sequence from Corollary \ref{HOMOTOPY FIBRATION EX SEQ COR}.  








