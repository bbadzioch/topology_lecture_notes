% !TEX root = mth727_lecture_notes.tex


\chapter[Weak Equivalences and Homology]{Weak Equivalences \\ and Homology}
\chaptermark{Weak Equivalences and Homology}
\label{WEAK EQUIVALENCES AND HOMOLOGY CHAPTER}
\thispagestyle{firststyle}

The main goal of this chapter is to show that the following holds:

\begin{theorem}
\label{WE HOMOLOGY ISO THM}
If $f\colon X\to Y$ is a weak equivalence then the induced homomorphisms 
$f_{\ast}\colon H_{i}(X) \to H_{i}(Y)$ and $f^{\ast}\colon H^{i}(X) \to H^{i}(Y)$
are isomorphisms for all $i\geq 0$. 
\end{theorem}


\begin{nn}{\bf Brief review of homological algebra.}
\begin{itemize}
\item A \emph{chain complex} $C_{\ast}$ consists of a sequence of abelian groups 
and group homomorphisms
\[
\dots \lra 
C_{n+1} \overset{\partial_{n+1}}{\lra}
C_{n} \overset{\partial_{n}}{\lra}
C_{n-1} \lra
\dots
\]
such that $\partial_{n}\partial_{n+1} = 0$ for all $n$. 
The homomorphisms $\partial_{n}$ are called \emph{differentials}.
\\[-3mm]

\item The \emph{$n$-th homology group} of a chain complex $C_{\ast}$ is the 
group $H_{n}(C_{\ast}) = \ker \partial_{n} /\Im \partial_{n+1}$. \\[-3mm]

\item A \emph{chain map} $f_{\ast}\colon C_{\ast}\to D_{\ast}$ is a sequence of 
homomophisms $f_{n}\colon C_{n}\to D_{n}$ such that 
$\partial_{n}f_{n} = f_{n-1}\partial_{n}$. \\[-3mm]

\item A chain map $f_{\ast}\colon C_{\ast}\to D_{\ast}$ induces homomorphisms 
of homology groups $f_{\ast}\colon H_{n}(C_{\ast})\to H_{n}(D_{\ast})$. \\[-2mm]

\item If $f_{\ast}, g_{\ast}\colon C_{\ast}\to D_{\ast}$ are chain maps, 
then a \emph{chain homotopy} $s_{\ast}\colon C_{\ast}\to D_{\ast}$ 
from $f_{\ast}$ to $g_{\ast}$ is a sequence of homomorphisms 
$s_{n}\colon C_{n}\to D_{n+1}$ such that 
$f_{n}-g_{n} = \partial_{n+1}s_{n} +  s_{n-1}\partial_{n}$ \\[-3mm]

\item If there exists a chain homotopy between chain maps 
$f_{\ast}, g_{\ast}\colon C_{\ast} \to D_{\ast}$ then $f_{\ast}$ and $g_{\ast}$
induce the same homorphism between homology groups 
$H_{\ast}(C_{\ast}) \to H_{\ast}(D_{\ast})$. 

\end{itemize}
\end{nn}

\begin{nn}{\bf Brief review of singular homology.}
\begin{itemize}
\item A \emph{singular chain complex} of a topological space $X$ is a chain complex
$C_{\ast}(X)$ such that $C_{n}(X)$ is the free abelian group generated by all singular 
simplices $\sigma\colon \Delta^{n}\to X$. \\[-3mm]

\item  Differentials in $C_{\ast}(X)$ are defined using 
face maps $d^{i}_{n}\colon \Delta^{n-1}\to \Delta^{n}$ for $i=0, \dots, n$:
$\partial_{n}\sigma = \sum_{i=0}^{n} (-1)^{i}\sigma d^{i}_{n}$. \\[-3mm]


\item \emph{Singular homology groups} of a space $X$ are homology groups of $C_{\ast}(X)$:
$H_{n}(X) := H_{n}(C_{\ast}(X))$.  \\[-3mm]

\item Any map of spaces $f\colon X \to Y$ defines a chain map of singular chain complexes 
$f_{\ast}\colon C_{\ast}(X) \to C_{\ast}(Y)$ given by $f_{\ast}(\sigma) = f\sigma$
for a singular simplex $\sigma\colon \Delta^{n}\to X$. This induces a homomorphism of 
homology groups $f_{\ast}\colon H_{\ast}(X) \to H_{\ast}(Y)$. \\[-3mm]

\item For a space $X$ let $i_{0}, i_{1}\colon X \to X\times [0, 1]$ denote the inclusions 
$i_{0}(x) = (x, 0)$ and $i_{1}(x) = (x, 1)$. There exists a chain homotopy 
$s^{X}_{\ast}\colon C_{\ast}(X) \to C_{\ast +1}(X\times [0, 1])$ from $i_{0\ast}$
to $i_{1\ast}$.  \\[-3mm]

\item The chain homotopy $s^{X}_{\ast}$ can be used to show that if $f, g\colon X \to Y$
are homotopic maps then they induce that same homomorphism of homology groups 
$H_{\ast}(X) \to H_{\ast}(Y)$.

\end{itemize}
\end{nn}


\begin{proof}[Proof of Theorem \ref{WE HOMOLOGY ISO THM}]
Assume first that $X \subseteq Y$ and that $f\colon X\hra Y$ is the inclusion map.
Notice that in this case $C_{\ast}(X)$ is a subcomplex of $C_{\ast}(Y)$ and 
the chain map $f_{\ast}\colon C_{\ast}(X) \to C_{\ast}(Y)$ is an inclusion. 

We will associate to each singular simplex $\sigma\colon \Delta^{n}\to Y$ a 
homotopy $h^{\sigma}\colon \Delta^{n} \times [0, 1] \to Y$ such that:
\begin{itemize}
\item[1)] $h^{\sigma}_{0} = \sigma$ and $h^{\sigma}_{1}(\Delta^{n})\subseteq X$.
\item[2)] If $\sigma(\Delta^{n})\subseteq X$ then $h^{\sigma}_{t} = \sigma$ for 
all $t\in [0, 1]$.
\item[3)] $h^{\sigma d^{i}_{n}} = h^{\sigma}(d^{i}_{n}\times \id_{[0, 1]})$
\end{itemize}
The homotopies $h^{\sigma}$ will be constructed by induction with respect to $n$.
For $n=0$ giving a simplex $\sigma\colon \Delta^{0} = \{\ast\} \to X$ 
is the same a giving a point $\sigma(\ast) = y\in Y$. 
Since $f$ is a weak equivalence, there exist a path 
$h^{\sigma}\colon \Delta^{0}\times [0, 1]\to Y$ such that $h^{\sigma}(\ast, 0) = y$
and $h^{\sigma}(\ast, 1)\in X$. If $y\in X$ take $h^{\sigma}$ to be the constant path. 

Assume that we have already constructed homotopies $h^{\tau}$ satisfying 1)-3) for 
all $\tau\colon \Delta^{k}\to Y$ with $k < n$, and let $\sigma\colon \Delta^{n}\to Y$. 
If $\sigma(\Delta^{n})\subseteq X$, define $h^{\sigma}$ using condition 2). 
Otherwise, let 
$\partial \Delta^{n} := \bigcup_{i=0}^{n}d^{i}_{n}(\Delta^{n-1})\subseteq \Delta^{n}$. 
Since homotopies $h^{\sigma d^{i}_{n}}$ are already defined, condition 3) determines
a map $h\colon \Delta^{n}\times \{0\} \cup \partial\Delta^{n}\times [0, 1]\to Y$ such that 
$h_{0} = \sigma$ and $h_{1}(\partial\Delta^{n})\subseteq X$.  
The pair $(\Delta^{n}, \partial \Delta^{n})$ is a relative CW complex, so by 
Proposition \ref{HELP PROP} we can extend $h$ to a homotopy 
$h^{\sigma}\colon \Delta^{n}\times [0, 1] \to Y$ such that 
$h^{\sigma}_{0} = \sigma$ and $h^{\sigma}_{1}(\Delta^{n}) \subseteq X$. 

Define a map $\varphi_{\ast}\colon C_{\ast}(Y) \to C_{\ast}(X)$ by 
$\varphi(\sigma) = h_{1}^{\sigma}$. Condition 3) implies that $\varphi_{\ast}$ is 
a chain map. Also, by condition 2) we obtain $\varphi_{\ast}f_{\ast} = \id_{C_{\ast}(X)}$.
Finally, a chain homotopy $\Phi_{\ast}\colon C_{\ast}(Y)\to C_{\ast}(Y)$ 
between $f_{\ast}\varphi_{\ast}$ and $\id_{C_{\ast}(Y)}$ can be obtained as follows.
Given a singular simplex $\sigma\colon \Delta^{n}\to Y$ the homotopy 
$h^{\sigma}$ induces a chain map 
$h^{\sigma}_{\ast}\colon C_{\ast}(\Delta^{n}\times [0, 1]) \to C_{\ast}(Y)$. 
We also have a chain homotopy 
$s^{\Delta^{n}}_{\ast}\colon C_{\ast}(\Delta^{n}) \to C_{\ast}(\Delta^{n}\times [0, 1])$. 
The identity map $\id_{\Delta^{n}}\colon \Delta^{n}\to \Delta^{n}$ is a singular 
simplex in $C_{n}(\Delta^{n})$. We set 
$\varphi(\sigma) = h^{\sigma}_{\ast}s^{\Delta^{n}}_{\ast}(\id_{\Delta^{n}})$.


For a general weak equivalence $f\colon X\to Y$ consider the commutative diagram 
\begin{equation*}
\begin{tikzpicture}
\matrix (m) 
[matrix of math nodes, row sep= 2em, column sep=1.5em, text height=1.5ex, text depth=0.25ex]
{
X & & Y \\
& M_{f} & \\ 
};
\path[->, thick, font=\scriptsize]
(m-1-1) 
edge node[above] {$f$} (m-1-3)
(m-2-2)
edge node[anchor= north west] {$r$} (m-1-3)
;
\path[right hook-latex, thick, font=\scriptsize]
(m-1-1) 
edge node[anchor=north east] {$i$} (m-2-2)
; 
\end{tikzpicture}
\end{equation*}
where $M_{f}$ is the mapping cylinder of $f$, $i$ is the inclusion map
and $r$ is the retraction. Since $f$ is a weak equivalence and $r$ is a homotopy 
equivalence, thus $i$ is a weak equivalence. Therefore, by the argument above, 
$i$ induces an isomorphism on homology groups 
$i_{\ast}\colon H_{\ast}(X)\overset{\cong}{\to} H_{\ast}(M_{f})$. 
Also, since every homotopy equivalence induces an isomorphism on homology, thus
we get an isomorphism $r_{\ast}\colon H_{\ast}(M_{f})\overset{\cong}{\to} H_{\ast}(Y)$. 
Therefore $f_{\ast} = r_{\ast}i_{\ast}\colon H_{\ast}(X)\to H_{\ast}(Y)$ is an 
isomorphism. 
  
  
The statement that a weak equivalence induces an isomorphism of cohomology groups 
follows from the Universal Coefficients Theorem, which implies that if a map 
$f\colon X\to Y$ gives an isomorphism on homology, then it also induces an
isomorphism on cohomology. 
\end{proof}

Using the same arguments as in the proof of Theorem \ref{WE HOMOLOGY ISO THM}, this 
theorem can be generalized as follows:

\begin{theorem}
\label{N-EQUIVALENCE HOMOLOGY THM}
% Compare tom Dieck 9.5.2 p. 236.
If $f\colon X \to Y$ is an $n$-equivalence for some $n\geq 1$ then then 
$f_{\ast}\colon H_{i}(X) \to H_{i}(Y)$ is an isomorphism for all $i < n$
and it is an epimorphism for $i=n$.
\end{theorem}
