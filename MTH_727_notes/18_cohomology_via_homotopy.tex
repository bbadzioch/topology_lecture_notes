% !TEX root = mth727_lecture_notes.tex


\chapter[Cohomology via Homotopy]{Cohomology \\ via Homotopy}
\chaptermark{Cohomology via Homotopy}
\label{HOMOLOGY VIA HOMOTOPY CHAPTER}
\thispagestyle{firststyle}


Recall (\ref{K(G, N) DEF}) that for an abelian group $G$ by $K(G, n)$ 
we denote the Eilenberg-MacLane space such that $\pi_{n}(K(G,n)) \cong G$. 
We will also denote by $K(G, 0)$ the discrete space consisting of elements 
of the group $G$. Notice that for every $n$ we have a weak equivalence
\[
K(G, n) 
\overset{\simeq}{\lra} \Omega K(G, n+1) 
\overset{\simeq}{\lra} \Omega^{2} K(G, n+2)
\]
For any pointed CW complex $X$ this induces a bijection of sets of pointed 
homotopy classes 
\[
[X, K(G, n)]_{\ast} \overset{\cong}{\lra} [X, \Omega^{2} K(G, n+2)]_{\ast}
\]
Since $[X, \Omega^{2} K(G, n+2)]_{\ast}$ has a natural structure of an 
abelian group (\ref{LOOP SPACE TO GROUP STRUCTURE NOTE}), we obtain in 
this way an abelian group structure on $[X, K(G, n)]_{\ast}$.

The main goal of this chapter is to show that the following holds:

\begin{theorem}
\label{COHOMOLOGY REPRESENTABILITY THM}
Let $G$ be an abelian group. 

1) For any pointed CW complex $X$ and $n \geq 0$ there exists 
an isomorphism 
\[
T_{X}\colon [X, K(G, n)]_{\ast} \overset{\cong}{\lra} \widetilde{H}^{n}(X; G)
\]
where $\widetilde{H}^{n}(X; G)$ is the $n$-th reduced singular cohomology group of 
$X$ with coefficients in $G$. 

2) These isomorphisms are natural. That is, if $f\colon X \to Y$ is a map 
of pointed CW complexes then the following diagram commutes: 
\begin{equation*}
\begin{tikzpicture}
\matrix (m) 
[matrix of math nodes, row sep=3em, column sep=3em, text height=1.5ex, text depth=0.25ex]
{
[X, K(G, n)]_{\ast} &  {[Y, K(G, n)]_{\ast}} \\
\widetilde{H}^{n}(X; G) & \widetilde{H}^{n}(Y; G) \\
};
\path[->, thick, font=\scriptsize]
(m-1-2) 
edge node[above] {$f^{\ast}$} (m-1-1)
edge node[anchor = west] {$T_{Y}$} node[anchor=east] {$\cong$} (m-2-2)
(m-2-2)
edge node[below] {$f^{\ast}$} (m-2-1)
(m-1-1)
edge node[anchor=  east] {$T_{X}$} node[anchor=west] {$\cong$} (m-2-1)
; 
\end{tikzpicture}
\end{equation*}
\end{theorem}


\begin{note}
Let $\varphi\colon X \to K(G, n)$ be a pointed map. By part 2) of 
Theorem \ref{COHOMOLOGY REPRESENTABILITY THM} we obtain a commutative diagram
\begin{equation*}
\begin{tikzpicture}
\matrix (m) 
[matrix of math nodes, row sep=3em, column sep=3em, text height=1.5ex, text depth=0.25ex]
{
[X, K(G, n)]_{\ast} &  {[K(G, n), K(G, n)]_{\ast}} \\
\widetilde{H}^{n}(X; G) & \widetilde{H}^{n}(K(G, n); G) \\
};
\path[->, thick, font=\scriptsize]
(m-1-2) 
edge node[above] {$\varphi^{\ast}$} (m-1-1)
edge node[anchor = west] {$T_{K(G, n)}$} node[anchor=east] {$\cong$} (m-2-2)
(m-2-2)
edge node[below] {$\varphi^{\ast}$} (m-2-1)
(m-1-1)
edge node[anchor=  east] {$T_{X}$} node[anchor=west] {$\cong$} (m-2-1)
; 
\end{tikzpicture}
\end{equation*}
This gives: 
\[
T_{X}([\varphi]) 
= T_{X}(\varphi^{\ast}([\id_{K(G, n)}])) 
= \varphi^{\ast}T_{K(G, n)}([\id_{K(G, n)}])
\]
This implies that for any pointed CW complex $X$ the bijection $T_{X}$
is determined by the cohomology class 
$\alpha_{n} = T_{K(G, n)}([\id_{K(G, n)}])\in \widetilde{H}^{n}(K(g, n); G)$.
This class is called the \emph{fundamental class}.
\end{note}

\begin{note}
An unpointed version of Theorem \ref{COHOMOLOGY REPRESENTABILITY THM} also 
holds: for any CW complex $X$ there exists a natural isomorphism 
$T_{X}\colon [X, K(G, n)] \overset{\cong}{\lra} H^{n}(X; G)$. This can be derived
from Theorem \ref{COHOMOLOGY REPRESENTABILITY THM} as follows. For a CW complex 
$X$ let $X_{+}$ denote the space obtained by adding one $0$-cell to $X$:
$X_{+} = X \sqcup \{ + \}$. We consider $+$ as the basepoint of $X_{+}$. 
We have bijections $[X, K(G, n)] \cong [X_{+}, K(G, n)]_{\ast}$ and 
$H^{n}(X; G) \cong \widetilde{H}^{n}(X_{+}; G)$. Thus if 
$[X_{+}, K(G, n)]_{\ast} \cong \widetilde{H}^{n}(X_{+}; G)$ then 
$[X, K(G, n)] \cong H^{n}(X; G)$.
\end{note}

\begin{example}
For any CW complex X we have:
\bit
\item $[X, S^{1}] \cong H^{1}(X, \Z)$
\item $[X, \CP^{\infty}] \cong H^{2}(X, \Z)$
\item $[X, \RP^{\infty}] \cong H^{2}(X, \Z/2)$
\eit
\end{example}



The proof of Theorem \ref{COHOMOLOGY REPRESENTABILITY THM} will proceed as follows. 
First, we will define a general notion of a cohomology theory, which 
consists of a sequence of functors $\{ h^{n} \}_{n\in \Z}$ from the category 
of pointed CW complexes to the category of abelian groups satisfying certain axioms.
We will show that both assignments $X \mapsto \widetilde{H}^{n}(X; G)$ and 
$X \mapsto [X, K(G, n)]_{\ast}$ satisfy this definition. Then
we will prove that if $\{h^{n}\}$ is any generalized cohomology 
theory such that $h^{n}(S^{0})\cong \widetilde{H}^{n}(S^{0} ; G)$ for all $n$, 
then for every pointed CW complex $X$ and every $n$ there is a natural isomorphism 
$h^{n}(X) \to H^{n}(X; G)$. Since the cohomology theory defined by Eilenberg-MacLane
space satisfies this property, Theorem \ref{COHOMOLOGY REPRESENTABILITY THM} will follow.   


\begin{definition}
Let $\CW_{\ast}$ denote the category of pointed CW complexes and basepoint 
preserving maps. A \emph{(reduced) cohomology theory} consists of: 
\benu
\item[\textbullet] A sequence contravariant functors 
$\{h^{n}\colon \CW_{\ast} \to \Ab\}_{n\in \Z}$. 
\item[\textbullet] For every $X\in \CW_{\ast}$ and every $n\in \Z$ a natural isomorphism 
$\Sigma \colon h^{n}(X) \to h^{n+1}(\Sigma X)$. Naturality means that for any map 
$f\colon X \to Y$ we have a commutative diagram 
\begin{equation*}
\begin{tikzpicture}
\matrix (m) 
[matrix of math nodes, row sep=3em, column sep=3em, text height=1.5ex, text depth=0.25ex]
{
h^{n}(Y) & h^{n+1}(\Sigma Y) \\
h^{n}(X) & h^{n+1}(\Sigma X) \\
};
\path[->, thick, font=\scriptsize]
(m-1-1) 
edge node[anchor= south] {$\Sigma$}  node[anchor= north] {$\cong$} (m-1-2)
edge node[anchor = east] {$f^{\ast}$} (m-2-1)
(m-1-2)
edge node[anchor=  west] {$\Sigma f ^{\ast}$}  (m-2-2)
(m-2-1)
edge node[anchor=  north] {$\Sigma$} node[anchor= south] {$\cong$} (m-2-2)
; 
\end{tikzpicture}
\end{equation*}
\eenu
Moreover, the following axioms are satisfied:
\benu
\item[\textbullet] ({\bf Homotopy axiom}) 
If $f, g\colon X\to Y$ are maps such that $f\simeq g$ then 
$f^{\ast} = g^{\ast}\colon h^{n}(Y) \to h^{n}(X)$ for all $n$.

\item[\textbullet] ({\bf Exactness axiom}) 
For any pair $(X, A)$ where $A\subseteq X$ is a subcomplex, 
$i\colon A \hra X$ is the inclusion and $q\colon X \to X/A$ is the quotient map, 
the following sequence is exact:
\[
h^{n}(A) \overset{i^{\ast}}{\lla} h^{n}(X)  \overset{q^{\ast}}{\lla} h^{n}(X/A)  
\]

\item[\textbullet] ({\bf Wedge axiom})  
For any family of pointed CW complexes $\{X_{i} \}_{i\in I}$
the inclusion maps $X_{j}\hra \bigvee_{i\in I} X_{i}$ induce isomorphisms 
$h^{n}\left(\bigvee_{i\in I} X_{i} \right) 
\overset{\cong}{\lra} \prod_{i\in I} h^{n}(X_{i})$ for all $n$. 
\eenu
\end{definition}


\begin{nn}{\bf Some consequences of the axioms.}
\bit
\item $h^{n}(\ast) = 0$ for all $n$.
\item For any pair $(X, A)$ where $A\subseteq X$ is a subcomplex, there is a long exact 
sequence 
\[
\dots \lla 
h^{n}(A) \overset{i^{\ast}}{\lla} h^{n}(X)  \overset{q^{\ast}}{\lla} h^{n}(X/A)  
\overset{\delta}{\lla} h^{n-1}(A)
\lla \dots
\]
The map $\delta \colon  h^{n-1}(A) \to h^{n}(X/A)$ is the composition of the suspension 
isomorphism $\Sigma \colon h^{n-1}(A) \to h^{n}(\Sigma A)$, the homomorphism induced by 
the quotient map $C_{i} \to C_{i}/X \cong \Sigma A$, where $C_{i}$ is the cone of the inclusion
$i\colon A \hra X$, and the  isomorphism induced by the homotopy equivalence 
$X/A \overset{\simeq}{\to} C_{i}$. 
\eit
\end{nn}

\begin{example}
Given an abelian group $G$, consider the reduced singular cohomology functors 
$X \mapsto \widetilde{H}^{n}(X; G)$. For $n < 0$ set $\widetilde{H}^{n}(X; G) = 0$ 
for all $X$. Then the functors $\{\widetilde{H}^{n}(-; G)\}$ define 
a cohomology theory.
\end{example}


\begin{example}
\label{K(G, N) COHOMOLOGY EXAMPLE}
For an abelian group $G$, let $h^{n}_{G}(X) = [X, K(G, n)]_{\ast}$. For $n <0$
we set $K(G, n) = \ast$. Then the functors $\{ h^{n}_{G}\}$ form 
a cohomology theory. To define the suspension isomorphism
\[
\Sigma \colon h^{n}_{G}(X) = [X, K(G, n)]_{\ast} \lra 
[\Sigma X, K(G, n+1)]_{\ast} = h^{n+1}_{G}(\Sigma X)
\]
choose a week equivalence $\varphi_{n}\colon K(G, n) \to \Omega K(G, n+1)$. 
This induces an isomorphism  
$\varphi_{n}^{\ast} \colon [X, K(G, n)]_{\ast} \to [X, \Omega K(G+1, n)]_{\ast}$. 
Then we compose it with the adjunction isomorphism 
$[X, \Omega K(G+1, n)]_{\ast} \overset{\cong}{\to} [\Sigma X, K(G+1, n)]_{\ast}$

It is obvious that $\{h_{G}^{n}\}$ satisfies the homotopy axiom.
The exactness axiom is also satisfied by
Proposition \ref{HOMOTOPY CLASSES COEXACT SEQ PROP}. The wedge axiom holds 
since for any family of well-pointed spaces $\{X_{i}\}_{i\in I}$ 
and any pointed space $Z$, inclusion maps induce a bijection 
$[\bigvee_{i\in I} X_{i}, Z]_{\ast} \to \prod_{i\in I} [X_{i}, Z]_{\ast}$.
\end{example}

Notice that the only property of the spaces $K(G, n)$ used in 
Example \ref{K(G, N) COHOMOLOGY EXAMPLE}  is that for each $n$ there exist a weak 
homotopy equivalence $\varphi_{n}\colon K(G, n) \to \Omega K(G, n+1)$. 
This motivates the following definition. 

\begin{definition}
An \emph{$\Omega$-spectrum} $(K_{n}, \varphi_{n})_{n\in\Z}$ is a sequence of pointed spaces 
$K_{n}$ and weak homotopy equivalences 
$\varphi_{n}\colon K_{n} \overset{\simeq}{\to} \Omega K_{n+1}$.  
\end{definition}

By the same argument as in Example \ref{K(G, N) COHOMOLOGY EXAMPLE} we obtain:

\begin{proposition}
Every $\Omega$-spectrum $(K_{n}, \varphi_{n})_{n\in\Z}$ defines a cohomology 
theory $\{h^{n}\}_{n\in \Z}$ given by $h^{n}(X) = [X, K_{n}]_{\ast}$.
\end{proposition}


\begin{definition}
\label{DIMENSION AXIOM DEF}
A cohomology theory $\{h^{n}\}$ satisfies the \emph{dimension axiom} if 
$h^{n}(S^{0}) = 0$ for $n\neq 0$.
\end{definition}

\begin{theorem}
\label{SINGULAR COHOMOLOGY UNIQUE THM}
Let $\{h_{1}^{n}\}_{n\in \Z}$ and $\{h_{2}^{n}\}_{n\in \Z}$ be cohomology theories 
that satisfy the dimension axiom and such that $h^{0}_{1}(S^{0}) \cong h^{0}_{2}(S^{0})$.
Then for each pointed CW complex there exists natural isomorphism 
$T_{X}\colon h_{1}^{n}(X) \overset{\cong}{\to} h_{2}^{n}(X)$.
Naturality means that each pointed map $f\colon X \to Y$ gives a commutative diagram
\begin{equation*}
\begin{tikzpicture}
\matrix (m) 
[matrix of math nodes, row sep=3em, column sep=3em, text height=1.5ex, text depth=0.25ex]
{
h_{1}^{\ast}(Y) & h_{1}^{\ast}(X) \\
h_{2}^{\ast}(Y) & h_{2}^{\ast}(X) \\
};
\path[->, thick, font=\scriptsize]
(m-1-1) 
edge node[anchor= south] {$f^{\ast}$}   (m-1-2)
edge node[anchor = east] {$T_{Y}$} node[anchor= west] {$\cong$} (m-2-1)
(m-1-2)
edge node[anchor=  west] {$T_{X}$} node[anchor= east] {$\cong$}  (m-2-2)
(m-2-1)
edge node[anchor=  north] {$f^{\ast}$}  (m-2-2)
; 
\end{tikzpicture}
\end{equation*}
\end{theorem}



\begin{proof}[Proof of Theorem \ref{COHOMOLOGY REPRESENTABILITY THM}]
For the reduced singular cohomology theory we have
\[
\widetilde{H}^{n}(S^{0}; G) \ \cong \ 
\begin{cases}
G & \text{if $n=0$} \\
0 & \text{otherwise} \\
\end{cases}
\]
Also, 
\[
[S^{0}, K(G, n)]_{\ast} \ \cong \ \pi_{0}(K(G, n)) \ \cong \ 
\begin{cases}
G & \text{if $n=0$} \\
0 & \text{otherwise} \\
\end{cases}
\]
Therefore we can apply Theorem \ref{SINGULAR COHOMOLOGY UNIQUE THM}.   
\end{proof}

The proof of Theorem \ref{SINGULAR COHOMOLOGY UNIQUE THM} will require some 
preparation. 

\begin{lemma} 
\label{COHOMOLOGY DIM AXIOM LEMMA}
Let $\{h^{n}\}$ be a cohomology theory satisfying the dimension 
axiom. Then: 
\benu
\item[1)] $h^{q}(\bigvee_{i\in I} S^{n}) = 0$ for $q \neq n$. 

\item[2)] For any CW complex $X$ the inclusion of the $n$-th skeleton
$j\colon X^{(n)} \hra X$ induces an isomorphism 
$j^{\ast}\colon h^{q}(X) \overset{\cong}{\lra} h^{q}(X^{(n)})$ for all $q< n$
Also, $h^{q}(X^{(n)}) = 0$ for $q>n$. 
\eenu
\end{lemma}

\begin{proof}
1) This follows from the isomorphisms 
\[
h^{q}(\bigvee_{i\in I }S^{n}) \cong 
\prod_{i\in I} h^{q}(S^{n}) \cong 
\prod_{i\in I} h^{q}(\Sigma^{n} S^{0}) \cong 
\prod_{i\in I} h^{q - n}(S^{0})
\]

2) For finite-dimensional CW complexes this can be proved by induction on skeleta 
of $X$, using cofibration sequences 
$X^{(k-1)} \hra X^{(k)} \to \bigvee S^{k}$. This can be generalized to 
the case $\dim X = \infty$ using the infinite telescope construction 
(see e.g. Hatcher, \emph{Algebraic Topology} pp. 138-139), which gives a 
cofibration sequence $\bigvee_{k} X^{(k)} \to X \to \bigvee_{k} \Sigma X^{(k)}$. 
\end{proof}


For abelian groups $G, H$ let $\Hom(G, H)$ denote the set of homomorphisms 
$G\to H$. This set has a group structure with addition defined by 
$(\varphi + \psi)(g) = \varphi(g) + \psi(g)$ for $\varphi, \psi\in \Hom(G, H)$.

\begin{proposition}
\label{COHOMOLOGY MAP ADDITIVITY PROP}
Let $\{h^{n}\}$ be a cohomology theory. For a pointed CW complex $X$ and $n\geq 1$ 
consider the map 
\[
\Phi\colon \pi_{n}(X) \to \Hom(h^{q}(X), h^{q}(S^{n}))
\]
that sends an element $[\varphi\colon S^{n} \to X]\in \pi_{n}(X)$ to the induced 
homomorphism $\varphi^{\ast}\colon h^{q}(X) \to h^{q}(S^{n})$. Then the map $\Phi$ 
is a homomorphism of groups.
\end{proposition}

\begin{proof}
The constant map $S^{n}\to X$ induces the trivial homomorphism 
$h^{q}(X) \to h^{q}(S^{n})$, so $\Phi$ preserves trivial elements.
Let $[\varphi], [\psi]\in \pi_{n}(X)$. The element 
$[\varphi]\cdot[\psi]\in \pi_{n}(X)$ is the homotopy class of the map 
\[
S^{n} \overset{p}{\lra} S^{n}\vee S^{n} \overset{\varphi\vee\psi}{\lra} S^{n}
\]  
where $p$ is the pinch map. We need to show that 
$p^{\ast}(\varphi\vee\psi)^{\ast} = \varphi^{\ast} + \psi^{\ast}\colon
h^{q}(X)\to h^{q}(S^{n})$. This follows from commutativity of the following diagram: 
\begin{equation*}
\begin{tikzpicture}
\matrix (m) 
[matrix of math nodes, row sep=3em, column sep=3em, text height=1.5ex, text depth=0.25ex]
{
h^{q}(X) & h^{q}(S^{n}\vee S^{n}) & h^{q}(S^{n}) \\
& h^{q}(S^{n}) \times h^{q}(S^{n}) & \\
};
\path[->, thick, font=\scriptsize]
(m-1-1) 
edge node[anchor= south] {$(\varphi\vee\psi)^{\ast}$} (m-1-2)
edge [bend right = 20] node[anchor= north east] {$\varphi^{\ast}\times \psi^{\ast}$} (m-2-2) 
(m-1-2)
edge node[anchor= west] {$\cong$} (m-2-2)
edge  node[anchor= south] {$p^{\ast}$} (m-1-3)
(m-2-2)
edge [bend right = 20] node[anchor= north west] {$\mu$} (m-1-3)
; 
\end{tikzpicture}
\end{equation*}
Here the isomorphism $h^{q}(S^{n}\vee S^{n})\to h^{q}(S^{n})\times h^{q}(S^{n})$
is induced by the inclusion maps and $\mu$ is given by $\mu(x, y) = x+y$. 
\end{proof}

\begin{corollary}
\label{COHOMOLOGY SN NATURAL COR}
Let $\{h^{n}_{1}\}$, $\{h^{n}_{2}\}$ be cohomology theories and let 
$T\colon h^{q}_{1}(S^{n}) \to h^{q}_{2}(S^{n})$ be an arbitrary homomorphism. 
Then for any map $f\colon S^{n}\to S^{n}$ the following diagram commutes:
\begin{equation*}
\label{COHOMOLOGY SN NATURALITY EQ}
\tag{$\ast$}
\begin{tikzpicture}[baseline=(current  bounding  box.center)]
\matrix (m) 
[matrix of math nodes, row sep=3em, column sep=3em, text height=1.5ex, text depth=0.25ex]
{
h_{1}^{q}(S^{n}) & h_{1}^{q}(S^{n}) \\
h_{2}^{q}(S^{n}) & h_{2}^{q}(S^{n}) \\
};
\path[->, thick, font=\scriptsize]
(m-1-1) 
edge node[anchor= south] {$f^{\ast}$}   (m-1-2)
edge node[anchor = east] {$T$}  (m-2-1)
(m-1-2)
edge node[anchor=  west] {$T$}  (m-2-2)
(m-2-1)
edge node[anchor=  north] {$f^{\ast}$}  (m-2-2)
; 
\end{tikzpicture}
\end{equation*}
\end{corollary}

\begin{proof}
Using Proposition \ref{COHOMOLOGY MAP ADDITIVITY PROP} 
we obtain that homotopy classes of maps $f$ for which the diagram 
(\ref{COHOMOLOGY SN NATURALITY EQ}) commutes form a subgroup of $\pi_{n}(S^{n})$. 
Since the homotopy class of the identity map $\id_{S^{n}}\colon S^{n}\to S^{n}$ 
belongs to this subgroup, the subgroup contains all elements of $\pi_{n}(S^{n})$.
\end{proof}




Let $\{h^{n}\}$ be a cohomology theory and let $X$ be a CW complex. 
For  $n\geq 0$ consider the map 
\[
\varphi_{n} \colon X^{(n+1)}/X^{(n)} \to \Sigma X^{(n)} \to \Sigma (X^{(n)}/X^{(n-1)})
\]
Let $d^{n}\colon h^{n}(X^{(n)}/X^{(n-1)}) \to h^{n+1}(X^{(n+1)}/X^{(n)})$
be a homomorphism given by the composition
\[
d^{n}\colon h^{n}(X^{(n)}/X^{(n-1)}) \overset{\Sigma}{\lra} 
h^{n+1}(\Sigma (X^{(n)}/X^{(n-1)})) \overset{\varphi_{n}^{\ast}}{\lra}
h^{n+1}(X^{(n+1)}/X^{(n)})
\]

\begin{proposition}
\label{CELLULAR CHAIN FOR GEN COHOMOLOGY PROP}
Let $\{h^{n}\}$ be a cohomology theory. For a CW complex $X$
consider the maps
\[
h^{n-1}(X^{(n-1)}/X^{(n-2)})\overset{d^{n-1}}{\lra}
h^{n}(X^{(n)}/X^{(n-1)}) \overset{d^{n}}{\lra}
h^{n+1}(X^{(n+1)}/X^{(n)})
\] 
Then $\Im(d^{n-1}) \subseteq \Ker(d^{n})$. Moreover, of $\{h^{n}\}$ satisfies
the dimension axiom then $h^{n}(X) \cong h^{n}(X^{(n+1)}) \cong \Ker(d^{n})/ \Im(d^{n-1})$.

\end{proposition}

\begin{proof}
Exercise. Use Lemma \ref{COHOMOLOGY DIM AXIOM LEMMA} and long exact sequences 
for the pairs $(X^{(n+1)}, X^{(n)})$, $(X^{(n)}, X^{(n-1)})$ and $(X^{(n-1)}, X^{(n-2)})$.
\end{proof}



\begin{proof}[Proof of Theorem \ref{SINGULAR COHOMOLOGY UNIQUE THM}]
We will construct natural isomorphisms $T_{X}\colon h_{1}^{\ast}(X) \to h_{2}^{\ast}(X)$
in a few steps. 

1) We define $T_{S^{n}} \colon h_{1}^{\ast}(S^{n}) \to h_{2}^{\ast}(S^{n})$
by induction with respect to $n$. By assumption we have isomorphisms 
$T_{S^{0}}\colon h_{1}^{\ast}(S^{0}) \to h_{2}^{\ast}(S^{0})$.
Assume that $T_{S^{n}}$ is already defined for some $n$. Choose a homeomorphism 
$f_{n+1}\colon S^{n+1} \to \Sigma S^{n}$ and define $T_{S^{n+1}}$ so that the 
following diagram commutes:
\begin{equation*}
\begin{tikzpicture}
\matrix (m) 
[matrix of math nodes, row sep=3em, column sep=3em, text height=1.5ex, text depth=0.25ex]
{
h_{1}^{\ast}(S^{n}) & h_{1}^{\ast}(S^{n+1}) \\
h_{2}^{\ast}(S^{n}) & h_{2}^{\ast}(S^{n+1}) \\
};
\path[->, thick, font=\scriptsize]
(m-1-1) 
edge node[anchor= south] {$f_{n}^{\ast}\Sigma$}  node[anchor=north] {$\cong$} (m-1-2)
edge node[anchor = east] {$T_{S^{n}}$} node[anchor= west] {$\cong$} (m-2-1)
(m-1-2)
edge node[anchor=  west] {$T_{S^{n+1}}$}   (m-2-2)
(m-2-1)
edge node[anchor=  north]{$f_{n}^{\ast}\Sigma$}  node[anchor=south] {$\cong$}  (m-2-2)
; 
\end{tikzpicture}
\end{equation*} 

2) By the definition of a cohomology theory, for any set $J$ inclusion maps 
$S^{n} \to \bigvee_{j\in J} S^{n}$ induce isomorphisms 
$h^{\ast}_{i}(\bigvee_{j\in J}S^{n}) \overset{\cong}{\lra} \prod_{j\in J} h_{i}^{\ast}(S^{n})$
Choose isomorphisms $T_{\bigvee_{j\in J}S^{n}}$ so that the following diagram commutes: 
\begin{equation*}
\begin{tikzpicture}
\matrix (m) 
[matrix of math nodes, row sep=3em, column sep=3em, text height=1.5ex, text depth=0.25ex]
{
h^{\ast}_{1}(\bigvee_{j\in J}S^{n}) & \prod_{j\in J} h_{1}^{\ast}(S^{n}) \\
h^{\ast}_{2}(\bigvee_{j\in J}S^{n}) & \prod_{j\in J} h_{2}^{\ast}(S^{n}) \\
};
\path[->, thick, font=\scriptsize]
(m-1-1) 
edge  node[anchor=north] {$\cong$} (m-1-2)
edge node[anchor = east] {$T_{\bigvee_{j\in J}S^{n}}$} node[anchor= west] {$\cong$} (m-2-1)
(m-1-2)
edge node[anchor=  west] {$\prod_{j\in J} T_{S^{n}}$} node[anchor= east] {$\cong$}  (m-2-2)
(m-2-1)
edge  node[anchor=south] {$\cong$}  (m-2-2)
; 
\end{tikzpicture}
\end{equation*} 

We claim that isomorphisms $T_{\bigvee_{j\in J}S^{n}}$ defined above are natural with 
respect to all maps $f\colon \bigvee_{j\in J}S^{n} \to \bigvee_{k\in K}S^{n}$. That is, for any 
such map the following diagram commutes:
\begin{equation*}
\begin{tikzpicture}
\matrix (m) 
[matrix of math nodes, row sep=3em, column sep=3em, text height=1.5ex, text depth=0.25ex]
{
h^{\ast}_{1}(\bigvee_{k\in K}S^{n}) & h^{\ast}_{1}(\bigvee_{j\in J}S^{n}) \\
h^{\ast}_{2}(\bigvee_{k\in K}S^{n}) & h^{\ast}_{2}(\bigvee_{j\in J}S^{n}) \\
};
\path[->, thick, font=\scriptsize]
(m-1-1) 
edge  node[anchor=south] {$f^{\ast}$} (m-1-2)
edge node[anchor = east] {$T_{\bigvee_{k\in K}S^{n}}$} node[anchor= west] {$\cong$} (m-2-1)
(m-1-2)
edge node[anchor=  west] {$T_{\bigvee_{j\in J}S^{n}}$} node[anchor= east] {$\cong$}  (m-2-2)
(m-2-1)
edge  node[anchor=north]  {$f^{\ast}$}  (m-2-2)
; 
\end{tikzpicture}
\end{equation*} 
Using the isomorphisms 
$h^{\ast}_{i}(\bigvee_{j\in J}S^{n}) \cong \prod_{j\in J} h_{i}^{\ast}(S^{n})$
and compactness of spheres, this can be reduced (exercise) to checking that for any 
pointed map $f\colon S^{n} \to S^{n}$ we have $f^{\ast}T_{S^{n}} = T_{S^{n}}f_{\ast}$. 
This,  however, follows from Corollary \ref{COHOMOLOGY SN NATURAL COR}. 

3) There are now two possible ways of obtaining an isomorphism 
$h^{\ast}_{1}(\Sigma \bigvee_{i\in I} S^{n}) \to h^{\ast}_{2}(\Sigma \bigvee_{i\in I} S^{n})$.
One is to use the suspension isomorphisms 
$\Sigma \colon h_{k}^{\ast}(\bigvee_{i\in I} S^{n}) 
\to  h_{k}^{\ast +1}(\Sigma \bigvee_{i\in I} S^{n})$ and the already defined 
isomorphism $T_{\bigvee_{i\in I} S^{n}}$. Another is to use the 
homeomorphism 
\[
\bigvee_{i\in I} S^{n+1} \overset{\bigvee f_{n+1}}{\lra}
\bigvee_{i\in I} \Sigma S^{n} \overset{\bigvee \Sigma j_{i}}{\lra} 
\Sigma \bigvee_{i\in I} S^{n}
\]
and the isomorphism $T_{\bigvee_{i\in I} S^{n+1}}$. Here 
$f_{n+1}\colon S^{n+1}\to \Sigma S^{n}$ is a homeomorphism and 
$\Sigma j_{i}$ is the suspension of the inclusion map 
$j_{i}\colon S^{n} \to \bigvee_{i\in I} S^{n}$. One can check that both these methods 
give the same isomorphism 
$T_{\Sigma \bigvee S^{n}} \colon 
h^{\ast}_{1}(\Sigma \bigvee_{i\in I} S^{n}) \to h^{\ast}_{2}(\Sigma \bigvee_{i\in I} S^{n})$.
Using naturality of isomorphisms $T_{\bigvee S^{n+1}}$ established in 2), we obtain that 
for any map $f\colon \bigvee_{j\in J} S^{n+1} \to \Sigma \bigvee_{k\in K} S^{n}$ we have 
a commutative diagram 
\begin{equation*}
\begin{tikzpicture}
\matrix (m) 
[matrix of math nodes, row sep=3em, column sep=3em, text height=1.5ex, text depth=0.25ex]
{
h^{\ast-1}_{1}(\bigvee_{k\in K}S^{n}) &
h^{\ast}_{1}(\Sigma \bigvee_{k\in K}S^{n}) & h^{\ast}_{1}(\bigvee_{j\in J}S^{n+1}) \\
h^{\ast-1}_{2}(\bigvee_{k\in K}S^{n}) &
h^{\ast}_{2}(\Sigma \bigvee_{k\in K}S^{n}) & h^{\ast}_{2}(\bigvee_{j\in J}S^{n+1}) \\
};
\path[->, thick, font=\scriptsize]
(m-1-1) 
edge node[anchor = east] {$T_{\bigvee_{k\in K}S^{n}}$}  
node[anchor= west] {$\cong$} (m-2-1)
edge node[anchor = south] {$\Sigma$}
node[anchor= north] {$\cong$} (m-1-2)
(m-2-1) 
edge node[anchor = north] {$\Sigma$}
node[anchor= south] {$\cong$} (m-2-2)
(m-1-2) 
edge  node[anchor=south] {$f^{\ast}$} (m-1-3)
edge node[anchor = east] {$T_{\Sigma \bigvee_{k\in K}S^{n}}$} 
node[anchor= west] {$\cong$} (m-2-2)
(m-1-3)
edge node[anchor=  west] {$T_{\bigvee_{j\in J}S^{n+1}}$} 
node[anchor= east] {$\cong$}  (m-2-3)
(m-2-2)
edge  node[anchor=north]  {$f^{\ast}$}  (m-2-3)
; 
\end{tikzpicture}
\end{equation*} 

4) Let now $X$ be an arbitrary pointed CW complex. 
By Proposition \ref{CELLULAR CHAIN FOR GEN COHOMOLOGY PROP} for $k=1, 2$
we have isomorphisms $h^{n}_{k}(X) \cong \Ker(d_{k}^{n})/ \Im(d_{k}^{n-1})$ where 
$d_{k}^{n}\colon h^{n}_{k}(X^{(n)}/X^{(n-1)}) \to h^{n+1}_{k}(X^{(n+1)}/X^{(n)})$. 
If we could find isomorphisms 
$T_{X, n} \colon h^{n}_{1}(X^{(n)}/X^{(n-1)}) \to h^{n}_{2}(X^{(n)}/X^{(n-1)})$ such that 
$d_{2}^{n}T_{X, n} = T_{X, n+1}d_{1}^{n}$, then they would induce isomorphisms 
\[
T_{X}\colon h_{1}^{n}(X) \cong \Ker(d_{1}^{n})/ \Im(d_{1}^{n-1})
\lra
\Ker(d_{2}^{n})/ \Im(d_{2}^{n-1}) \cong  h_{2}^{n}(X)
\]
Such isomorphisms $T_{X, n}$ can be constructed as follows. For each $n$ choose a 
homeomorphism $f_{n}\colon \bigvee_{i\in I_{n}} S^{n} \to X^{(n)}/X^{(n-1)}$.
Then define $T_{X, n}$ so that the following diagram commutes: 
\begin{equation*}
\begin{tikzpicture}
\matrix (m) 
[matrix of math nodes, row sep=3em, column sep=3em, text height=1.5ex, text depth=0.25ex]
{
h^{n}_{1}(X^{(n)}/X^{(n-1)}) & h^{n}_{1}(\bigvee_{i\in I_{n}}S^{n})  \\
h^{n}_{2}(X^{(n)}/X^{(n-1)}) & h^{n}_{2}(\bigvee_{i\in I_{n}}S^{n}) \\
};
\path[->, thick, font=\scriptsize]
(m-1-1) 
edge  node[anchor=south] {$f_{n}^{\ast}$} node[anchor= north] {$\cong$} (m-1-2)
edge node[anchor = east] {$T_{X, n}$}  (m-2-1)
(m-1-2)
edge node[anchor=  west] {$T_{\bigvee_{i\in I_{n}}S^{n}}$} 
node[anchor= east] {$\cong$}  (m-2-2)
(m-2-1)
edge  node[anchor=north]  {$f_{n}^{\ast}$}  node[anchor= south] {$\cong$} (m-2-2)
; 
\end{tikzpicture}
\end{equation*}
Commutativity of isomorphisms $T_{X, n}$ with the maps $d^{n}_{k}$ follows from 
commutativity of the following diagram: 


\begin{equation*}
\begin{tikzpicture}
\matrix (m) 
[matrix of math nodes, row sep=3em, column sep=3.5em, text height=1.5ex, text depth=0.25ex]
{
h^{n}_{1}(X^{(n)}/X^{(n-1)}) & 
h^{n+1}_{1}(\Sigma (X^{(n)}/X^{(n-1)})) & 
h^{n+1}_{1}(X^{(n+1)}/X^{(n)}) \\
h^{n}_{1}(\bigvee_{i\in I_{n}} S^{n}) & 
h^{n+1}_{1}(\Sigma \bigvee_{i\in I_{n}} S^{n}) & 
h^{n+1}_{1}(\bigvee_{i\in I_{n+1}} S^{n}) \\
h^{n}_{2}(\bigvee_{i\in I_{n}} S^{n}) & 
h^{n+1}_{2}(\Sigma \bigvee_{i\in I_{n}} S^{n}) & 
h^{n+1}_{2}(\bigvee_{i\in I_{n+1}} S^{n}) \\
h^{n}_{2}(X^{(n)}/X^{(n-1)}) & 
h^{n+1}_{2}(\Sigma (X^{(n)}/X^{(n-1)})) & 
h^{n+1}_{2}(X^{(n+1)}/X^{(n)}) \\
};
\path[->, thick, font=\scriptsize]
(m-1-1.south west)
edge [bend right = 20] node[anchor=east] {$T_{X, n}$} (m-4-1.north west)
(m-1-3.south east)
edge [bend left = 20] node[anchor=west] {$T_{X, n+1}$} (m-4-3.north east)
(m-1-1) 
edge [bend left = 20] node[anchor=south] {$d^{n}_{1}$} (m-1-3)
edge  node[anchor=south] {$\Sigma$} node[anchor= north] {$\cong$} (m-1-2)
edge node[anchor = east] {$f_{n}^{\ast}$}  node[anchor= west] {$\cong$} (m-2-1)
(m-1-2)
edge node[anchor=  west] {$(\Sigma f_{n})^{\ast}$} node[anchor= east] {$\cong$}  (m-2-2)
edge  node[anchor=south]  {$\varphi_{n}^{\ast}$} (m-1-3)
(m-1-3)
edge node[anchor = west] {$f_{n+1}^{\ast}$}  node[anchor= east] {$\cong$} (m-2-3)
(m-2-1)
edge  node[anchor=south]  {$\Sigma$}  node[anchor= north] {$\cong$} (m-2-2)
edge node[anchor = east] {$T_{\bigvee_{i\in I^{n}} S^{n}}$}  node[anchor= west] {$\cong$} (m-3-1)
(m-2-2)
edge  node[anchor=south]  {$((\Sigma f_{n})^{-1}\varphi_{n}f_{n+1})^{\ast}$}   (m-2-3)
edge node[anchor = east] {$T_{\Sigma \bigvee_{i\in I^{n}} S^{n}}$}  node[anchor= west] {$\cong$} (m-3-2)
(m-2-3)
edge node[anchor = west] {$T_{\bigvee_{i\in I^{n+1}} S^{n+1}}$}  node[anchor= east] {$\cong$} (m-3-3)
(m-3-1)
edge  node[anchor=south] {$\Sigma$} node[anchor= north] {$\cong$} (m-3-2)
(m-3-2)
edge node[anchor=south]  {$((\Sigma f_{n})^{-1}\varphi_{n}f_{n+1})^{\ast}$} (m-3-3) 
(m-4-1)
edge [bend right = 20] node[anchor=north] {$d^{n}_{2}$} (m-4-3)
edge node[anchor = east] {$f_{n}^{\ast}$}  node[anchor= west] {$\cong$} (m-3-1)
edge  node[anchor=south] {$\Sigma$} node[anchor= north] {$\cong$} (m-4-2)
(m-4-2)
edge  node[anchor=south]  {$\varphi_{n}^{\ast}$} (m-4-3)
edge node[anchor=  west] {$(\Sigma f_{n})^{\ast}$} node[anchor= east] {$\cong$}  (m-3-2)
(m-4-3)
edge node[anchor = west] {$f_{n+1}^{\ast}$}  node[anchor= east] {$\cong$} (m-3-3)
;
\end{tikzpicture}
\end{equation*}
The maps $\varphi_{n}$ are defined as in 
Proposition \ref{CELLULAR CHAIN FOR GEN COHOMOLOGY PROP}. The middle squares commute 
by 3). 

To check that the isomorphisms $T_{X}$ are natural with respect to maps $f\colon X \to Y$, 
notice that we can assume that $f$ is cellular and so it induces homomorphisms 
$f^{\ast}\colon h^{n}_{k}(Y^{(n)}/Y^{(n-1)}) \to h^{n}_{k}(X^{(n)}/X^{(n-1)})$ 
which commute with the maps $d^{n}_{k}$. Then it remains check that 
$f^{\ast}T_{Y, n} = T_{X, n}f^{\ast}$. This can be verified using naturality of 
the isomorphisms $T_{\bigvee S^{n}}$ established in 2).






\end{proof}