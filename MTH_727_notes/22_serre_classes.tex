% !TEX root = mth727_lecture_notes.tex


\chapter[Serre classes]{Serre classes}
\chaptermark{Serre classes}
\label{SERRE CLASSES CHAPTER}
\thispagestyle{firststyle}


The motivation for this chapter is to show that the following holds.

\begin{theorem}
\label{HOMOTOPY GPS SPHERES THM}
The homotopy groups $\pi_{n}(S^{m})$ are finitely generated for all $n, m \geq 1$.
\end{theorem}

This will follow from a more general result that will be stated in terms 
of Serre classes.

\begin{definition}
A \emph{Serre class} is a non-empty collection $\ccal$ of abelian groups satisfying 
the property that if 
\[
0 \to A \to B \to C \to 0
\]
is a short exact sequence of abelian groups then $B\in \ccal$ if and only if 
$A, C \in \ccal$

We will say that a Serre class $\ccal$ is a \emph{Serre ring} if in addition 
it satisfies that if $A, B \in \ccal$ then $A\otimes B \in \ccal$ and 
$\Tor(A, B)\in \ccal$. 

We will also say that a Serre class is \emph{acyclic} if for every group $A\in \ccal$
we have $H_{q}(K(A, 1)) \in \ccal$ for all $q >0$. 
\end{definition}

\begin{proposition}
Let $\ccal$ is a Serre class. The following hold:
\benu
\item $0 \in \ccal$.
\item If $A \in \ccal$ and $A’\cong A$ then $A’ \in \ccal$.
\item If $B\subseteq A$ then $A\in \ccal$ if and only if $B, A/B \in \ccal$.
\item If $A \to B \to C$ is an exact sequence and $A, C \in \ccal$ then $B\in \ccal$.
\item If $0 = A_{-1} \subseteq A_{1} \subseteq A_{2} \subseteq {\dots} \subseteq A_{n}$
then $A_{n}\in \ccal$ if and only if $A_{i}/A_{i-1} \in \ccal$ for all $i$. 
\eenu
\end{proposition}

\begin{proof}
Exercise.
\end{proof}

\begin{proposition}
\label{SERRE RING HOMOLOGY PROP}
Let $\ccal$ is a Serre ring. If $X$ is a path connected space such that 
$H_{q}(X) \in \ccal$ for all $0 < q < p$ then $H_{p}(X; G) \in \ccal$ for any group 
$G\in \ccal$. 
\end{proposition}

\begin{proof}
By the Universal Coefficient Theorem we have 
\[ 
H_{p}(X; G) \cong (H_{p}(X)\otimes G) \oplus \Tor(H_{p-1}(X), G)
\]
This immediately gives that $H_{p}(X; G)\in \ccal$ for $p\geq 2$
For $p=0$ we gave $H_{0}(X; G) \cong G \in \ccal$ while for 
$p=1$ we obtain $H_{1}(X; G) \cong H_{1}(X)\otimes G \in \ccal$.
\end{proof}




\begin{proposition} All of the following are acyclic Serre rings:
\ 
\begin{itemize}
\item $\cfin = $ the class of all finite abelian groups.
\item $\cfg = $ the class of all finitely generated abelian groups.
\item $\ctor = $ the class of all torsion abelian groups.
\item $\ccal_{p} = $ the class of all p-torsion abelian groups for 
a given prime $p$.
\end{itemize}
\end{proposition}


\begin{theorem}
\label{2 OF 3 SERRE CLASS HOMOLOGY THM}
Let $F \to E \overset{p}{\to} B$ be a Serre fibration with a simply connected space 
$B$, and let $\ccal$ be a Serre ring. If for two of the spaces $F$, $E$, $B$
the homology groups $H_{q}(-)$ are in $\ccal$ for all $q>0$ then the same holds 
for the third space.
\end{theorem}

\begin{proof}\ 
There are three cases to consider. 

\emph{Case 1:} $H_{q}(F), H_{q}(B) \in \ccal$ for all $q > 0$.

Consider the Serre spectral sequence of the fibration $p$. 
We have $E^{2}_{p, q} = H_{p}(B, H_{q}(F))$ so by 
Proposition \ref{SERRE RING HOMOLOGY PROP} we get that $E^{2}_{p, q} \in \ccal$
for all $(p, q) \neq (0, 0)$. Next, since groups $E^{3}_{p, q}$ are obtained 
by taking quotients of subgroups of the groups $E^{2}_{p, q}$, we get that 
$E^{3}_{p, q}\in \ccal$ for all $(p, q) \neq (0, 0)$. Inductively, we obtain 
that $E^{r}_{p, q}\in \ccal$ for all $r \geq 2$ and $(p, q)\neq (0, 0)$, 
and so also $E^{\infty}_{p, q}\in \ccal$ for $(p, q)\neq (0, 0)$. 
For $q > 0$ the groups $H_{q}(E)$ admit a finite filtration such filtration
quotients are isomorphic to groups $E^{\infty}_{p, q}$ with $(p, q)\neq 0$. 
This implies that $H_{q}(E)\in \ccal$.

\emph{Case 2:} $H_{q}(F), H_{q}(E) \in \ccal$ for all $q > 0$.

Since all groups $E^{\infty}_{p, q}$ are quotients of subgroups of $H_{p+q}(E)$, 
we have $E^{\infty}_{p, q}\in \ccal$ for all $(p, q) \neq (0, 0)$. 
We will show that $E^2_{p, q}\in \ccal$ for $(p, q)\neq (0, 0)$ by induction with 
respect to $p$. For $p=0$ this holds since $E^{2}_{0, q}\cong H_{q}(F)$. 
Assume that it also holds for $E^{2}_{i, q}$ for all $i < p$. It follows that 
$E^{r}_{i, q}\in \ccal$ for all $i < p$ and all $r\geq 2$. 

Since all differentials 
terminating at $E^{r}_{p, 0}$ are trivial, for each $r$ we have an exact 
sequence 
\[
E^{r+1}_{p, 0} \to E^{r}_{p, 0} \overset{d^{r}}{\to} E^{r}_{p-r, r-1}
\]
By assumption $E^{r}_{p-r, r-1}\in \ccal$, so if $E^{r+1}_{p, 0} \in \ccal$
then the same is true for $E^{r}_{p, 0}$. Since 
$E^{p+1}_{p, q} = E^{\infty}_{p, q} \in \ccal$, arguing inductively over 
decreasing values of $r$ we obtain that $E^{r}_{p, 0} \in\ccal$ for all $r\geq 2$. 
In particular, $H_{p}(B) = E^{2}_{p, 0}\in \ccal$. Using Proposition 
\ref{SERRE RING HOMOLOGY PROP} we obtain that $E^2_{p, q} = H_{p}(B, H_{q}(F))\in \ccal$
for all $q\geq 0$. 

\emph{Case 3:} $H_{q}(B), H_{q}(E) \in \ccal$ for all $q > 0$.

This is similar to case 2.
\end{proof}


\begin{proposition}
\label{HOMOLOGY K(GN) ACYCLIC SERRE CLASS PROP}
If $\ccal$ is an acyclic Serre ring then for every $A\in \ccal$ and $n\geq 1$
we have $H_{q}(K(A, n)) \in \ccal$. 
\end{proposition}

\begin{proof}
We argue by induction with respect to $n$. 
The case $n=1$ holds by definition of acyclicity of a Serre class. Assume 
that the statement is true for some $n\geq 1$. For $A\in \ccal$ consider 
the homotopy fibration sequence $K(A, n) = \Omega K(A, n+1) \to \ast \to K(A, n+1)$. 
Since $H_{q}(K(A, n)), H_{q}(\ast) \in \ccal$ for all $q>0$, by   
Theorem \ref{2 OF 3 SERRE CLASS HOMOLOGY THM} we obtain that $H_{q}(K(A, n+1))\in \ccal$. 
\end{proof}



\begin{theorem}
\label{HOMOTOPY VS HOMOLOGY SERRE CLASS THM}
Let $\ccal$ be an acyclic Serre ring. If $X$ is a simply connected space then the following 
conditions are equivalent:
\benu
\item $\pi_{n}(X)\in \ccal$ for all $n\geq 1$ 
\item $H_{n}(X)\in \ccal$ for all $n\geq 1$
\eenu  
\end{theorem}


The proof of Theorem \ref{HOMOTOPY VS HOMOLOGY SERRE CLASS THM} will make 
use of the notion of Postnikov sections:

\begin{definition}
\label{POSTNIKOV SECTION DEF}
Let $X$ be a path connected space. The $n$-th Postnikov section of $X$ 
is a space $X_{n}$ together with a map $f\colon X \to X_{n}$ such that
\benu
\item[1)] $f_{\ast}\colon \pi_{q}(X) \to \pi_{q}(X_{n})$ is an isomorphism 
for $q \leq n$
\item[2)] $\pi_{q}(X) = 0$ for $q > n$. 
\eenu
\end{definition} 

The $n$-th Postnikov section of a space $X$ can be constructed glueing to 
$X$ cells in dimensions $n+1$ and higher to kill all homotopy groups above 
$\pi_{n}(X)$. The map $f\colon X\to X_{n}$ is then given by the inclusion.


\begin{proof}[Proof of Theorem \ref{HOMOTOPY VS HOMOLOGY SERRE CLASS THM}]\ 

1) $\Ra$ 2) Let $X_{n}$ denote the $n$-th  Postnikov section of $X$. 
By Theorem \ref{N-EQUIVALENCE HOMOLOGY THM} we have $H_{q}(X)\cong H_{q}(X_{n})$
for all $q< n$, so it will be enough to show that $H_{q}(X_{n})\in \ccal$ for all
$n, q > 0$. We will prove this by induction with respect to $n$.  For $n=2$ we have 
$X_{2} = K(\pi_{2}(X), 2)$, so the statement holds by 
Proposition \ref{HOMOLOGY K(GN) ACYCLIC SERRE CLASS PROP}. Assume that it also holds 
for some $n\geq 2$. Notice that we have a fibration sequence 
\[
K(\pi_{n+1}(X), n+1) \to X_{n+1} \to X_{n}
\]
Using Proposition \ref{HOMOLOGY K(GN) ACYCLIC SERRE CLASS PROP} again we get 
that $H_{q}(K(\pi_{n+1}(X), n+1) ) \in \ccal$ for $q>0$, so using 
Theorem \ref{2 OF 3 SERRE CLASS HOMOLOGY THM} we obtain that $H_{q}(X_{n+1}) \in \ccal$
for $q>0$. 

2) $\Ra$ 1) We will show that $\pi_{n}(X)\in \ccal$ by induction with respect to $n$. 
Since $X$ is simply connected, for $n=2$ by the Hurewicz Isomorphism Theorem we
get  $\pi_{2}(X) \cong H_{2}(X) \in \ccal$. Next, assume that $\pi_{q}(X)\in \ccal$
for all $q\leq n$ and consider the fibration sequence 
\[
\hofib f \to X \to X_{n}
\]
where $X_{n}$ is the $n$-th Postnikov section of $X$.  Notice that
\[
\pi_{q}(\hofib f) = 
\begin{cases}
0 & { if } q\leq n \\
\pi_{q}(X) & { if } q > n \\
\end{cases}
\]
Since $\pi_{q}(X_{n})\in \ccal$ for all $q$, thus by part 1) $\Ra$ 2) we get that 
$H_{q}(X_{n})\in \ccal$ for all $q>0$. By assumption $H_{q}(X)\in \ccal$
for $q>0$. Therefore, using Theorem \ref{2 OF 3 SERRE CLASS HOMOLOGY THM} we obtain 
that $H_{q}(\hofib f)\in \ccal$ for $q>0$. Since  $\hofib f$ is $n$-connected, 
by the Hurewicz Isomorphism Theorem we get 
$H_{n+1}(\hofib f)\cong \pi_{n+1}(\hofib f) \cong \pi_{n+1}(X)$. This gives 
$\pi_{n+1}(X)\in \ccal$.



\end{proof}




