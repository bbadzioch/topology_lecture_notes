%%%%%%%%%%%%%%%%%%%%%%%%%%%%%%%%%%%%%%%%%%%%%
%  
%   (C)  Bernard Badzioch 
%.  This work is licensed under CC BY-NC-SA 4.0. To view a copy of this license, visit 
%   https://creativecommons.org/licenses/by-nc-sa/4.0
%
%%%%%%%%%%%%%%%%%%%%%%%%%%%%%%%%%%%%%%%%%%%%%

% !TEX root = mth727_lecture_notes.tex

\documentclass[11pt, letterpaper, oneside]{report}
\usepackage{amsmath,amssymb,amsthm,eucal, graphicx, units, stackrel}
\usepackage{framed}
\usepackage{accents}
\usepackage{color}
\usepackage{pifont}
\usepackage[headings]{fullpage}
\usepackage{tocloft}
\usepackage{wrapfig}
\usepackage{microtype}
\usepackage{tikz}
\usetikzlibrary{calc,through,intersections, arrows, shapes, decorations.markings, matrix, patterns, cd}
\usetikzlibrary{decorations.pathmorphing}
\usetikzlibrary{arrows.meta}
\usetikzlibrary{bending}
\usepackage{pgfplots}
\pgfplotsset{compat=1.12}
\tikzset{>=latex}
\usetikzlibrary{external} 
\tikzexternalize[prefix=tikz/] 



%%%%%%%%%%%%%%%%%%
%   RUBICS
%%%%%%%%%%%%%%%%%%

\usepackage{etoolbox}
\newtoggle{ShowRubrics}
\newcommand{\Rubrics}[1]{ \iftoggle{ShowRubrics} { {\footnotesize\color{red}{#1}   } } {}     }
\settoggle{ShowRubrics}{false}
%\settoggle{ShowRubrics}{true}


%%%%%%%%%%%%%%%%%%%%%%%%%%%%
%   COLORS
%%%%%%%%%%%%%%%%%%%%%%%%%%%%

\definecolor{mypink}{RGB}{255, 178, 164}
\definecolor{myblue}{RGB}{0, 0, 238}

\definecolor{mygray1}{RGB}{238, 238, 238}
\definecolor{mygray2}{RGB}{221, 221, 221}
\definecolor{mygray3}{RGB}{187, 187, 187}
\definecolor{mygray4}{RGB}{170, 170, 170}
\definecolor{mygray5}{RGB}{119, 119, 119}
\definecolor{mygray6}{RGB}{85, 85, 85}
\definecolor{mygray7}{RGB}{68, 68, 68}






%%%%%%%%%%%%%%%%%%
%  GRID WITH COORDS
%  NEEDED FOR TIKZ GRID
%%%%%%%%%%%%%%%%%%
\makeatletter
\def\grd@save@target#1{%
  \def\grd@target{#1}}
\def\grd@save@start#1{%
  \def\grd@start{#1}}
\tikzset{
  grid with coordinates/.style={
    to path={%
      \pgfextra{%
        \edef\grd@@target{(\tikztotarget)}%
        \tikz@scan@one@point\grd@save@target\grd@@target\relax
        \edef\grd@@start{(\tikztostart)}%
        \tikz@scan@one@point\grd@save@start\grd@@start\relax
        \draw[minor help lines] (\tikztostart) grid (\tikztotarget);
        \draw[major help lines] (\tikztostart) grid (\tikztotarget);
        \grd@start
        \pgfmathsetmacro{\grd@xa}{\the\pgf@x/1cm}
        \pgfmathsetmacro{\grd@ya}{\the\pgf@y/1cm}
        \grd@target
        \pgfmathsetmacro{\grd@xb}{\the\pgf@x/1cm}
        \pgfmathsetmacro{\grd@yb}{\the\pgf@y/1cm}
        \pgfmathsetmacro{\grd@xc}{\grd@xa + \pgfkeysvalueof{/tikz/grid with coordinates/major step}}
        \pgfmathsetmacro{\grd@yc}{\grd@ya + \pgfkeysvalueof{/tikz/grid with coordinates/major step}}
        \foreach \x in {\grd@xa,\grd@xc,...,\grd@xb}
        \node[anchor=north] at (\x,\grd@ya) {\pgfmathprintnumber{\x}};
        \foreach \y in {\grd@ya,\grd@yc,...,\grd@yb}
        \node[anchor=east] at (\grd@xa,\y) {\pgfmathprintnumber{\y}};
      }
    }
  },
  minor help lines/.style={
    help lines,
    step=\pgfkeysvalueof{/tikz/grid with coordinates/minor step}
  },
  major help lines/.style={
    help lines,
    line width=\pgfkeysvalueof{/tikz/grid with coordinates/major line width},
    step=\pgfkeysvalueof{/tikz/grid with coordinates/major step}
  },
  grid with coordinates/.cd,
  minor step/.initial=.2,
  major step/.initial=1,
  major line width/.initial=1pt,
}
\makeatother
%%%% END GRID WITH COORDS




%%%%%%%%%%%%%%%%
% CHAPTER HEADINGS
%%%%%%%%%%%%%%%%
\usepackage[T1]{fontenc}
\usepackage{titlesec}
\usepackage{cyklop}
\definecolor{gray75}{gray}{0.5}
\newcommand{\hsp}{\hspace{20pt}}
\titleformat{\chapter}[hang]{\fontfamily{cyklop}\Huge\bfseries}{\thechapter
\hsp\textcolor{red}{\raisebox{3pt}|}\hsp}{0pt}{\Huge\bfseries}[\vskip 10mm]


\usepackage{enumitem}
\setlist{itemsep= 0pt, parsep= 2pt, topsep=-3pt, partopsep= 0pt}
\renewcommand{\labelenumi}{\theenumi )} % redefines enumerate labels to 1), 2) 3) etc.


\definecolor{linkcol}{rgb}{0, 0, 0.7}
\usepackage[ linkcolor=gray75, citecolor=linkcol, urlcolor= gray75, colorlinks=true, hypertexnames=false]
{hyperref}



%%%%%%%%%%%%%%%%
% HEADERS
%%%%%%%%%%%%%%%%
\setlength{\headheight}{15pt} 
\newcommand{\cyk}{\fontfamily{cyklop}\fontsize{9}{11}\selectfont}
\usepackage{fancyhdr}
\pagestyle{fancy}
\fancyhf{} % clear all header and footer fields
\fancyhead[L]{
\cyk
\thechapter. \leftmark}
\fancyhead[R]{
\cyk
\thepage}
\renewcommand{\headrulewidth}{0.8pt}
\renewcommand{\footrulewidth}{0pt}
\renewcommand{\chaptermark}[1]{\markboth{#1}{}}

\fancypagestyle{firststyle}
{
\fancyhf{}
\fancyfoot[C]{
\cyk \thepage
}
\renewcommand{\headrulewidth}{0pt}
}



%%%%%%%%%%%%%%%%
% EXERCISES HEADING
%%%%%%%%%%%%%%%%
\newcommand{\exercises}{
\vskip 10mm
\begin{tikzpicture}[scale=1]
\draw node[inner sep = 4pt, right,text width=\textwidth -8pt, fill= black!25]
{
\fontfamily{cyklop}\selectfont  \raisebox{-8pt}{Exercises to Chapter \arabic{chapter}}
 };
\end{tikzpicture}
}



%%%%%%%%%%%%%%%%
% FONTS
%%%%%%%%%%%%%%%%
\usepackage{pifont}
\usepackage[math]{iwona}


%%%%%%%%%%%%%%%%%%%%%%%%%%%%%%
% THIS FIXES THE MISSING NORM FONT WITH IWONA
%%%%%%%%%%%%%%%%%%%%%%%%%%%%%%
\usepackage{mathtools}
\usepackage{xparse}
\DeclarePairedDelimiter\xnorm{\lVert}{\rVert}
\NewDocumentCommand{\norm}{som}
 {\IfBooleanTF{#1}
   {\xnorm*{#3}}
   {\IfNoValueTF{#2}
     {\mathopen{|\mkern-.8mu|}#3\mathclose{|\mkern-.8mu|}}
     {\xnorm[#2]{#3}}%
   }
 }



%%%%%%%%%%%%%%%%
% IMPROVED \overline
%%%%%%%%%%%%%%%%
\makeatletter
\newsavebox\myboxA
\newsavebox\myboxB
\newlength\mylenA

\newcommand*\xov[2][0.75]{%
    \sbox{\myboxA}{$\m@th#2$}%
    \setbox\myboxB\null% Phantom box
    \ht\myboxB=\ht\myboxA%
    \dp\myboxB=\dp\myboxA%
    \wd\myboxB=#1\wd\myboxA% Scale phantom
    \sbox\myboxB{$\m@th\overline{\copy\myboxB}$}%  Overlined phantom
    \setlength\mylenA{\the\wd\myboxA}%   calc width diff
    \addtolength\mylenA{-\the\wd\myboxB}%
    \ifdim\wd\myboxB<\wd\myboxA%
       \rlap{\hskip 0.5\mylenA\usebox\myboxB}{\usebox\myboxA}%
    \else
        \hskip -0.5\mylenA\rlap{\usebox\myboxA}{\hskip 0.5\mylenA\usebox\myboxB}%
    \fi}
\makeatother
%%%%%%%%%%%%%%%%
% END IMPROVED \overline
%%%%%%%%%%%%%%%%


%%% PARSKIP AND PARINDENT 
\setlength{\parindent}{0in}
\setlength{\parskip}{3mm}

\setcounter{secnumdepth}{1}
\setcounter{tocdepth}{2}



%%%%%%%%%%%%%%%%%%%%%%%%%%%%
%   ENVIRONMENTS
%%%%%%%%%%%%%%%%%%%%%%%%%%%%

%\swapnumbers

\newtheoremstyle{pplain}% name of the style to be used
  {5mm}% measure of space to leave above the theorem. E.g.: 3pt
  {5mm}% measure of space to leave below the theorem. E.g.: 3pt
  {\itshape}% name of font to use in the body of the theorem
  {0pt}% measure of space to indent
  {\bfseries}% name of head font
  {.}% punctuation between head and body
  {5pt plus 1pt minus 1pt}% space after theorem head; " " = normal interword space
  {#2 #1}% Manually specify head
 %{\llap{ {\scriptsize  #2 \ } \  {\color{gray75}\ding{121}} \  }#1}% Manually specify head


\newtheoremstyle{ddefinition}% name of the style to be used
 {5mm}% measure of space to leave above the theorem. E.g.: 3pt
 {5mm}% measure of space to leave below the theorem. E.g.: 3pt
 {}% name of font to use in the body of the theorem
 {0pt}% measure of space to indent
 {\bfseries}% name of head font
 {.}% punctuation between head and body
 {5pt plus 1pt minus 1pt}% space after theorem head; " " = normal interword space
  {#2 #1}% Manually specify head
  %{\llap{ {\scriptsize  #2 \ } \  {\color{gray75}\ding{121}} \  }#1}% Manually specify head


\newtheoremstyle{eexercise}% name of the style to be used
 {3mm}% measure of space to leave above the theorem. E.g.: 3pt
 {0mm}% measure of space to leave below the theorem. E.g.: 3pt
 {}% name of font to use in the body of the theorem
 {0pt}% measure of space to indent
 {\bfseries}% name of head font
 {.}% punctuation between head and body
 {5pt plus 1pt minus 1pt}% space after theorem head; " " = normal interword space
 {E#2 #1}% Manually specify head
 %{\llap{ {\scriptsize  E.#2 \ } \  {\color{gray75}\ding{121}} \  }#1}% Manually specify head
 
 \newtheoremstyle{nnn}% name of the style to be used
 {5mm}% measure of space to leave above the theorem. E.g.: 3pt
 {5mm}% measure of space to leave below the theorem. E.g.: 3pt
 {}% name of font to use in the body of the theorem
 {0pt}% measure of space to indent
 {\bfseries}% name of head font
 {}% punctuation between head and body
 {0pt}% space after theorem head; " " = normal interword space
  {#2 #1}% Manually specify head
  %{\llap{ {\scriptsize  #2 \ } \  {\color{gray75}\ding{121}} \  }#1}% Manually specify head
 
 


\theoremstyle{pplain}
\newtheorem{theorem}{Theorem}[chapter]
\newtheorem{lemma}[theorem]{Lemma}
\newtheorem{proposition}[theorem]{Proposition}
\newtheorem{corollary}[theorem]{Corollary}


\newtheorem{CELLAPPROXTHM}[theorem]{Cellular Approximation Theorem}
\newtheorem{INDUCTIVEHOMOTLEMMA}[theorem]{Inductive Homotopy Lemma}
\newtheorem{ECKMAN-HILTON THM}[theorem]{Eckmann–Hilton Theorem}
\newtheorem{LIFTCRITTHM}[theorem]{Theorem (Lifting Criterion)}
\newtheorem{HOMOTEXCISIONTHM}[theorem]{Excision Theorem}
\newtheorem{FREUDENTHALTHM}[theorem]{Freudenthal Suspension Theorem}
\newtheorem{HUREWICZISO THM}[theorem]{Hurewicz Isomorphism Theorem}
\newtheorem{INVHUREWICZISO THM}[theorem]{Inverse Hurewicz Isomorphism Theorem}
\newtheorem{RELINVHUREWICZISO THM}[theorem]{Relative Inverse Hurewicz Isomorphism Theorem}

\theoremstyle{ddefinition}
\newtheorem{definition}[theorem]{Definition}
\newtheorem{cordef}[theorem]{Corollary/Definition}
\newtheorem{example}[theorem]{Example}
\newtheorem{examples}[theorem]{Examples}
\newtheorem{notation}[theorem]{Notation}
\newtheorem{remark}[theorem]{Remark}
\newtheorem{note}[theorem]{Note}
\newtheorem{conjecture}[theorem]{Conjecture}
\newtheorem{problem}[theorem]{Problem}     
\newtheorem{definition/proposition}[theorem]{Definition/Proposition}  
 
 
\theoremstyle{nnn}
\newtheorem{nn}[theorem]{}    
 
 
\theoremstyle{eexercise}
\newtheorem{exercise}{Exercise}[chapter]     

%%%%%%%%%%%%%%%%%%%%%%%%%%%%
%   MACROS
%%%%%%%%%%%%%%%%%%%%%%%%%%%%




%%% ARROWS

\newcommand{\noi}{\noindent}
\newcommand{\ra}{\rightarrow}
\newcommand{\la}{\leftarrow}
\newcommand{\lra}{\longrightarrow}
\newcommand{\lla}{\longleftarrow}
\newcommand{\Ra}{\Rightarrow}
\newcommand{\La}{\Leftarrow}
\newcommand{\hra}{\hookrightarrow}



%%% RINGS AND FIELDS

\newcommand{\N}{{\mathbb N}}
\newcommand{\Z}{{\mathbb Z}}
\newcommand{\Q}{{\mathbb Q}}
\newcommand{\R}{{\mathbb R}}
\newcommand{\C}{{\mathbb C}}
\newcommand{\HH}{{\mathbb H}}
\newcommand{\F}{{\mathbb F}}


%%% MORPHISM SETS

\newcommand{\Mor}{\mathrm{Mor}}
\newcommand{\Hom}{\mathrm{Hom}}
\newcommand{\Map}{\mathrm{Map}}
\newcommand{\Aut}{\operatorname{Aut}}
\newcommand{\Inn}{\mathrm{Inn}}
\newcommand{\Out}{\mathrm{Out}}
\newcommand{\End}{\mathrm{End}}
\newcommand{\Perm}{\mathrm{Perm}}
\newcommand{\Iso}{\mathrm{Iso}}
\newcommand{\Orb}{\mathrm{Orb}}

\newcommand{\Int}{\operatorname{Int}}
\newcommand{\Bd}{\operatorname{Bd}}

\newcommand{\Ob}{\mathrm{Ob}}

\newcommand{\id}{\mathrm{id}}
\newcommand{\inv}{\mathrm{inv}}


\renewcommand{\Im}{\operatorname{Im}}
\newcommand{\Ker}{\operatorname{Ker}}
\newcommand{\Coker}{\operatorname{Coker}}
\newcommand{\Tor}{\operatorname{Tor}}
\newcommand{\subgp}[1]{\langle#1\rangle}
\newcommand{\rank}{\operatorname{rank}}
\newcommand{\sgn}{\operatorname{sgn}}
\newcommand{\irr}{\operatorname{irr}}
\newcommand{\tr}{\operatorname{tr}}
\newcommand{\nil}{\operatorname{nil}}
\newcommand{\hofib}{\operatorname{hofib}}


%%% CATEGORIES
\renewcommand{\AA}{{\mathbf A}}
\newcommand{\BB}{{\mathcal B}}
\newcommand{\CC}{{\mathbf C}}
\newcommand{\DD}{{\mathbf D}}
\newcommand{\FF}{{\mathbf F}}
\newcommand{\MM}{{\mathcal M}}
\renewcommand{\SS}{{\mathcal S}}
\newcommand{\UU}{{\mathcal U}}
\newcommand{\VV}{{\mathcal V}}
\newcommand{\TT}{{\mathcal T}}
\newcommand{\YY}{{\mathcal Y}}
\newcommand{\XX}{{\mathcal X}}
\newcommand{\ZZ}{{\mathcal Z}}
\newcommand{\Ab}{{\mathbf{Ab}}}
\newcommand{\Gr}{{\mathbf{Gr}}}
\newcommand{\Ring}{{\mathcal Ring}}
\newcommand{\Top}{{\mathbf{Top}}}
\newcommand{\Set}{{\mathbf{Set}}}
\newcommand{\TSet}{{\mathbf{TSet}}}
\newcommand{\Cov}{{\mathbf{Cov}}}
\newcommand{\PCov}{{\mathbf{PCov}}}
\newcommand{\Cat}{{\mathbf{Cat}}}
\newcommand{\CW}{{\mathbf{CW}}}

\newcommand{\Mod}{{\mathcal Mod}}
\newcommand{\ccal}{{\mathcal C}}
\newcommand{\cfin}{{\mathcal{C}_{\text{fin}}}}
\newcommand{\cfg}{{\mathcal{C}_{\text{fg}}}}
\newcommand{\ctor}{{\mathcal{C}_{\text{tor}}}}
\newcommand{\cptor}[1]{{\mathcal{C}_{#1\text{-tor}}}}

%%% SPACES
\newcommand{\pint}{{\partial I}}


%%% VARIA

\newcommand{\ssmin}{\smallsetminus}
\renewcommand{\setminus}{\ssmin}

\newcommand{\bpm}{\begin{pmatrix}}
\newcommand{\epm}{\end{pmatrix}}

\newcommand{\benu}{\begin{enumerate}}
\newcommand{\eenu}{\end{enumerate}}

\newcommand{\bit}{\begin{itemize}}
\newcommand{\eit}{\end{itemize}}


\newcommand{\ntilde}{\skew{3}\tilde}
\newcommand{\nwidetilde}{\skew{4}\widetilde}


\newcommand{\vs}{\vskip 5mm}
\newcommand{\msp}{\phantom{-}}     

\newcommand{\RP}{{\mathbb R\mathbb P}}
\newcommand{\CP}{{\mathbb C\mathbb P}}
\DeclareMathOperator\supp{supp}
\DeclareMathOperator\colim{colim}
\DeclareMathOperator\pr{pr}
\DeclareMathOperator\ev{ev}
\DeclareMathOperator\adj{adj}
\newcommand{\rel}{\mathrm{rel\ }}


%%%%%%%%%%%%%%%%%%%%%%%%%%%%
%   END MACROS
%%%%%%%%%%%%%%%%%%%%%%%%%%%%