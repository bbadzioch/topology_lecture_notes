% !TEX root = mth727_lecture_notes.tex


\chapter[Serre Spectral Sequence]{Serre Spectral \\ Sequence}
\chaptermark{Serre Spectral Sequence}
\label{SERRE SPECTRAL SEQUENCE}
\thispagestyle{firststyle}


The Serre spectral sequence is a special case of the spectral sequence
associated to a filtration described in Theorem \ref{SPACE FILTRATION SS THM}. 

\begin{definition}
Let $p\colon E \to B$ is a Serre fibration where $B$ is a connected 
CW complex. Let 
\[
\varnothing = B^{(-1)} \subseteq B^{(0)} \subseteq {\dots} B
\]
be the filtration of $B$ by skeleta. Taking $E^{p} \coloneq p^{-1}(B^{(k)})$ 
we obtain a filtration of the space $E$:
\[ 
\varnothing = E^{-1} \subseteq E^{0} \subseteq {\dots} \subseteq E
\]
The \emph{Serre spectral sequence} of the fibration $p$ is the 
spectral sequence associated to this filtration.
\end{definition}

By Theorem \ref{SPACE FILTRATION SS THM} we get that 
$E^{1}_{p, q} = H_{p+q}(E^{p}, E^{p-1})$ and that $E^{r}_{\ast\ast}$ converges to 
$H_{\ast}(E)$. The advantage of the Serre spectral sequence is that we can 
explicitly describe its second page:

\begin{theorem}
Let $E^{r}_{\ast\ast}$ be the Serre spectral sequence of a fibration $p\colon E\to B$.
Let $F = p^{-1}(b_{0})$ for some $b_{0}\in B$. 
If the space $B$ is simply connected then $E^{2}_{p, q} \cong H_{p}(B, H_{q}(F))$. 
\end{theorem}

While we will skip the proof of this result, it is useful to point out that the 
assumption that $B$ is simply connected is needed in order to obtain a 
canonical identification between fibers of $p$ taken over different points.
Assume for a moment $p$ is a Hurewicz fibration and that $b_{0}, b_{1}\in B$, 
Let $F_{i} = p^{-1}(b_{i})$ for $i=0, 1$. 
Given a path $\omega\colon [0, 1]\to B$ such that $\omega(0) = b_{0}$ 
$\omega(1) = b_{1}$, consider the diagram 
\begin{equation*}
\begin{tikzpicture}
\matrix (m) 
[matrix of math nodes, row sep=3em, column sep=3em, text height=1.5ex, text depth=0.25ex]
{
F_{0} \times \{0\} & E \\
F_{0} \times [0, 1] &  B \\
};
\path[->, thick, font=\scriptsize]
(m-1-2) 
edge node[anchor = west] {$p$} (m-2-2)
(m-1-1) 
edge node[above] {$i_{0}$} (m-1-2)
(m-2-1) 
edge node[below] {$\omega\pr$} (m-2-2)
;
\path[right hook-latex, thick, font=\scriptsize]
(m-1-1) 
edge  (m-2-1);
\path[dashed, ->,  thick, font=\scriptsize]
(m-2-1) 
edge node[anchor=south east] {$h$} (m-1-2);
\end{tikzpicture}
\end{equation*}
where $\pr\colon F_{0} \times [0, 1] \to [0, 1]$ is the projection map and 
$i_{0}\colon F_{0} \to E$ is the inclusion. A lift $h$ of $\omega\pr$ gives 
a homotopy in $E$ between the map $i_{0}$ and a certain map $h_{1}\colon F_{0}\to F_{1}$. 
One can show that this map $h_{1}$ is a homotopy equivalence 
and that its homotopy class depends only on the homotopy class of the 
path $\omega$ (relative its endpoints). If the space $B$ is simply connected, all paths 
joining $b_{0}$ and $b_{1}$ are homotopic, so the homotopy class of  
$h_{1}$ is uniquely defined. In particular, we obtain canonical isomorphisms 
of homology groups $h_{1\ast}\colon H_{q}(F_{0}) \overset{\cong}{\lra} H_{q}(F_{1})$.  
If $p$ is a Serre fibration, we can use the same argument, but in order to get the lift
$h$ we replace $F_{0}$ by its CW approximation. 

We have seen already one application of the Serre spectral sequence in Theorem 
\ref{HOMOLOGY LOOP SPHERE THM}. Here is another one:

\begin{proposition}
Let $S^{k} \to S^{m} \overset{p}{\to} S^{n}$ be a homotopy fibration sequence with $n\geq 1$.
Then $k=n-1$ and $m=2n-1$.
\end{proposition} 

\begin{proof}
If $n=1$ then the long exact sequence of homotopy groups shows that 
we must have $m = 1$ and $k=0$. Assume then that $n\geq 2$. Consider the Serre spectral 
sequence of this fibration. Its second page $E^{2}_{p, q} \cong H_{p}(S^{n}, H_{q}(S^{k}))$
has only four non-zero terms, all isomorphic 
to $\Z$:
\begin{equation*}
\begin{tikzpicture}[scale=0.8]
\matrix (m) [matrix of math nodes,
    nodes in empty cells,
    row sep=0.1em, 
    column sep=0.5em,
    nodes={minimum width=5ex, anchor=center, minimum height=5ex,outer sep=-1pt,
    font=\footnotesize}
    ]{          
k &[-3mm]  E^{2}_{0,k}   &[15mm]   E^{2}_{n, k} \\
  &                &      \\
0 &   E^{2}_{0, 0} &   E^{2}_{n, 0}  \\[-3mm]
  &   0            &   n  \\};
\draw[very thick] ([yshift=2mm]m-1-1.east) -- ([yshift=-1mm]m-4-1.east);
\draw[very thick] ([yshift=-1mm]m-4-1.north) -- ([yshift=-1mm, xshift=3mm]m-4-3.north);
\end{tikzpicture}
\end{equation*}
All differentials originating and terminating at $E^{r}_{0, 0}$ and $E^{r}_{k, n}$ 
are trivial, so $E^{2}_{, 0} = E^{\infty}_{0, 0}$ and  $E^{2}_{n, k} = E^{\infty}_{n, k}$.
The page $E^{\infty}_{\ast\ast}$ can have non-zero terms $E^{\infty}_{p, q}$ only if
$(p, q) = (0, 0)$ or $p+q = m$. It follows that $k+n = m$. The terms 
$E^{2}_{0, k}$ and $E^{2}_{n, 0}$ must kill each other, so they must be connected by 
a differential. This is possible only if  $k=n-1$. Taken together these observations 
imply that $p$ is a fibration sequence of the form 
$S^{n-1} \to S^{2n-1} \overset{p}{\to} S^{n}$.
\end{proof}

Hopf bundles give examples of fibration sequences 
$S^{n-1} \to S^{2n-1} \to S^{n}$ for $n=1, 2, 4, 8$. A theorem of 
Adams implies that these are the only fibration sequences where all three spaces 
are spheres.

