% !TEX root = mth727_lecture_notes.tex


\chapter[Hurewicz Theorem]{Hurewicz Theorem}
\chaptermark{Hurewicz Theorem}
\label{HUREWICZ THEOREM CHAPTER}
\thispagestyle{firststyle}


Hurewicz homomorphism is a map that connects homotopy and homology groups. 
Recall that $H_{n}(S^{n})\cong \Z$ We will denote by $\gamma_{n}$ a chosen 
generator of $H_{n}(S^{n})$. Given an element 
$[\varphi\colon (S^{n}, s_{0}) \to (X, x_{0})]\in \pi_{n}(X, x_{0})$ 
consider the homomorphism $\varphi_{\ast}\colon H_{\ast}(S^{n}) \to H_{n}(X)$. 
This homomorphism depends only on the homotopy class of $\varphi$.

\begin{definition}
The \emph{Hurewicz homomorphism} is a function 
\[
h\colon \pi_{n}(X, x_{0}) \to H_{n}(X)
\]
given by $h([\varphi]) = \varphi_{\ast}(\gamma_{n})$.
\end{definition}


\begin{proposition}
\label{HUREWICZ NATURAL PROP}
For any function $f\colon X \to Y$ the following diagram commutes:
\begin{equation*}
\begin{tikzpicture}
\matrix (m) 
[matrix of math nodes, row sep=3em, column sep=3em, text height=1.5ex, text depth=0.25ex]
{
\pi_{n}(X, x_{0}) & \pi_{n}(Y, f(x_{0})) \\
H_{n}(X) & H_{n}(Y) \\
};
\path[->, thick, font=\scriptsize]
(m-1-1) 
edge node[auto] {$f_{\ast}$} (m-1-2)
edge node[anchor = east] {$h$}  (m-2-1)
(m-1-2)
edge node[anchor=  west] {$h$}  (m-2-2)
(m-2-1)
edge node[anchor=  north] {$f_{\ast}$} (m-2-2)
; 
\end{tikzpicture}
\end{equation*}
\end{proposition}

\begin{proof}
For $[\varphi]\in\pi_{n}(X, x_{0})$ we have
\[
hf_{\ast}([\varphi]) = h([f\varphi]) = (f\varphi)_{\ast}(\gamma_{n})
= f_{\ast}\varphi_{\ast}(\gamma_{n}) = f_{\ast}h([\varphi])
\]
\end{proof}


\begin{proposition}
The Hurewicz homorphism is a group homomorphism. 
\end{proposition}

\begin{proof}
Let $\varphi, \psi\colon (S^{n}, s_{0}) \to (X, x_{0})$ where $n\geq 1$. 
Recall that the element 
$[\varphi]\cdot[\psi]\in \pi_{n}(X, x_{0})$ is the homotopy class of the map
\[
S^{n} \overset{p}{\lra} S^{n}\vee S^{n} \overset{\varphi\vee\psi}{\lra} X
\]
where $p\colon S^{n}\to S^{n}\vee S^{n}$ is the pinch map. 
Let $r_{1}, r_{2}\colon S^{n}\vee S^{n}\to S^{n}$ be the retractions of $S^{n}\vee S^{n}$
onto the first and, respectively, the second copy of $S^{n}$. 
We have a commutative diagram
\begin{equation*}
\begin{tikzpicture}
\matrix (m) 
[matrix of math nodes, row sep=3.5em, column sep=3em, text height=1.5ex, text depth=0.25ex]
{
H_{n}(S^{n}) & H_{n}(S^{n}\vee S^{n}) & H_{n}(X)\\
& H_{n}(S^{n}) \oplus H_{n}(S^{n}) &  \\
};
\path[->, thick, font=\scriptsize]
(m-1-1) 
edge node[auto] {$p_{\ast}$} (m-1-2)
edge node[anchor = north east] {$\id_{\ast}\oplus \id_{\ast}$}  (m-2-2)
(m-1-2)
edge node[anchor= south] {$(\varphi\vee\psi)_{\ast}$}  (m-1-3)
edge node[anchor=  west] {$r_{1\ast}\oplus r_{2\ast}$}
node[anchor=  east] {$\cong$} (m-2-2)
(m-2-2)
edge node[anchor=  north west] {$\varphi_{\ast} + \psi_{\ast}$} (m-1-3)
; 
\end{tikzpicture}
\end{equation*}
This gives: 
\[
h([\varphi]\cdot[\psi]) = ((\varphi\vee\psi)p)_{\ast}(\gamma_{n})
= (\varphi_{\ast}+\psi_{\ast})(\id_{\ast}\oplus\id_{\ast})(\gamma_{n})
= \varphi_{\ast}(\gamma_{n}) + \psi_{\ast}(\gamma_{n}) = h([\varphi]) + h([\psi])  
\]
\end{proof}

\begin{HUREWICZISO THM}
\label{HUREWICZ ISOMORPHISM THM}
Let $X$ be a path connected space such that for some $n\geq  2$ we have  
$\pi_{i}(X) = 0$ for $i<n$. Then $H_{i}(X)=0$ for $0< i < n$ and 
the Hurewicz homomorphism 
\[
h\colon \pi_{n}(X, x_{0}) \to H_{n}(X)
\]
is an isomorphism. 
\end{HUREWICZISO THM}

\begin{proof}
Assume first that $X = S^{n}$. We have $H_{i}(S^{n}) = 0$ for $0< i < n$. 
In degree $n$ the Hurewicz homomorphism is a map
$h\colon \Z\cong \pi_{n}(S^{n}) \to H_{n}(S^{n})\cong \Z$. The group 
$\pi_{n}(S^{n})$ is generated by the homotopy class of the identity map 
$\id_{S^{n}}\colon S^{n}\to S^{n}$ (\ref{PIN SN NOTE}). We have 
$h([\id_{S^{n}}]) = \id_{S^{n}\ast}(\gamma_{n}) = \gamma_{n}$. Therefore 
$h$ maps a generator of $\pi_{n}(S^{n})$ to a generator of $H^{n}(S^{n})$, 
and so it is an isomorphism.


Next, assume that $X = \bigvee_{i\in I} S^{n}$. Again, in this case
$H_{i}(\bigvee_{i\in I} S^{n}) = 0$ for $0< i < n$. Also, the retraction maps 
$r_{i}\colon \bigvee_{i\in I} S^{n} \to S^{n}$ give a commutative diagram
\begin{equation*}
\begin{tikzpicture}
\matrix (m) 
[matrix of math nodes, row sep=3em, column sep=3em, text height=1.5ex, text depth=0.25ex]
{
\pi_{n}(\bigvee_{i\in I} S^{n}) & \bigoplus_{i\in I}\pi_{n}(S^{n}) \\
H_{n}(\bigvee_{i\in I} S^{n}) & \bigoplus_{i\in I}H_{n}(S^{n}) \\
};
\path[->, thick, font=\scriptsize]
(m-1-1) 
edge node[anchor= south] {$\bigoplus r_{i\ast}$} 
node[anchor= north] {$\cong$}(m-1-2)
edge node[anchor = east] {$h$}  (m-2-1)
(m-1-2)
edge node[anchor=  west] {$\bigoplus_{i\in I} h$}  node[anchor= east] {$\cong$} (m-2-2)
(m-2-1)
edge node[anchor= north] {$\bigoplus r_{i\ast}$} 
node[anchor= south] {$\cong$} (m-2-2)
; 
\end{tikzpicture}
\end{equation*}
The the map $\bigoplus_{i\in I}h$ is an isomorphism by the previous case, 
so the left vertical map $h$ is also an isomorphism. 

For the next step, assume that $X$ is an arbitrary CW complex with $\pi_{i}(X)=0$ 
for $i<n$. By Proposition \ref{N SKELETON FOR N CONNECTED PROP} we can assume that 
$X^{(n-1)} = \ast$, which gives $H_{i}(X) = 0$ for $0< i < n$. 

Let $j\colon X^{(n+1)}\hra X$ be the inclusion of the $(n+1)$-skeleton 
of $X$. By Proposition \ref{HUREWICZ NATURAL PROP} we have a commutative diagram
\begin{equation*}
\begin{tikzpicture}
\matrix (m) 
[matrix of math nodes, row sep=3em, column sep=3em, text height=1.5ex, text depth=0.25ex]
{
\pi_{n}(X^{(n+1)}) & \pi_{n}(X) \\
H_{n}(X^{(n+1)}) & H_{n}(X)  \\
};
\path[->, thick, font=\scriptsize]
(m-1-1) 
edge node[anchor= south] {$j_{\ast}$} node[anchor= north] {$\cong$} (m-1-2)
edge node[anchor = east] {$h$}  (m-2-1)
(m-1-2) 
edge node[anchor = east] {$h$}  (m-2-2)
(m-2-1)
edge node[anchor=  north] {$j_{\ast}$} node[anchor= south] {$\cong$} (m-2-2)
; 
\end{tikzpicture}
\end{equation*}
The upper homomorphism $j_{\ast}$ is an isomorphism by 
Proposition \ref{PIN FOR CW SKELETON PROP}, and the lower 
$j_{\ast}$ is an isomorphism by properties of homology groups. 
As a consequence, it is enough to show that 
$h\colon \pi_{n}(X^{(n+1)})\to H_{n}(X^{(n+1)})$ is an isomorphism. 


Since $X^{(n-1)} = \ast$, it follows that $X^{(n)} = \bigvee_{i\in I} S^{n}$ and 
$X^{(n+1)} = X^{(n)} \cup \bigcup_{k\in K} e_{k}^{(n+1)}$
where $\{e_{k}^{(n+1)}\}_{k\in K}$ are $(n+1)$-cells of $X$. 
Let $\varphi_{k}\colon S^{n}\to X^{(n)}$ be the attaching map of the cell $e^{n+1}_{k}$, 
and let $i\colon X^{(n)}\hra X^{(n+1)}$ denote the inclusion map. 
We have a commutative diagram
\begin{equation*}
\begin{tikzpicture}
\matrix (m) 
[matrix of math nodes, row sep=3em, column sep=3em, text height=1.5ex, text depth=0.25ex]
{
\pi_{n}(\bigvee_{k\in K}S^{n}) & \pi_{n}(X^{(n)}) & \pi_{n}(X^{(n+1)}) & 0 \\
H_{n}(\bigvee_{k\in K}S^{n}) & H_{n}(X^{(n)}) & H_{n}(X^{(n+1)}) &H_{n-1}(\bigvee_{k\in K}S^{n})\\
};
\path[->, thick, font=\scriptsize]
(m-1-1) 
edge node[anchor= south] {$(\bigvee_{k\in K}\varphi_{k})_{\ast}$} (m-1-2)
edge node[anchor = east] {$h$}  node[anchor= west] {$\cong$} (m-2-1)
(m-1-2) 
edge node[anchor= south] {$i_{\ast}$}  (m-1-3)
edge node[anchor = east] {$h$}  node[anchor= west] {$\cong$} (m-2-2)
(m-1-3)
edge node[anchor=  west] {$h$}  (m-2-3)
(m-2-1)
edge node[anchor=  north] {$(\bigvee_{k\in K}\varphi_{k})_{\ast}$}  (m-2-2)
(m-2-2)
edge node[anchor=  north] {$i_{\ast}$}  (m-2-3)
(m-1-3)
edge (m-1-4)
(m-2-3)
edge (m-2-4)
(m-1-4)
edge node[anchor= west] {$\cong$} (m-2-4)
; 
\end{tikzpicture}
\end{equation*}
The upper row of this diagram is exact by Proposition \ref{1CONN PIN XCUPE PROP}, 
and the lower row is exact by the long homology sequence associated to the map 
$\bigvee_{k\in K}\varphi_{k}$. By the Five Lemma we obtain that 
$h\colon \pi_{n}(X^{(n+1)})\to H_{n}(X^{(n+1)})$ is an isomorphism.



Finally, let $X$ be an arbitary space with $\pi_{i}(X) = 0$ for $i< n$.
Let $f\colon Y \to X$ be a CW approximation of $X$ (\ref{CW APPROX DEF}). 
Using Theorem \ref{WE HOMOLOGY ISO THM} and the previous case we get 
$H_{i}(X) \cong H_{i}(Y) = 0$ for $0 < i < n$. 

By Proposition \ref{HUREWICZ NATURAL PROP} we have a commutative diagram
\begin{equation*}
\begin{tikzpicture}
\matrix (m) 
[matrix of math nodes, row sep=3em, column sep=3em, text height=1.5ex, text depth=0.25ex]
{
\pi_{n}(Y) & \pi_{n}(X) \\
H_{n}(Y) & H_{n}(X) \\
};
\path[->, thick, font=\scriptsize]
(m-1-1) 
edge node[anchor= south] {$f_{\ast}$} node[anchor= north] {$\cong$} (m-1-2)
edge node[anchor = east] {$h$}  node[anchor= west] {$\cong$} (m-2-1)
(m-1-2)
edge node[anchor=  west] {$h$}  (m-2-2)
(m-2-1)
edge node[anchor=  north] {$f_{\ast}$} node[anchor= south] {$\cong$} (m-2-2)
; 
\end{tikzpicture}
\end{equation*}
Since $f$ is a weak equivalence, the upper homomorphism $f_{\ast}$ is an isomorphism 
by definition, and the lower $f_{\ast}$ is an isomorphism 
by Theorem \ref{WE HOMOLOGY ISO THM}. Also, since $Y$ is a CW complex the left vertical 
map is an isomorphism by the previous case. Therefore $h\colon \pi_{n}(X)\to H_{n}(X)$
is an isomorphism.  
\end{proof}

\begin{INVHUREWICZISO THM}
\label{INVERSE HUREWICZ ISO THM}
Let $X$ be a simply connected space, and let $H_{i}(X) = 0$ for $1\leq i < n$ for some 
$n \geq 2$. Then $\pi_{i}(X) = 0$ for $i < n$ and the Hurewicz homomorphism 
$h\colon \pi_{n}(X) \to H_{n}(X)$ is an isomorphism. 
\end{INVHUREWICZISO THM}

\begin{proof}
Exercise.
\end{proof}



Since all homology groups $H_{i}(X)$ are abelian but the fundamental group 
$\pi_{1}(X)$ need not be abelian, in general the Hurewicz homomorphism 
$h\colon \pi_{1}(X) \to H_{1}(X)$ is not an isomorphism. However,
a version of Theorem \ref{HUREWICZ ISOMORPHISM THM} still holds with the following 
modification. Recall that if $G$ is a group then the commutator of $G$ is the 
subgroup $[G, G]\subseteq G$ generated by all elements of the form $ghg^{-1}h^{-1}$
for $g, h\in G$. The commutator is a normal subgroup of $G$, and the quotient group 
$G^{\text{ab}} := G/[G, G]$ is an abelian group. The group  $G^{\text{ab}}$ is called
the abelianization of $G$. 

If $H$ is an abelian group then any homomorphism
$\varphi\colon G \to H$ defines a unique homomorphism $\xov{\varphi}\colon G^{\text{ab}} \to H$
such that $\varphi = \xov{\varphi}\eta$ where $\eta\colon G\to G^{\text{ab}}$ is 
the quotient homomorphism. Also, if $\psi\colon G \to H$ is a homomorphism of 
arbitary groups, then $\psi([G, G])\subseteq [H, H]$, and so $\psi$ induces a homomorphism 
of abelianizations $\psi^{\text{ab}}\colon G^{\text{ab}}\to H^{\text{ab}}$.

\begin{theorem}
\label{HUREWICZ PI1 THM}
Let $X$ be a path connected space and let
$h\colon \pi_{1}(X, x_{0}) \to H_{1}(X)$ be the Hurewicz homomorphism. 
Then the induced homomorphism $\xov{h}\colon \pi_{1}(X, x_{0})^{\text{ab}} \to H_{1}(X)$
is an isomorphism. 
\end{theorem}

The proof will use the following algebraic fact. 

\begin{lemma}
\label{ABELIANIZATION EXACTNESS LEMMA}
Consider a sequence of group homomorphisms
\[
G \overset{\varphi}{\lra} H \overset{\psi}{\lra} K 
\]
such that $\psi$ is onto and $\ker \psi = N_{H}(\Im\varphi)$
where $N_{H}(\Im\varphi)$ is the normalizer of $\Im\varphi$ in $H$. 
Then the induced sequence 
\[
G^{\text{ab}} \overset{\varphi^{\text{ab}}}{\lra} H^{\text{ab}} \overset{\psi^{\text{ab}}}{\lra} K^{\text{ab}} \lra 0
\]
is exact.
\end{lemma}

\begin{proof}
Exercise.
\end{proof}



\begin{proof}[Proof of Theorem \ref{HUREWICZ PI1 THM}]
Take $X=S^{1}$. As in the proof of Theorem \ref{HUREWICZ ISOMORPHISM THM} we obtain
that $h\colon \pi_{1}(S^{1}) \to H_{1}(S^{1})$ is an isomorphism. Also, since
$\pi_{1}(S^{1})\cong \Z$ is an abelian group, thus 
$\pi_{1}(S^{1})\cong \pi_{1}(S^{1})^{\text{ab}}$ and, up to this isomorphism, 
$\xov{h}$ coincides with $h$. 

Next, take $X= \bigvee_{i\in I} S^{1}$ and let $r_{i}\colon \bigvee_{i\in I} S^{1} \to S^{1}$
be retraction maps. We have a commutative diagram 
\begin{equation*}
\begin{tikzpicture}
\matrix (m) 
[matrix of math nodes, row sep=3em, column sep=3em, text height=1.5ex, text depth=0.25ex]
{
\pi_{1}(\bigvee_{i\in I} S^{1}) & \bigoplus_{i\in I}\pi_{1}(S^{1}) \\
H_{n}(\bigvee_{i\in I} S^{1}) & \bigoplus_{i\in I}H_{n}(S^{1}) \\
};
\path[->, thick, font=\scriptsize]
(m-1-1) 
edge node[anchor= south] {$\bigoplus r_{i\ast}$} 
(m-1-2)
edge node[anchor = east] {$h$}  (m-2-1)
(m-1-2)
edge node[anchor=  west] {$\bigoplus_{i\in I} h$}  node[anchor= east] {$\cong$} (m-2-2)
(m-2-1)
edge node[anchor= north] {$\bigoplus r_{i\ast}$} 
node[anchor= south] {$\cong$} (m-2-2)
; 
\end{tikzpicture}
\end{equation*}
The upper map $\bigoplus r_{i\ast}$ essentially conincides with the abelianization 
of $\pi_{1}(\bigvee_{i\in I}S^{1})$, and the map $\bigoplus_{i\in I} h$ 
coincides, up to an isomorphism, with 
$\xov{h}\colon \pi_{1}(\bigvee_{i\in I}S^{1})^{\text{ab}} \to H_{1}(\bigvee_{i\in I})$. 
It remains to notice that 
$\bigoplus_{i\in I} h$ is an isomorphism by the previous case. 


As in the proof of Theorem \ref{HUREWICZ ISOMORPHISM THM}, it remains to consider 
the case where $X$ is a 2-dimensional CW complex of the form 
$X = \bigvee_{i\in I} S^{1} \cup \bigcup_{k\in K} e^{2}_{k}$.   
Let $\varphi_{k}\colon S^{1}\to \bigvee_{i\in I} S^{1}$ 
be the attaching map of the cell $e^{2}_{k}$. Denote 
$\psi := \bigvee_{k\in K}\varphi_{k}\colon \bigvee_{k\in K}S^{1} \to
\bigvee_{i\in I}S^{1}$ Also, let $j\colon \bigvee_{i\in I} S^{1} \hra X$ 
be the inclusion of the 1-skeleton of $X$. We have a sequence of group homomorphisms
\[ 
\pi_{1}(\bigvee_{k\in K}S^{1}) \overset{\psi_{\ast}}{\lra} 
\pi_{1}(\bigvee_{i\in I}S^{1}) \overset{j_{\ast}}{\lra} \pi_{1}(X)
\]
By van Kampen’s Theorem the hoomorphism $j_{\ast}$ is onto and 
$\ker j_{\ast}  = N_{\pi_{1}(\bigvee_{i\in I}S^{1})}(\Im \psi_{\ast})$.
Cosider the commutative diagram
\begin{equation*}
\begin{tikzpicture}
\matrix (m) 
[matrix of math nodes, row sep=3em, column sep=3em, text height=1.5ex, text depth=0.25ex]
{
\pi_{1}(\bigvee_{k\in K}S^{1})^{\text{ab}} & 
\pi_{1}(\bigvee_{i\in I}S^{1})^{\text{ab}} & 
\pi_{1}(X)^{\text{ab}} & 0 \\
H_{1}(\bigvee_{i\in I}S^{1}) & 
H_{1}(\bigvee_{i\in I}S^{1}) & 
H_{1}(X) & 0\\
};
\path[->, thick, font=\scriptsize]
(m-1-1) 
edge node[anchor= south] {$\psi_{\ast}$} (m-1-2)
edge node[anchor = east] {$\xov{h}$}  node[anchor= west] {$\cong$} (m-2-1)
(m-1-2) 
edge node[anchor= south] {$i_{\ast}$}  (m-1-3)
edge node[anchor = east] {$\xov{h}$}  node[anchor= west] {$\cong$} (m-2-2)
(m-1-3)
edge node[anchor=  west] {$\xov{h}$}  (m-2-3)
(m-2-1)
edge node[anchor=  north] {$\psi_{\ast}$}  (m-2-2)
(m-2-2)
edge node[anchor=  north] {$i_{\ast}$}  (m-2-3)
(m-1-3)
edge (m-1-4)
(m-2-3)
edge (m-2-4)
(m-1-4)
edge node[anchor= west] {$\cong$} (m-2-4)
; 
\end{tikzpicture}
\end{equation*}
The upper row is exact by Lemma \ref{ABELIANIZATION EXACTNESS LEMMA}
and the lower row is exact by the long exact homology sequence associated to
$\psi_{\ast}$. Using the Five Lemma we obtain that 
$\xov{h}\colon \pi_{1}(X)^{\text{ab}} \to H_{1}(X)$ is an isomorphism. 
\end{proof}


\begin{nn}{\bf Relative Hurewicz Homomorphism.} 
Recall that $H_{n}(D^{n}, S^{n-1}) \cong \Z$. Let $\bar{\gamma}_{n}$ denote 
a chosen generator of $H_{n}(D^{n}, S^{n-1})$. Given a pointed pair $(X, A, x_{0})$,
and an element 
$[\varphi\colon (D^{n}, S^{n-1}, s_{0}) \to (X, A, x_{0})]\in \pi_{n}(X, A, x_{0})$
consider the function $\varphi_{\ast}\colon H_{n}(D^{n}, S^{n-1}) \to H_{n}(X, A)$. 
\end{nn}

\begin{definition}
The \emph{relative Hurewicz homomorphism} is a function
\[
h \colon \pi_{n}(X, A, x_{0}) \to H_{n}(X, A)
\]
given by $h([\varphi]) = \varphi_{\ast}(\bar{\gamma}_{n})$.
\end{definition}

\begin{proposition}
The relative Hurewicz homomorphism is a group homomorphism for $n \geq 2$.
\end{proposition}


\begin{RELINVHUREWICZISO THM}
\label{REL INV HUREWICZ ISOMORPHISM THM}
Let $(X, A)$ be a pair of simply connected CW complexes. 
If $H_{i}(X, A) = 0$ for all $0 < i< n$ for some $n\geq 2$  
then $\pi_{i}(X, A)=0$ for all $i< n$ and the Hurewicz homomorphism 
$h\colon \pi_{n}(X, A) \to H_{n}(X, A)$ is an isomorphism.   
\end{RELINVHUREWICZISO THM}

\begin{proof}
See tom Dieck, Theorem 20.1.3 p. 497. Uses commutativity of the diagram 
\begin{equation*}
\begin{tikzpicture}
\matrix (m) 
[matrix of math nodes, row sep=3em, column sep=3em, text height=1.5ex, text depth=0.25ex]
{
\pi_{i}(X, A) & \pi_{i}(X/A) \\
H_{i}(X, A) & H_{i}(X/A) \\
};
\path[->, thick, font=\scriptsize]
(m-1-1) 
edge   (m-1-2)
edge node[anchor = east] {$h$}  (m-2-1)
(m-1-2)
edge node[anchor=  west] {$h$}  (m-2-2)
(m-2-1)
edge  (m-2-2)
; 
\end{tikzpicture}
\end{equation*}
and the Inverse Hurewicz Theorem \ref{INVERSE HUREWICZ ISO THM} applied 
to the space $X/A$.
\end{proof}


\begin{theorem}
\label{WHITEHEAD ISO THM}
Let $X, Y$ be simply connected CW complexes and let $f\colon X \to Y$ be a map such 
that for some $n\geq 2$ the homorphism $f_{\ast}\colon H_{i}(X) \to H_{i}(Y)$ 
is an isomorphism for $i< n$ and epimorphism for $i = n$. Then 
$f_{\ast}\colon \pi_{i}(X) \to \pi_{i}(Y)$ is an isomorphism for $i< n$ and 
epimorphism for $i = n$. 
\end{theorem}

\begin{proof}
Let $M_{f}$ be the mapping cylinder of $f$. The assumption about $f$ is 
equivalent to the condition that $H_{i}(M_{f}, X) = 0$ for $i<n$. By 
Theorem \ref{REL INV HUREWICZ ISOMORPHISM THM} this gives $\pi_{i}(M_{f}, X) = 0$ for $i< n$.
The statement then follows from the long exact sequence of homotopy groups of the 
pair $(M_{f}, X)$.
\end{proof}

\begin{corollary}
Let $f\colon X \to Y$ be a map of simply connected CW complexes such that 
$f_{\ast}\colon H_{i}(X) \to H_{i}(Y)$ is an isomorphism for all $i\geq 0$. 
Then $f$ is a homotopy equivalence.
\end{corollary}


Let $p_{X}\colon \widetilde{X} \to X$ denote the universal cover 
of a space $X$. Given a map $f\colon X\to Y$ we can find a map 
$\widetilde{f}\colon \widetilde{X} \to \widetilde{Y}$ such that the following diagram commutes:
\begin{equation*}
\begin{tikzpicture}
\matrix (m) 
[matrix of math nodes, row sep=3em, column sep=3em, text height=1.5ex, text depth=0.25ex]
{
\widetilde{X} & \widetilde{Y} \\
X & Y \\
};
\path[->, thick, font=\scriptsize]
(m-1-1) 
edge node[anchor= south] {$\widetilde{f}$}  (m-1-2)
edge node[anchor = east] {$p_{X}$} (m-2-1)
(m-1-2)
edge node[anchor=  west] {$p_{Y}$}  (m-2-2)
(m-2-1)
edge node[anchor=  north] {$f$} (m-2-2)
; 
\end{tikzpicture}
\end{equation*}


\begin{theorem}
Let $f\colon X \to Y$ be a map of path connected CW complexes. 
If the homomorphisms $f_{\ast}\colon\pi_{1}(X)\to \pi_{1}(Y)$ and 
$\widetilde{f}_{\ast}\colon H_{i}(\widetilde{X}) \to  H_{i}(\widetilde{Y})$
for all $i\geq 0$ are isomorphisms then $f$ is a homotopy equivalence.
\end{theorem}


\begin{proof}
By Theorem \ref{WHITEHEAD ISO THM} the map 
$\widetilde{f}_{\ast}\colon \pi_{i}(\widetilde{X}) \to \pi_{i}(\widetilde{Y})$
is an isomorphism for all $i\geq 0$. Since $p_{X\ast}\colon \pi_{i}(\widetilde{X}) \to 
\pi_{i}(X)$ and  $p_{Y\ast}\colon \pi_{i}(\widetilde{Y}) \cong \pi_{i}(Y)$ are 
isomorphisms for $i \geq 2$, this gives that $f_{\ast}\colon \pi_{i}(X) \to \pi_{i}(Y)$ 
is an isomorphism for $i\geq 2$. By assumption, $f_{\ast}\colon \pi_{1}(X) \to \pi_{1}(Y)$
is an isomorphism as well, so $f$ is a weak equivalence and thus a homotopy equivalence.
\end{proof}





