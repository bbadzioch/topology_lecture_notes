% !TEX root = mth727_lecture_notes.tex


\chapter[Excision]{Excision}
\chaptermark{Excision}
\label{EXCISION}
\thispagestyle{firststyle}

One of the main properties of homology groups is excision. 
It can stated as follows: 

\begin{theorem}
Let $X$ be a space and $X_{1}, X_{2}\subseteq X$ be open sets such that 
$X= X_{1}\cup X_{2}$. Then the map of pairs 
$i\colon (X_{1}, X_{1}\cap X_{2}) \to (X, X_{2})$ induces an 
isomorphism
\[
i_{\ast}\colon H_{n}(X_{1}, X_{1}\cap X_{2}) 
\overset{\cong}{\lra} 
H_{n}(X, X_{2})
\]
for all $n\geq 0$.
\end{theorem}


The same property does not holds in general for homotopy groups. However, 
it does hold under some extra assumptions. In order to make this precise we
will need a definition.

\begin{definition}
\label{N CONNECTED PAIR DEF}
Let $A\subseteq X$ and let $0 \leq n\leq \infty$. The pair $(X, A)$ is \emph{$n$-connected}
if the map $\pi_{0}(A) \to \pi_{0}(X)$ is onto and $\pi_{k}(X, A, x_{0}) = \{1\}$ 
for all $x_{0}\in A$ and all $1 \leq k\leq n$.
\end{definition}


\begin{proposition}
\label{N CONNECTED PAIR PROP}
Let $A\subseteq X$. The following conditions are equivalent.
\benu
\item[1)] $(X, A)$ is $n$-connected.
\item[2)] The homomorphism $i_{\ast}\colon \pi_{k}(A, x_{0})  \to \pi_{k}(X, x_{0})$ 
induced by the inclusion map $i\colon A\hra X$ is an isomorphism for all $x_{0}\in A$
and all $k < n$ and it is an epimorphism for $k=n$.  
\item[3)] For $k\leq n$, any map $(I^{k}, \pint^{k}) \to (X, A)$ is homotopic 
relative to $\pint^{k}$ to a map $I^{k} \to A$. 
\item[4)] For $k\leq n$, any map $h\colon I^{k}\cup (\partial I^{k} \times I)  \to X$ 
such that $h(\partial I^{k} \times \{ 1 \}) \subseteq A$ can be extended to a map 
$\xov{h}\colon I^{k}\times I \to X$ such that $h(I^{k}\times \{1\}) \subseteq A$. 
\eenu
\end{proposition}

\begin{proof}
Exercise.
\end{proof}


\begin{HOMOTEXCISIONTHM}
\label{HOMOTOPY  EXCISION THM}
Let $X$ be a space and
$X_{1}, X_{2}\subseteq X$ be open such that 
$X=  X_{1}\cup X_{2}$. Assume that 
\begin{itemize}
\item $(X_{1}, X_{1}\cap X_{2})$ is $m$-connected 
\item $(X_{2}, X_{1}\cap X_{2})$ is $n$-connected 
\end{itemize}
for some $m, n \geq 0$. Then for any $x_{0}\in X_{1}\cap X_{2}$ the homomorphism 
\[
i_{\ast}\colon \pi_{k}(X_{1}, X_{1}\cap X_{2}, x_{0}) \to \pi_{k}(X, X_{2}, x_{0})
\]
induced by  the inclusion map, is an isomorphism for $1 \leq k < m+n$ and 
it is onto for $k = m+n$.
\end{HOMOTEXCISIONTHM}

In this chapter we will explore some consequences Theorem \ref{HOMOTOPY  EXCISION THM}, 
an we will return to its proof in Chapter \ref{PROOF EXCISION CHAPTER}.


\begin{proposition}
\label{PIK QUOTIENT PROP}
Let $(X, A)$ be a pair with the homotopy extension property and let 
$q\colon X \to X/A$ be the quotient map. Let $x_{0}\in A$ and 
$\ast = q(A) \in X/A$. If $(X, A)$  is $m$-connected
and the space $A$ is $n$-connected for some $m, n \geq 0$ then the homomorphism 
\[
q_{\ast}\colon \pi_{k}(X, A, x_{0}) \to \pi_{k}(X/A, \ast, \ast) = \pi_{k}(X/A, \ast)
\]
is an isomorphism for $k \leq m+n$ and it is an epimorphism for $k = m+n+1$. 
\end{proposition}

\begin{proof}
Let $j\colon A \hra X$ be the inclusion map. Let $M$ denote the the mapping cylinder of $j$:
\[
M = (A \times [0, 1] \sqcup X) / {\sim}
\]
where $(x, 0) \sim x$ for all $x\in A$.
Also, let $C = M/(A\times \{1\})$ be the mapping cone of $j$. In other words, 
$C$ is obtained by attaching the cone $CA = A\times [0, 1]/ (A\times \{1\})$ to $X$. 

Take the quotient map $\varphi \colon M \to C$. Denote by $M_{1}, M_{2}\subseteq M$
the subspaces of $M$ given by $M_{1} = X \cup A\times [0, \frac{3}{4}]$ and 
$M_{2} = A\times [\frac{1}{4}, 1]$, and let $C_{i} = \varphi(M_{i})$ for $i=1, 2$.  
Also, let $\xov{x}_{0} = (x_{0}, \frac{1}{2})\in M_{1}\cap M_{2}$. 
\begin{equation*}
\begin{tikzpicture}[
    xscale = 1.2,
    yscale = 1.2,
    d0/.style = {line width = 1.6}
]

\draw[d0]  (-0.4, 0) --(1.4, 0)  
node[pos=0, anchor=south] {$X$}
node[yshift = -2mm, anchor = north, pos=0.5] {\small $M$};
\draw[d0, fill = mygray1] (0,0) -- (0.0, 2) -- (1.0, 2) -- (1, 0) -- cycle;
\draw[d0, red] (0, 0) --(1.0, 0) node[pos= 0.5, anchor = south] {\small $A$};

\draw[blue, line width = 1.4pt, dotted] (-0.8,2) -- (1.2, 2) (-0.8, 0.5) -- (1.2, 0.5);
\draw[line width = 1.4pt,  color = blue, decorate, decoration={brace, amplitude = 5pt, mirror}] 
(-0.8, 2) -- (-0.8, 0.5) node [midway,xshift=-18pt]  {\small $M_{2}$};

\draw[blue, line width = 1.4pt, dotted] (1.8,1.5) -- (-0.2, 1.5) (1.8, 0.0) -- (1.4, 0.0);
\draw[line width = 1.4pt,  color = blue, decorate, decoration={brace, amplitude = 5pt}] 
(1.8, 1.5) -- (1.8, 0) node [midway,xshift=18pt]  {\small $M_{1}$};

\fill (0.5, 0.9) circle (0.05) node[above] {\small $\xov{x}_{0}$};

\draw[thick, ->, >=latex] (3.0, 1) -- (4.0, 1)
node [above, pos=0.5] {\small $\varphi$};

\begin{scope}[shift ={(6, 0)}]
\draw[d0] (-0.4, 0) --(1.4, 0) 
node[pos=0, anchor=south] {$X$}
node[yshift = -2mm, anchor = north, pos=0.5] {\small $C$};
\draw[d0, fill = mygray1] (0, 0.0) -- (0.5, 2) -- (1, 0) -- cycle;
\draw[d0, red] (0, 0) --(1.0, 0) node[pos= 0.5, anchor = south] {\small $A$};

\draw[blue, line width = 1.4pt, dotted] (-0.8,2) -- (1.2, 2) (-0.8, 0.5) -- (1.2, 0.5);
\draw[line width = 1.4pt,  color = blue, decorate, decoration={brace, amplitude = 5pt, mirror}] 
(-0.8, 2) -- (-0.8, 0.5) node [midway,xshift=-18pt]  {\small $C_{2}$};

\draw[blue, line width = 1.4pt, dotted] (1.8,1.5) -- (-0.2, 1.5) (1.8, 0.0) -- (1.4, 0.0);
\draw[line width = 1.4pt,  color = blue, decorate, decoration={brace, amplitude = 5pt}] 
(1.8, 1.5) -- (1.8, 0) node [midway,xshift=18pt]  {\small $C_{1}$};

\end{scope}
\end{tikzpicture}
\end{equation*}
Let $r\colon M \to X$ be the retraction map, and let $s\colon C \to X/A$ be the map
that sends the cone $CA\subseteq C$ to the point $\ast\in X/A$. Both $r$ and $s$ are 
homotopy equivalences. For $s$ this follows from Proposition \ref{CONTR QUOTIENT WITH HEP PROP}  
using the fact that since $(X, A)$ has the homotopy extension property, then $(C, CA)$ 
also has this property.

For any $k\geq 1$  the following diagram commutes: 

\begin{equation*}
\begin{tikzpicture}
\matrix (m) 
[matrix of math nodes, row sep=3em, column sep=2em, text height=1.5ex, text depth=0.25ex]
{
\pi_{k}(X, A, x_{0})  & \pi_{k}(X/A, \ast, \ast) \\
\pi_{k}(M, M_{2}, \xov{x}_{0}) & \pi_{k}(C, C_{2}, \varphi(\xov{x}_{0})) \\
\pi_{k}(M_{1}, M_{1}\cap M_{2}, \xov{x}_{0}) & \pi_{k}(C_{1}, C_{1}\cap C_{2}, \varphi(\xov{x}_{0})) \\
};
\path[->, thick, font=\scriptsize]
(m-1-1) 
edge node[auto] {$q_{\ast}$} (m-1-2)
(m-2-1)
edge 
node[anchor = east] {$r_{\ast}$} 
node[anchor = west] {$\cong$} (m-1-1)
edge 
node[anchor=  south] {$\varphi_{\ast}$} (m-2-2)
(m-2-2)
edge 
node[anchor=  west] {$s_{\ast}$} 
node[anchor=  east] {$\cong$} (m-1-2)
(m-3-1)
edge 
node[anchor=  west] {$\cong$} 
node[anchor=  east] {$i_{\ast}$} (m-2-1)
edge 
node[anchor=  south] {$\varphi|_{M_{1}\ast}$} 
node[anchor=  north] {$\cong$} (m-3-2)
(m-3-2)
edge 
node[anchor=  west] {$i’_{\ast}$} (m-2-2)
; 
\end{tikzpicture}
\end{equation*}
Here $\varphi|_{M_{1}}$ is the restriction of $\varphi$ and 
the homomorphisms $i_{\ast}$, $i’_{\ast}$ are induced by inclusions.
Since $i\colon (M_{1}, M_{1}\cap M_{2}) \to (M_{j}, M_{1}\cap M_{2})$ is a homotopy 
equivalence and $\varphi|_{M_{1}}\colon (M_{1}, M_{1}\cap M_{2}) \to (C_{1}, C_{1}\cap C_{2})$
is a homeomorphism, $i_{\ast}$ and $\varphi|_{M_{1}\ast}$ are isomorphisms. It follows that 
$q_{\ast}$ is an isomorphism or epimorphism if and only if $i’_{\ast}$ has the same 
property. 

From the above diagram we also obtain that 
$\pi_{k}(C_{1}, C_{1}\cap C_{2}, \varphi(\xov{x}_{0})) \cong \pi_{k}(X, A, x_{0})$ for all 
$k$, so $(C_{1}, C_{1}\cap C_{2})$ is $m$-connected. Also, since $C_{2}$ is a contractible 
space, from the long exact sequence of the pair $(C_{2}, C_{1}\cap C_{2})$ we get 
\[
\pi_{k}(C_{2}, C_{1}\cap C_{2}, \varphi(\xov{x}_{0})) \cong 
\pi_{k-1}(C_{1}\cap C_{2}, \varphi(\xov{x}_{0})) \cong 
\pi_{k-1}(A, x_{0})
\]
Since by assumption $A$ is $n$-connected, thus $(C_{2}, C_{1}\cap C_{2})$ is $(n+1)$-connected. 
By the Excision Theorem \ref{HOMOTOPY  EXCISION THM} we obtain that 
$i’_{\ast}$ (and thus also $q_{\ast}$) is an isomorphism for $k\leq m + n$ and 
an epimorphism for $k = m+n+1$.
\end{proof}



Let $(X, x_{0})$ be a pointed space and let $\omega\colon (I^{n}, \pint^{n}) \to (X, x_{0})$
represent an element $[\omega] \in \pi_{n}(X, x_{0})$. Let $\Sigma X$ be the reduced suspension 
of $X$. Consider the map 
$\Sigma’\omega\colon I^{n+1}\to \Sigma X$ obtained the composition 
\[
\Sigma’\omega\colon I^{n+1} = I^{n}\times [0, 1] \overset{q}{\lra} \Sigma I^{n}
\overset{\Sigma \omega}{\lra} \Sigma X
\]
where $q$ is the quotient map. One can check that $\Sigma’\omega$ represents an element 
of $\pi_{n+1}(\Sigma X, \xov{x}_{0})$. 

\begin{definition/proposition}
The assignment $[\omega] \mapsto [\Sigma’\omega]$
defines a homomorphism of groups 
\[
\Sigma_{\ast}\colon \pi_{n}(X, x_{0}) \to \pi_{n+1}(\Sigma X, \xov{x}_{0})
\]
which is called the \emph{suspension homomorphism}.
\end{definition/proposition}

\begin{proof}
The function $\Sigma_{\ast}$ is well defined since the suspension functor preserves 
homotopy classes of maps. It remains to check that $\Sigma_{\ast}$ is a group homomorphism 
(exercise).
\end{proof}


\begin{FREUDENTHALTHM}
\label{FREUDENTHAL THM}
Let $(X, x_{0})$ be a well-pointed, $n$-connected space. Let $\xov{x}_{0}$
denote the basepoint in the reduced suspension $\Sigma X$. The suspension 
homomorphism 
\[
\Sigma_{\ast}\colon \pi_{k}(X, x_{0}) \to \pi_{k+1}(\Sigma X, \xov{x}_{0})
\]
is an isomorphism for $k \leq 2n$ and it is an epimorphism for $k= 2n + 1$.
\end{FREUDENTHALTHM}

\begin{proof}
First, let $CX = X\times [0, 1]/X\times \{1\}$ be the cone on $X$. Identifying 
$X$ with $X\times \{0\}$ we can consider it as a subspace of $CX$. Since $CX$
is a contractible space, in the long exact sequence of the pair $(CX, X)$ the 
homomorphism 
$\partial \colon \pi_{k+1}(CX, X, x_{0}) \to \pi_{k}(X, x_{0})$ 
is an isomorphism for all $k\geq 0$.

One can check (exercise) that if $(X, x_{0})$ is a well-pointed space, 
then for any $k\geq 0$ the following diagram commutes:
\begin{equation*}
\begin{tikzpicture}
\matrix (m) 
[matrix of math nodes, row sep=3em, column sep=3em, text height=1.5ex, text depth=0.25ex]
{
\pi_{k}(X, x_{0})  & \pi_{k+1}(\Sigma X, \xov{x}_{0}) \\
\pi_{k+1}(CX, X, x_{0}) & \pi_{k+1}(CX/X, \xov{x}_{0}) \\
};
\path[->, thick, font=\scriptsize]
(m-1-1) 
edge node[auto] {$\Sigma_{\ast}$} (m-1-2)
(m-2-2)
edge 
node[anchor=  west] {$q’_{\ast}$} 
node[anchor=  east] {$\cong$} (m-1-2)
(m-2-1)
edge node[anchor=  north] {$q_{\ast}$} (m-2-2)
edge 
node[anchor = east] {$\partial$} 
node[anchor = west] {$\cong$} (m-1-1)
; 
\end{tikzpicture}
\end{equation*}
Here $q_{\ast}$ and $q’_{\ast}$ are induced by the quotient maps $q\colon CX \to CX/X$
and $q’\colon CX/X = SX \to \Sigma X$. 

Since $(X, x_{0})$ is well-pointed, the  map $q’$ is a homotopy equivalence, and thus
$q’_{\ast}$ is an isomorphism. 
It follows that $\Sigma_{\ast}$ is an isomorphism or epimorphism if and only if this 
holds for $q_{\ast}$. Since $X$ is $n$-connected and $CX$ is contractible, the pair
$(CX, X)$ is $n+1$-connected. Therefore, by Proposition \ref{PIK QUOTIENT PROP},
$q_{\ast}$ is an isomorphism for $k + 1 \leq 2n + 1$ (or $k \leq 2n$)
and an epimorphism for $k + 1 = 2n + 2$ (i.e. $k=2n + 1$)

\end{proof}


Since the sphere $S^{n}$ is $(n-1)$-connected, by Theorem \ref{FREUDENTHAL THM} 
we obtain: 

\begin{corollary}
\label{FREUDENTHAL SN THM}
The suspension homomorphism 
\[
\Sigma_{\ast}\colon \pi_{k}(S^{n}) \to \pi_{k+1}(\Sigma S^{n}) \cong \pi_{k+1}(S^{n+1})
\]
is an isomorphism for $k \leq 2n-2$ and an epimorphism for $k = 2n -1$.  
\end{corollary}



\begin{corollary}
\label{PIN SN COROLLARY}
For any $n\geq 1$ we have $\pi_{n}(S^{n})\cong \Z$. 
\end{corollary}

\begin{proof}

We argue by induction with respect to $n$. We already know that $\pi_{1}(S^{1})\cong \Z$. 
Also, by Theorem \ref{PI2 S2 THM} we have $\pi_{2}(S^{2}) \cong \Z$. 

Next, assume that $\pi_{n}(S^{n})\cong \Z$ for some $n\geq 2$. 
In such case $2n - 2 \geq n$, so by Corollary \ref{FREUDENTHAL SN THM} we obtain 
$\Z \cong \pi_{n}(S^{n})\cong \pi_{n+1}(S^{n+1})$. 
\end{proof}

\begin{note}
\label{PIN SN NOTE}
1) By Corollary \ref{FREUDENTHAL SN THM} the suspension homomorphism 
$\Sigma_{\ast}\colon \pi_{n}(S^{n}) \to \pi_{n+1}(S^{n+1})$
is an isomorphism for all $n\geq 2$. By the same corollary 
$\Sigma_{\ast}\colon \pi_{1}(S^{1}) \to \pi_{2}(S^{2})$ is onto,  
and since every epimorphism $\Z \to \Z$ is an isomorphism, it follows 
that this is an isomorphism as well. 

2) The generator of the group $\pi_{n}(S^{n})$ is represented by 
the identity map $\id\colon S^{n}\to S^{n}$. For $n=1$ it follows 
from the direct computation of $\pi_{1}(S^{1})$, and for $n> 1$
it holds since the suspension isomorphism maps the homotopy class 
of $\id_{S^{n-1}}$ to the homotopy class of $\id_{S^{n}}$
\end{note}


\begin{corollary}
$\pi_{3}(S^{2}) \cong \Z$ and the generator of $\pi_{3}(S^{2})$ is given by the 
homotopy class of the Hopf bundle map (\ref{HOPF BUNDLE EXAMPLE}). 
\end{corollary}

\begin{proof}
The long exact sequence of the Hopf fibration $S^{1}\to S^{3} \overset{p}{\to} S^{2}$ gives
an exact sequence:
\[
0 = \pi_{3}(S^{1}) {\lra}
\pi_{3}(S^{3}) \overset{p_{\ast}}{\lra}
\pi_{3}(S^{2}) \overset{\partial}{\lra} 
\pi_{2}(S^{1}) = 0
\]
Therefore $p_{\ast}$ is an isomorphism and so 
$\pi_{3}(S^{2}) \cong \pi_{3}(S^{3})\cong \Z$. Also, since 
$[\id_{S^{3}}]$ is a generator of $\pi_{3}(S^{3})$, thus 
$p_{\ast}([\id_{S^{3}}]) = [p]$ is a generator of $\pi_{3}(S^{2})$.
\end{proof}

\begin{note}
Notice that since  $\pi_{2}(S^{1}) = 0$, the suspension homomorphism 
$\Sigma_{\ast}\colon \pi_{2}(S^{1}) \to \pi_{3}(S^{2})$ is not an isomorphism.
\end{note}


\begin{corollary}
For $n\geq 1$ the group $\pi_{n+1}(S^{n})$ is cyclic. 
\end{corollary}

\begin{proof}
We have $\pi_{2}(S^{1}) = 0$ and $\pi_{3}(S^{2})\cong \Z$. 
By Corollary \ref{FREUDENTHAL SN THM} the suspension homomorphism 
$\Z\cong \pi_{3}(S^{2}) \to \pi_{4}(S^{3})$ is onto, so $\pi_{4}(S^{3})$ is a cyclic group. 
By the same corollary we have $\pi_{n+1}(S^{n}) \cong \pi_{n+2}(S^{n+1})$ for all 
$n\geq 3$.
\end{proof}







