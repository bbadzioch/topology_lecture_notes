% !TEX root = mth727_lecture_notes.tex


\chapter[All Groups Are Homotopy Groups]{All Groups Are\\ Homotopy Groups}
\chaptermark{All Groups Are Homotopy Groups}
\label{ALL GROUPS}
\thispagestyle{firststyle}


Recall that van Kampen’s Theorem implies that for any group $G$ we can find 
a space $X$ such that $\pi_{1}(X)\cong G$. The goal of this chapter is to extend 
this result to higher homotopy groups. Since all groups $\pi_{n}(X)$ with $n\geq 2$
are abelian (\ref{HIGHER HOMOT GPS ABELIAN THM}), we will show that the following 
holds:

\begin{theorem}
\label{GROUP REALIZATION THM}
For any abelian group $G$ and any $n\geq 2$ there exists a space $X$ such that 
$\pi_{n}(X)\cong G$. Moreover, such space $X$ can be constructed in such way, 
that $X$ is a CW complex and $X^{(n-1)} = \ast$.

\end{theorem}

For every abelian group $G$ there exists an epimorphism 
$\varphi \colon \bigoplus_{i\in I}\Z \to G$ for some set $I$. Indeed, it is enough
to take $I = G$, the set of elements of the group $G$. Then we can define 
$\varphi$ by $\varphi(e_{g}) = g$, where $e_{g}$ is the generator of the copy 
of $\Z\subseteq \bigoplus_{h\in G}\Z$ indexed by $g$. Given such a homomorphism $\varphi$
we get $G \cong \bigoplus_{i\in I} \Z /\ker(\varphi)$.    


Based on this, in order to prove Theorem \ref{GROUP REALIZATION THM} 
it will suffice to show that:
\benu
\item[1)] for any set $I$ and $n\geq 2$ there exists a space $X$ such that 
$\pi_{n}(X)\cong \bigoplus_{i\in I}\Z$. 
\item[2)] for any subgoup $H\subseteq \bigoplus_{i\in I}\Z$ and any $n\geq 2$ there 
exists a space $X$ such that $\pi_{n}(X)\cong \bigoplus_{i\in I}\Z/H$. 
\eenu


\begin{lemma} 
\label{PIN VEE XI EPI LEMMA}
Let $\{(X_{i}, \xov{x}_{i})\}_{i\in I}$ be a family of pointed Hasdorff spaces. 
Let $X = \bigvee_{i\in I} X_{i}$, and let $\ast\in X$ denote the basepoint. For $k\in I$ 
let $r_{k}\colon X \to X_{k}$ be the retraction map. Then for any $n\geq 2$ 
we an epimorphism $\varphi\colon \pi_{n}(X, \ast) \to 
\bigoplus_{i\in I} \pi_{n}(X_{i}, \xov{x}_{i})$
given by $\varphi([\omega]) = \sum_{i\in I} r_{i\ast}([\omega])$. 
\end{lemma}
 
\begin{proof}
For each $k\in I$ let $j_{k}\colon X_{k} \to X$ be the inclusion map. 
We have a homomorphism
\[
\textstyle
\psi := \bigoplus_{i\in I} j_{i\ast}\colon \bigoplus_{i\in I} \pi_{n}(X_{i}, \xov{x}_{i}) \to 
\pi_{n}(X, \ast)
\]
The retractions $r_{i}$ define a map
\[
\textstyle
\varphi := \prod_{i\in I} r_{i\ast}\colon 
\pi_{n}(X, \ast) \to 
\prod_{i\in I}\pi_{n}(X_{i}, \xov{x}_{i})
\]
We claim that $\Im(\varphi) \subseteq \bigoplus_{i\in I} \pi_{n}(X_{i}, \xov{x}_{i}) \subseteq 
\prod_{i\in I}\pi_{n}(X_{i}, \xov{x}_{i})$. Indeed, if 
$\omega\colon (I^{n}, \pint^{n}) \to (X, \ast)$ is a map 
representing an element $[\omega]\in \pi_{n}(X, \ast)$, then, 
by compactness of $I^{n}$, we have $\omega(I^{n})\cap X_{i} \neq \ast$ for finitely many 
$i\in I$ only, and so $r_{i\ast}([\omega]) \neq 0$ for finitely many $i\in I$. 
Thus $\varphi([\omega]) \subseteq \bigoplus_{i\in I} \pi_{n}(X_{i},\xov{x}_{i})$. 
It follows that we can consider $\varphi$ as a homomorphism 
$\pi_{n}(X, \ast)\to \bigoplus_{i\in I}\pi_{n}(X_{i}, \xov{x}_{i})$. 

Since $r_{i}j_{i} = \id_{X_{i}}$ for all $i\in I$, and $r_{i’}j_{i}$ is the constant map 
for all $i\neq i’$, it follows that $\varphi\psi$
is the identity homomorphism, and so $\varphi$ is onto. 
\end{proof}

\begin{note}
In general, the epimorphism $\varphi$ in Lemma \ref{PIN VEE XI EPI LEMMA}
is not an isomorphism. For example, recall (\ref{PIN S1vSm EXAMPLE}) that for $n\geq 2$
we have $\pi_{n}(S^{1}\vee S^{n}) \cong \pi_{n}(\bigvee_{i\in \Z} S^{n})$. By 
Lemma \ref{PIN VEE XI EPI LEMMA} we get an epimorphism
\[
\textstyle
\pi_{n}(S^{1}\vee S^{n}) \cong \pi_{n}(\bigvee_{i\in \Z} S^{n}) \to 
\bigoplus_{i\in \Z}\Z
\]
which shows that the group $\pi_{n}(S^{1}\vee S^{n})$ is not finitely generated. 
Therefore  $\pi_{n}(S^{1}\vee S^{n})\not \cong \pi_{n}(S^{1})\oplus \pi_{n}(S^{n}) \cong \Z$. 
\end{note}




\begin{proposition} 
\label{PIN VEE XI MONO PROP}
Let $\{(X_{i}, \xov{x}_{i})\}_{i\in I}$ be a family of pointed CW-complexes. 
Given $n\geq 1$, assume that each complex $X_{i}$ is $n$-connected. Then the homomorphism 
$\varphi \colon \pi_{m}(\bigvee_{i\in I}X_{i}, \ast) \to 
\bigoplus_{i\in I} \pi_{m}(X_{i}, \xov{x}_{i})$
is an isomorphism for $m\leq 2n$. 
\end{proposition}

\begin{proof}
For each CW complex $X_{i}$ we can assume that $\xov{x}_{i}$ is a $0$-cell of 
$X_{i}$. Also, by Proposition \ref{N SKELETON FOR N CONNECTED PROP} we can assume that 
$X_{i}$ has no other $0$-cells, and no $k$-cells for $k \leq n$
\footnote{This uses the fact that if $X_{i}\simeq X’_{i}$ for all $i\in I$ then 
$\bigvee_{i\in I} X_{i} \simeq \bigvee X’_{i\in I}$. This holds for well-pointed, path connected spaces.}. 


By Proposition \ref{PIN VEE XI MONO PROP} $\varphi$ is onto. It will suffice 
to show that $\ker \varphi = 0$ for $m \leq 2n$.

Assume first, that the set $I$ is finite, so 
$\bigvee_{i\in I}X_{i} = X_{1}\vee \dots \vee X_{k}$ 
for some $k\geq 0$. Take the product $X_{1} \times {\dots}\times X_{k}$. The inclusion 
maps $\psi_{j}\colon X_{j} \to X_{1} \times {\dots}\times X_{k}$ 
given by $\psi_{j}(x) = 
(\xov{x}_{1}, \dots, \xov{x}_{j-1}, x, \xov{x}_{j+1}, \dots \xov{x}_{k})$
define an embedding 
$\psi \colon X_{1} \vee{\dots}\vee X_{k} \to 
X_{1} \times {\dots}\times X_{k}$. 
This gives a commutative diagram:

\begin{equation*}
\begin{tikzpicture}
\matrix (m) 
[matrix of math nodes, row sep=3em, column sep=3em, text height=1.5ex, text depth=0.25ex]
{
\pi_{m}(X_{1} \vee{\dots}\vee X_{k}) & 
\pi_{m}(X_{1} \times {\dots}\times X_{k}) \\
\bigoplus_{j=1}^{k} \pi_{k+1}(X_{i}) & \prod_{j=1}^{k} \pi_{k+1}(X_{i}) \\
};
\path[->, thick, font=\scriptsize]
(m-1-1) 
edge node[auto] {$\psi_{\ast}$} (m-1-2)
edge node[anchor = east] {$\varphi$} (m-2-1)
(m-1-2)
edge node[anchor = east] {$\cong$} (m-2-2)
(m-2-1)
edge node[anchor = south] {$=$} (m-2-2)
; 
\end{tikzpicture}
\end{equation*}
This shows that $\varphi$ is a monomorphism if any only if $\psi_{\ast}$ is one. 
If $X_{1}, \dots, X_{k}$ are finite CW complexes, then the space 
$X_{1}\times \dots \times X_{k}$ also has the structure of a CW complex, with cells 
given by products $e_{1}\times \dots \times e_{k}$ where $e_{i}$ is a cell in $X_{i}$.  
All cells of $X_{1}\times \dots \times X_{k}$ that are not contained 
in $X_{1}\vee \dots \vee X_{k}$ have dimension $2n + 2$ or higher, so 
$X_{1}\vee \dots \vee X_{k}$ is the $(2n + 1)$-skeleton of $X_{1}\times \dots \times X_{k}$. 
Thus, by Proposition \ref{PIN FOR CW SKELETON PROP},  $\psi_{\ast}$ is an isomorphism 
for all $m \leq 2n$. 

Next, assume that the set $I$ is infinite, and let 
$\omega\colon (I^{m}, \pint^{m}) \to (\bigvee_{i\in I} X_{i}, \ast)$ be a map 
such that $\varphi([\omega]) = 0$. 
By compactness of $I^{m}$ we have 
$\omega(I^{m})\cap X_{i}\neq \ast$ for finitely many $i\in I$ only. Thus 
we can consider $\omega$ as a map 
$\omega\colon (I^{m}, \pint^{m}) \to 
(X_{i_{1}} \vee{\dots}\vee X_{i_{k}}, \ast)$ for some $i_{1}, \dots, i_{k} \in I$. 
Since $\varphi([\omega]) = 0$, the homomorphism 
$\pi_{m}(X_{i_{1}} \vee{\dots}\vee X_{i_{k}}) \to \bigoplus_{j=1}^{k} \pi_{k+1}(X_{i_{j}})$
also maps $[\omega]$ to $0$. By the finite case this means that $[\omega] = 0$.
\end{proof}


\begin{note}
The proof of Proposition \ref{PIN VEE XI MONO PROP} uses the fact that if 
$X$ and $Y$ are CW complexes, then $X\times Y$ has the structure of a CW 
complex with cells given by products of cells in $X$ and $Y$. An issue with 
this statement is that the topology induced on $X\times Y$ by this cell structure
(where a set $U\subseteq X\times Y$ is open if and only if its intersection with 
each cell is an open subset of the cell) need not be the same as the product 
topology on $X\times Y$. The topology induced by the cell structure 
on $X\times Y$ is called the compactly generated topology. Let $X\times_{cg} Y$
denote the product taken with this topology, and let $X\times Y$ denote the product 
taken with the product topology. Every open set in $X\times Y$ is also open in 
$X\times_{cg} Y$, so the identity map $\id \colon X\times_{cg} Y \to X\times Y$
is continuous. Moreover, this map induces an isomorphism of homotopy groups 
$\pi_{n}(X\times_{cg}Y)\overset{\cong}{\lra}\pi_{n}(X\times Y)$ for all $n$. 
For this reason this change of topology does not affect the proof of 
Proposition \ref{PIN VEE XI MONO PROP}. 
\end{note}


\begin{corollary}
\label{WEDGE SN PIN COR}
For any set $I$ and any $n\geq 2$ we have an isomorphism 
\[
\textstyle
\pi_{n}(\bigvee_{i\in I} S^{n})\cong \bigoplus_{i\in I}\Z
\]
Moreover, the group $\pi_{n}(\bigvee_{i\in I} S^{n})$ is generated by
elements $[j_{k}]$ for $k\in I$ where $j_{k}\colon S^{n}\hra \bigvee_{i\in I} S^{n}$
is the inclusion of the $k$-th copy of $S^{n}$. 
\end{corollary}


\begin{proposition}
\label{1CONN PIN XCUPE PROP}
Let $(X, x_{0})$ be a simply connected space, and let 
$\varphi_{i}\colon (S^{n}, s_{0}) \to (X, x_{0})$ be maps representing  elements of 
$\pi_{n}(X, x_{0})$ for some $n\geq 2$. Consider the space $Y = X\cup \bigcup_{i} e_{i}^{n+1}$ 
obtained by attaching $(n+1)$-cells to $X$ using $\varphi_{i}$ 
as the attaching maps. If $j\colon X \hra Y$ is the inclusion map, then the induced homomorphism 
\[
j_{\ast}\colon \pi_{k}(X, x_{0}) \to \pi_{k}(Y, x_{0})
\]
is an isomorphism for $k < n$ and an epimorphism for $k=n$. Moreover, 
$\ker(j_{\ast}\colon \pi_{n}(X, x_{0}) \to \pi_{n}(Y, x_{0}))$ is the subgroup of 
$\pi_{n}(X, x_{0})$ generated by the elements $[\varphi_{i}]$. 
\end{proposition}

\begin{proof}
We can consider the pair $(Y, X)$ as a relative CW complex with the $n$-skeleton 
given by $X$. Then $j_{\ast}$ is an isomorphism for $k<n$ and epimorphism for 
$k=n$ by Proposition \ref{PIN FOR CW SKELETON PROP}. Notice that by 
Proposition \ref{N CONNECTED PAIR PROP} this is equivalent to saying that 
the pair $(X, Y)$ is $n$-connected.

It remains to verify the statement about the kernel of $j_{\ast}$ for $k=n$. 
Consider the exact sequence of the pair $(Y, X)$:
\[
\dots \to \pi_{n+1}(Y, X) \overset{\partial}{\lra} 
\pi_{n}(X) \overset{j_{\ast}}{\lra} \pi_{n}(Y) \to \pi_{n}(Y, X) \to \dots
\]  
We have $\ker j_{\ast} = \Im \partial$. Since the pair $(X, Y)$ is 
$n$-connected and, by assumption, the space $X$
is 1-connected, from Theorem \ref{PIK QUOTIENT PROP} we obtain that the quotient
map $q\colon Y \to Y/X$ induces an isomorphism 
\[
q_{\ast}\colon \pi_{n+1}(Y, X) \overset{\cong}{\lra} \pi_{n+1}(Y/X) \cong 
\pi_{n+1}(\textstyle{\bigvee_{i}S^{n+1}}) \cong \oplus_{i}\Z
\]
This implies that $\pi_{n+1}(Y, X)$ is generated by homotopy classes of maps 
$\xov{\varphi}_{i}\colon D^{n+1} \to Y$ which are the characteristic maps 
of the cells $e_{i}^{n+1}$. The boundary homorphism is given by 
$\partial[\xov{\varphi}_{i}] = [\varphi_{i}]$. Therefore $\Im \partial = \ker j_{\ast}$
is the subgroup of $\pi_{n}(X)$ generated by the elements $[\varphi_{i}]$.
\end{proof}


Proposition \ref{1CONN PIN XCUPE PROP} can be generalized to non-simply connected 
spaces as follows. Recall (\ref{PI1 ACTION NN})  that higher homotopy groups admit 
the action of the fundamental group:
\begin{alignat*}{2}
\pi_{1}(X, x_{0}) \times \pi_{n}(X, x_{0}) & \  \to\  & \pi_{n}(X, x_{0}) \\
([\tau]\ ,\  [\omega]) \ \ \ \ \ \ \  &  \ \mapsto\  & [\tau]\odot [\omega] \\
\end{alignat*}
\vskip -10mm
We have: 

\begin{proposition}
Let $(X, x_{0})$ be a space which is connected, locally path connected, and semi-locally 
simply connected. Let 
$\varphi_{i}\colon (S^{n}, s_{0}) \to (X, x_{0})$ be maps representing  elements of 
$\pi_{n}(X, x_{0})$ for some $n\geq 2$. Consider the space $Y = X\cup \bigcup_{i} e_{i}^{n+1}$ 
obtained by attaching $(n+1)$-cells to $X$ using $\varphi_{i}$ 
as the attaching maps. If $j\colon X \hra Y$ is the inclusion map, then the induced homomorphism 
\[
j_{\ast}\colon \pi_{k}(X, x_{0}) \to \pi_{k}(Y, x_{0})
\]
is an isomorphism for $k\leq n$ and an epimorphism for $k=n$. Moreover, 
$\ker(j_{\ast}\colon \pi_{n}(X, x_{0}) \to \pi_{n}(Y, x_{0}))$ is the subgroup of 
$\pi_{n}(X, x_{0})$ generated by the elements 
$[\omega]\odot[\varphi_{i}]$ for all $[\omega]\in \pi_{1}(X, x_{0})$.
\end{proposition}

\begin{proof}
The only non-trivial part is the statement about $\ker j_{\ast}$. The conditions 
on the space $X$ guarantee that it has a universal covering $p_{X}\colon \widetilde{X} \to X$. 
Let $p_{X}^{-1}(x_{0}) = \{\widetilde{x}_{k}\}_{k\in K}$ and let 
$\widetilde{\varphi}_{i, k}\colon S^{n} \to \widetilde{X}$ denote the lift of $\varphi_{i}$
such that $\widetilde{\varphi}_{i, k}(s_{0}) = \widetilde{x}_{k}$. 
Let $\widetilde{Y} = \widetilde{X}\cup \bigcup_{i,j} e^{n+1}_{i, k}$ be the space 
obtained by attaching $(n+1)$-cells to $\widetilde X$ using $\varphi_{i, k}$ as 
attaching maps. The natural map $p_{Y}\colon \widetilde{Y} \to Y$ is a universal covering 
of $Y$. We get a commutative diagram:
\begin{equation*}
\begin{tikzpicture}
\matrix (m) 
[matrix of math nodes, row sep=3em, column sep=3em, text height=1.5ex, text depth=0.25ex]
{
\pi_{n}(\widetilde{X}, \widetilde{x}_{0}) & \pi_{n}(\widetilde{Y}, \widetilde{x}_{0}) \\
\pi_{n}(X, x_{0}) & \pi_{n}(Y, x_{0}) \\
};
\path[->, thick, font=\scriptsize]
(m-1-1) 
edge node[auto] {$\tilde{j}_{\ast}$} (m-1-2)
edge node[anchor = east] {$p_{X\ast}$} node[anchor=west] {$\cong$} (m-2-1)
(m-1-2)
edge node[anchor=  west] {$p_{Y\ast}$} node[anchor=east] {$\cong$} (m-2-2)
(m-2-1)
edge node[anchor=  north] {$j_{\ast}$} (m-2-2)
; 
\end{tikzpicture}
\end{equation*}
where $\tilde{j}\colon \widetilde{X} \to \widetilde{Y}$ is the inclusion and 
$\widetilde{x}_{0}\in p_{X}^{-1}(x_{0})$. Since $p_{X\ast}$ and $p_{Y\ast}$
are isomorphisms (\ref{COVERING PIN PROP}), we obtain that 
$\ker j_{\ast} = p_{X\ast}(\ker \tilde{j}_{\ast})$. 

For each $\widetilde{x}_{k}\in p^{-1}(x_{0})$ let $\widetilde{\omega}_{k}$ be a path 
in $\widetilde X$ such that $\widetilde{\omega}_{k}(0) = \widetilde{x}_{0}$ and 
$\widetilde{\omega}_{k}(1) = \widetilde{x}_{k}$. Then for each 
$[\omega]\in \pi_{1}(X, x_{0})$ we have $[\omega] = [p_{X}\widetilde{\omega}_{k}]$
for some $k$. Let 
$s_{k}\colon  \pi_{n}(\widetilde{X}, \widetilde{x}_{k}) \to 
\pi_{n}(\widetilde{X}, \widetilde{x}_{0})$ be the change of the basepoint 
isomorphism defined by $\widetilde{\omega}_{k}$ (\ref{STAU DEF}). 
Since $\widetilde X$ is simply connected, using Proposition \ref{1CONN PIN XCUPE PROP}
we obtain that $\ker \tilde{j}_{\ast}$ is generated by the elements
$s_{k}[\widetilde{\varphi}_{i, k}]$ for all $i, k$. Thus 
$\ker j_{\ast}$ is generated by elements $p_{X\ast}s_{k}[\widetilde{\varphi}_{i, k}]$. 
It remains to notice that $p_{X\ast}s_{k}[\widetilde{\varphi}_{i, k}] = 
[p_{X}\omega_{k}]\odot [p_{X}\widetilde{\varphi}_{i, k}] = 
[p_{X}\omega_{k}]\odot [\varphi_{i}]$ (exercise).

\end{proof}


\begin{proof}[Proof of Theorem \ref{GROUP REALIZATION THM}]
Given an abelian group $G$ and $n\geq 2$, we can find a set $I$ and 
an epimorphism
\[
\Phi \colon \pi_{n}(\bigvee_{i\in I}S^{n}) \cong \bigoplus_{i\in I}\Z \to G 
\]
Let $\ker \Phi = \{[\varphi_{k}\colon S^{n}\to \bigvee_{i\in I}S^{n}]\}_{k\in K}$, 
and let $X$ be the space obtained by attaching $(n+1)$-cells to $\bigvee_{i\in }S^{n}$
using the maps $\varphi_{i}$. By Proposition \ref{1CONN PIN XCUPE PROP} we obtain
$\pi_{n}(X)\cong \pi_{n}(\bigvee_{i\in I}S^{n})/\ker \Phi \cong G$.  
\end{proof}

\begin{definition}
\label{K(G, N) DEF}
Given a group $G$ and an integer $n\geq 1$, an \emph{Eilenberg-MacLane space} 
of the type $K(G, n)$ is a path connnected CW complex $X$ such that 
\[
\pi_{i}(X) \cong 
\begin{cases}
G & \text{if } i = n\\
0 & \text{otherwise}
\end{cases}
\]
\end{definition}

\begin{note}
Eilenberg-MacLane spaces are not uniquely defined, but as we will see later 
(\ref{EM SPACES HOMOT UNIQUE PROP}), 
they are unique up to homotopy equivalence. By abuse of notation we will 
write $X=K(G, n)$ to indicate that $X$ has the type of $K(G, n)$.
\end{note}

\begin{example}
$S^{1} = K(\Z, 1)$. 
\end{example}

\begin{example}
Recall that the $n$-dimensional real projective space $\RP^{n}$ is the 
quotient space of $S^{n}$ obtained by identifying antipodal points: 
$\RP^{n} = S^{n}/\sim$ where $x\sim -x$ for all $x\in S^{n}$. The quotient 
map $q\colon S^{n} \to \RP^{n}$ is the 2-fold universal cover of $\RP^{n}$. 
It follows that 
\[
\pi_{i}(\RP^{n})
\cong
\begin{cases}
\Z/2 & \text{if } i = 1\\
\pi_{i}(S^{n}) & i\geq 2
\end{cases}
\]
Embeddings of spheres $S^{1} \hra S^{2} \hra \dots$ induce embeddings of 
projective spaces $\RP^{1} \hra \RP^{2} \hra \dots$. Take 
$S^{\infty} = \bigcup_{n=1}^{\infty} S^{n}$ and 
$\RP^{\infty} = \bigcup_{n=1}^{\infty} \RP^{n}$. The quotient map 
$q\colon S^{\infty} \to \RP^{\infty}$ is a 2-fold universal covering of $\RP^{\infty}$. 
Since $S^{\infty}$ is a contractible space (\ref{SINFTY CONTRACTIBLE PROP}), we obtain 
\[
\pi_{i}(\RP^{\infty})
\cong
\begin{cases}
\Z/2 & \text{if } i = 1\\
0 & i \text{if } \geq 2
\end{cases}
\]
Therefore $\RP^{\infty} = K(\Z/2, 1)$. 
\end{example}

\begin{example}
Recall (\ref{CPN BUNDLE EXAMPLE}) that for a complex projective space the quotient 
map $p\colon S^{2n+1} \to \CP^{n}$ is a Serre fibration with the fiber $S^{1}$. The long 
exact sequence of this fibration gives 
\[
\pi_{i}(\CP^{n})
\cong
\begin{cases}
0 & \text{if } i = 1\\
\Z & \text{if } i = 2\\
\pi_{i}(S^{2n+1}) & \text{if } i \geq 3 \\
\end{cases}
\]
\end{example}
The embedding maps $S^{3} \hra S^{5} \hra \dots$ induce embeddings
$\CP^{1}\hra \CP^{2} \hra \dots$. We again have $S^{\infty} = \bigcup_{n=1}^{\infty}S^{2n+1}$. 
Also, define $\CP^{\infty} = \bigcup_{n=1}^{\infty}\CP^{n}$. The map 
$p\colon S^{\infty} \to \CP^{\infty}$ is again a Serre fibration with fiber $S^{1}$. 
Since $S^{\infty}$ is contractible, the long exact sequence of this fibration gives
\[
\pi_{i}(\CP^{\infty})
\cong
\begin{cases}
0 & \text{if } i = 1\\
\Z & \text{if } i = 2\\
0 &  \text{if } i\geq 3 \\
\end{cases}
\]
Thus $\CP^{\infty} = K(\Z, 2)$.



\begin{proposition}
\label{EM SPACES EXIST PROP}
For any $n\geq 1$ and any group $G$ (abelian if $n\geq 2$) there exists 
an Eilenberg-MacLane space $K(G, n)$. Moreover, it is possible to construct 
such space so that $K(G, n)^{(n-1)} = \ast$. 
\end{proposition}





\begin{proof}
By Theorem \ref{GROUP REALIZATION THM}, if $n\geq 2$ then we can find a path 
connected CW complex $(X_{n}, x_{0})$ such that $X_{n}^{(n-1)} = \ast$ and
\[
\pi_{i}(X_{n}, x_{0}) \cong 
\begin{cases}
G & \text{if } i = n\\
0 & \text{if } i < n\\
\end{cases}
\]
For $n=1$ such CW complex can be constructed using van Kampen’s theorem.
Let $X_{n+1}$ be the space obtained by attaching an $(n+2)$-cells to $X_{n}$
using all possible maps $(S^{n+1}, s_{0}) \to (X_{n}, x_{0})$. 
Then $X_{n}\subseteq X_{n+1}$, and using Proposition \ref{PIN FOR CW SKELETON PROP} 
we obtain 
\[
\pi_{i}(X_{n+1}, x_{0}) \cong 
\begin{cases}
0 & \text{if } i = n+1\\
G & \text{if } i = n\\
0 & \text{if } i < n\\
\end{cases}
\]
In the same way, for any $m > n+1$ we can can inductively construct a space $X_{m}$
such that $X_{m}$ is obtained by attaching $(m+1)$-cells to $X_{m-1}$ and  
\[
\pi_{i}(X_{m}, x_{0}) \cong 
\begin{cases}
0 & \text{if } n < i \leq m\\
G & \text{if } i = n\\
0 & \text{if } i < n\\
\end{cases}
\]
Then we can take $K(G, n) = \bigcup_{m=n}^{\infty} X_{m}$.
\end{proof}

\begin{corollary}
For any sequence of groups $G_{1}, G_{2}, \dots$ such that $G_{i}$ is abelian for 
$i\geq 2$, there exists a path connected CW complex $X$ such that 
$\pi_{i}(X) \cong G_{i}$ for all $i\geq 1$. 

\end{corollary}

\begin{proof}
Take $X = \prod_{i=1}^{\infty} K(G_{i}, i)$. 
\end{proof}



