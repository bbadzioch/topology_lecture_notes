% !TEX root = mth727_lecture_notes.tex


\chapter[Fibrations]{Fibrations}
\chaptermark{Fibrations}
\label{FIBRATIONS CHAPTER}
\thispagestyle{firststyle}


\begin{definition}
\label{HOMOTOPY LIFTING PROPERTY DEF}
A map $p\colon E \to B$ has the \emph{homotopy lifting property} for a space $X$
if for any commutative diagram of the form 
\begin{equation*}
\begin{tikzpicture}
\matrix (m) 
[matrix of math nodes, row sep=3em, column sep=3em, text height=1.5ex, text depth=0.25ex]
{
X \times \{0\} & E \\
X \times [0, 1] &  B \\
};
\path[->, thick, font=\scriptsize]
(m-1-2) 
edge node[anchor = west] {$p$} (m-2-2)
(m-1-1) 
edge node[above] {$\xov{f}$} (m-1-2)
(m-2-1) 
edge node[below] {$h$} (m-2-2)
;
\path[right hook-latex, thick, font=\scriptsize]
(m-1-1) 
edge node[left] {$i$} (m-2-1);
\path[dashed, ->,  thick, font=\scriptsize]
(m-2-1) 
edge node[anchor=south east] {$\xov{h}$} (m-1-2);
\end{tikzpicture}
\end{equation*}
there exists a map $\xov{h}\colon X\times [0, 1] \to E$ such that 
$\xov{h}i = \xov{f}$ and $p\xov{h} = h$.
\end{definition}

In the setting of Definition \ref{HOMOTOPY LIFTING PROPERTY DEF} we will say 
that $\xov{h}$ is a lift of $h$ beginning at $\xov{f}$.


\begin{definition}
A map $p\colon E \to B$ is 
\benu
\item[\textbullet] a \emph{Hurewicz fibration} if it has the homotopy lifting 
property for any space $X$. 
\item[\textbullet] a \emph{Serre fibration} if it has the homotopy lifting 
property for any CW complex $X$. 
\eenu
\end{definition}

\begin{note}
Every Hurewicz fibration is a Serre fibration.
\end{note}

\begin{example}
\label{PRODUCT FIBRATION EXAMPLE}
For any spaces $B, F$ the projection map $\pr_{B} \colon B\times F \to B$ is a Hurewicz 
fibration. Indeed, assume that we have a commutative diagram
\begin{equation*}
\begin{tikzpicture}
\matrix (m) 
[matrix of math nodes, row sep=3em, column sep=3em, text height=1.5ex, text depth=0.25ex]
{
X \times \{0\} & B\times F \\
X \times [0, 1] &  B \\
};
\path[->, thick, font=\scriptsize]
(m-1-2) 
edge node[anchor = west] {$\pr_{B}$} (m-2-2)
(m-1-1) 
edge node[above] {$\xov{f}$} (m-1-2)
(m-2-1) 
edge node[below] {$h$} (m-2-2)
;
\path[right hook-latex, thick, font=\scriptsize]
(m-1-1) 
edge node[left] {$i$} (m-2-1);
\end{tikzpicture}
\end{equation*}
\ 
\vskip -5mm
\ 
Let $\pr_{F}\colon B\times F \to F$ be the projection onto $F$.
We can define $\xov{h}\colon X\times [0, 1] \to B\times F$ by 
\[
\xov{h}(x, t) = (h(x, t),\  \pr_{F}\xov{f}(x, 0))
\]
\end{example}


\begin{example}
Every covering map $p\colon E\to B$ is a Hurewicz fibration.
\end{example}

%
%\begin{proposition}
%\label{HOMOTOPY LIFTING INVARIANCE PROP}
%Let $h, h’\colon X\times [0, 1] \to B$ be two homotopies between maps 
%$f, g\colon X \to B$. Assume that these homotopies are themselves homotopic 
%relative endpoints. That is, there exists 
%a map 
%\[
%H\colon X\times [0,1] \times [0, 1] \to B
%\] 
%such that $H|_{X\times [0, 1]\times \{0\}} = h$,  $H|_{X\times [0, 1]\times \{1\}} = h’$,
%and for each $t\in [0, 1]$ the map  $H|_{X\times [0, 1]\times \{t\}}$ is a homotopy 
%between $f$ and $g$. Assume also that we have commutative diagrams
%
%\begin{equation*}
%\begin{tikzpicture}
%\begin{scope}
%\matrix (m) 
%[matrix of math nodes, row sep=3em, column sep=3em, text height=1.5ex, text depth=0.25ex]
%{
%X \times \{0\} & E \\
%X \times [0, 1] &  B \\
%};
%\path[->, thick, font=\scriptsize]
%(m-1-2) 
%edge node[anchor = west] {$p$} (m-2-2)
%(m-1-1) 
%edge node[above] {$\xov{f}$} (m-1-2)
%(m-2-1) 
%edge node[below] {$h$} (m-2-2)
%;
%\path[right hook-latex, thick, font=\scriptsize]
%(m-1-1) 
%edge node[left] {$i$} (m-2-1);
%\path[dashed, ->,  thick, font=\scriptsize]
%(m-2-1) 
%edge node[anchor=south east] {$\xov{h}$} (m-1-2);
%\end{scope}
%
%\begin{scope}[xshift=45mm]
%\matrix (m) 
%[matrix of math nodes, row sep=3em, column sep=3em, text height=1.5ex, text depth=0.25ex]
%{
%X \times \{0\} & E \\
%X \times [0, 1] &  B \\
%};
%\path[->, thick, font=\scriptsize]
%(m-1-2) 
%edge node[anchor = west] {$p$} (m-2-2)
%(m-1-1) 
%edge node[above] {$\xov{f}$} (m-1-2)
%(m-2-1) 
%edge node[below] {$h’$} (m-2-2)
%;
%\path[right hook-latex, thick, font=\scriptsize]
%(m-1-1) 
%edge node[left] {$i$} (m-2-1);
%\path[dashed, ->,  thick, font=\scriptsize]
%(m-2-1) 
%edge node[anchor=south east] {$\xov{h’}$} (m-1-2);
%\end{scope}
%\end{tikzpicture}
%\end{equation*}
%If $p$ is a Hurewicz fibration then the maps $\xov{h}_{1}, \xov{h’}_{1}\colon X \to E$
%are homotopic via homotopy $\varphi\colon X\times [0, 1] \to E$ such that 
%$p\varphi_{t} = g$ for all $t\in [0, 1]$. 
%The same holds if $p$ is a Serre fibration and $X$ is a CW complex. 
%\end{proposition}
%
%\begin{proof}
%Exercise.
%\end{proof}
%
%
%
%\begin{definition}
%$p\colon E \to B$ be a Hurewicz or Serre fibration. 
%For $b\in B$ the space $p^{-1}(b)\subseteq E$ is called the \emph{fiber} of $p$ over $b$. 
%\end{definition}
%
%
%\begin{proposition}
%Let $p\colon E \to B$ be a Hurewicz fibration. 
%If $b_{0}, b_{1}$ are points in the same path connected component of 
%$B$, then $p^{-1}(b_{0}) \simeq p^{-1}(b_{1})$. 
%\end{proposition}
%
%\begin{proof}
%Let $\tau\colon [0, 1]\to B$ be a path such that $\tau(0) = b_{0}$ and $\tau(1) = b_{1}$. 
%We have commutative diagrams
%\begin{equation*}
%\begin{tikzpicture}
%\begin{scope}
%\matrix (m) 
%[matrix of math nodes, row sep=3em, column sep=3em, text height=1.5ex, text depth=0.25ex]
%{
%p^{-1}(b_{0}) \times \{0\} & E \\
%p^{-1}(b_{0}) \times [0, 1] &  B \\
%};
%\path[->, thick, font=\scriptsize]
%(m-1-2) 
%edge node[anchor = west] {$p$} (m-2-2)
%(m-1-1) 
%edge node[above] {$\xov{f}$} (m-1-2)
%(m-2-1) 
%edge node[below] {$h$} (m-2-2)
%;
%\path[right hook-latex, thick, font=\scriptsize]
%(m-1-1) 
%edge  (m-2-1);
%\path[dashed, ->,  thick, font=\scriptsize]
%(m-2-1) 
%edge node[anchor=south east] {$\xov{h}$} (m-1-2);
%\end{scope}
%
%\begin{scope}[xshift=55mm]
%\matrix (m) 
%[matrix of math nodes, row sep=3em, column sep=3em, text height=1.5ex, text depth=0.25ex]
%{
%p^{-1}(b_{1}) \times \{0\} & E \\
%p^{-1}(b_{1}) \times [0, 1] &  B \\
%};
%\path[->, thick, font=\scriptsize]
%(m-1-2) 
%edge node[anchor = west] {$p$} (m-2-2)
%(m-1-1) 
%edge node[above] {$\xov{f’}$} (m-1-2)
%(m-2-1) 
%edge node[below] {$h’$} (m-2-2)
%;
%\path[right hook-latex, thick, font=\scriptsize]
%(m-1-1) 
%edge (m-2-1);
%\path[dashed, ->,  thick, font=\scriptsize]
%(m-2-1) 
%edge node[anchor=south east] {$\xov{h’}$} (m-1-2);
%\end{scope}
%\end{tikzpicture}
%\end{equation*}
%where $\xov{f}(x, 0) = x$, and $h(x, t) = \tau(t)$, 
%$\xov{f’}(x, 0) = x$, and $h’(x, t) = \tau(1-t)$. 
%This gives maps $\xov{h}_{1}\colon p^{-1}(b_{0}) \to p^{-1}(b_{1})$ and
%$\xov{h’}_{1}\colon p^{-1}(b_{1}) \to p^{-1}(b_{1})$. 
%Using Proposition \ref{HOMOTOPY LIFTING INVARIANCE PROP} one can show that 
%these maps are inverse homotopy equivalences.
%\end{proof}



\begin{definition}
Let $A\subseteq X$. A map $p\colon E \to B$ has the 
\emph{relative homotopy lifting property} for the pair  $(X, A)$
if for any commutative diagram of the form 
\begin{equation*}
\begin{tikzpicture}
\matrix (m) 
[matrix of math nodes, row sep=3em, column sep=3em, text height=1.5ex, text depth=0.25ex]
{
X \times \{0\} \cup A\times [0, 1] & E \\
X \times [0, 1] &  B \\
};
\path[->, thick, font=\scriptsize]
(m-1-2) 
edge node[anchor = west] {$p$} (m-2-2)
(m-1-1) 
edge node[above] {$\xov{f}$} (m-1-2)
(m-2-1) 
edge node[below] {$h$} (m-2-2)
;
\path[right hook-latex, thick, font=\scriptsize]
(m-1-1) 
edge node[left] {$i$} (m-2-1);
\path[dashed, ->,  thick, font=\scriptsize]
(m-2-1) 
edge node[anchor=south east, pos=0.45] {$\xov{h}$} (m-1-2);
\end{tikzpicture}
\end{equation*}

there exists a map $\xov{h}\colon X\times [0, 1] \to E$ such that 
$\xov{h}i = \xov{f}$ and $p\xov{h} = h$.
\end{definition}

\begin{theorem}
\label{SERRE FIBR EQUIV COND THM}
Let $p\colon E\to B$ be a map. The following conditions are equivalent:
\benu
\item[1)] $p$ is a Serre fibration;
\item[2)] $p$ has the homotopy lifting property for $D^{n}$ for all $n\geq0$;
\item[3)] $p$ has the relative homotopy lifting property for $(D^{n}, S^{n-1})$ 
for all $n\geq0$;
\item[4)] $p$ has the relative homotopy lifting property for all relative 
CW-complexes $(X, A)$.  
\eenu
\end{theorem}

\begin{proof}
1) $\Ra$ 2) Obvious.

2) $\Ra$ 3) Assume that we have a diagram 
\begin{equation*}
\begin{tikzpicture}
\matrix (m) 
[matrix of math nodes, row sep=3em, column sep=3em, text height=1.5ex, text depth=0.25ex]
{
D^{n} \times \{0\} \cup S^{n-1}\times [0, 1] & E \\
D^{n} \times [0, 1] &  B \\
};
\path[->, thick, font=\scriptsize]
(m-1-2) 
edge node[anchor = west] {$p$} (m-2-2)
(m-1-1) 
edge node[above] {$\xov{f}$} (m-1-2)
(m-2-1) 
edge node[below] {$h$} (m-2-2)
;
\path[right hook-latex, thick, font=\scriptsize]
(m-1-1) 
edge  (m-2-1);
\path[dashed, ->,  thick, font=\scriptsize]
(m-2-1) 
edge node[anchor=south east, pos=0.45] {$\xov{h}$} (m-1-2);
\end{tikzpicture}
\end{equation*}
We want to show that the map $\xov h$ exists.

We can construct a homeomorphism 
$\varphi\colon D^{n}\times [0, 1] \to D^{n}\times [0, 1]$ such that 
$\varphi(D^{n}\times\{0\}) = D^{n}\times\{0\}\cup S^{n-1}\times [0, 1]$. 
This gives a commutative diagram
\begin{equation*}
\begin{tikzpicture}
\matrix (m) 
[matrix of math nodes, row sep=3em, column sep=3em, text height=1.5ex, text depth=0.25ex]
{
D^{n} \times \{0\}  &   D^{n}\times\{0\}\cup S^{n-1}\times [0, 1] & E \\
D^{n} \times [0, 1] &  D^{n} \times [0, 1] & B \\
};
\path[->, thick, font=\scriptsize]
(m-1-1)
edge node[above] {$\varphi$} (m-1-2)
(m-1-3) 
edge node[anchor = west] {$p$} (m-2-3)
(m-1-2) 
edge node[above] {$\xov{f}$} (m-1-3)
(m-2-1)
edge node[below] {$\varphi$} (m-2-2)
(m-2-2) 
edge node[below] {$h$} (m-2-3)
;
\path[right hook-latex, thick, font=\scriptsize]
(m-1-1)
edge (m-2-1)
(m-1-2) 
edge (m-2-2);
\path[dashed, ->,  thick, font=\scriptsize]
(m-2-1) 
edge node[anchor=south east, pos=0.3] {$h’$} (m-1-3);
\end{tikzpicture}
\end{equation*}
The map $h’\colon D^{n}\times [0, 1] \to E$ exists by 2). Then we can take 
$\xov{h} = h’\varphi^{-1}$.

3) $\Ra$ 4) Let $(X, A)$ be a relative complex, and assume that we have a commutative 
diagram
\begin{equation*}
\begin{tikzpicture}
\matrix (m) 
[matrix of math nodes, row sep=3em, column sep=3em, text height=1.5ex, text depth=0.25ex]
{
X \times \{0\} \cup A\times [0, 1] & E \\
X \times [0, 1] &  B \\
};
\path[->, thick, font=\scriptsize]
(m-1-2) 
edge node[anchor = west] {$p$} (m-2-2)
(m-1-1) 
edge node[above] {$\xov{f}$} (m-1-2)
(m-2-1) 
edge node[below] {$h$} (m-2-2)
;
\path[right hook-latex, thick, font=\scriptsize]
(m-1-1) 
edge (m-2-1);
\path[dashed, ->,  thick, font=\scriptsize]
(m-2-1) 
edge node[anchor=south east, pos=0.45] {$\xov{h}$} (m-1-2);
\end{tikzpicture}
\end{equation*}
We want to show that the map $\xov h$ exists.

Assume that $X$ is obtained by attaching a single $n$-dimensional 
cell $e^{n}$ to $A$ using an attaching map $\varphi\colon S^{n-1}\to A$, i.e. 
$X = A\cup_{\varphi} e^{n}$. Let $\xov{\varphi}\colon D^{n}\to X$ be the characteristic 
map of $e^{n}$ (\ref{CW TERMINOLOGY NN}).
Then the above diagram can be extended as follow: 
\begin{equation*}
\begin{tikzpicture}
\matrix (m) 
[matrix of math nodes, row sep=3em, column sep=3em, text height=1.5ex, text depth=0.25ex]
{
D^{n}\times \{0\}\cup S^{n-1}\times[0, 1]  &[20mm]  X\times\{0\}\cup A\times [0, 1] & E \\
D^{n} \times [0, 1] & X \times [0, 1] & B \\
};
\path[->, thick, font=\scriptsize]
(m-1-1)
edge node[above] {$\xov{\varphi}\times \{0\}\cup \varphi\times [0, 1]$} (m-1-2)
(m-1-3) 
edge node[anchor = west] {$p$} (m-2-3)
(m-1-2) 
edge node[above] {$\xov{f}$} (m-1-3)
(m-2-1)
edge node[below] {$\xov{\varphi}\times [0, 1]$} (m-2-2)
(m-2-2) 
edge node[below] {$h$} (m-2-3)
;
\path[right hook-latex, thick, font=\scriptsize]
(m-1-1)
edge (m-2-1)
(m-1-2) 
edge (m-2-2);
\path[dashed, ->,  thick, font=\scriptsize]
(m-2-1) 
edge node[anchor=south east, pos=0.3] {$h’$} (m-1-3);
\end{tikzpicture}
\end{equation*}
The map $h’$ exists by 3). Since  $X\times [0, 1]$ is a quotient space of 
$A\times [0, 1] \sqcup D^{n} \times [0, 1]$, 
the map 
\[
\xov{f} \sqcup h’ \colon A\times [0, 1] \sqcup D^{n} \times [0, 1] \to E
\] 
defines the desired map $\xov{h}\colon X\times [0, 1] \to E$. 
The general statement can be obtained from here by induction with respect to 
cell attachments. 



4) $\Ra$ 1) Let $X$ be a CW complex and $A=\varnothing$. Then the relative lifting 
property for $(X, A)$ is the same as the lifting property for $X$. 
\end{proof}

\begin{note}
\label{REL HOMOT LIFT FOR CUBES NOTE}
Property 3) in Theorem \ref{SERRE FIBR EQUIV COND THM} can be equivalently 
stated as follows. Given a cube $I^{n}$, let $K$ be a subset of $\pint^{n}$ 
consisting of all but one face of $I^{n}$. Then for any commutative diagram 
\begin{equation*}
\begin{tikzpicture}
\matrix (m) 
[matrix of math nodes, row sep=3em, column sep=3em, text height=1.5ex, text depth=0.25ex]
{
K & E \\
I^{n} &  B \\
};
\path[->, thick, font=\scriptsize]
(m-1-2) 
edge node[anchor = west] {$p$} (m-2-2)
(m-1-1) 
edge node[above] {$\xov{f}$} (m-1-2)
(m-2-1) 
edge node[below] {$h$} (m-2-2)
;
\path[right hook-latex, thick, font=\scriptsize]
(m-1-1) 
edge (m-2-1);
\path[dashed, ->,  thick, font=\scriptsize]
(m-2-1) 
edge node[anchor=south east] {$\xov{h}$} (m-1-2);
\end{tikzpicture}
\end{equation*}
there exists a map $\xov{h}\colon I^{n} \to E$ such that 
this diagram commutes.
\end{note}


\begin{lemma}
\label{REL HOMOT ISO FIBER HOMOT LEMMA}
Let $p\colon E\to B$ be a Serre fibration. Let $e_{0}\in E$ and $b_{0}\in B$
be points such that $p(e_{0}) = b_{0}$, and let $F = p^{-1}(b_{0})$. For any $n\geq 1$
the map $p\colon (E, F, e_{0}) \to (B, b_{0}, b_{0})$ induces an isomorphism 
of homotopy groups 
\[
p_{\ast} \colon \pi_{n}(E, F, e_{0}) \lra \pi_{n}(B, b_{0}, b_{0}) = \pi_{n}(B, b_{0})
\]
\end{lemma}


\begin{proof}
To check that $p_{\ast}\colon \pi_{n}(E, F, e_{0}) \to \pi_{n}(B, b_{0})$ is 
onto, take a map $\omega \colon (I^{n}, \pint^{n}) \to (B, b_{0})$. By the 
relative homotopy lifting property for $(I^{n-1}, \pint^{n-1})$, we can find 
a map $\xov{\omega}\colon  I^{n} \to E$
such that $\xov{\omega}(J^{n-1}) = e_{0}$ and $p\xov{\omega} = \omega$.
\begin{equation*}
\begin{tikzpicture}
\matrix (m) 
[matrix of math nodes, row sep=3em, column sep=3em, text height=1.5ex, text depth=0.25ex]
{
J^{n-1} = I^{n-1} \times \{1\} \cup \pint^{n-1}\times [0, 1] & E \\
I^{n} = I^{n-1} \times [0, 1] &  B \\
};
\path[->, thick, font=\scriptsize]
(m-1-2) 
edge node[anchor = west] {$p$} (m-2-2)
(m-1-1) 
edge node[above] {$c_{e_{0}}$} (m-1-2)
(m-2-1) 
edge node[below] {$\omega$} (m-2-2)
;
\path[right hook-latex, thick, font=\scriptsize]
(m-1-1) 
edge (m-2-1);
\path[dashed, ->,  thick, font=\scriptsize]
(m-2-1) 
edge node[anchor=south east, pos=0.45] {$\xov{\omega}$} (m-1-2);
\end{tikzpicture}
\end{equation*}
Then $\xov{\omega}$ represents and element of $\pi_{n}(E, F, e_{0})$, 
and $p_{\ast}([\xov{\omega}]) = [\omega]$.

It remains to verify that $p_{\ast}$ is 1-1. Assume that 
$\omega_{0}, \omega_{1}\colon (I^{n}, \pint^{n}, J^{n-1})\to (E, F, e_{0})$
be maps such that $p_{\ast}([\omega_{0}]) = p_{\ast}([\omega_{1}])$. 
Then there exists a homotopy $h\colon I^{n}\times I \to B$ with
$h_{0} = p\omega_{0}$ and $h_{1} = p\omega_{1}$, 
and such that $h(\pint^{n}\times [0, 1]) = b_{0}$.
Take the subset $K\subseteq I^{n}\times I$ given by 
\[
K = I^{n}\times \{0, 1\} \cup J^{n-1}\times [0, 1]
\]
Notice that $K$ consists of all faces of the cube $I^{n+1} = I^{n}\times [0, 1]$, 
except for the face $I^{n-1}\times \{0\} \times [0, 1]$. 
Define $\xov{f}\colon X \to E$ by 
\[
\xov{f}(x) = 
\begin{cases}
\omega_{0}(x) & \text{for $x\in I^{n}\times \{0\}$} \\
\omega_{1}(x) & \text{for $x\in I^{n}\times \{1\}$} \\
e_{0} & \text{for $x\in J^{n-1}\times [0, 1]$} \\
\end{cases}
\]
By (\ref{REL HOMOT LIFT FOR CUBES NOTE}) we can find a map 
$\xov{h}\colon I^{n+1} \to E$ such that $\xov{h}|_{K} = \xov{f}$
and $p\xov{h} = h$. Such map $\xov{h}$ gives a homotopy between 
$\omega_{0}$ and $\omega_{1}$. Therefore $[\omega_{0}] = [\omega_{1}]$
in $\pi_{n}(E, F, e_{0})$. 
\end{proof}



\begin{theorem}
\label{EXACT SEQ FIBRATION THM}
Let $p\colon E\to B$ be a Serre fibration. Let $e_{0}\in E$ and $b_{0}\in B$
be  such that $p(e_{0}) = b_{0}$, and let $F = p^{-1}(b_{0})$. Let 
$i\colon F \to E$ be the inclusion map. For any $n\geq 1$ define a homomorphism 
$\partial \colon \pi_{n}(B, b_{0})\to \pi_{n-1}(F, e_{0})$ given by 
\[
\partial \colon \pi_{n}(B, b_{0}) \overset{p_{\ast}^{-1}}{\lra} \pi_{n}(E, F, e_{0}) 
\overset{\partial}{\lra} \pi_{n-1}(F, e_{0})
\] 
Then the following sequence is exact: 
\begin{multline*}
{\dots}\lra \pi_{n+1}(B, b_{0}) \overset{\partial}{\lra} 
\pi_{n}(F, e_{0}) \overset{i_{\ast}}{\lra}
\pi_{n}(E, e_{0}) \overset{p_{\ast}}{\lra}
\pi_{n}(B, b_{0}) \overset{\partial}{\lra} {\dots} \\
{\dots} \overset{p_{\ast}}{\lra} \pi_{1}(B, b_{0}) \overset{\partial}{\lra} 
\pi_{0}(F, e_{0}) \overset{i_{\ast}}{\lra} 
\pi_{0}(E, e_{0}) \overset{p_{\ast}}{\lra} 
\pi_{0}(B, b_{0}) 
\end{multline*}
\end{theorem}


\begin{proof}
Exactness in almost all places follows from the exactness of the 
long exact sequence of the triple $(E, F, e_{0})$, and the commutativity of the 
following diagram:
\begin{equation*}
\begin{tikzpicture}
\matrix (m) 
[matrix of math nodes, row sep=3em, column sep=2em, text height=1.5ex, text depth=0.25ex]
{
{\dots} & \pi_{n}(F, e_{0}) & \pi_{n}(E, e_{0}) & \pi_{n}(B, b_{0}) 
& \pi_{n-1}(F, e_{0}) &  \pi_{n-1}(E, e_{0}) & {\dots} \\
{\dots} & \pi_{n}(F, e_{0}) & \pi_{n}(E, e_{0}) & \pi_{n}(E, F, e_{0}) 
& \pi_{n-1}(F, e_{0}) &  \pi_{n-1}(E, e_{0}) & {\dots} \\
};
\path[->, thick, font=\scriptsize]
(m-1-1)
edge (m-1-2)
(m-1-2) 
edge node[above] {$i_{\ast}$} (m-1-3)
(m-1-3) 
edge node[above] {$p_{\ast}$} (m-1-4)
(m-1-4) 
edge node[above] {$\partial$} (m-1-5)
(m-1-5) 
edge node[above] {$i_{\ast}$} (m-1-6)
(m-1-6) 
edge (m-1-7)

(m-2-1)
edge (m-2-2)
(m-2-2) 
edge node[below] {$i_{\ast}$} (m-2-3)
edge node[pos=0.45, anchor = south, rotate=90] {$=$} (m-1-2)
(m-2-3) 
edge node[below] {$j_{\ast}$} (m-2-4)
edge node[pos=0.45, anchor = south, rotate=90] {$=$} (m-1-3)
(m-2-4) 
edge node[below] {$\partial$} (m-2-5)
edge node[left] {$p_{\ast}$} node[right] {$\cong$}(m-1-4)
(m-2-5) 
edge node[below] {$i_{\ast}$} (m-2-6)
edge node[pos=0.45, anchor = south, rotate=90] {$=$} (m-1-5)
(m-2-6) 
edge (m-2-7)
edge node[pos=0.45, anchor = south, rotate=90] {$=$} (m-1-6)
;
\end{tikzpicture}
\end{equation*}
\ 
\vskip -5mm
\ 
Since the long exact sequence of $(E, F, x_{0})$ ends at $\pi_{0}(E, e_{0})$, exactness
of the sequence 
\[
\pi_{0}(F, e_{0}) \overset{i_{\ast}}{\lra} 
\pi_{0}(E, e_{0}) \overset{p_{\ast}}{\lra} 
\pi_{0}(B, b_{0}) 
\]
needs to be checked separately (exercise).
\end{proof}


\begin{note}
The map $\partial\colon\pi_{n}(B, b_{0})\to \pi_{n-1}(F, e_{0})$ can be described 
directly as follows. Take a map $\omega\colon (I^{n}, \pint^{n}) \to (B, x_{0})$.
Since $p\colon E\to B$ is a Serre fibration, by (\ref{REL HOMOT LIFT FOR CUBES NOTE})
we can find $\xov{\omega}\colon I^{n}\to E$ 
such that $p\xov{\omega} = \omega$, and $\xov{\omega}(J^{n-1}) = e_{0}$. 
Then $\partial([\omega]) = [\xov{\omega}|_{I^{n-1}\times \{0\}}]$.

\begin{equation*}
\begin{tikzpicture}[scale=1.3]
\filldraw[line width=2pt, fill=mygray1] (0, 0) rectangle (1, 1) node[pos=0.5] {\small $I^{n}$};
\path (0, 0) --  (1, 0) node[midway,below] {\small $I^{n-1}\times \{0\}$};

\draw[red, line width=2pt, cap=rect] (0, 0) -- (0, 1) -- node[above] {\small $J^{n-1}$}
(1, 1) -- (1, 0);


\filldraw[line width=2pt, fill=mygray1] 
(3, 1.2) rectangle (4, 2.2);
\path (3, 1.2) --  (4, 1.2) node[midway,below] {\small $F$};
\draw[red, line width=2pt, cap=rect] 
(3, 1.2) -- node[left] {\small $e_{0}$} 
(3, 2.2) -- node[above] {\small $e_{0}$} 
(4, 2.2) -- node[right] {\small $e_{0}$} (4, 1.2);


\filldraw[line width=2pt, fill=mygray1] 
(3, -1.2) rectangle (4, -0.2);
\path (3, -1.2) --  (4, -1.2) node[midway,below] {\small $b_{0}$};
\draw[red, line width=2pt, cap=rect] 
(3, -1.2) -- node[left] {\small $b_{0}$} 
(3, -0.2) -- node[above] {\small $b_{0}$}
(4, -0.2) -- node[right] {\small $b_{0}$} (4, -1.2);

\draw[thick, ->, >=latex] 
(1.4, 0.8) -- node[anchor=south east] {\small $\xov{\omega}$} (2.6, 1.4);
\draw[thick, ->, >=latex] 
(1.4, 0.2) -- node[anchor=north east] {\small $\omega$} (2.6, -0.4);
\draw[thick, ->, >=latex] 
(3.5, 0.7) -- node[anchor=west] {\small $p$} (3.5, 0.3);		

\end{tikzpicture}
\end{equation*}

\end{note}






\begin{example}
Consider the product fibration $\pr_{B}\colon B\times F \to B$. For $b_{0}\in B$
we have $\pr_{B}^{-1}(b_{0}) = \{b_{0}\}\times F\cong F$. This for $f_{0}\in F$
the exact sequence looks as follows: 

\begin{equation*}
{\dots} \to \pi_{n+1}(B, b_{0}) \overset{\partial}{\lra} 
\pi_{n}(F, f_{0}) \overset{i_{\ast}}{\lra}
\pi_{n}(B\times F, (b_{0}, f_{0})) \overset{p_{\ast}}{\lra}
\pi_{n}(B, b_{0}) \overset{\partial}{\lra}
\pi_{n-1}(F,f_{0}) \overset{i_{\ast}}{\to} {\dots}
\end{equation*}

The projection map $\pr_{F}\colon B\times F \to F$ induces homomorphisms 
$\pr_{F\ast}\colon \pi_{n}(B\times F, (b_{0}, f_{0})) \to \pi_{n}(F, f_{0})$
such that $\pr_{F\ast}i_{\ast} = \id_{\pi_{n}(F, f_{0})}$. This means that 
$\Im{\partial} = \Ker i_{\ast} = 0$. Therefore for each $n\geq 1$ we obtain 
a split short exact sequence 
\begin{equation*}
0 \lra 
\pi_{n}(F, f_{0}) \stackrel[\pr_{F\ast}]{i_{\ast}}{\overrightarrow{\underleftarrow{\hspace{0.7cm}}}}
\pi_{n}(B\times F, (b_{0}, f_{0})) \overset{p_{\ast}}{\lra}
\pi_{n}(B, b_{0}) \lra 0
\end{equation*}
This shows that 
$\pi_{n}(B\times F, (b_{0}, f_{0})) \cong 
\pi_{n}(B, b_{0}) \times \pi_{n}(F, f_{0})$,
which is a special case of the product formula (\ref{PIN  PROD THM}).
\end{example}


\begin{example}
Let $p\colon E \to B$ be a covering, let $b_{0}\in B$ and let $e_{0}\in p^{-1}(b_{0})$. 
The space $F = p^{-1}(b_{0})$ is discrete, so $\pi_{n}(F) = 0$ for all 
$n \geq 1$. Therefore the exact sequence of the fibration becomes 
\begin{multline*}
{\dots}\lra
\ 0\  {\lra}
\pi_{n}(E, e_{0}) \overset{p_{\ast}}{\lra}
\pi_{n}(B, b_{0}) {\lra}
\ 0\  \lra {\dots} \\
{\dots}{\lra}\  0\  {\lra}
\pi_{1}(E, e_{0}) \overset{p_{\ast}}{\lra}
\pi_{1}(B, b_{0}) \overset{\partial}{\lra}
\pi_{0}(F, e_{0}) \overset{i_{\ast}}{\lra}
\pi_{0}(E, e_{0}) \overset{p_{\ast}}{\lra}
\pi_{0}(B, b_{0})
\end{multline*}
This shows that $p_{\ast}\colon \pi_{n}(E, e_{0}) \to \pi_{n}(B, b_{0})$
is an isomorphism for all $n\geq 2$, and it is a monomorphism for $n=1$. 
This recovers the statement of Proposition \ref{COVERING PIN PROP}. 

The image of $p_{\ast}\colon \pi_{1}(E, e_{0}) \to \pi_{1}(B, b_{0})$
coincides with $\Ker(\partial\colon \pi_{1}(B, b_{0}) \to \pi_{0}(F, e_{0}))$. 
By the definition of the map $\partial$, an element $[\omega]\in \pi_{1}(B, b_{0})$
is in $\Ker \partial$ if $\omega\colon [0, 1]\to B$ has a lift 
$\xov{\omega}\colon [0, 1]\to E$ such that $\xov{\omega}(1) = e_{0}$ and 
$\omega(1)$ is in the same path connected component of $F$ as $e_{0}$. Since 
$F$ is discrete, it means that $\xov{\omega}(1) = e_{0} = \xov{\omega}(0)$. 
As a consequence, we obtain that $\Im(p_{\ast}\colon \pi_{1}(E, e_{0}) \to \pi_{1}(B, b_{0})$
consists of elements $[\omega]\in \pi_{1}(B, b_{0})$ such that the lift of $\omega$
ending at $e_{0}$ is a loop. 
\end{example}


\begin{theorem}
\label{LOC TO GLOB SERRE FIB THM}
Let $p\colon E\to B$ be map and let  $\{U_{i}\}_{i\in I}$ be an open cover of $B$. 
Assume that for each $i\in I$ the map $p_{i}\colon p^{-1}(U_{i}) \to U_{i}$, which is
the restriction of $p$ is a Serre fibration. Then $p$ is a Serre fibration. 
\end{theorem}

\begin{note}
An analogous fact is true for Hurewicz fibrations, under the assumption that $B$ is 
a paracompact space. 
\end{note}

\begin{proof}[Proof of Theorem \ref{LOC TO GLOB SERRE FIB THM}]
See e.g. Hatcher \emph{Algebraic Topology}, Proposition 4.48 p. 379.
\end{proof}


\begin{definition} 
\label{FIBRE BUNDLE DEF}
A map $p\colon E\to B$ is a \emph{fiber bundle} with fiber $F$ if for every point 
$b\in B$ there exists an open neighborhood $b\in U\subseteq B$ and a homeomorphism 
$h_{U}\colon p^{-1}(U) \to U \times F$ such that the following 
diagram commutes: 
\begin{equation*}
\begin{tikzpicture}
\matrix (m) 
[matrix of math nodes, row sep= 2em, column sep=1.5em, text height=1.5ex, text depth=0.25ex]
{
p^{-1}(U) & & U \times F \\
& U & \\ 
};
\path[->, thick, font=\scriptsize]
(m-1-1) 
edge node[auto] {$h_{U}$} (m-1-3)
edge node[anchor=north east] {$p$} (m-2-2)
(m-1-3)
edge node[anchor= north west] {$\pr_{1}$} (m-2-2)
; 
\end{tikzpicture}
\end{equation*}
Here $\pr_{1} \colon U\times F \to U$ is the projection map $\pr_{1}(x, y) = x$. 
\end{definition}


\begin{proposition}
Every fiber bundle is a Serre fibration. 
\end{proposition}

\begin{proof}
This follows from Theorem \ref{LOC TO GLOB SERRE FIB THM} and Example \ref{PRODUCT FIBRATION EXAMPLE}.
\end{proof}

\begin{example}
Every covering space $p\colon E\to B$ is a fiber bundle whose fiber is a discrete space.
\end{example}


\begin{example}
{\color{red} Mobius band}
\end{example}

\begin{example}
{\color{red} Klein bottle}
\end{example}


\begin{example}
\label{CPN BUNDLE EXAMPLE}
Consider $S^{2n+1}$ as a subspace of the complex space $\C^{n}$:
\[
S^{2n+1} = \{(z_{0}, \dots, z_{n})\in \C^{n+1}\ | 
\ \textstyle{\sum_{i=0}^{n} \norm{z_{i}}^{2} = 1}\}
\]
In particular, $S^{1} = \{z \in \C \ | \ \norm{z} = 1\}$.
The \emph{$n$-dimensional complex projective space} is the quotient space 
\[
\CP^{n} = S^{2n+1}/\sim
\]
where $(z_{1}, \dots, z_{n})\sim  \lambda(z_{0}, \dots , z_{n})$ for all 
$\lambda \in S^{1}$. We will denote by $[z_{0}, \dots, z_{n}]\in \CP^{n}$
the equivalence class of $(z_{0}, \dots, z_{n})$. 
Let $p\colon S^{2n+1} \to \CP^{n}$ be the quotient map
$p(z_{0}, \dots, z_{n}) = [z_{0}, \dots, z_{n}]$.

We will show that $p\colon S^{2n+1} \to \CP^{n}$ is a fiber bundle with fiber 
$S^{1}$. Let $b = [z_{0}, \dots, z_{n}]\in \CP^{n}$. Choose $0\leq i\leq n$ 
such that $z_{i}\neq 0$, 
and take $U_{i} = \{[w_{0}, \dots, w_{n}]\in \CP^{n} \ |\  w_{i}\neq 0\}$. This 
set is an open neighborhood of $b$ in $\CP^{n}$. We have  
\[
p^{-1}(U_{i}) = \{(w_{0}, \dots, w_{n})\in S^{2n+1} \ | \ w_{i}\neq 0\}
\]
Define a map $h_{i}\colon p^{-1}(U_{i}) \to U_{i}\times S^{1}$ by
$h_{i}(w_{0}, \dots, w_{n}) = ([w_{0}, \dots, w_{n}], w_{i}/\norm{w_{i}})$. 
This is a homeomorphism, with the inverse given by 
\[
h_{i}^{-1}([v_{0}, \dots, v_{n}], \lambda) = 
\frac{\norm{v_{i}}}{v_{i}}\cdot \lambda \cdot(v_{0}, \dots, v_{n})
\]

Let $n\geq 1$. The long exact sequence of the bundle $p\colon S^{2n+1} \to \CP^{n}$
has the form 
\begin{multline*}
{\dots}\lra 
\pi_{m}(S^{1}) \overset{i_{\ast}}{\lra}
\pi_{m}(S^{2n+1}) \overset{p_{\ast}}{\lra}
\pi_{m}(\CP^{n}) \overset{\partial}{\lra} 
\pi_{m-1}(S^{1})\lra{\dots} \\
{\dots}\lra 
\pi_{2}(S^{2n+1}) \overset{p_{\ast}}{\lra}
\pi_{2}(\CP^{n}) \overset{\partial}{\lra}
\pi_{1}(S^{1}) \overset{i_{\ast}}{\lra}
\pi_{1}(S^{2n+1})\overset{p_{\ast}}{\lra} 
\pi_{1}(\CP^{n}, b_{0}) \overset{\partial}{\lra} 
\pi_{0}(S^{1}) = 0
\end{multline*}
Since $\pi_{m}(S^{1}) = 0$ for $m >1$, 
we obtain that  $\pi_{m}(\CP^{n})\cong \pi_{m}(S^{2n+1})$ for $m\geq 3$. 
Also, since $\pi_{1}(S^{1})\cong \Z$ and 
$\pi_{m}(S^{2n + 1}) = 0$ for $m<2n+1$, thus 
$\pi_{2}(\CP^{n}) \cong \pi_{1}(S^{1})\cong \Z$ and $\pi_{1}(\CP^{n}) = 0$. 
\end{example}

\begin{example}
\label{HOPF BUNDLE EXAMPLE}
As a special case of Example \ref{CPN BUNDLE EXAMPLE}, take $n=1$. 
In this case, we have a homeomorphism $\CP^{1} \cong S^{2}$. 
To see this, define a map $h\colon \CP^{1}\setminus \{[1, 0]\}\to \C$ by 
$h([z_{0}, z_{1}]) = \frac{z_{0}}{z_{1}}$. This is a homeomorphism 
with the inverse given by $h^{-1}(z) = \frac{1}{1 + \norm{z}}\cdot [z, 1]$. 
Since $S^{2}$ is homeomorphic to the one-point compactification of $\C$, i.e. 
$S^{2}\cong \C \cup \{\infty\}$, the map $h$ can be extended to a homeomomrphism 
$h\colon  \CP^{1}\to S^{2}$ by setting $h([1, 0]) = \infty$.  

Under the identification $\CP^{1}\cong S^{2}$ the bundle 
$S^{1}\to S^{3}\overset{p}{\to} \CP^{1}$ becomes $S^{1}\to S^{3}\overset{p}{\to} S^{2}$. 
This bundle is called the \emph{Hopf bundle} (or the \emph{Hopf fibration}). 
\end{example}

Using the long exact sequence of the Hopf fibration we obtain:

\begin{theorem}
\label{PI2 S2 THM}
$\pi_{2}(S^{2}) \cong \Z$.
\end{theorem}


