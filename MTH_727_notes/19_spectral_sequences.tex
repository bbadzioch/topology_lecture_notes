% !TEX root = mth727_lecture_notes.tex


\chapter[Spectral Sequences]{Spectral Sequences}
\chaptermark{Spectral Sequences}
\label{SPECTRAL SEQUENCES CHAPTER}
\thispagestyle{firststyle}

\begin{nn}{\bf Motivation.}
Hurewicz Isomorphism Theorem \ref{HUREWICZ ISOMORPHISM THM} lets us compute
the first non-trivial homotopy group of a space $X$ using homological methods. 
In order to extend this to to higher homotopy groups of $X$, one can attempt the following 
approach. Assume that $\pi_{k}(X) = 0$ for $k < n$ and that we know 
homology groups of $X$. This in particular gives us $\pi_{n}(X) \cong H_{n}(X)$.
We can construct a map $f_{n}\colon X \to K(\pi_{n}(X), n)$ which induces an isomorphism 
on $n$-th homotopy groups. Let $X_{n}$ denote the homotopy fiber of $f_{n}$. 
The long exact sequence of a fibration shows that $\pi_{k}(X_{n}) = 0$ for $k < n+1$
and $\pi_{k}(X) \cong \pi_{k}(X_{n})$ for $k\geq n+1$. In particular using 
the Hurewicz Isomorphism Theorem we obtain 
\[
\pi_{n+1}(X) \cong \pi_{n+1}(X_{n}) \cong H_{n+1}(X_{n})
\]
Thus computations of $\pi_{n+1}(X)$ are reduced to computing a homology group of
the space $X_{n}$.

This procedure can be repeated: once we know $ \pi_{n+1}(X_{n})$, 
we can construct a map $f_{n+1} \colon X_{n} \to K(\pi_{n+1}(X_{n}), n+1)$
that induces an isomorphism on $(n+1)$-st homotopy groups. Taking $X_{n+1}$
to be the homotopy fiber of this map we obtain isomorphisms
\[
\pi_{n+2}(X) \cong \pi_{n+2}(X_{n}) \cong \pi_{n+2}(X_{n+1}) \cong H_{n+2}(X_{n+1})
\]
Proceeding recursively, we obtain that in order to compute homotopy groups of $X$ 
it suffices to compute homology groups of spaces $X_{k}$ for $k\geq n-1$ such 
that $X_{n-1} = X$ and which are connected by fibration sequences
\[
X_{k+1} \to X_{k} \to K(\pi_{k+1}(X), k+1)
\]   
In order to carry out this program we would need to; 
\bit
\item calculate homology groups of Eilenberg-MacLane spaces $K(G, k)$;
\item given a fibration sequence $F \to E \to B$ find a relationship between 
homology groups of the spaces $F$, $E$, and $B$. 
\eit
\end{nn}
\vskip -6mm

Spectral sequences provide a tool for achieving the second of these objectives. 
They are helpful with the first one as well.



In this chapter we give the definition of
a spectral sequence and some examples how spectral sequences are used. 
Explanation in which circumstances spectral sequences occur 
is left for later.

\begin{definition}
A \emph{bigraded} abelian group $G_{\ast\ast}$ is a collection of abelian groups
$G_{p, q}$ for $p, q\in \Z$.  
\end{definition}

\begin{definition}
A \emph{(first quadrant, homological) spectral sequence} $(E^{r}_{\ast\ast}, d^{r})$ 
is a sequence of bigraded abelian groups 
$E_{\ast\ast}^{r}$ for $r = 1, 2, \dots$ such that:
\benu
\item[1)] $E^{r}_{p, q} = 0$ if $p< 0$ or $q < 0$.
\item[2)] Each $E_{\ast\ast}^{r}$ is equipped with homomorphisms (\emph{differentials})
\[
d^{r}\colon E^{r}_{p, q} \to E^{r}_{p-r, q+r-1}
\]
satisfying $d^{r}d^{r} = 0$.
\item[3)] For each $r\geq 0$ we have $E^{r+1}_{p, q} \cong H_{p, q}(E^{r}_{\ast\ast})$
where 
\[
H_{p, q}(E^{r}_{\ast\ast})
= \frac{\Ker(d^{r}\colon E^{r}_{p, q} \to E^{r}_{p-r, q+r-1})}
{\Im(d^{r}\colon E^{r}_{p+r, q-r+1} \to E^{r}_{p, q})}
\]
\eenu
\end{definition}

\begin{note}
The bigraded group $E^{r}_{\ast\ast}$ is called the \emph{$r$-th page} of the spectral sequence.
\end{note}

Below are pictures of the first three pages of a spectral sequence. 
Notice that the differentials $d^{r}$ always go between  groups 
$E^{r}_{p, q}$ where $p+q = n$ for some $n$ and groups where $p+q=n-1$. 

\begin{equation*}
\begin{tikzpicture}[scale=0.8]
\matrix (m) [matrix of math nodes,
    nodes in empty cells,
    row sep=0.5em, 
    column sep=0.5em,
    nodes={minimum width=5ex, anchor=center, minimum height=5ex,outer sep=-1pt, font=\footnotesize}
    ]{          
  &[-3mm]  \vdots &  \vdots       & \vdots        &  \vdots       &  \vdots &[0mm]  \\[-3mm]
2 &  E^{1}_{2, 0} &  E^{1}_{2, 1} & E^{1}_{2, 2}  &  E^{1}_{2, 3} &  E^{1}_{2, 3}& \cdots \\
1 &  E^{1}_{1, 0} &  E^{1}_{1, 1} & E^{1}_{1, 2}  &  E^{1}_{1, 3} &  E^{1}_{2, 3}& \cdots \\
0 &  E^{1}_{0, 0} &  E^{1}_{0, 1} & E^{1}_{0, 2}  &  E^{1}_{0, 3} &  E^{1}_{2, 3}& \cdots \\[-5mm]
  &   0  &  1  &  2  &  3 & 4 & \\};

\path[->, font=\scriptsize]
(m-4-3.west) edge   (m-4-2.east)
(m-3-3.west) edge   (m-3-2.east)
(m-2-3.west) edge   (m-2-2.east)
(m-4-4.west) edge   (m-4-3.east)
(m-3-4.west) edge   (m-3-3.east)
(m-2-4.west) edge   (m-2-3.east)
(m-4-5.west) edge   (m-4-4.east)
(m-3-5.west) edge   (m-3-4.east)
(m-2-5.west) edge   (m-2-4.east)
(m-4-6.west) edge   (m-4-5.east)
(m-3-6.west) edge   (m-3-5.east)
(m-2-6.west) edge   (m-2-5.east)
(m-4-7.west) edge   (m-4-6.east)
(m-3-7.west) edge   (m-3-6.east)
(m-2-7.west) edge   (m-2-6.east)
;
\draw[line width=3mm, color=red, opacity=0.2, shorten >= 2.5mm, shorten <=  0.6mm] 
(m-2-6.north west) -- (m-2-6.south east);
\draw[line width=3mm, color=red, opacity=0.2, shorten >= 2.5mm, shorten <=  0mm] 
(m-4-2.north west) -- (m-4-2.south east);
\draw[line width=3mm, color=red, opacity=0.2, shorten >=-3mm, shorten <= -4mm] 
(m-3-2.center) -- (m-4-3.center);
\draw[line width=3mm, color=red, opacity=0.2, shorten >=-3mm, shorten <= -4mm] 
(m-2-2.center) -- (m-4-4.center);
\draw[line width=3mm, color=red, opacity=0.2, shorten >=-3mm, shorten <= -4mm] 
(m-2-3.center) -- (m-4-5.center);
\draw[line width=3mm, color=red, opacity=0.2, shorten >=-3mm, shorten <= -4mm] 
(m-2-4.center) -- (m-4-6.center);
\draw[line width=3mm, color=red, opacity=0.2, shorten >= 2.3mm, shorten <=  1.5mm] 
(m-2-5.north west) -- (m-3-6.south east);

\draw[very thick] ([yshift=2mm]m-1-1.east) -- ([yshift=-1mm]m-5-1.east);
\draw[very thick] ([yshift=-1mm]m-5-1.north) -- ([yshift=-1mm, xshift=3mm]m-5-7.north);
\end{tikzpicture}
\end{equation*}


\begin{equation*}
\begin{tikzpicture}[scale=0.8]
\matrix (m) [matrix of math nodes,
    nodes in empty cells,
    row sep=0.5em, 
    column sep=0.5em,
    nodes={minimum width=5ex, anchor=center, minimum height=5ex,outer sep=-1pt, font=\footnotesize}
    ]{          
  &[-3mm]  \vdots &  \vdots       & \vdots        &  \vdots       &  \vdots &[0mm]  \\[-3mm]
2 &  E^{2}_{2, 0} &  E^{2}_{2, 1} & E^{2}_{2, 2}  &  E^{2}_{2, 3} &  E^{2}_{2, 3}& \cdots \\
1 &  E^{2}_{1, 0} &  E^{2}_{1, 1} & E^{2}_{1, 2}  &  E^{2}_{1, 3} &  E^{2}_{2, 3}& \cdots \\
0 &  E^{2}_{0, 0} &  E^{2}_{0, 1} & E^{2}_{0, 2}  &  E^{2}_{0, 3} &  E^{2}_{2, 3}& \cdots \\[-5mm]
  &   0  &  1  &  2  &  3 & 4 & \\};
\path[->, font=\scriptsize]
(m-3-6) edge[shorten >=-1mm, shorten <= -1mm]    (m-2-4)
(m-4-6) edge[shorten >=-1mm, shorten <= -1mm]    (m-3-4)
(m-4-4) edge[shorten >=-1mm, shorten <= -1mm]    (m-3-2)
(m-4-5) edge[shorten >=-1mm, shorten <= -1mm]    (m-3-3)
(m-3-5) edge[shorten >=-1mm, shorten <= -1mm]    (m-2-3)
(m-3-4) edge[shorten >=-1mm, shorten <= -1mm]    (m-2-2)
(m-3-7) edge[shorten >=-1mm, shorten <= -1mm]    (m-2-5)
(m-4-7) edge[shorten >=-1mm, shorten <= -1mm]    (m-3-5)
;
\draw[line width=3mm, color=red, opacity=0.2, shorten >= 2.5mm, shorten <=  0.6mm] 
(m-2-6.north west) -- (m-2-6.south east);
\draw[line width=3mm, color=red, opacity=0.2, shorten >= 2.5mm, shorten <=  0mm] 
(m-4-2.north west) -- (m-4-2.south east);
\draw[line width=3mm, color=red, opacity=0.2, shorten >=-3mm, shorten <= -4mm] 
(m-3-2.center) -- (m-4-3.center);
\draw[line width=3mm, color=red, opacity=0.2, shorten >=-3mm, shorten <= -4mm] 
(m-2-2.center) -- (m-4-4.center);
\draw[line width=3mm, color=red, opacity=0.2, shorten >=-3mm, shorten <= -4mm] 
(m-2-3.center) -- (m-4-5.center);
\draw[line width=3mm, color=red, opacity=0.2, shorten >=-3mm, shorten <= -4mm] 
(m-2-4.center) -- (m-4-6.center);
\draw[line width=3mm, color=red, opacity=0.2, shorten >= 2.3mm, shorten <=  1.5mm] 
(m-2-5.north west) -- (m-3-6.south east);

\draw[very thick] ([yshift=2mm]m-1-1.east) -- ([yshift=-1mm]m-5-1.east);
\draw[very thick] ([yshift=-1mm]m-5-1.north) -- ([yshift=-1mm, xshift=3mm]m-5-7.north);
\end{tikzpicture}
\end{equation*}

\begin{equation*}
\begin{tikzpicture}[scale=0.8]
\matrix (m) [matrix of math nodes,
    nodes in empty cells,
    row sep=0.5em, 
    column sep=0.5em,
    nodes={minimum width=5ex, anchor=center, minimum height=5ex,outer sep=-1pt, font=\footnotesize}
    ]{          
  &[-3mm]  \vdots &  \vdots       & \vdots        &  \vdots       &  \vdots &[0mm]  \\[-3mm]
2 &  E^{3}_{2, 0} &  E^{3}_{2, 1} & E^{0}_{2, 2}  &  E^{3}_{2, 3} &  E^{3}_{2, 3}& \cdots \\
1 &  E^{3}_{1, 0} &  E^{3}_{1, 1} & E^{0}_{1, 2}  &  E^{3}_{1, 3} &  E^{3}_{2, 3}& \cdots \\
0 &  E^{3}_{0, 0} &  E^{3}_{0, 1} & E^{0}_{0, 2}  &  E^{3}_{0, 3} &  E^{3}_{2, 3}& \cdots \\[-5mm]
  &   0  &  1  &  2  &  3 & 4 & \\};

\path[->, font=\scriptsize]
(m-4-5) edge  (m-2-2)
(m-4-6) edge  (m-2-3)
(m-4-7) edge  (m-2-4)
;
\draw[line width=3mm, color=red, opacity=0.2, shorten >= 2.5mm, shorten <=  0.6mm] 
(m-2-6.north west) -- (m-2-6.south east);
\draw[line width=3mm, color=red, opacity=0.2, shorten >= 2.5mm, shorten <=  0mm] 
(m-4-2.north west) -- (m-4-2.south east);
\draw[line width=3mm, color=red, opacity=0.2, shorten >=-3mm, shorten <= -4mm] 
(m-3-2.center) -- (m-4-3.center);
\draw[line width=3mm, color=red, opacity=0.2, shorten >=-3mm, shorten <= -4mm] 
(m-2-2.center) -- (m-4-4.center);
\draw[line width=3mm, color=red, opacity=0.2, shorten >=-3mm, shorten <= -4mm] 
(m-2-3.center) -- (m-4-5.center);
\draw[line width=3mm, color=red, opacity=0.2, shorten >=-3mm, shorten <= -4mm] 
(m-2-4.center) -- (m-4-6.center);
\draw[line width=3mm, color=red, opacity=0.2, shorten >= 2.3mm, shorten <=  1.5mm] 
(m-2-5.north west) -- (m-3-6.south east);

\draw[very thick] ([yshift=2mm]m-1-1.east) -- ([yshift=-1mm]m-5-1.east);
\draw[very thick] ([yshift=-1mm]m-5-1.north) -- ([yshift=-1mm, xshift=3mm]m-5-7.north);
\end{tikzpicture}
\end{equation*}


Since all groups $E^{r}_{p, q}$ with negative $p$ or $q$ are trivial, the differentials 
$d^{r}$ originating at $E^{r}_{p, q}$ are trivial for $r > p$. Likewise, the differentials 
$d^{r}$ terminating at $E^{r}_{p, q}$ are trivial if $r > q+1$. As a consequence, 
for $r \leq \text{max(p+1, q+2)}$ we get
\[
E^{r}_{p, q} = E^{r+1}_{p, q} = E^{r+2}_{p, q} = \dots
\]
For each $p, q$, let $E^{\infty}_{p, q}$ denote this recurring group. These groups 
form a bigraded group $E^{\infty}_{\ast\ast}$.  

In typical applications of spectral sequences, $E^{\infty}_{\ast\ast}$ is related to some 
object of interest, e.g. homology groups of some space. This is done as follows. 
We start with a graded abelian group $H_{\ast}$ i.e. a collection of abelian groups 
$H_{n}$ for $n\in \Z$. A \emph{filtration} of $H_{\ast}$ is a sequence of graded subgrops: 
\[
0 = F_{-1}H_{\ast} \subseteq F_{0}H_{\ast} \subseteq F_{1}H_{\ast} 
\subseteq {\dots} \subseteq H_{\ast}
\]
such that $\bigcup_{p=0}^{\infty} F_{p}H_{\ast} = H_{\ast}$. 

\begin{definition}
We say that a spectral sequence $(E^{r}_{\ast\ast}, d^{r})$ \emph{converges} to 
a graded group $H_{\ast}$ if there exists a filtration of $H_{\ast}$ such that 
\[
E^{\infty}_{p, q} \cong F_{p}H_{p+q}/F_{p-1}H_{p+q}
\]
for all $p, q$.
\end{definition}


Results on existence spectral sequences usually say that there exists
a spectral sequence for which we can say describe in some useful way groups 
$E^{r}_{p, q}$ for some fixed $r$, and that this sequence converges to some 
interesting graded group $H_{\ast}$. Here is one example of such a statement: 

\begin{theorem}
\label{SERRE SS THM}
Let $p\colon E \to B$ be a Serre fibration and let $F = p^{-1}(b_{0})$ for some $b_{0}\in B$. 
If the space $B$ is simply connected then there exists a spectral sequence 
$(E^{r}_{\ast\ast}, d^{r})$ such that 
\[
E^{2}_{p, q} \cong H_{p}(B, H_{q}(F))
\]
for all $p, q$, and which converges to $H_{\ast}(E)$. 
\end{theorem}

The spectral sequence described in this theorem is called the 
\emph{Serre spectral sequence} of the fibration $p$.


The next result provides an example how spectral sequences are used in 
computations.
 
\begin{theorem}
\label{HOMOLOGY LOOP SPHERE THM}
If $n\geq 2$ then 
\[
H_{m}(\Omega S^{n}) \cong 
\begin{cases}
\Z & \text{ if } (n-1)|m \\
0  & \text{otherwise}
\end{cases}
\]
\end{theorem}

\begin{proof}
The space $\Omega S^{n}$ is the fiber of a Serre fibration $p\colon P \to S^{n}$
with a contractible space $P$. Consider the Serre spectral sequence of this fibration. 
We have 
\[
E^{2}_{p, q} \cong H_{p}(S^{n}, H_{q}(\Omega S^{n})) \cong 
\begin{cases}
H_{q}(\Omega S^{n}) & \text{ if } p=0, n \\
0 & \text{otherwise}
\end{cases}
\]
For example, for $n=4$ the second page of this spectral sequence looks as follows:
\begin{equation*}
\begin{tikzpicture}[scale=0.8]
\matrix (m) [matrix of math nodes,
    nodes in empty cells,
    row sep=0.1em, 
    column sep=0.5em,
    nodes={minimum width=5ex, anchor=center, minimum height=5ex,outer sep=-1pt,
    font=\footnotesize}
    ]{          
5 &  H_{5}(\Omega S^{4}) &   &    &   &  H_{5}(\Omega S^{4}) \\
4 &  H_{4}(\Omega S^{4}) &   &    &   &  H_{4}(\Omega S^{4}) \\
3 &  H_{3}(\Omega S^{4}) &   &    &   &  H_{3}(\Omega S^{4}) \\
2 &  H_{2}(\Omega S^{4}) &   &    &   &  H_{2}(\Omega S^{4}) \\
1 &  H_{1}(\Omega S^{4}) &   &    &   &  H_{1}(\Omega S^{4}) \\
0 &  H_{0}(\Omega S^{4}) &   &    &   &  H_{0}(\Omega S^{4})  \\[-3mm]
  &   0  &  1  &  2  &  3 & 4 & \\};
\path[->, thick, font=\scriptsize]
(m-6-6) edge[shorten >=-1mm, shorten <= -1mm] node[above]{$d_{4}$}    (m-3-2)
(m-5-6) edge[shorten >=-1mm, shorten <= -1mm] node[above]{$d_{4}$}   (m-2-2)
(m-4-6) edge[shorten >=-1mm, shorten <= -1mm] node[above]{$d_{4}$}   (m-1-2)

;
\draw[very thick] ([yshift=2mm]m-1-1.east) -- ([yshift=-1mm]m-7-1.east);
\draw[very thick] ([yshift=-1mm]m-7-1.north) -- ([yshift=-1mm, xshift=3mm]m-7-7.north);
\end{tikzpicture}
\end{equation*}
All differentials in the spectral sequence are trivial, except, possibly
$d^{n}\colon E^{n}_{p, q} \to E^{n}_{0, q+n-1}$. It follows that 
$E^{2}_{\ast\ast} = E^{n}_{\ast\ast}$ and 
$E^{n+1}_{\ast\ast} = E^{\infty}_{\ast\ast}$.
The total space $P$ of the fibration is contractible, so
$H_{0}(P) = \Z$ and $H_{p}(P) = 0$ for $p> 0$. By Theorem \ref{SERRE SS THM}
we have $E^{\infty}_{p, q} \cong F_{p}H_{p+q}(P)/F_{p-1}H_{p+q}(P)$ for some 
filtration $\{F_{p}H_{\ast}(P)\}$ of $H_{\ast}(P)$. It follows that 
\[
E^{n+1}_{p, q} = E^{\infty}_{p, q} \cong
\begin{cases}
\Z & \text{ if } (p, q) = (0, 0) \\
0 & \text{otherwise}
\end{cases}
\]
Since $E^{n+1}_{p, q} \cong H_{p, q}(E^{2}_{\ast\ast})$ we obtain that 
$H_{0}(\Omega S^{n}) \cong \Z$ and $H_{p}(\Omega S^{n}) = 0$ for $0 < p \leq n-2$. 
Also, all differentials $d^{n}$ must be isomorphisms. 
This gives:
\[
H_{p}(\Omega S^{n}) \cong H_{p+(n-1)}(\Omega S^{n}) \cong H_{p+2(n-1)}(\Omega S^{n}) 
\cong H_{p+3(n-1)}(\Omega S^{n}) \cong
\dots 
\]
Taking $p=0$ we obtain that $H_{m}(\Omega S^{n})\cong \Z$ if $(n-1)|m$. In all other cases
$H_{m}(\Omega S^{n}) \cong H_{p}(\Omega S^{n})$ for some $0< p \leq n-2$, and so it is 
a trivial group. 
\end{proof}

\begin{note}
The proof of Theorem \ref{HOMOLOGY LOOP SPHERE THM} used the observation 
that all differentials $d^{r}$ in the Serre spectral sequence of the fibration 
$p\colon P \to S^{n}$ were trivial for $r\geq n+1$. 
A situation like this appears frequently in computations involving spectral sequences, 
which motivates the next definition.
\end{note}

\begin{definition}
We say that a spectral sequence \emph{collapses} at the page $r_{0}$ if all differentials 
$d^{r}$ are trivial for $r\geq r_{0}$.
\end{definition}

If a spectral sequence collapses at the page $r_{0}$ then we have 
$E^{r_{0}}_{p, q} = E^{r_{0}+1}_{p, q} = {\dots} = E^{\infty}_{p, q}$.






